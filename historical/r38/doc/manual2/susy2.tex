\chapter{SUSY2: Super Symmetry}
\label{SUSY2}
\typeout{{SUSY2: Super Symmetry}}

{\footnotesize
\begin{center}
Ziemowit Popowicz \\
Institute of Theoretical Physics, University of Wroclaw\\
pl. M. Borna 9 50-205 Wroclaw, Poland \\
e-mail: ziemek@ift.uni.wroc.pl
\end{center}
}
\ttindex{SUSY2}


This package deals with supersymmetric functions and with algebra
of supersymmetric operators in the extended N=2 as well as in the
nonextended N=1 supersymmetry. It allows us
to make the realization of SuSy algebra of differential operators,
compute the gradients of given SuSy Hamiltonians and to obtain
SuSy version of soliton equations using the SuSy Lax approach. There
are also many additional procedures encountered in the SuSy soliton
approach, as for example: conjugation of a given SuSy operator, computation
of general form of SuSy Hamiltonians (up to SuSy-divergence equivalence),
checking of the validity of the Jacobi identity for some SuSy
Hamiltonian operators.

To load the package, type \quad {\tt load susy2;} \\
\\
For full explanation and further examples, please refer to the 
detailed documentation and the susy2.tst which comes with this package.

\section{Operators}

\subsection{Operators for constructing Objects}

The superfunctions are represented in this package by \f{BOS}(f,n,m) for superbosons
and \f{FER}(f,n,m) for superfermions. The first index denotes the name of the given
superobject, the second denotes the value of SuSy derivatives, and the last gives the
value of usual derivative. \\
In addition to the definitions of the superfunctions, also the inverse and the exponential 
of superbosons can be defined (where the inverse is defined as \f{BOS}(f,n,m,-1)
with the property {\it bos(f,n,m,-1)*bos(f,n,m,1)=1}). The exponential of the superboson
function is \f{AXP}(\f{BOS}(f,0,0)). \\
The operator \f{FUN} and \f{GRAS} denote the classical and the Grassmann function. \\
Three different realizations of supersymmetric derivatives are implemented. To select
traditional realization declare \f{LET TRAD}. In order to select chiral or chiral1 algebra
declare \f{LET CHIRAL} or \f{LET CHIRAL1}. For usual differentiation the operator
\f{D}(1) stands for right and \f{D}(2) for left differentiation. SuSy derivatives are 
denoted as {\it der} and {\it del}. \f{DER} and \f{DEL} are one component argument operations
and represent the left and right operators. The action of these operators on the 
superfunctions depends on the choice of the supersymmetry algebra. 

\flushleft
{\small\begin{center}
\begin{tabular}{ l l l l l l}
 \f{BOS}(f,n,m)\ttindex{BOS} & \f{BOS}(f,n,m,k)\ttindex{BOS} & 
 \f{FER}(f,n,m)\ttindex{FER} & \f{AXP}(f)\ttindex{AXP} &
 \f{FUN}(f,n)\ttindex{FUN} & \f{FUN}(f,n,m)\ttindex{FUN} \cr
 \f{GRAS}(f,n)\ttindex{GRAS} & \f{AXX}(f)\ttindex{AXX}  &
 \f{D}(1)\ttindex{D} & \f{D}(2)\ttindex{D} &
 \f{D}(3)\ttindex{D}  & \f{D}(-1)\ttindex{D} \cr
 \f{D}(-2)\ttindex{D}   & \f{D}(-3)\ttindex{D} &
 \f{D}(-4)\ttindex{D} & \f{DR}(-n)\ttindex{DR} &
 \f{DER}(1)\ttindex{DER} & \f{DER}(2)\ttindex{DER} \cr
 \f{DEL}(1)\ttindex{DEL} & \f{DEL}(2)\ttindex{DEL}
\end{tabular}
\end{center} }
\vspace{1cm}

{\bf Example}:
\begin{verbatim}
1: load susy2;  

2: bos(f,0,2,-2)*axp(fer(k,1,2))*del(1);  %first susy derivative

2*fer(f,1,2)*bos(f,0,2,-3)*axp(fer(k,1,2))  

 - bos(k,0,3)*bos(f,0,2,-2)*axp(fer(k,1,2))

 + del(1)*bos(f,0,2,-2)*axp(fer(k,1,2))

3: sub(del=der,ws);

bos(f,0,2,-2)*axp(fer(k,1,2))*der(1)

\end{verbatim}

\subsection{Commands}

There are plenty of operators on superfunction objects. Some of them are introduced  
here briefly. 
\begin{itemize}
\item By using the operators \f{FPART}, \f{BPART}, \f{BFPART} and \f{BF\_PART} 
      it is possible to compute the coordinates of the arbitrary SuSy expressions. 
\item With \f{W\_COMB}, \f{FCOMB} and \f{PSE\_ELE} there are three operators to be able to 
      construct different possible combinations of superfunctions and 
      super-pseudo-differential elements with the given conformal dimensions . 
\item The three operators \f{S\_PART}, \f{D\_PART} and \f{SD\_PART} are implemented to 
      obtain the components of the (pseudo)-SuSy element.
\item \f{RZUT} is used to obtain the projection onto the invariant subspace (with respect
      to commutator) of algebra of pseudo-SuSy-differential algebra.
\item To obtain the list of the same combinations of some superfunctions and (SuSy) 
      derivatives from some given operator-valued expression, the operators
      \f{LYST}, \f{LYST1} and \f{LYST2} are constructed.
\end{itemize}


\begin{center}
\begin{tabular}{ l l}
 \f{FPART}(expression)\ttindex{FPART} &
 \f{BPART}(expression)\ttindex{BPART} \cr 
 \f{BF\_PART}(expression,n)\ttindex{BF\_PART} &
 \f{B\_PART}(expression,n)\ttindex{B\_PART} \cr
 \f{PR}(n,expression)\ttindex{PR} &
 \f{PG}(n,expression)\ttindex{PG} \cr
 \f{W\_COMB}(\{\{f,n,x\},...\},m,z,y)\ttindex{W\_COMB} &
 \f{FCOMB}(\{\{f,n,x\},...\},m,z,y)\ttindex{FCOMB} \cr
 \f{PSE\_ELE}(n,\{\{f,n\},...\},z)\ttindex{PSE\_ELE} \cr
 \f{S\_PART}(expression,n)\ttindex{S\_PART} &
 \f{D\_PART}(expression,n)\ttindex{D\_PART} \cr
 \f{SD\_PART}(expression,n,m)\ttindex{SD\_PART} &
 \f{CP}(expression)\ttindex{CP} \cr
 \f{RZUT}(expression,n)\ttindex{RZUT} &
 \f{LYST}(expression)\ttindex{LYST} \cr
 \f{LYST1}(expression)\ttindex{LYST1} &
 \f{LYST2}(expression)\ttindex{LYST2} \cr
 \f{CHAN}(expression)\ttindex{CHAN} &
 \f{ODWA}(expression)\ttindex{ODWA} \cr
 \f{GRA}(expression,f)\ttindex{GRA} &
 \f{DYW}(expression,f)\ttindex{DYW} \cr
 \f{WAR}(expression,f)\ttindex{WAR} &
 \f{DOT\_HAM}(equations,expression)\ttindex{DOT\_HAM} \cr
 \f{N\_GAT}(operator,list)\ttindex{N\_GAT} &
 \f{FJACOB}(operator,list)\ttindex{FJACOB} \cr
 \f{JACOB}(operator,list,\{$\alpha,\beta,\gamma$\})\ttindex{JACOB} &
 \f{MACIERZ}(expression,x,y)\ttindex{MACIERZ} \cr
 \f{S\_INT}(number,expression,list)\ttindex{S\_INT}
\end{tabular}
\end{center}
\vspace{1cm}

{\bf Example}:
\begin{verbatim}
4: xxx:=fer(f,2,3);

xxx := fer(f,2,3)

5: fpart(xxx);  % all components

                - fun(f0,4) + 2*fun(f1,3)      gras(ff2,4)
{gras(ff2,3), ----------------------------,0, -------------}
                           2                        2
6: bpart(xxx);  % bosonic sector

     - fun(f0,4) + 2*fun(f1,3)
{0,----------------------------,0,0}
                2

9: b_part(xxx,1); %the given component in the bosonic sector

  - fun(f0,4) + 2*fun(f1,3)
----------------------------
             2
\end{verbatim}

\section{Options}
The are several options defined in this package. Please note that they are 
activated by typing \f{let $<$option$>$}. See also above. \\
The \f{TRAD}, \f{CHIRAL} and \f{CHIRAL1} select the different realizations of the
supersymmetric derivatives. By default traditional algebra is selected. \\
If the command  {\tt LET INVERSE} is used, then three indices {\it bos} objects are
transformed onto four indices objects.
\begin{center}
\begin{tabular}{ l l l l l l }
\f{TRAD}\ttindex{TRAD} & \f{CHIRAL}\ttindex{CHIRAL} &
\f{CHIRAL1}\ttindex{CHIRAL1} & \f{INVERSE}\ttindex{INVERSE} & 
\f{DRR}\ttindex{DRR} & \f{NODRR}\ttindex{NODRR}
\end{tabular}
\end{center}
\vspace{1cm}

{\bf Example}:
\begin{verbatim}
10: let inverse;

11: bos(f,0,3)**3*bos(k,3,1)**40*bos(f,0,3,-2);

bos(k,3,1,40)*bos(f,0,3,1);

12: clearrules inverse;

13: xxx:=fer(f,1,2)*bos(k,0,2,-2);

xxx := fer(f,1,2)*bos(k,0,2,-2)

14: pr(1,xxx); % first susy derivative

- 2*fer(k,1,2)*fer(f,1,2)*bos(k,0,2,-3) + bos(k,0,2,-2)*bos(f,0,3)

15: pr(2,xxx); %second susy derivative

- 2*fer(k,2,2)*fer(f,1,2)*bos(k,0,2,-3) - bos(k,0,2,-2)*bos(f,3,2)

16: clearrules trad;

17: let chiral; % changing to chiral algebra

18: pr(1,xxx);

- 2*fer(k,1,2)*fer(f,1,2)*bos(k,0,2,-3)
\end{verbatim}

