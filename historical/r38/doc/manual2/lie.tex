\chapter[LIE: Classification of Lie algebras]%
 {LIE: Functions for the classification of real n-dimensional Lie algebras}
\label{LIE}
\typeout{{LIE: Functions for the classification of real n-dimensional
Lie algebras}}

{\footnotesize
\begin{center}
Carsten and Franziska Sch\"obel\\
The Leipzig University,  Computer Science Department \\
Augustusplatz 10/11, \\
O-7010 Leipzig, Germany \\[0.05in]
e--mail: cschoeb@aix550.informatik.uni-leipzig.de
\end{center}
}
\ttindex{LIE}

{\bf LIE} is a package of functions for the classification of real
n-dimensional Lie algebras. It consists of two modules: {\bf liendmc1}
and {\bf lie1234}.

\section{liendmc1}

With the help of the functions in this module real n-dimensional Lie
algebras $L$ with a derived algebra $L^{(1)}$ of dimension 1 can be
classified.  $L$ has to be defined by its structure constants
 $c_{ij}^k$ in the basis $\{X_1,\ldots,X_n\}$ with
 $[X_i,X_j]=c_{ij}^k X_k$.  The user must define an ARRAY
LIENSTRUCIN($n,n,n$) with n being
the dimension of the Lie algebra $L$.  The structure constants
LIENSTRUCIN($i,j,k$):=$c_{ij}^k$ for $i<j$ should be given. Then the
procedure LIENDIMCOM1 can be called. Its syntax is:\ttindex{LIENDIMCOM1}
\begin{verbatim}
   LIENDIMCOM1(<number>).
\end{verbatim}
{\tt <number>} corresponds to the dimension $n$.  The procedure simplifies
the structure of $L$ performing real linear transformations.  The returned
value is a list of the form
\begin{verbatim}
   (i) {LIE_ALGEBRA(2),COMMUTATIVE(n-2)} or
   (ii) {HEISENBERG(k),COMMUTATIVE(n-k)}
\end{verbatim}
with $3\leq k\leq n$, $k$ odd.

The returned list is also stored as\ttindex{LIE\_LIST}{\tt
LIE\_LIST}. The matrix LIENTRANS gives the transformation from the
given basis $\{X_1,\ldots ,X_n\}$ into the standard basis
$\{Y_1,\ldots ,Y_n\}$: $Y_j=($LIENTRANS$)_j^k X_k$.

\section{lie1234}

This part of the package classifies real low-dimensional Lie algebras $L$
of the dimension $n:={\rm dim}\,L=1,2,3,4$. $L$ is also given by its
structure constants $c_{ij}^k$ in the basis $\{X_1,\ldots,X_n\}$ with
$[X_i,X_j]=c_{ij}^k X_k$.  An ARRAY
LIESTRIN($n,n,n$) has to be defined and LIESTRIN($i,j,k$):=$c_{ij}^k$ for
$i<j$ should be given. Then the procedure LIECLASS can be performed
whose syntax is:\ttindex{LIECLASS}
\begin{verbatim}
   LIECLASS(<number>).
\end{verbatim}
{\tt <number>} should be the dimension of the Lie algebra $L$.  The
procedure stepwise simplifies the commutator relations of $L$ using
properties of invariance like the dimension of the centre, of the
derived algebra, unimodularity {\em etc.}  The returned value has the form:
\begin{verbatim}
   {LIEALG(n),COMTAB(m)},
\end{verbatim}
where the value $m$ corresponds to the number of the standard form (basis:
$\{Y_1, \ldots ,Y_n\}$) in an enumeration scheme.

This returned value is also stored as LIE\_CLASS.  The linear
transformation from the basis $\{X_1,\ldots,X_n\}$ into the basis of
the standard form $\{Y_1,\ldots,Y_n\}$ is given by the matrix LIEMAT:
 $Y_j=($LIEMAT$)_j^k X_k$.

