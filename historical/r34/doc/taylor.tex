\documentstyle[11pt]{article}

\newcommand{\REDUCE}{{\sf REDUCE}}
\newcommand{\MACSYMA}{{\sf MACSYMA}}
\newcommand{\MAPLE}{{\sf MAPLE}}
\newcommand{\Mathematica}{{\sf Mathematica}}
\newcommand{\PSL}{{\sf PSL}}
\renewcommand{\today}{October 5, 1990}

%%%%%%%%%%%%%%%%%%%%%%%%%%%%%%%%%%%%%%%%%%%%%%%%%%%%%%%%%%%%%%%%%
%
%    Warning: an overfull \hbox is produced in the first
%       verbatim environment.
%
%%%%%%%%%%%%%%%%%%%%%%%%%%%%%%%%%%%%%%%%%%%%%%%%%%%%%%%%%%%%%%%%%

\begin{document}

\title{A \REDUCE{} package\\
       for manipulation of Taylor series}

\author{Rainer Sch\"opf\\
        Konrad-Zuse-Zentrum f\"ur Informationstechnik Berlin\\
        Heilbronner Str.\ 10\\
        1000 Berlin 31\\
        Federal Republic of Germany\\
        Email: \verb|<SCHOEPF@SC.ZIB-Berlin.DE>|}

\maketitle

\begin{abstract}
  This short note describes a package of \REDUCE{} procedures
  that allow Taylor expansion in one or more variables and efficient
  manipulation of the resulting Taylor series. Capabilities include
  basic operations (addition, subtraction, multiplication
  and division) and also application of certain algebraic and 
  transcendental functions.
\end{abstract}

\section{Introduction}

The Taylor package was written to provide \REDUCE{} with some of
the facilities
that \MACSYMA's \verb+TAYLOR+ function offers,
but most of all I needed it to be faster and
more space-efficient.
Especially I wanted procedures that would return the logarithm or
arc tangent of a Taylor series, again as a Taylor series.
This turned out be more work than expected. The features absolutely
required were (as usual) those that were hardest to implement,
e.g., arc tangent applied to a Taylor expansion in more than
one variable.

This package is still undergoing development.
I'll be happy if it is of any use for you.
Tell me if you think that there is something missing.
I invite everybody to criticize and comment and will eagerly try to 
correct any errors found.

\section{How to use it}

The most important operator is `\verb+TAYLOR+'.
It is used as follows:
\begin{verbatim}
  TAYLOR(exp:exprn[,var:kernel,var0:exprn,order:integer]...):exprn
\end{verbatim}
where \verb|exp| is the expression to be expanded. It can be any
\REDUCE{} object, even an expression containing other Taylor kernels.
\verb|var| is the kernel with respect to
which \verb|exp| is to be expanded. \verb|var0| denotes the point
about
which and \verb|order| the order up to which expansion is to take
place.
If more than one (\verb|var|,\verb|var0|,\verb|order|) triple is
specified \verb|TAYLOR| will expand its first argument independently
with respect to all the variables.
For example,
\begin{verbatim}
  TAYLOR(E^(X^2+Y^2),X,0,2,Y,0,2);
\end{verbatim}
will calculate the Taylor expansion up to order $X^{2}*Y^{2}$.
Note that once the expansion has been done it is not possible to 
calculate higher orders.
Instead of a kernel, \verb|var| may also
be a list of kernels. In this case expansion will take place in a way
so that the {\em sum\/} of the degrees of the kernels does not exceed
\verb|order|.
If \verb|var0| evaluates to the special identifier \verb|INFINITY|
\verb|TAYLOR| tries to expand \verb|exp| in a series in \verb|1/var|.

The expansion is performed variable per variable, i.e.\ in the example
above by first expanding $\exp(x^{2}+y^{2})$ with respect to $x$ and
then expanding every coefficient with respect to $y$.

If the switch \verb|TAYLORKEEPORIGINAL| is set to \verb|ON| the
original expression \verb|exp| is kept for later reference.
It can be recovered by means of the operator
\begin{verbatim}
  TAYLORORIGINAL(exp:exprn):exprn
\end{verbatim}
An error is signalled if \verb|exp| is not a Taylor kernel or
if the original expression was not kept, i.e.\ if 
\verb|TAYLORKEEPORIGINAL| was \verb|OFF| during expansion.
The template of a Taylor kernel, i.e.\ the list of all
variables with respect to which expansion took place together with
expansion point and order can be extracted using
\begin{verbatim}
  TAYLORTEMPLATE(exp:exprn):list
\end{verbatim}
This returns a list of lists with the three elements
\verb|{var,var0,order}|.
As with \verb|TAYLORORIGINAL|,
an error is signalled if \verb|exp| is not a Taylor kernel.
\begin{verbatim}
  TAYLORTOSTANDARD(exp:exprn):exprn
\end{verbatim}
converts all Taylor kernels in \verb|exp| into standard form and
resimplifies the result.
\begin{verbatim}
  TAYLORSERIESP(exp:exprn):boolean
\end{verbatim}
may be used to determine if \verb|exp| is a Taylor kernel.
Note that this operator is subject to the same restrictions as,
e.g., \verb|ORDP| or \verb|NUMBERP|, i.e.\ it may only be 
used in boolean expressions in \verb|IF| or \verb|LET| statements.
Finally there is
\begin{verbatim}
  TAYLORCOMBINE(exp:exprn):exprn
\end{verbatim}
which tries to combine all Taylor kernels found in \verb|exp| into one.
Operations currently possible are:
\begin{itemize}
  \item Addition, subtraction, multiplication, and division.
  \item Roots, exponentials, and logarithms.
  \item Trigonometric and hyperbolic functions and their inverses.
\end{itemize}
Application of unary operators like \verb|LOG| and \verb|ATAN| will
nearly always succeed. For binary operations their arguments have to be
Taylor kernels with the same template. This means that the expansion
variable and the expansion point must match. Expansion order is not so
important, different order usually means that one of them is truncated
before doing the operation.

If \verb|TAYLORKEEPORIGINAL| is set to \verb|ON| and if all Taylor
kernels in \verb|exp| have their original expressions kept
\verb|TAYLORCOMBINE| will also combine these and store the result
as the original expression of the resulting Taylor kernel.
There is also the switch \verb|TAYLORAUTOEXPAND| (see below).

There are a few restrictions to avoid mathematically undefined 
expressions:
it is not possible to take the logarithm of a Taylor kernel whose 
constant
term is zero, or to divide by a Taylor kernel that consists only 
of the constant zero.
There are, however, some provisions made to detect singularities during
expansion: poles that arise because the denominator has zeros at
the expansion point are
detected and properly treated, i.e.\ the Taylor kernel will
start with a negative power.  (This is accomplished by expanding
numerator and denominator separately and combining the results.)
It has been observed, however, that this does {\em not\/} work if the
\verb|MCD| switch is set to \verb|OFF|.  This seems to be a
limitation of \REDUCE{} version 3.3.
Essential singularities are not handled at all which means that usually
some sort of error will be signalled. Maybe I can improve this later.

\verb|TAYLOR| does not always detect non-analytical expressions in
its first argument.
In this case a wrong result will be given that depends on the order
of Taylor variables in the call to \verb|TAYLOR|.
An example for this behaviour is given by the function $xy/(x+y)$ that is
not analytical in the neighborhood of $(x,y) = (0,0)$:
Trying to calculate
\begin{verbatim}
  TAYLOR(X*Y/(X+Y),X,0,2,Y,0,2);
\end{verbatim}
we get as result $X-X^{2}/Y$.
The reason for this is as follows:
\verb|TAYLOR| first expands it with respect to $X$ about $0$
up to order $2$ giving $X - X^{2}/Y$.
This has only a simple pole in $Y$ at $0$ and is therefore returned as
result.
If we interchange \verb|X| and \verb|Y| in the call to \verb|TAYLOR|
they are also interchanged in the result.
At the moment I don't know a general method to detect non-analytical
expressions in the argument to \verb|TAYLOR|.

Differentiation of a Taylor expression is possible.  If you
differentiate with respect to one of the Taylor variables the order
will decrease by one. 

Substitution is a bit restricted: Taylor variables can only be replaced
by other kernels.  There is one exception to this rule: you can always
substitute a Taylor variable by an expression that evaluates to a
constant.  Note that \REDUCE{} will not always be able to determine
that an expression is constant:  an example is \verb|SIN(ACOS(4))|.

This packages introduces a number of new switches:
\begin{itemize}

\item If you set \verb|TAYLORAUTOCOMBINE| to \verb|ON| \REDUCE{}
    automatically combines Taylor expressions during the simplification
    process.  This is equivalent to applying \verb|TAYLORCOMBINE| to
    every expression that contains Taylor kernels.

\item \verb|TAYLORAUTOEXPAND| makes Taylor expressions ``contagious''
    in the sense that \verb|TAYLORCOMBINE| tries to Taylor expand
    all non-Taylor subexpressions and to combine the result with the
    rest.

\item \verb|TAYLORKEEPORIGINAL|, if set to \verb|ON|, forces the
    package to keep the original expression, i.e.\ the expression
    that was Taylor expanded.  All operations performed on the
    Taylor kernels are also applied to this expression  which can
    be recovered using the operator \verb|TAYLORORIGINAL|.

\item \verb|TAYLORPRINTORDER|, if set to \verb|ON|, causes the
    remainder to be printed in big-$O$ notation.  Otherwise, three
    dots are printed.

\item There is also the switch \verb|VERBOSELOAD|.  If it is set to
    \verb|ON|
    \REDUCE{} will print some information when the Taylor package is
    loaded.  This switch is already present in \PSL{} systems.

\end{itemize}
All switches are normally off.

\section{Warnings and error messages}

\begin{itemize}

\item \verb|Branch point detected in ...|\\
    This occurs if you take a rational power of a Taylor kernel
    and raising the lowest order term of the kernel to this
    power yields a non analytical term (i.e.\ a fractional power).

\item \verb|Cannot expand further... truncation done|\\
    You will get this warning if you try to expand a Taylor kernel to
    a higher order.

\item \verb|Error during expansion (possible singularity)|\\
    The expression you are trying to expand caused an error.
    As far as I know this can only happen if it contains a function
    with a pole or an essential singularity at the expansion point.
    (But one can never be sure.)

\item \verb|Essential singularity in ...|\\
    An essential singularity was detected while applying a
    special function to a Taylor kernel.
    This error occurs, for example, if you try to take
    the logarithm of a Taylor kernel that starts with a negative
    power in one of its variables, i.e.\ that has a pole
    at the expansion point.

\item \verb|Expansion point lies on branch cut in ...|\\
    The only functions with branch cuts this package knows of
    are (natural) logarithm, inverse circular and hyperbolic
    tangent and cotangent.
    The branch cut of the logarithm is assumed to lie on the negative
    real axis.  Those of the arc tangent and arc cotangent
    functions are chosen to be compatible with this:
    both have essential singularities at the points $\pm i$.
    The branch cut of arc tangent is the straight line along the
    imaginary axis connecting $+1$ to $-1$ going through $\infty$ whereas
    that of arc cotangent goes through the origin.
    Consequently, the branch cut of the inverse hyperbolic
    tangent resp.\ cotangent lies on the real axis and goes from
    $-1$ to $+1$, that of the latter across
    $0$, the other across $\infty$.
    
    The error message can currently only appear when you try to
    calculate the inverse tangent or cotangent of a Taylor
    kernel that starts with a negative degree.
    The case of a logarithm of a Taylor kernel whose constant term
    is a negative real number is not caught since it is
    difficult to detect this in general.

\item \verb|Not a unity in ...|\\
    This will happen if you try to divide by or take the logarithm of 
    a Taylor series whose constant term vanishes.

\item \verb|Not implemented yet (...)|\\
    Sorry, but I haven't had the time to implement this feature.
    Tell me if you really need it, maybe I have already an improved
    version of the package.

\item \verb|Substitution of dependent variables ...|\\
    You tried to substitute a variable that is already present in the
    Taylor kernel or on which one of the Taylor variables depend.

\item \verb|Taylor kernel doesn't have an original part|\\
    The Taylor kernel upon which you try to use \verb|TAYLORORIGINAL|
    was created with the switch \verb|TAYLORKEEPORIGINAL|
    set to \verb|OFF|
    and does therefore not keep the original expression.

\item \verb|Wrong number of arguments (TAYLOR)|\\
    You try to use the operator \verb|TAYLOR| with a wrong number of
    arguments.

\item \verb|Zero denominator in Taylor substitution|\\
    That's exactly what the message says.  As an example consider the
    case of a Taylor kernel containing the term \verb|1/x| and you try
    to substitute \verb|x| by \verb|0|.

\item \verb|... invalid as kernel|\\
    You tried to expand with respect to an expression that is not a
    kernel.

\item \verb|... invalid as order of expansion|\\
    The order parameter you gave to \verb|TAYLOR| is not an integer.

\item \verb|... invalid as Taylor kernel|\\
    You tried to apply \verb|TAYLORORIGINAL| or \verb|TAYLORTEMPLATE|
    to an expression that is not a Taylor kernel.

\item \verb|... invalid as Taylor variable|\\
    You tried to substitute a Taylor variable by an expression that is
    not a kernel.

\item \verb|TAYLOR PACKAGE (...): this can't happen ...|\\
    This message shows that an internal inconsistency was detected.
    This is not your fault, at least as long as you did not try to
    work with the internal data structures of \REDUCE. Send input
    and output to me, together with the version information that is
    printed out.

\end{itemize}

\section{Comparison to other packages}

At the moment there is only one \REDUCE{} package that I know of:
the truncated power series package by Alan Barnes and Julian Padget.
In my opinion there are two major differences:
\begin{itemize}
  \item The interface. They use the domain mechanism for their power
        series, I decided to invent a special kind of kernel. Both
        approaches have advantages and disadvantages: with domain
        modes, it is easier
        to do certain things automatically, e.g., conversions.
  \item The concept of a truncated series. Their idea is to remember
        the original expression and to compute more coefficients when
        more of them are needed. My approach is to truncate at a
        certain order and forget how the unexpanded expression
        looked like.  I think that their method is more widely
        usable, whereas mine is more efficient when you know in
        advance exactly how many terms you need.
\end{itemize}

\MACSYMA{} has Taylor and power series packages.  I don't know much
about the general power series package but the Taylor package has some
features that are still lacking here, e.g., correct treatment of
known essential singularities.  In \MACSYMA{} a Taylor series is a
special object, a sort of extended rational expression recognized by
all simplification functions.  They also have a better user interface. 
E.g., you may define the Taylor expansion of an unknown function. 

\Mathematica's \verb|series| function can only handle power series of
one variable.  However, it is better in its handling of singularities.

(I should add a comparison to \MAPLE.)
\end{document}
