\section{Syntax}

\begin{Command}[semicolon]{;}
The semicolon is a statement delimiter, indicating results are to be printed
when used in interactive mode.

\begin{Examples}
(x+1)**2;                     &  X^{2} + 2*X + 1 \\
df(x**2 + 1,x);               &    2*X
\end{Examples}

\begin{Comments}
Entering a \key{Return} without a semicolon or dollar sign results in a
prompt on the following line.  A semicolon or dollar sign can be
added at this point to execute the statement.  In interactive mode, a
statement that is ended with a semicolon and \key{Return} has its results
printed on the screen.

Inside a group statement \name{<<}\ldots\name{>>}
or a \name{begin}\ldots\name{end} block, a
semicolon or dollar sign separates individual REDUCE statements.  Since
results are not printed from a block without a specific \name{return}
statement, there is no difference between using the semicolon or dollar
sign.  In a group statement, the last value produced is the value
returned by the group statement.  Thus, if a semicolon or dollar sign is
placed between the last statement and the ending brackets, the group
statement returns the value 0 or {\em nil}, rather than the value of the
last statement.
\end{Comments}
\end{Command}


\begin{Command}[dollar]{$}
The dollar sign is a statement delimiter, indicating results are not to be
printed when used in interactive mode.

\begin{Examples}

(x+1)**2$ &
\explanationo{The workspace is set to $x^{2} + 2x + 1$
              but nothing shows on the screen} \\
ws; &            X^{2}  + 2*X + 1
\end{Examples}

\begin{Comments}
Entering a \key{Return} without a semicolon or dollar sign results in a
prompt on the following line.  A semicolon or dollar sign can
be added at this point to execute the statement.  In interactive mode, a
statement that ends with a dollar sign \name{\$} and a \key{Return} is
executed, but the results not printed.

Inside a \nameref{group} statement \name{<<}\ldots\name{>>}
or a \name{begin}\ldots\name{end} \nameref{block}, a
semicolon or dollar sign separates individual REDUCE statements.  Since
results are not printed from a \nameref{block} without a specific 
\nameref{return}
statement, there is no difference between using the semicolon or dollar
sign.

In a group statement, the last value produced is the value returned by the
group statement.  Thus, if a semicolon or dollar sign is placed between the
last statement and the ending brackets, the group statement returns the
value 0 or {\em nil}, rather than the value of the last statement.

\end{Comments}
\end{Command}

\begin{Command}[percent]{%}
The percent sign is used to precede comments; everything from a percent
to the end of the line is ignored.

\begin{Examples}

df(x**3 + y,x);\% This is a comment \key{Return} &       3*X^{2} \\
int(3*x**2,x) \%This is a comment; \key{Return} \\

\explanation{A prompt is given, waiting for the semicolon that was not
detected in the comment}
\end{Examples}

\begin{Comments}
Statement delimiters \name{;} and \name{\$} are not detected between a
percent sign and the end of the line.
\end{Comments}
\end{Command}


% \begin{Operator}[ampersand]{&}
%
% ***** To be added *****
%
% \end{Operator}
%
%
\begin{Operator}[dot]{.}
\index{list}
The . (dot) infix binary operator adds a new item to the beginning of an
existing \nameref{list}.  In high energy physics expressions, 
it can also be used
to represent the scalar product of two Lorentz four-vectors.

\begin{Syntax}
\meta{item} \name{.} \meta{list}
\end{Syntax}

\meta{item} can be any REDUCE scalar expression, including a list;
\meta{list} must be a \nameref{list} to avoid producing an error message.  
The dot operator is right associative.

\begin{Examples}

liss := a . \{\};             &     LISS := \{A\} \\
liss := b . liss;             &     LISS := \{B,A\} \\
newliss := liss . liss;       &     NEWLISS := \{\{B,A\},B,A\} \\
firstlis := a . b . \{c\};    &     FIRSTLIS := \{A,B,C\} \\
secondlis := x . y . \{z\};   &     SECONDLIS := \{X,Y,Z\} \\
for i := 1:3 sum part(firstlis,i)*part(secondlis,i);
                              &     A*X + B*Y + C*Z
\end{Examples}
\end{Operator}


\begin{Operator}[assign]{:=}
\index{assign}
The \name{:=} is the assignment operator, assigning the value on the right-hand
side to the identifier or other valid expression on the left-hand side.

\begin{Syntax}
  \meta{restricted\_expression} \name{:=} \meta{expression}
\end{Syntax}

\meta{restricted\_expression} is ordinarily a single identifier, though simple
expressions may be used (see Comments below). \meta{expression} is any
valid REDUCE expression.  If \meta{expression} is a \nameref{matrix} 
identifier, then
\meta{restricted\_expression} can be a matrix identifier (redimensioned if
necessary) which has each element set to the corresponding elements
of the identifier on the right-hand side.

\begin{Examples}
a := x**2 + 1;                &          A := X^{2}  + 1 \\
a;                            &          X^{2} + 1 \\
first := second := third;     &          FIRST := SECOND := THIRD \\
first;                        &          THIRD \\
second;                       &          THIRD \\
b := for i := 1:5 product i;  &          B := 120 \\
b;                            &          120 \\
w + (c := x + 3) + z;         &          W + X + Z + 3 \\
c;                            &          X + 3 \\
y + b := c;                   &          Y + B := C \\
y;                            &           - (B - C)
\end{Examples}

\begin{Comments}
The assignment operator is right associative, as shown in the second and
third examples.  A string of such assignments has all but the last
item set to the value of the last item.  Embedding an assignment statement
in another expression has the side effect of making the assignment, as well
as causing the given replacement in the expression.

Assignments of values to expressions rather than simple identifiers (such as in
the last example above) can also be done, subject to the following remarks:
\begin{itemize}

\item[(i)]
If the left-hand side is an identifier, an operator, or a power, the
substitution rule is added to the rule table.

\item[(ii)]
If the operators \name{- + /} appear on the left-hand side, all but the first
term of the expression is moved to the right-hand side.

\item[(iii)]
If the operator \name{*} appears on the left-hand side, any constant terms are
moved to the right-hand side, but the symbolic factors remain.
\end{itemize}

Assignment is valid for \nameref{array} elements, but not for entire arrays.
The assignment operator can also be used to attach functionality to operators.

A recursive construction such as \name{a := a + b} is allowed, but when
\name{a} is referenced again, the process of resubstitution continues
until the expression stack overflows (you get an error message).
Recursive assignments can be done safely inside controlled loop
expressions, such as \nameref{for}\ldots or \nameref{repeat}\ldots\name{until}.

\end{Comments}
\end{Operator}


\begin{Operator}[equalsign]{=}
The \name{=} operator is a prefix or infix equality comparison operator.

\begin{Syntax}
  \name{=}\(\meta{expression}\name{,}\meta{expression}\)
  or
  \meta{expression} \name{=} \meta{expression}
\end{Syntax}

\meta{expression} can be any REDUCE scalar expression.

\begin{Examples}
a := 4;                       &         A := 4 \\
if =(a,10) then write "yes" else write "no";
                              &         no \\
b := c;                       &         B := C \\
if b = c then write "yes" else write "no";
                              &         yes \\
on rounded; \\
if 4.0 = 4 then write "yes" else write "no";
                              &         yes
\end{Examples}

\begin{Comments}
This logical equality operator can only be used inside a conditional
statement, such as \nameref{if}\ldots\name{then}\ldots\name{else}
or \nameref{repeat}\ldots\name{until}. In other places the equal
sign establishes an algebraic object of type \nameref{equation}.

\end{Comments}
\end{Operator}


\begin{Operator}[replace]{=>}

The \name{=>} operator is a binary operator used in \ref{rule} lists to
denote replacements.

\begin{Examples}
operator f; \\
let f(x) => x^2; \\
f(x); & x^{2}
\end{Examples}
\end{Operator}


\begin{Operator}[plussign]{+}
The \name{+} operator is a prefix or infix n-ary addition operator.

\begin{Syntax}
\meta{expression} \{ \name{+}\meta{expression}\}\repeated

{\em or} \name{+}\(\meta{expression} \{,\meta{expression}\}\repeated\)
\end{Syntax}

\meta{expression} may be any valid REDUCE expression.

\begin{Examples}
x**4 + 4*x**2 + 17*x + 1;     &   X^{4} + 4*X^{2} + 17*X + 1 \\
14 + 15 + x;                  &   X + 29 \\
+(1,2,3,4,5);                 &   15
\end{Examples}

\begin{Comments}
\name{+} is also valid as an addition operator for \nameref{matrix} variables
that are of the same dimensions and for \nameref{equation}s.
\end{Comments}
\end{Operator}


\begin{Operator}[minussign]{-}
The \name{-} operator is a prefix or infix binary subtraction operator, as well
as the unary minus operator.

\begin{Syntax}
\meta{expression} \name{-} \meta{expression}
or \name{-}\(\meta{expression},\meta{expression}\)
\end{Syntax}

\meta{expression} may be any valid REDUCE expression.

\begin{Examples}
15 - 4;                      &   11 \\
x*(-5);                      &   - 5*X \\
a - b - 15;                  &   A - B - 15 \\
-(a,4);                      &   A - 4
\end{Examples}

\begin{Comments}
The subtraction operator is left associative, so that a - b - c is equivalent
to (a - b) - c, as shown in the third example.  The subtraction operator is
also valid with \nameref{matrix} expressions of the correct dimensions
and with \nameref{equation}s.
\end{Comments}
\end{Operator}


\begin{Operator}[asterisk]{*}
The \name{*} operator is a prefix or infix n-ary multiplication operator.

\begin{Syntax}
\meta{expression} \{ \name{*} \meta{expression}\}\repeated

 or \name{*}\(\meta{expression} \{,\meta{expression}\}\repeated\)
\end{Syntax}

\meta{expression} may be any valid REDUCE expression.

\begin{Examples}
15*3;                        &   45 \\
24*x*yvalue*2;               &   48*X*YVALUE \\
*(6,x);                      &   6*X \\
on rounded; \\
3*1.5*x*x*x;                 &   4.5*X^{3} \\
off rounded; \\
2x**2;                       &   2*X^{2}
\end{Examples}

\begin{Comments}
REDUCE assumes you are using an implicit multiplication operator when an
identifier is preceded by a number, as shown in the last line above.  Since
no valid identifiers can begin with numbers, there is no ambiguity in
making this assumption.

The multiplication operator is also valid with \nameref{matrix} expressions 
of the
proper dimensions: matrices \IFTEX{$A$}{A} and \IFTEX{$B$}{B} 
can be multiplied if
\IFTEX{$A$}{A} is \IFTEX{$n \times m$}{n x m} and \IFTEX{$B$}{B} is
\IFTEX{$m \times p$}{m x p}.  Matrices and \nameref{equation}s can also be 
multiplied by scalars: the
result is as if each element was multiplied by the scalar.
\end{Comments}
\end{Operator}


\begin{Operator}[slash]{/}
The \name{/} operator is a prefix or infix binary division operator or
prefix unary \nameref{recip}rocal operator.

\begin{Syntax}
\meta{expression}\name{/}\meta{expression} {\em or}
 \name{/}\meta{expression}

or \name{/}\(\meta{expression},\meta{expression}\)
\end{Syntax}

\meta{expression} may be any valid REDUCE expression.

\begin{Examples}
20/5;                        &   4  \\
100/6;                       &   \rfrac{50}{3} \\
16/2/x;                      &   \rfrac{8}{X} \\
/b;                          &   \rfrac{1}{B} \\
/(y,5);                      &   \rfrac{Y}{5} \\
on rounded; \\
35/4;                        &   8.75 \\
/20;                         &   0.05
\end{Examples}

\begin{Comments}
The division operator is left associative, so that \name{a/b/c} is equivalent
to \name{(a/b)/c}.  The division operator is also valid with square
\nameref{matrix} expressions of the same dimensions: With \IFTEX{$A$}{A} and
\IFTEX{$B$}{B} both \IFTEX{$n \times n$}{n x n} matrices and \IFTEX{$B$}{B}
invertible, \IFTEX{$A/B$}{A/B} is
given by \IFTEX{$A \times B^{-1}$}{A*B^{-1}}.
Division of a matrix by a scalar is defined, with the results being the
division of each element of the matrix by the scalar.  Division of a
scalar by a matrix is defined if the matrix is invertible, and has the
effect of multiplying the scalar by the inverse of the matrix.  When
\name{/} is used as a reciprocal operator for a matrix, the inverse of
the matrix is returned if it exists.
\end{Comments}
\end{Operator}


\begin{Operator}[power]{**}
The \name{**} operator is a prefix or infix binary exponentiation operator.

\begin{Syntax}
\meta{expression} \name{**}\meta{expression}
     or \name{**}\(\meta{expression},\meta{expression}\)
\end{Syntax}

\meta{expression} may be any valid REDUCE expression.

\begin{Examples}
x**15;                       &        X^{15} \\
x**y**z;                     &        X^{Y*Z} \\
x**(y**z);                   &        X^{Y^{Z}} \\
 **(y,4);                    &        Y^{4} \\
on rounded; \\
2**pi;                       &        8.82497782708
\end{Examples}

\begin{Comments}
The exponentiation operator is left associative, so that \name{a**b**c} is
equivalent to \name{(a**b)**c}, as shown in the second example.  Note
that this is {\em not} \name{a**(b**c)}, which would be right associative.

When \nameref{nat} is on (the default), REDUCE output produces raised
exponents, as shown.  The symbol \name{^}, which is the upper-case 6 on
most keyboards, may be used in the place of \name{**}.

A square \nameref{matrix} may also be raised to positive and negative powers
with the exponentiation operator (negative powers require the matrix to be
invertible).  Scalar expressions and \nameref{equation}s may be raised to 
fractional and floating-point powers.
\end{Comments}
\end{Operator}

\begin{Operator}[caret]{^}
The \name{^} operator is a prefix or infix binary exponentiation operator.
It is equivalent to \nameref{power} or **.

\begin{Syntax}
\meta{expression} \name{^}\meta{expression}
      or \name{^}\(\meta{expression},\meta{expression}\)
\end{Syntax}

\meta{expression} may be any valid REDUCE expression.

\begin{Examples}
x^15;                       &        X^{15} \\
x^y^z;                     &        X^{Y*Z} \\
x^(y^z);                   &        X^{Y^{Z}} \\
^(y,4);                      &        Y^{4} \\
on rounded; \\
2^pi;                       &        8.82497782708
\end{Examples}

\begin{Comments}
The exponentiation operator is left associative, so that \name{a^b^c} is
equivalent to \name{(a^b)^c}, as shown in the second example.  Note
that this is \meta{not} \name{a^(b^c)}, which would be right associative.

When \nameref{nat} is on (the default), REDUCE output produces raised
exponents, as shown.

A square \nameref{matrix} may also be raised to positive 
and negative powers with
the exponentiation operator (negative powers require the matrix to be
invertible).  Scalar expressions and \nameref{equation}s 
may be raised to fractional and floating-point powers.
\end{Comments}
\end{Operator}



\begin{Operator}[geqsign]{>=}
\name{>=} is an infix binary comparison operator, which returns {\em true} if
its first argument is greater than or equal to its second argument.

\begin{Syntax}
\meta{expression} \name{>=} \meta{expression}
\end{Syntax}

\meta{expression} must evaluate to an integer or floating-point number.

\begin{Examples}
if (3 >= 2) then yes;        &     yes \\
a := 15;                     &       A := 15 \\
if a >= 20 then big else small;
                             &     small \\
\end{Examples}

\begin{Comments}

The binary comparison operators can only be used for comparisons between
numbers or variables that evaluate to numbers.  The truth values returned
by such a comparison can only be used inside programming constructs, such
as \nameref{if}\ldots\name{then}\ldots\name{else}
or \nameref{repeat}\ldots\name{until} or \nameref{while}\ldots\name{do}.
\end{Comments}
\end{Operator}


\begin{Operator}[greater]{>}
The \name{>} is an infix binary comparison operator that returns
{\em true} if its first argument is strictly greater than its second.

\begin{Syntax}
\meta{expression} \name{>} \meta{expression}
\end{Syntax}

\meta{expression} must evaluate to a number, e.g., integer, rational or
floating point number.

\begin{Examples}
on rounded; \\
if 3.0 > 3 then write "different" else write "same";    &     same \\
off rounded; \\
a := 20;                                                &     A := 20 \\
if a > 20 then write "bigger" else write "not bigger";  &     not bigger \\
\end{Examples}

\begin{Comments}
The binary comparison operators can only be used for comparisons between
numbers or variables that evaluate to numbers.  The truth values returned
by such a comparison can only be used inside programming constructs, such
as \nameref{if}\ldots\name{then}\ldots\name{else} or
\nameref{repeat}\ldots\name{until} or \nameref{while}\ldots\name{do}.
\end{Comments}
\end{Operator}


\begin{Operator}[leqsign]{<=}
\name{<=} is an infix binary comparison operator that returns
{\em true} if its first argument is less than or equal to its second argument.

\begin{Syntax}
\meta{expression} \name{<=} \meta{expression}
\end{Syntax}

\meta{expression} must evaluate to a number, e.g., integer, rational or
floating point number.

\begin{Examples}
a := 10;                     &     A := 10 \\
if a <= 10 then true;        &     true
\end{Examples}

\begin{Comments}

The binary comparison operators can only be used for comparisons between
numbers or variables that evaluate to numbers.  The truth values returned
by such a comparison can only be used inside programming constructs, such
as \nameref{if}\ldots\name{then}\ldots\name{else} or
\nameref{repeat}\ldots\name{until} or \nameref{while}\ldots\name{do}.
\end{Comments}
\end{Operator}


\begin{Operator}[less]{<}
\name{<} is an infix binary logical comparison operator that
returns {\em true} if its first argument is strictly less than its second
argument.

\begin{Syntax}
\meta{expression} \name{<} \meta{expression}
\end{Syntax}

\meta{expression} must evaluate to a number, e.g., integer, rational or
floating point number.

\begin{Examples}
f := -3;                                               &       F := -3 \\
if f < -3 then write "yes" else write "no";            &       no
\end{Examples}

\begin{Comments}
The binary comparison operators can only be used for comparisons between
numbers or variables that evaluate to numbers.  The truth values returned
by such a comparison can only be used inside programming constructs, such
as \nameref{if}\ldots\name{then}\ldots\name{else}
or \nameref{repeat}\ldots\name{until} or
\nameref{while}\ldots\name{do}.
\end{Comments}
\end{Operator}

\begin{Operator}[tilde]{~}
The \name{~} is used as a unary prefix operator in the left-hand
sides of \nameref{rule}s to mark \nameref{free variable}s. A double tilde
marks an optional \nameref{free variable}.
\end{Operator}


\begin{Command}[group]{<<}
The \name{<<}\ldots\name{>>} command is a group statement,
used to group statements
together where REDUCE expects a single statement.

%%%INCONSISTENT??? \name{or}
\begin{Syntax}
\name{<<}\meta{statement}\{; \meta{statement} \name{or}
                           \$\meta{statement}\}\optional \name{>>}
\end{Syntax}

\meta{statement} may be any valid REDUCE statement or expression.

\begin{Examples}
a := 2;                                                &    A := 2 \\
if a < 5 then <<b := a + 10; write b>>;                &    12 \\
<<d := c/15; f := d + 3; f**2>>;
                             &      \rfrac{C^{2} + 90*C + 202}{225}
\end{Examples}

\begin{Comments}
The value returned from a group statement is the value of the last
individual statement executed inside it.  Note that when a semicolon is
placed between the last statement and the closing brackets, 0 or
{\em nil} is returned.  Group statements are often used in the
consequence portions of \nameref{if}\ldots\name{then},
\nameref{repeat}\ldots\name{until}, and
\nameref{while}\ldots\name{do}
clauses.  They may also be used in interactive
operation to execute several statements at one time.  Statements inside
the group statement are separated by semicolons or dollar signs.

\end{Comments}
\end{Command}


\begin{Operator}{AND}
The \name{and} binary logical operator returns {\em true} if both of its
arguments are {\em true}.

\begin{Syntax}
\meta{logical\_expression} \name{and} \meta{logical\_expression}
\end{Syntax}

\meta{logical\_expression} must evaluate to {\em true} or {\em nil}.

\begin{Examples}
a := 12;                   &     A := 12 \\
if numberp a and a < 15 then write a**2 else write "no";
                           &     144 \\
clear a; \\
if numberp a and a < 15 then write a**2 else write "no";
                           &     no
\end{Examples}

\begin{Comments}
Logical operators can only be used inside conditional statements, such as
\nameref{while}\ldots\name{do} or
\nameref{if}\ldots\name{then}\ldots\name{else}. \name{and} examines each of
its arguments in order, and quits, returning {\em nil}, on finding an
argument that is not {\em true}.  An error results if it is used in other
contexts.

\name{and} is left associative: \name{x and y and z} is equivalent to
\name{(x and y) and z}.
\end{Comments}
\end{Operator}


\begin{Command}{BEGIN}
\name{begin} is used to start a \nameref{block} statement, which is closed with
\name{end}.

\begin{Syntax}
\name{begin} \meta{statement}\{\name{;} \meta{statement}\}\optional \ \name{end}
\end{Syntax}

\meta{statement} is any valid REDUCE statement.

\begin{Examples}
begin for i := 1:3 do write i end; &
\begin{multilineoutput}{1cm}
1
2
3     
\end{multilineoutput} \\
begin scalar n;n:=1;b:=for i:=1:4 product(x-i);return n end;
                     &         1 \\
b;                   &
                     X^{4}  - 10*X^{3}  + 35*X^{2}  - 50*X  + 24
\end{Examples}

\begin{Comments}
A \name{begin}\ldots\name{end} block can do actions (such as \name{write}), but 
does not
return a value unless instructed to by a \nameref{return} statement, which must
be the last statement executed in the block.  It is unnecessary to insert
a semicolon before the \name{end}.

Local variables, if any, are declared in the first statement immediately
after \name{begin}, and may be defined as \name{scalar, integer,} or
\name{real}.  \nameref{array} variables declared 
within a \name{begin}\ldots\name{end} block
are global in every case,  and \nameref{let} statements have global
effects.  A \nameref{let} statement involving a formal parameter affects
the calling parameter that corresponds to it.  \nameref{let} statements
involving local variables make global assignments, overwriting outside
variables by the same name or creating them if they do not exist.  You
can use this feature to affect global variables from procedures, but be
careful that you do not do it inadvertently.

\end{Comments}
\end{Command}


\begin{Command}{block}
A \name{block} is a sequence of statements enclosed by
commands \nameref{begin} and \nameref{end}.

\begin{Syntax}
\name{begin} \meta{statement}\{\name{;} \meta{statement}\}\optional \ \name{end}
\end{Syntax}
For more details see \nameref{begin}.
\end{Command}


\begin{Command}{COMMENT}
Beginning with the word \name{comment}, all text until the next statement
terminator (\name{;} or \name{\$}) is ignored.

\begin{Examples}

x := a**2 comment--a is the velocity of the particle;;
                             &      X := A^{2}
\end{Examples}

\begin{Comments}
Note that the first semicolon ends the comment and the second one
terminates the original REDUCE statement.

Multiple-line comments are often needed in interactive files.  The
\name{comment} command allows a normal-looking text to accompany the
REDUCE statements in the file.
\end{Comments}
\end{Command}


\begin{Operator}{CONS}
The \name{cons} operator adds a new element to the beginning of a 
\nameref{list}.  Its
operation is identical to the symbol \nameref{dot} (dot).  It can be used
infix or prefix.

\begin{Syntax}
\name{cons}\(\meta{item},\meta{list}\) or \meta{item} \name{cons} \meta{list}
\end{Syntax}

\meta{item} can be any REDUCE scalar expression, including a list; \meta{list}
must be a list.

\begin{Examples}

liss := cons(a,{b});         &        \{A,B\} \\

liss := c cons liss;         &        \{C,A,B\} \\

newliss := for each y in liss collect cons(y,list x);
                             &        NEWLISS := \{\{C,X\},\{A,X\},\{B,X\}\} \\

for each y in newliss sum (first y)*(second y);
                             &        X*(A + B + C)
\end{Examples}

\begin{Comments}
If you want to use \name{cons} to put together two elements into a new list,
you must make the second one into a list with curly brackets or the \name{list}
command.  You can also start with an empty list created by \name{\{\}}.

The \name{cons} operator is right associative: \name{a cons b cons c} is valid
if \name{c} is a list; \name{b} need not be a list.  The list produced is
\name{\{a,b,c\}}.

\end{Comments}
\end{Operator}


\begin{Command}{END}
The command \name{end} has two main uses:
\begin{itemize}
\item[(i)]
as the ending of a \nameref{begin}\ldots\name{end} \nameref{block}; and
\item[(ii)]
to end input from a file.
\end{itemize}

\begin{Comments}
In a \name{begin}\ldots\name{end} \nameref{block}, there need not be a delimiter
(\name{;} or \name{\$}) before the \name{end}, though there must be one
after it, or a right bracket matching an earlier left bracket.

Files to be read into REDUCE should end with \name{end;}, which must be
preceded by a semicolon (usually the last character of the previous line).
The additional semicolon avoids problems with mistakes in the files.  If
you have suspended file operation by answering \name{n} to a \name{pause}
command, you are still, technically speaking, ``in" the file.  Use
\name{end} to exit the file.

An \name{end} at the top level of a program is ignored.
\end{Comments}
\end{Command}


\begin{Type}{EQUATION}
\index{equation}\index{equal}\index{arithmetic}
An \name{equation} is an expression where two algebraic expressions
are connected by the (infix) operator \nameref{equal} or by \nameindex{=}.
For access to the components of an \name{equation} the operators
\nameref{lhs}, \nameref{rhs} or \nameref{part} can be used. The
evaluation of the left-hand side of an \name{equation} is controlled
by the switch \nameref{evallhseqp}, while the right-hand side is
evaluated unconditionally. When an \name{equation} is part of a
logical expression, e.g. in a \nameref{if} or \nameref{while} statement,
the equation is evaluated by subtracting both sides can comparing
the result with zero.

Equations occur in many contexts, e.g. as arguments of the \nameref{sub}
operator and in the arguments and the results
of the operator \nameref{solve}. An equation can be member of a \nameref{list}
and you may assign an equation to a variable. Elementary arithmetic is supported
for equations: if \nameref{evallhseqp} is on, you may add and subtract
equations, and you can combine an equation with a scalar expression by
addition, subtraction, multiplication, division and raise an equation
to a power. 
\begin{Examples}
on evallhseqp;\\
u:=x+y=1$\\
v:=2x-y=0$\\
2*u-v; &  - 3*y=-2\\
ws/3; & y=\rfrac{2}{3}\\
\end{Examples}
Important: the equation must occur in the leftmost term of such an expression.
For other operations, e.g. taking function values of both sides, use the
\nameref{map} operator.

\end{Type}

\begin{Operator}{FIRST}
\index{list}\index{decomposition}
The \name{first} operator returns the first element of a \nameref{list}.
\begin{Syntax}
\name{first}\(\meta{list}\) or \name{first} \meta{list}
\end{Syntax}

\meta{list} must be a non-empty list to avoid an error message.

\begin{Examples}
alist := \{a,b,c,d\};          &    ALIST := \{A,B,C,D\} \\
first alist;                 &  A \\
blist := \{x,y,\{ww,aa,qq\},z\}; &  BLIST := \{X,Y,\{WW,AA,QQ\},Z\} \\
first third blist;           &  WW
\end{Examples}
\end{Operator}


\begin{Command}{FOR}
\index{loop}
The \name{for} command is used for iterative loops.  There are many
possible forms it can take.
\begin{INFO}
{
\begin{verbatim}
                   /                   \
   /               |STEP <number> UNTIL|        \
   |<var>:=<number>|                   |<number>|
FOR|               |         :         |        |<action> <exprn>
   |               \                   /        |
   |EACH <var> IN <list>                        |
   \                                            /

 where <action> ::= DO|PRODUCT|SUM|COLLECT|JOIN.
\end{verbatim}
}
\end{INFO}
\begin{TEX}
\begin{Syntax}
     \name{for}
      \begin{alternative}
         \meta{var} \name{:=} \meta{start} \name{:} \meta{stop}\\
         \meta{var} \name{:=} \meta{start} \name{step} \meta{inc}
                              \name{until} \meta{stop}\\
         \name{each} \meta{var} \name{in} \meta{list}
      \end{alternative}
      \begin{alternative}
         \name{collect}\\
         \name{do}\\
         \name{join}\\
         \name{product}\\
         \name{sum}
      \end{alternative}
    \meta{expression}
\end{Syntax}
\end{TEX}

\meta{var} can be any valid REDUCE identifier except \name{t} or
\name{nil}, \meta{inc}, \meta{start} and \meta{stop} can be any expression
that evaluates to a positive or negative integer. \meta{list} must be a
valid \nameref{list} structure.  
The action taken must be one of the actions shown
above, each of which is followed by a single REDUCE expression, statement
or a \nameref{group} (\name{<<}\ldots\name{>>}) or \nameref{block}
(\nameref{begin}\ldots\nameref{end}) statement.

\begin{Examples}
for i := 1:10 sum i;                                    
                            &     55 \\
for a := -2 step 3 until 6 product a;
							&     -8 \\
a := 3;                     &     A := 3 \\
for iter := 4:a do write iter; \\
m := 0;                     &     M := 0 \\
for s := 10 step -1 until 3 do <<d := 10*s;m := m + d>>; \\
m;                          &     520 \\
for each x in {q,r,s} sum x**2;  &     Q^{2} + R^{2} + S^{2} \\
for i := 1:4 collect 1/i;                              
     &     \{1,\rfrac{1}{2},\rfrac{1}{3},\rfrac{1}{4}\} \\
for i := 1:3 join list solve(x**2 + i*x + 1,x);         
     & \begin{multilineoutput}{5cm}
\{\{X= -\rfrac{SQRT(3)*I + 1}{2},
  X= -\rfrac{SQRT(3)*I - 1}{2}\}
 \{X=-1\},
 \{X= - \rfrac{SQRT(5) + 3}{2},X=\rfrac{SQRT(5) - 3}{2}\}\}
\end{multilineoutput}
\end{Examples}

\begin{Comments}
The behavior of each of the five action words follows:

\begin{TEX}
\begin{center}
\begin{tabular}{|l|p{5cm}|p{5cm}|}
\hline
\multicolumn{3}{|c|}{Action Word Behavior}\\
\hline
\multicolumn{1}{|c|}{Keyword} &
   \multicolumn{1}{c|}{Argument Type} & \multicolumn{1}{c|}{Action} \\
\hline
do & statement, command, group or block &
Evaluates its argument once for each iteration of the loop, not saving
results \\
collect & expression, statement, command, group, block, list &
Evaluates its argument once for each iteration of the loop, storing the results
in a list which is returned by the \verb|for| statement when done \\
join & list or an operator which produces a list &
Evaluates its argument once for each iteration of the loop, appending the
elements in each individual result list onto the overall result list \\
product & expression, statement, command, 
\nameref{group} or \nameref{block} &
Evaluates its argument once for each iteration of the loop, multiplying the
results together and returning the overall product \\
sum & expression, statement, command, group or block &
Evaluates its argument once for each iteration of the loop, adding the results
together and returning the overall sum\\
\hline
\end{tabular}
\end{center}
\end{TEX}
\begin{INFO}
{\begin{verbatim}
                           Action Word Behavior
Keyword   Argument Type                    Action
   do    statement, command, group   Evaluates its argument once
         or block                    for each iteration of the loop,
                                     not saving results
collect expression, statement,       Evaluates its argument once for
        command, group, block, list  each iteration of the loop,
                                     storing the results in a list
                                     which is returned by the for
                                     statement when done
 join   list or an operator which    Evaluates its argument once for
        produces a list              each iteration of the loop,
                                     appending the elements in each
                                     individual result list onto the
                                     overall result list
product expression, statement,       Evaluates its argument once for
        command, group or block      each iteration of the loop,
                                     multiplying the results together
                                     and returning the overall product
  sum   expression, statement,       Evaluates its argument once for
        command, group or block      each iteration of the loop,
                                     adding the results together and
                                     returning the overall sum
\end{verbatim} }
\end{INFO}

For number-driven \name{for} statements, if the ending limit is smaller
than the beginning limit (larger in the case of negative steps) the action
statement is not executed at all.  The iterative variable is local to the
\name{for} statement, and does not affect the value of an identifier with
the same name.  For list-driven \name{for} statements, if the list is
empty, the action statement is not executed, but no error occurs.

You can use nested \name{for} statements, with the inner \name{for}
statement after the action keyword.   You must make sure that your inner
statement returns an expression that the outer statement can handle.
\end{Comments}
\end{Command}


\begin{Command}{FOREACH}
\index{loop}
\name{foreach} is a synonym for the \name{for each} variant of the
\nameref{for} construct.  It is designed to iterate down a list, and an
error will occur if a list is not used.  The use of \name{for each} is
preferred to \name{foreach}.

\begin{Syntax}
\name{foreach} \meta{variable} in \meta{list} \meta{action} \meta{expression} \\
 where \meta{action} ::= \name{do | product | sum | collect | join}
\end{Syntax}

\begin{Examples}
foreach x in {q,r,s} sum x**2;  &   Q^{2} + R^{2} + S^{2}
\end{Examples}

\end{Command}


\begin{Operator}{GEQ}
The \name{geq} operator is a binary infix or prefix logical operator.  It
returns true if its first argument is greater than or equal to its second 
argument.  As an infix operator it is identical with \name{>=}.
\begin{Syntax}
\name{geq}\(\meta{expression},\meta{expression}\) or \meta{expression}
\name{geq} \meta{expression}
\end{Syntax}

\meta{expression} can be any valid REDUCE expression that evaluates to a
number.

\begin{Examples}
a := 20;                     &          A := 20 \\
if geq(a,25) then write "big" else write "small";
			     &          small \\
if a geq 20 then write "big" else write "small";
			     &          big  \\
if (a geq 18) then write "big" else write "small";
			     &          big
\end{Examples}

\begin{Comments}
Logical operators can only be used in conditional statements such as \\
\nameref{if}\ldots\name{then}\ldots\name{else} or 
\nameref{repeat}\ldots\name{until}.
\end{Comments}
\end{Operator}


\begin{Command}{GOTO}
Inside a \name{begin}\ldots\name{end} \nameref{block}, \name{goto}, or 
preferably, \name{go to}, transfers flow of control to a labeled statement.
\begin{Syntax}
\name{go to} \meta{labeled_statement} or \name{goto} \meta{labeled_statement}
\end{Syntax}
\meta{labeled_statement} is of the form \meta{label} \name{:}\meta{statement}

\begin{Examples}
\begin{multilineinput}
     procedure dumb(a);
        begin scalar q;
           go to lab;
           q := df(a**2 - sin(a),a);
           write q;
      lab: return a
        end;
\end{multilineinput}         &         DUMB \\

dumb(17);                    &         17
\end{Examples}

\begin{Comments}
\name{go to} can only be used inside a \name{begin}\ldots\name{end} 
\nameref{block}, and inside
the block only statements at the top level can be labeled, not ones inside
\name{<<}\ldots\name{>>}, \nameref{while}\ldots\name{do}, etc.
\end{Comments}
\end{Command}


\begin{Operator}{GREATERP}
The \name{greaterp} logical operator returns true if its first argument is
strictly greater than its second argument.  As an infix operator it is
identical with \name{>}.
\begin{Syntax}
\name{greaterp}\(\meta{expression},\meta{expression}\) or \meta{expression}
\name{greaterp} \meta{expression}
\end{Syntax}

\meta{expression} can be any valid REDUCE expression that evaluates to a
number.

\begin{Examples}

a := 20;                     &          A := 20 \\
if greaterp(a,25) then write "big" else write "small";
			     &          small \\
if a greaterp 20 then write "big" else write "small";
			     &          small \\
if (a greaterp 18) then write "big" else write "small";
			     &          big
\end{Examples}

\begin{Comments}
Logical operators can only be used in conditional statements such as \\
\nameref{if}\ldots\name{then}\ldots\name{else} 
or \nameref{repeat}\ldots\nameref{while}.
\end{Comments}
\end{Operator}


\begin{Command}{IF}
The \name{if} command is a conditional statement that executes a statement
if a condition is true, and optionally another statement if it is not.
\begin{Syntax}
\name{if} \meta{condition} \name{then} \meta{statement}
\ \&option\(\name{else}\ \meta{statement}\)
\end{Syntax}

\meta{condition} must be a logical or comparison operator that evaluates to
a \nameref{boolean value}.  
\meta{statement} must be a single REDUCE statement or a
\nameref{group} (\name{<<}\ldots\name{>>}) or 
\nameref{block} (\name{begin}\ldots\name{end}) statement.

\begin{Examples}
if x = 5 then a := b+c else a := d+f;
			     &         D + F \\
x := 9;                      &         X := 9 \\
if numberp x and x<20 then y := sqrt(x) else write "illegal";
			     &         3  \\
clear x; \\
if numberp x and x<20 then y := sqrt(x) else write "illegal";
			     &          illegal \\
x := 12;                     &          X := 12 \\
a := if x < 5 then 100 else 150;
			     &          A := 150 \\
b := u**(if x < 10 then 2);
			     &          B := 1 \\
bb := u**(if x > 10 then 2);
			     &          BB := U^{2}
\end{Examples}

\begin{Comments}
An \name{if} statement may be used inside an assignment statement and sets
its value depending on the conditions, or used anywhere else an
expression would be valid, as shown in the last example.  If there is no
\nameindex{else} clause, the value is 0 if a number is expected, and nothing
otherwise.

The \name{else} clause may be left out if no action is to be taken if the
condition is false.

The condition may be a compound conditional statement using \nameref{and} or
\nameref{or}.  If a non-conditional statement, such as a constant, is used by
accident, it is assumed to have value {\em true}.
 
Be sure to use \nameref{group} or \nameref{block} statements after 
\nameindex{then} or \name{else}.

The \name{if} operator is right associative.  The following constructions are
examples:
\begin{itemize}
\item[(1)]
\begin{Syntax}
\name{if} \meta{condition} \name{then} \name{if} \meta{condition} \name{then}
          \meta{action} \name{else} \meta{action}
\end{Syntax}
%\end{itemize}

which is equivalent to
\begin{Syntax}
\name{if} \meta{condition} \name{then} \(\name{if}\ \meta{condition}
\ \name{then}\ \meta{action}\ \name{else}\ \meta{action}\);
\end{Syntax}
%\begin{itemize}

\item[(2)]
\begin{Syntax}
\name{if} \meta{condition} \name{then} \meta{action} \name{else if}
 \meta{condition} \name{then} \meta{action} \name{else} \meta{action}
\end{Syntax}
which is equivalent to
\begin{Syntax}
\name{if}\ \meta{condition} \name{then} \meta{action} \name{else} \\
    \(\name{if}\ \meta{condition}\ \name{then}\ \meta{action}
\ \name{else}\ \meta{action}\).
\end{Syntax}
\end{itemize}
\end{Comments}
\end{Command}


\begin{Operator}{LIST}
\index{list}
The \name{list} operator constructs a list from its arguments.
\begin{Syntax}
\name{list}\(\meta{item} \{,\meta{item}\}\optional\) or
 \name{list}\(\) to construct an empty list.
\end{Syntax}

\meta{item} can be any REDUCE scalar expression, including another list.
Left and right curly brackets can also be used instead of the operator
\name{list} to construct a list.

\begin{Examples}
liss := list(c,b,c,\{xx,yy\},3x**2+7x+3,df(sin(2*x),x));
	 &    LISS := \{C,B,C,\{XX,YY\},3*X^{2} + 7*X + 3,2*COS(2*X)\} \\
length liss;                &    6 \\
liss := \{c,b,c,\{xx,yy\},3x**2+7x+3,df(sin(2*x),x)\};
	 &    LISS := \{C,B,C,\{XX,YY\},3*X^{2} + 7*X + 3,2*COS(2*X)\} \\
emptylis := list();         &    EMPTYLIS := \{\} \\
a . emptylis;               &    \{A\}
\end{Examples}

\begin{Comments}
Lists are ordered, hierarchical structures.  The elements stay where you
put them, and only change position in the list if you specifically change
them.  Lists can have nested sublists to any (reasonable) level.  The 
\nameref{part} operator can be used to access elements anywhere within a list
hierarchy.  The \nameref{length} operator counts the 
number of top-level elements
of its list argument; elements that are themselves lists still only 
count as one element.
\end{Comments}
\end{Operator}


\begin{Operator}{OR}
The \name{or} binary logical operator returns {\it true} if either one or
both of its arguments is {\it true}.
\begin{Syntax}
\meta{logical expression} \name{or} \meta{logical expression}
\end{Syntax}

\meta{logical expression} must evaluate to {\it true} or {\it nil}.

\begin{Examples}
a := 10;                    &         A := 10 \\
\begin{multilineinput}
if a<0 or a>140 then write "not a valid human age" else
   write "age = ",a;
\end{multilineinput} \\
&         age = 10 \\
a := 200;                   &         A := 200 \\
if a < 0 or a > 140 then write "not a valid human age";
			    &         not a valid human age
\end{Examples}

\begin{Comments}
The \name{or} operator is left associative: \name{x or y or z} is equivalent to
\name{(x or y)} \name{or z}.

Logical operators can only be used in conditional expressions, such as \\
\nameref{if}\ldots\name{then}\ldots\name{else} 
and \nameref{while}\ldots\name{do}.
\name{or} evaluates its arguments in order and quits, returning {\it true},
on finding the first {\it true} statement.
\end{Comments}
\end{Operator}


\begin{Command}{PROCEDURE}
The \name{procedure} command allows you to define a mathematical operation as a
function with arguments.
\begin{Syntax}
\&\meta{option} \name{procedure} \meta{identifier}
  \(\meta{arg}\{,\meta{arg}\}\repeated\)\name{;}\meta{body}
\end{Syntax}

The \meta{option} may be \nameref{algebraic} or \nameref{symbolic}, 
indicating the
mode under which the procedure is executed, or \nameref{real} or
\nameref{integer}, indicating the type of answer expected.  The default is
algebraic.  Real or integer procedures are subtypes of algebraic
procedures; type-checking is done on the results of integer procedures, but
not on real procedures (in the current REDUCE release). \meta{identifier}
may be any valid REDUCE identifier that is not already a procedure name,
operator, \nameref{array} or \nameref{matrix}.  
\meta{arg} is a formal parameter that may be any
valid REDUCE identifier.  \meta{body} is a single statement (a \nameref{group} 
or \nameref{block} statement may be used) with the desired activities in it.

\begin{Examples}
\begin{multilineinput}
procedure fac(n);
   if not (fixp(n) and n>=0)
     then rederr "Choose nonneg. integer only"
    else for i := 0:n-1 product i+1;
\end{multilineinput}
			      &        FAC \\
fac(0);                       &        1 \\
fac(5);                       &        120 \\
fac(-5);                      &        ***** choose nonneg. integer only
\end{Examples}

\begin{Comments}
Procedures are automatically declared as operators upon definition.  When
REDUCE has parsed the procedure definition and successfully converted it to
a form for its own use, it prints the name of the procedure.  Procedure
definitions cannot be nested.  Procedures can call other procedures, or can
recursively call themselves.  Procedure identifiers can be cleared as you
would clear an operator.  Unlike \nameref{let} statements, new definitions
under the same procedure name replace the previous definitions completely.

Be careful not to use the name of a system operator for your own procedure.
REDUCE may or may not give you a warning message.  If you redefine a system
operator in your own procedure, the original function of the system operator
is lost for the remainder of the REDUCE session.

Procedures may have none, one, or more than one parameter.  A REDUCE
parameter is a formal parameter only; the use of {\it x} as a parameter in
a \name{procedure} definition has no connection with a value of {\it x} in
the REDUCE session, and the results of calling a procedure have no effect
on the value of {\it x}.  If a procedure is {\it called} with {\it x} as a
parameter, the current value of {\it x} is used as specified in the
computation, but is not changed outside the procedure.
Making an assignment statement by \name{:=} with a
formal parameter on the left-hand side only changes the value of the
calling parameter within the procedure.

Using a \nameref{let} statement inside a procedure always changes the value
globally: a \name{let} with a formal parameter makes the change to the calling
parameter.  \name{let} statements cannot be made on local variables inside
\nameref{begin}\ldots\name{end} \nameref{block}\name{s}.  
When \nameref{clear} statements are used on formal
parameters, the calling variables associated with them are cleared globally too.
The use of \name{let} or \name{clear} statements inside procedures 
should be done with extreme caution.

Arrays and operators may be used as parameters to procedures.  The body of the
procedure can contain statements that appropriately manipulate these
arguments.  Changes are made to values of the calling arrays or operators.
Simple expressions can also be used as arguments, in the place of scalar
variables.  Matrices may {\it not} be used as arguments to procedures.

A procedure that has no parameters is called by the procedure name,
immediately followed by empty parentheses.  The empty parentheses may be left
out when writing a procedure with no parameters, but must appear in a call of
the procedure.  If this is a nuisance to you, use a \nameref{let} statement on
the name of the procedure (i.e., \name{let noargs = noargs()}) after which
you can call the procedure by just its name.

Procedures that have a single argument can leave out the parentheses around
it both in the definition and procedure call.  (You can use the parentheses if
you wish.)  Procedures with more than one argument must use parentheses, with
the arguments separated by commas.

Procedures often have a \name{begin}\ldots\name{end} block in them. Inside the
block, local variables are declared using \name{scalar}, \name{real} or 
\name{integer} declarations.  
The declarations must be made immediately after the word
\name{begin}, and if more than one type of declaration is made, they are
separated by semicolons.  REDUCE currently does no type checking on local
variables; \name{real} and \name{integer} are treated just like \name{scalar}.
Actions take place as specified in the statements inside the block statement.
Any identifiers that are not formal parameters or local variables are treated
as global variables, and activities involving these identifiers are global in
effect.

If a return value is desired from a procedure call, a specific
\nameref{return} command must be the last statement executed before exiting
from the procedure.  If no \name{return} is used, a procedure returns a
zero or no value.

Procedures are often written in a file using an editor, then the file
is input using the command \nameref{in}.  This method allows easy changes in
development, and also allows you to load the named procedures whenever
you like, by loading the files that contain them.
\end{Comments}
\end{Command}


\begin{Command}{REPEAT}
\index{loop}
The \nameref{repeat} command causes repeated execution of a statement 
\nameindex{until}
the given  condition is found to be true.  The statement is always executed
at least once.
\begin{Syntax}
\name{repeat} \meta{statement} \name{until} \meta{condition}
\end{Syntax}

\meta{statement} can be a single statement, \nameref{group} statement, or
a \name{begin}\ldots\name{end} \nameref{block}.  \meta{condition} must be 
a logical operator that evaluates to {\it true} or {\it nil}.

\begin{Examples}
<<m := 4; repeat <<write 100*x*m;m := m-1>> until m = 0>>;
			     & \begin{multilineoutput}{6cm}
400*X
300*X
200*X
100*X
\end{multilineoutput}\\

<<m := -1; repeat <<write m; m := m-1>> until m <= 0>>;
			     &           -1

\end{Examples}
\begin{Comments}
\name{repeat} must always be followed by an \name{until} with a condition.
Be careful not to generate an infinite loop with a condition that is never
true.  In the second example, if the condition had been \name{m = 0}, it
would never have been true since \name{m} already had value -2 when the
condition was first evaluated.
\end{Comments}
\end{Command}


\begin{Operator}{REST}
\index{list}\index{decomposition}
The \name{rest} operator returns a \nameref{list} containing all but the first 
element of the list it is given.
\begin{Syntax}
\name{rest}\(\meta{list}\) or \name{rest} \meta{list}


\end{Syntax}
\meta{list} must be a non-empty list, but need not have more than one element.

\begin{Examples}
alist := {a,b,c,d};          &     ALIST := \{A,B,C,D\}; \\
rest alist;                  &     \{B,C,D\} \\
blist := {x,y,{aa,bb,cc},z}; &     BLIST := \{X,Y,\{AA,BB,CC\},Z\} \\
second rest blist;           &     \{AA,BB,CC\} \\
clist := {c};                &     CLIST := C \\
rest clist;                  &     \{\}
\end{Examples}

\end{Operator}


\begin{Command}{RETURN}
The \name{return} command causes a value to be returned from inside a
\name{begin}\ldots\name{end} \nameref{block}.
\begin{TEX}
\begin{Syntax}
\name{begin} \meta{statements} \name{return} \meta{\&option(expression)}
 \name{end}
\end{Syntax}    
\end{TEX}
\begin{INFO}
{\begin{Syntax}
\name{begin} \meta{statements} \name{return} \meta{(expression)}
 \name{end}
\end{Syntax}
}\end{INFO}

\meta{statements} can be any valid REDUCE statements.  The value of
\meta{expression} is returned.

\begin{Examples}
begin write "yes"; return a end;                       & 
\begin{multilineoutput}{5cm}
yes
A
\end{multilineoutput}\\
\begin{multilineinput}
procedure dumb(a);
  begin if numberp(a) then return a else return 10 end;
\end{multilineinput}
						       &      DUMB \\
dumb(x);                                               &      10 \\
dumb(-5);                                              &      -5  \\
\begin{multilineinput}
procedure dumb2(a);
  begin c := a**2 + 2*a + 1; d := 17; c*d; return end;
\end{multilineinput}		                       &      DUMB2 \\
dumb2(4); \\
c;                                                     &      25 \\
d;                                                     &      17
\end{Examples}

\begin{Comments}
Note in \name{dumb2} above that the assignments were made as requested, but
the product \name{c*d} cannot be accessed.  Changing the procedure to read
\name{return c*d} would remedy this problem.

The \name{return} statement is always the last statement executed before
leaving the block.  If \name{return} has no argument, the block is exited but
no value is returned.  A block statement does not need a \name{return} ;
the statements inside terminate in their normal fashion without one.  
In that case no value is returned, although the specified actions inside the 
block take place.

The \name{return} command can be used inside \name{<<}\ldots\name{>>} 
\nameref{group} statements and
\nameref{if}\ldots\name{then}\ldots\name{else} commands that 
are inside \name{begin}\ldots\name{end} \nameref{block}s.
It is not valid in these constructions that are not inside 
a \name{begin}\ldots\name{end}
 block. It is not valid inside \nameref{for}, 
\nameref{repeat}\ldots\name{until} or \nameref{while}\ldots\name{do}
 loops in any construction.  To force early termination from loops, the
\name{go to}(\nameref{goto}) command must be used. 
When you use nested block statements, a
\name{return} from an inner block exits returning a value to the next-outermost
block, rather than all the way to the outside.
\end{Comments}
\end{Command}


\begin{Operator}{REVERSE}
\index{list}
The \name{reverse} operator returns a \nameref{list} that is the reverse of the 
list it is given.
\begin{Syntax}
\name{reverse}\(\meta{list}\) or \name{reverse} \meta{list}
\end{Syntax}

\meta{list} must be a \nameref{list}.

\begin{Examples}
aa := \{c,b,a,\{x**2,z**3\},y\}; &    AA := \{C,B,A,\{X^{2},Z^{3}\},Y\} \\
reverse aa;                  &    \{Y,\{X^{2},Z^{3}\},A,B,C\} \\
reverse(q . reverse aa);  &    \{C,B,A,\{X^{2},Z^{3}\},Y,Q\}
\end{Examples}

\begin{Comments}
\name{reverse} and \nameref{cons} can be used together to add a new element to
the end of a list (\name{.} adds its new element to the beginning).  The
\name{reverse} operator uses a noticeable amount of system resources,
especially if the list is long.  If you are doing much heavy-duty list
manipulation, you should probably design your algorithms to avoid much
reversing of lists.  A moderate amount of list reversing is no problem.
\end{Comments}
\end{Operator}


\begin{Type}{RULE}
\index{rule}\index{rule list}
A \name{rule} is an instruction to replace an algebraic expression
or a part of an expression by another one. 
\begin{Syntax}
\meta{lhs} => \meta{rhs} or
\meta{lhs} => \meta{rhs} \name{when} \meta{cond}
\end{Syntax}
\meta{lhs} is an algebraic expression used as search pattern and
\meta{rhs} is an algebraic expression which replaces matches of
\meta{rhs}. \name{=>} is the operator \nameref{replace}.

\meta{lhs} can contain \nameref{free variable}s which are
symbols preceded by a tilde \nameindex{~} in their leftmost position
in  \meta{lhs}. 
A double tilde marks an \nameref{optional free variable}.
If a rule has a \name{when} \meta{cond}
part it will fire only if the evaluation of \meta{cond} has a
result \nameref{true}.   \meta{cond} may contain references to
free variables of \meta{lhs}.

Rules can be collected in a \nameref{list} which then forms a
\nameindex{rule list}. \name{Rule lists} can be used to collect
algebraic knowledge for a  specific evaluation context.

\name{Rules} and \name{rule lists} are globally activated and 
deactivated  by \nameref{let}, \nameref{forall}, \nameref{clearrules}.
For a single evaluation they can be locally activate by \nameref{where}. 
The active rules for an operator can be visualized by \nameref{showrules}.

\begin{Examples}
operator f,g,h; \\
let f(x) => x^2; \\
f(x); & x^{2}\\
g_rules:={g(~n,~x)=>h(n/2,x) when evenp n,\\
g(~n,~x)=>h((1-n)/2,x) when not evenp n}$\\
let g_rules;\\
g(3,x); & h(-1,x)
\end{Examples}

\end{Type}

\begin{Type}{Free Variable}
\index{variable}
A variable preceded by a tilde is considered as \name{free variable}
and stands for an arbitrary part in an algebraic form during
pattern matching. Free variables occur in the left-hand sides
of \nameref{rule}s, in the side relations for \nameref{compact}
and in the first arguments of \nameref{map} and \nameref{select}
calls. See \nameref{rule} for examples.

In rules also \nameref{optional free variable}s may occur.
\end{Type}

\begin{Type}{Optional Free Variable}
\index{variable}
A variable preceded by a double tilde is considered as 
\name{optional free variable}
and stands for an arbitrary part part in an algebraic form during
pattern matching. In contrast to ordinary \nameref{free variable}s
an operator pattern with an \name{optional free variable}
matches also if the operand for the variable is missing. In such
a case the variable is bound to a neutral value.
Optional free variables can be used as

   term in a sum: set to 0 if missing,

   factor in a product: set to 1 if missing,

   exponent: set to 1 if missing

\begin{Examples}
   sin(~~u + ~~n * pi) => sin(u) when evenp u;
\end{Examples}

Optional free variables are allowed only in the left-hand sides
of \nameref{rule}s.
\end{Type}

\begin{Operator}{SECOND}
\index{list}\index{decomposition}
The \name{second} operator returns the second element of a list.
\begin{Syntax}
\name{second}\(\meta{list}\) or \name{second} \meta{list}


\end{Syntax}

\meta{list} must be a list with at least two elements, to avoid an error
message.

\begin{Examples}
alist := \{a,b,c,d\};          &      ALIST := \{A,B,C,D\} \\
second alist;                &      B \\
blist := \{x,\{aa,bb,cc\},z\};   &      BLIST := \{X,\{AA,BB,CC\},Z\} \\
second second blist;         &      BB
\end{Examples}
\end{Operator}


\begin{Operator}{SET}
\index{assign}
The \name{set} operator is used for assignments when you want both sides of
the assignment statement to be evaluated.
%%% INCONSISTENT??? hyphen after restricted
\begin{Syntax}
\name{set}\(\meta{restricted\_expression},\meta{expression}\)
\end{Syntax}

\meta{expression} can be any REDUCE expression; \meta{restricted\_expression}
must be an identifier or an expression that evaluates to an identifier.

\begin{Examples}
a := y;                      &           A := Y \\
set(a,sin(x^2));             &           SIN(X^{2}) \\
a;                           &           SIN(X^{2}) \\
y;                           &           SIN(X^{2}) \\
a := b + c;                  &           A := B + C \\
set(a-c,z);                  &           Z \\
b;                           &           Z
\end{Examples}

\begin{Comments}
Using an \nameref{array} or \nameref{matrix} reference as the first 
argument to \name{set} has
the result of setting the {\it contents} of the designated element to
\name{set}'s second argument.  You should be careful to avoid unwanted
side effects when you use this facility.
\end{Comments}
\end{Operator}


\begin{Operator}{SETQ}
\index{assign}
The \name{setq} operator is an infix or prefix binary assignment operator.
It is identical to \name{:=}.
\begin{Syntax}
\name{setq}\(\meta{restricted\_expression},\meta{expression}\) or \\
\meta{restricted\_expression} \name{setq} \meta{expression}
\end{Syntax}

\meta{restricted expression} is ordinarily a single identifier, though
simple expressions may be used (see Comments below). \meta{expression} can
be any valid REDUCE expression.  If \meta{expression} is a \nameref{matrix}
identifier, then \meta{restricted\_expression} can be a matrix identifier
(redimensioned if necessary), which has each element set to the
corresponding elements of the identifier on the right-hand side.

\begin{Examples}
setq(b,6);                   &           B := 6 \\
c setq sin(x);               &           C := SIN(X) \\
w + setq(c,x+3) + z;         &           W + X + Z + 3 \\
c;                           &           X + 3 \\
setq(a1 + a2,25);            &           A1 + A2 := 25 \\
a1;                          &           - (A2 - 25)
\end{Examples}
\begin{Comments}
Embedding a \name{setq} statement in an expression has the side effect of making
the assignment, as shown in the third example above.

Assignments are generally done for identifiers, but may be done for simple
expressions as well, subject to the following remarks:
\begin{itemize}

\item[(i)]
If the left-hand side is an identifier, an operator, or a power, the rule
is added to the rule table.

\item[(ii)]
If the operators \name{- + /} appear on the left-hand side, all but the first
term of the expression is moved to the right-hand side.

\item[(iii)]
If the operator \name{*} appears on the left-hand side, any constant terms are
moved to the right-hand side, but the symbolic factors remain.
\end{itemize}

Be careful not to make a recursive \name{setq} assignment that is not
controlled inside a loop statement.  The process of resubstitution
continues until you get a stack overflow message. \name{setq} can be used
to attach functionality to operators, as the \name{:=} does.
\end{Comments}
\end{Operator}


\begin{Operator}{THIRD}
\index{list}\index{decomposition}
The \name{third} operator returns the third item of a \nameref{list}.
\begin{Syntax}
\name{third}\(\meta{list}\) or \name{third} \meta{list}

\end{Syntax}


\meta{list} must be a list containing at least three items to avoid an error
message.

\begin{Examples}
alist := \{a,b,c,d\};          &      ALIST := \{A,B,C,D\} \\
third alist;                 &      C \\
blist := \{x,\{aa,bb,cc\},y,z\}; &      BLIST := \{X,\{AA,BB,CC\},Y,Z\}; \\
third second blist;          &      CC \\
third blist;                 &      Y
\end{Examples}

\end{Operator}

\begin{Operator}{WHEN}
\index{rule}
The \name{when} operator is used inside a \name{rule} to make the
execution of the rule depend on a boolean condition which is
evaluated at execution time. For the use see \nameref{rule}. 
\end{Operator}

