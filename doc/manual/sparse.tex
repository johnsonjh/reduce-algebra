

\newcommand\sparselineqlist {lin\_eqn$_{1}$,lin\_eqn$_{2}$, \ldots ,lin\_eqn$_{n}$}
\newcommand\sparsematlist   {mat$_{1}$,mat$_{2}$, \ldots ,mat$_{n}$}
\newcommand\sparseveclist   {vec$_{1}$,vec$_{2}$, \ldots ,vec$_{n}$}

\newcommand\sparselazyfootnote{\footnote{The \{\}'s can be omitted.}}

\index{Linear Algebra package}

\subsection{Introduction}
A very powerful feature of \REDUCE{} is the ease with which matrix 
calculations can be performed.
This package extends the available matrix feature to enable calculations
with sparse matrices. This package also provides a selection of 
functions that are useful in the world of linear algebra with respect to
sparse matrices.

\subsubsection*{Loading the Package}
The package is loaded by: \texttt{load\_package sparse;}

\subsection{Sparse Matrix Calculations}
To extend the the syntax to this class of calculations we need to add an
expression type \texttt{sparse}.

\subsubsection{Sparse Variables}
\ttindextype{SPARSE}{declaration}
An identifier may be declared a sparse variable by the declaration
\texttt{SPARSE}.
The size of the sparse matrix must be declared explicitly in the matrix
declaration. For example,
\begin{verbatim}
sparse aa(10,1),bb(200,200);
\end{verbatim}
declares \texttt{AA} to be a 10 x 1 (column) sparse matrix and \texttt{Y} to 
be a 200 x 200 sparse matrix.
The declaration \texttt{SPARSE} is similar to the declaration \texttt{MATRIX}.
Once a symbol is declared to name a sparse matrix, it can not also be 
used to name an array, operator, procedure, or used as an ordinary 
variable. For more information see the Matrix Variables
section (\ref{sec:core-matrix-variables}).

\subsubsection{Assigning Sparse Matrix Elements}
Once a matix has been declared a sparse matrix all elements of the 
matrix are initialized to 0. Thus when a sparse matrix is initially 
referred to the message
\begin{verbatim}
"The matrix is dense, contains only zeros"
\end{verbatim}
is returned. When printing out a matrix only the non-zero elements are 
printed. This is due to the fact that only the non-zero elements of the 
matrix are stored. 
To assign the elements of the declared matrix we use the following 
syntax. Assuming \texttt{AA} and \texttt{BB} have been declared as spasre 
matrices, we simply write,
\begin{verbatim}
aa(1,1):=10;
bb(100,150):=a;
\end{verbatim}
etc. This then sets the element in the first row and first column to 10,
or the element in the 100th row and 150th column to \texttt{a}.

\subsubsection{Evaluating Sparse Matrix Elements}
Once an element of a sparse matrix has been assingned, it may be referred
to in standard array element notation. Thus \texttt{aa(2,1)} refers to the 
element in the second row and first column of the sparse matrix \texttt{AA}.

\subsection{Sparse Matrix Expressions}
These follow the normal rules of matrix algebra. Sums and products must 
be of compatible size; otherwise an error will result during evaluation.
Similarly, only square matrices may be raised to a power.
A negative power is computed as the inverse of the matrix raised to the
corresponding positive power. For more information and the syntax for 
matrix algebra see the Matrix Expressions section (\ref{sec:core-matrix-expressions}).

\subsection{Operators with Sparse Matrix Arguments}
The operators in the Sparse Matix Package are the same as those in the
Matrix Packge with the exception that the \texttt{NULLSPACE} operator is 
not defined. See section Operators with Matrix Arguments 
(\ref{sec:core-matrix-operators}) for more details.

\subsubsection{Examples}
In the examples the matrix $\mathcal{AA}$ will be 

\begin{flushleft}
\begin{math}
\mathcal{AA} = \left( \begin{array}{cccc} 1 & 0 & 0 & 0 \\ 0 & 3 & 0 & 0 \\ 
0 & 0 & 5 & 0 \\ 0 & 0 & 0 & 9
\end{array} \right)
\end{math}
\end{flushleft}
\begin {verbatim}
det ppp;

135

trace ppp;

18

rank ppp;

4

spmateigen(ppp,eta);

{{eta - 1,1,

  spm(1,1) := arbcomplex(4)$
  },

 {eta - 3,1,

  spm(2,1) := arbcomplex(5)$
  },

 {eta - 5,1,

  spm(3,1) := arbcomplex(6)$
  },

 {eta - 9,1,

  spm(4,1) := arbcomplex(7)$
  }}
\end{verbatim}

\subsection{The Linear Algebra Package for Sparse Matrices}
This package is an extension of the Linear Algebra Package for \REDUCE{}
described in section \ref{LINALG}.
These functions are described 
alphabetically in section \ref{sec:sparse-available-functions}.
They can be classified into four sections(n.b: the numbers after 
the dots signify the function label in section 6).
\subsubsection{Basic matrix handling}
\begin{center}
\begin{tabular}{l l l l l l}
spadd\_columns     & \ldots & \ref{sparse:spadd_columns}  & 
spadd\_rows        & \ldots & \ref{sparse:spadd_rows}  \\
spadd\_to\_columns & \ldots & \ref{sparse:spadd_to_columns}  &
spadd\_to\_rows    & \ldots & \ref{sparse:spadd_to_rows}  \\
spaugment\_columns & \ldots & \ref{sparse:spaugment_columns}  &
spchar\_poly       & \ldots & \ref{sparse:spchar_poly}  \\
spcol\_dim         & \ldots & \ref{sparse:spcol_dim}  &
spcopy\_into       & \ldots & \ref{sparse:spcopy_into} \\
spdiagonal         & \ldots & \ref{sparse:spdiagonal} &
spextend           & \ldots & \ref{sparse:spextend} \\
spfind\_companion  & \ldots & \ref{sparse:spfind_companion}  &
spget\_columns     & \ldots & \ref{sparse:spget_columns} \\
spget\_rows        & \ldots & \ref{sparse:spget_rows} &
sphermitian\_tp    & \ldots & \ref{sparse:sphermitian_tp} \\
spmatrix\_augment  & \ldots & \ref{sparse:spmatrix_augment} &
spmatrix\_stack    & \ldots & \ref{sparse:spmatrix_stack} \\
spminor            & \ldots & \ref{sparse:spminor} &
spmult\_columns    & \ldots & \ref{sparse:spmult_columns} \\ 
spmult\_rows       & \ldots & \ref{sparse:spmult_rows} &
sppivot            & \ldots & \ref{sparse:sppivot} \\
spremove\_columns  & \ldots & \ref{sparse:spremove_columns} &
spremove\_rows     & \ldots & \ref{sparse:spremove_rows} \\
sprow\_dim         & \ldots & \ref{sparse:sprow_dim} &
sprows\_pivot      & \ldots & \ref{sparse:sprows_pivot} \\
spstack\_rows      & \ldots & \ref{sparse:spstack_rows} &
spsub\_matrix      & \ldots & \ref{sparse:spsub_matrix} \\
spswap\_columns    & \ldots & \ref{sparse:spswap_columns} &
spswap\_entries    & \ldots & \ref{sparse:spadd_entries} \\
spswap\_rows       & \ldots & \ref{sparse:spswap_rows} &
\end{tabular}
\end{center}

\subsubsection{Constructors}

Functions that create sparse matrices.

\begin{center}
\begin{tabular}{l l l l l l}
spband\_matrix       & \ldots & \ref{sparse:spband_matrix} & 
spblock\_matrix      & \ldots & \ref{sparse:spblock_matrix} \\
spchar\_matrix       & \ldots & \ref{sparse:spcoeff_matrix} & 
spcoeff\_matrix      & \ldots & \ref{sparse:spcoeff_matrix} \\ 
spcompanion          & \ldots & \ref{sparse:spcompanion} & 
sphessian            & \ldots & \ref{sparse:sphessian} \\
spjacobian           & \ldots & \ref{sparse:spjacobian} &
spjordan\_block      & \ldots & \ref{sparse:spjordan_block} \\ 
spmake\_identity     & \ldots & \ref{sparse:spmake_identity} &
\end{tabular}
\end{center}

\subsubsection{High level algorithms}

\begin{center}
\begin{tabular}{l l l l l l}
spchar\_poly       & \ldots & \ref{sparse:spchar_poly} & 
spcholesky         & \ldots & \ref{sparse:spcholesky} \\ 
spgram\_schmidt    & \ldots & \ref{sparse:spgram_schmidt} & 
splu\_decom        & \ldots & \ref{sparse:splu_decom} \\
sppseudo\_inverse  & \ldots & \ref{sparse:sppseudo_inverse} & 
spsvd              & \ldots & \ref{sparse:spsvd} 
\end{tabular}
\end{center}

\subsubsection{Predicates}

\begin{center}
\begin{tabular}{l l l l l l}
matrixp     & \ldots & \ref{sparse:matrixp} & 
sparsematp  & \ldots & \ref{sparse:sparsematp} \\
squarep     & \ldots & \ref{sparse:squarep} &
symmetricp  & \ldots & \ref{sparse:symmetricp} 
\end{tabular}
\end{center}

\subsubsection*{Note on examples:} 

In the examples the matrix $\mathcal{A}$ will be 

\begin{flushleft}
\begin{math}
\mathcal{A} = \begin{pmatrix} 1 & 0 & 0 \\ 0 & 5 & 0 \\ 0 & 0 & 9 \end{pmatrix}
\end{math}
\end{flushleft}
Unfortunately, due to restrictions of size, it is not practical to use
``large'' sparse matrices in the examples. As a result the examples 
shown may appear trivial, but they give an idea of how the functions 
work.  

\subsubsection*{Notation}

Throughout $\mathcal{I}$ is used to indicate the identity matrix and 
$\mathcal{A}^T$ to indicate the transpose of the matrix $\mathcal{A}$.

\subsection{Available Functions}
\label{sec:sparse-available-functions}

\subsubsection{spadd\_columns, spadd\_rows}
\label{sparse:spadd_columns}
\ttindextype{SPADD\_COLUMNS}{operator}

\begin{description}
\item[Syntax:]\mbox{}\\
\texttt{spadd\_columns($\mathcal{A}$,c1,c2,expr);} 

\begin{tabular}{l l l}
$\mathcal{A}$ & :- & a sparse matrix. \\
$c1,c2$      & :- & positive integers. \\
expr       & :- & a scalar expression. 
\end{tabular}

\item[Synopsis:]\mbox{}\\
\texttt{spadd\_columns} replaces column $c2$ of $\mathcal{A}$ by\\
$\texttt{expr} * \texttt{column($\mathcal{A}$,c1)} + \texttt{column($\mathcal{A}$,c2)}$.\\
\texttt{add\_rows} performs the equivalent task on the rows of $\mathcal{A}$.

\item[Examples:]
\begin{flalign*}
&\texttt{spadd\_columns}(\mathcal{A},1,2,x)  =  
  \begin{pmatrix} 1 & x & 0 \\ 0 & 5 & 0 \\ 0 & 0 & 9 \end{pmatrix} && \\[2mm]
& \texttt{spadd\_rows}(\mathcal{A},2,3,5)  = 
\begin{pmatrix} 1 & 0 & 0 \\ 0 & 5 & 0 \\ 0 & 25 & 9 \end{pmatrix} &&
\end{flalign*}

\item[Related functions:]\mbox{}\\
\texttt{spadd\_to\_columns}, \texttt{spadd\_to\_rows}, 
\texttt{spmult\_columns}, \texttt{spmult\_rows}.
\end{description}

\subsubsection{spadd\_rows}
\label{sparse:spadd_rows}
\ttindextype{SPADD\_ROWS}{operator}

See: \texttt{spadd\_columns}.


\subsubsection{spadd\_to\_columns, spadd\_to\_rows}
\label{sparse:spadd_to_columns}
\ttindextype{SPADD\_TO\_COLUMNS}{operator}

\begin{description}
  \item[Syntax:]\mbox{}\\*
\texttt{spadd\_to\_columns($\mathcal{A}$,column\_list,expr);}\\[2mm]
\begin{tabular}{l l l}
$\mathcal{A}$   &:-& a sparse matrix. \\
column\_list &:-& a positive integer or a list of positive integers. \\
expr        &:-& a scalar expression.
\end{tabular}

\item[Synopsis:]\mbox{}\\*
\texttt{spadd\_to\_columns} adds expr to each column specified in 
column\_list of $\mathcal{A}$.  

\texttt{spadd\_to\_rows} performs the equivalent task on the rows of 
$\mathcal{A}$.

\item[Examples:]
\begin{flalign*}  
& \texttt{spadd\_to\_columns}(\mathcal{A},\{1,2\},10)  = 
\begin{pmatrix} 11 & 10 & 0 \\ 10 & 15 & 0 \\ 10 & 10 & 9 \end{pmatrix} && \\[2mm]
&\texttt{spadd\_to\_rows}(\mathcal{A},2,-x)  = 
\begin{pmatrix} 1 & 0 & 0 \\ -x & -x+5 & -x \\ 0 & 0 & 9 \end{pmatrix}&&
\end{flalign*}

\item[Related functions:]\mbox{}\\
\texttt{spadd\_columns}, \texttt{spadd\_rows}, \texttt{spmult\_rows}, 
\texttt{spmult\_columns}.
\end{description}

\subsubsection{spadd\_to\_rows}
\label{sparse:spadd_to_rows}
\ttindextype{SPADD\_TO\_ROWS}{operator}

See: \texttt{spadd\_to\_columns}.


\subsubsection{spaugment\_columns, spstack\_rows}
\label{sparse:spaugment_columns}
\ttindextype{SPAUGMENT\_COLUMNS}{operator}

\begin{description}
\item[Syntax:]\mbox{}\\
 \texttt{spaugment\_columns($\mathcal{A}$,column\_list);}\\[2mm]
\begin{tabular}{l l l}
$\mathcal{A}$  &:-& a sparse matrix. \\
column\_list &:-&  either a positive integer or a list of positive 
                   integers. 
\end{tabular}

\item[Synopsis:]\mbox{}\\
\texttt{spaugment\_columns} gets hold of the columns of $\mathcal{A}$ specified 
in column\_list and sticks them together. 

\texttt{spstack\_rows} performs the same task on rows of 
                $\mathcal{A}$.

\item[Examples:]
\begin{flalign*}  
&\texttt{spaugment\_columns}(\mathcal{A},\{1,2\})  = 
\begin{pmatrix} 1 & 0 \\ 0 & 5 \\ 0 & 0 \end{pmatrix} &&  \\[2mm]
& \texttt{spstack\_rows}(\mathcal{A},\{1,3\})  = 
\begin{pmatrix} 1 & 0 & 0 \\ 0 & 0 & 9 \end{pmatrix} &&
\end{flalign*}

\item[Related functions:]\mbox{}\\
\texttt{spget\_columns}, \texttt{spget\_rows}, 
\texttt{spsub\_matrix}.
\end{description}

\subsubsection{spband\_matrix}
\label{sparse:spband_matrix}
\ttindextype{SPBAND\_MATRIX}{operator}

\begin{description}
\item[Syntax:]\mbox{}\\
\texttt{spband\_matrix(expr\_list,square\_size);}\\[2mm]
\begin{tabular}{l l p{.72\linewidth}}
expr\_list  &:-& 
either a single scalar expression or a list of an odd number of scalar
expressions.\\
square\_size &:-& a positive integer.
\end{tabular}

\item[Synopsis:]\mbox{}\\
                \texttt{spband\_matrix} creates a sparse square matrix of 
                dimension square\_size. 

\item[Examples:]
\begin{flushleft}  
\begin{math}  
\texttt{spband\_matrix}(\{x,y,z\},6) = 
\begin{pmatrix} y & z & 0 & 0 & 0 & 0 \\ x & y & z & 0 & 0
& 0 \\ 0 & x & y & z & 0 & 0 \\ 0 & 0 & x & y & z & 0 \\ 0 & 0 & 0 & x &
 y & z \\ 0 & 0 & 0 & 0 & x & y 
\end{pmatrix} 
\end{math}  
\end{flushleft}

\item[Related functions:]\mbox{}\\
 \texttt{spdiagonal}.
\end{description}

\subsubsection{spblock\_matrix}
\label{sparse:spblock_matrix}
\ttindextype{SPBLOCK\_MATRIX}{operator}

\begin{description}

\item[Syntax:]\mbox{}\\
\texttt{spblock\_matrix(r,c,matrix\_list);}\\[2mm]
\begin{tabular}{l l l}
r,c          &:-& positive integers. \\
matrix\_list &:-& a list of matrices of either sparse or matrix type. 
\end{tabular}

\item[Synopsis:]\mbox{}\\
\texttt{spblock\_matrix} creates a sparse matrix that consists of r by c matrices 
filled from the matrix\_list row wise.

\item[Examples:]
\begin{flalign*}  
&\mathcal{B} = \begin{pmatrix} 1 & 0 \\ 0 & 1 \end{pmatrix}, \,\,
 \mathcal{C} = \begin{pmatrix} 5 \\ 0 \end{pmatrix}, \,\,
 \mathcal{D} = \begin{pmatrix} 22 & 0 \\ 0 & 0 \end{pmatrix} && \\[2mm]
&\texttt{spblock\_matrix}(2,3,\{\mathcal{B,C,D,D,C,B}\})  =  
\begin{pmatrix} 1 & 0 & 5 & 22 & 0 \\ 0 & 1 & 0 & 0 & 0 \\
22 & 0 & 5 & 1 & 0 \\ 0 & 0 & 0 & 0 & 1
\end{pmatrix} &&
\end{flalign*}
\end{description}

\subsubsection{spchar\_matrix}
\label{sparse:spchar_matrix}
\ttindextype{SPCHAR\_MATRIX}{operator}

\begin{description}
\item[Syntax:]\mbox{}\\
\texttt{spchar\_matrix($\mathcal{A},\lambda$);}\\[2mm]
\begin{tabular}{l l l}
$\mathcal{A}$ &:-& a square sparse matrix. \\
$\lambda$  &:-& a symbol or algebraic expression. 
\end{tabular}

\item[Synopsis:]\mbox{}\\
\texttt{spchar\_matrix} creates the characteristic matrix $\mathcal{C}$ of 
$\mathcal{A}$.

This is $\mathcal{C} = \lambda * \mathcal{I} - \mathcal{A}$. 

\item[Examples:]
\begin{flushleft}  
\begin{math}  
\texttt{spchar\_matrix}(\mathcal{A},x) = 
\begin{pmatrix} x-1 & 0 & 0 \\ 0 & x-5 & 0 \\ 0 & 0 & x-9 \end{pmatrix} 
\end{math}  
\end{flushleft}

\item[Related functions:]\mbox{}\\
\texttt{spchar\_poly}. 
\end{description}

\subsubsection{spchar\_poly}
\label{sparse:spchar_poly}
\ttindextype{SPCHAR\_POLY}{operator}

\begin{description}
\item[Syntax:]\mbox{}\\
\texttt{spchar\_poly($\mathcal{A},\lambda$);}\\[2mm]
\begin{tabular}{l l l}
$\mathcal{A}$ &:-& a sparse square matrix. \\
$\lambda$ &:-& a symbol or algebraic expression.
\end{tabular}

\item[Synopsis:]\mbox{}\\
\texttt{spchar\_poly} finds the characteristic polynomial of
                $\mathcal{A}$.  

This is the determinant of $\lambda * \mathcal{I} - \mathcal{A}$.

\item[Examples:]\mbox{}\\
\texttt{spchar\_poly($\mathcal{A}$,$x$) $= x^3-15*x^2-59*x-45$} 

\item[Related functions:]\mbox{}\\
\texttt{spchar\_matrix}. 
\end{description}

\subsubsection{spcholesky}
\label{sparse:spcholesky}
\ttindextype{SPCHOLESKY}{operator}

\begin{description}
\item[Syntax:]\mbox{}\\
\texttt{spcholesky($\mathcal{A}$);}\\[2mm]
\begin{tabular}{l l l}
$\mathcal{A}$ &:-& a positive definite sparse matrix containing numeric entries.
\end{tabular}

\item[Synopsis:]\mbox{}\\
\texttt{spcholesky} computes the cholesky decomposition of $\mathcal{A}$.

It returns \{$\mathcal{L,U}$\} where $\mathcal{L}$
is a lower matrix, $\mathcal{U}$ is an upper matrix, \\ $\mathcal{A} = 
\mathcal{LU}$, and $\mathcal{U} = \mathcal{L}^T$.

\item[Examples:]
\begin{flalign*}  
&\mathcal{F} = \begin{pmatrix} 1 & 0 & 0 \\ 0 & 5 & 0 \\ 0 & 0 & 9 \end{pmatrix}&&\\[2mm]
&\f{cholesky}(\mathcal{F})  =  
\left\{
   \begin{pmatrix} 1 & 0 & 0 \\ 0 & \sqrt{5} & 0 \\ 0 & 0& 3 \end{pmatrix},  
   \begin{pmatrix} 1 & 0 & 0 \\ 0 & \sqrt{5} & 0 \\ 0 & 0 & 3 \end{pmatrix}
\right\} && 
\end{flalign*}

\item[Related functions:]\mbox{}\\
\texttt{splu\_decom}.
\end{description}

\subsubsection{spcoeff\_matrix}
\label{sparse:spcoeff_matrix}
\ttindextype{SPCOEFF\_MATRIX}{operator}

\begin{description}
\item[Syntax:]\mbox{}\\
\texttt{spcoeff\_matrix(\{\sparselineqlist{}\});} \\[2mm]
\begin{tabular}{l l p{.435\linewidth}}
\sparselineqlist  &:-& linear equations. Can be 
of the form \textit{equation $=$ number} or just \textit{equation} which is
equivalent to \textit{equation $=$ 0}.
\end{tabular}

\item[Synopsis:]\mbox{}\\
\texttt{spcoeff\_matrix} creates the coefficient matrix 
                $\mathcal{C}$ of the linear equations. 

It returns \{$\mathcal{C,X,B}$\} such that $\mathcal{CX} = \mathcal{B}$.

\item[Examples:]
\begin{flalign*}
& \texttt{spcoeff\_matrix}(\{y-20*w=10,y-z=20,y+4+3*z,w+x+50\}) = && \\[3mm]
& \left\{ \begin{pmatrix} 1 & -20 & 0 & 0 \\ 1 & 0 & -1 & 0 \\
 1 & 0 & 3 & 0 \\ 0 & 1 & 0 & 1 
\end{pmatrix},
 \begin{pmatrix} y \\ w \\ z \\ x \end{pmatrix}, 
 \begin{pmatrix} 10 \\ 20 \\ -4 \\ 50 \end{pmatrix} \right\} &&
\end{flalign*}
\end{description}

\subsubsection{spcol\_dim, sprow\_dim}
\label{sparse:spcol_dim}
\ttindextype{SPCOL\_DIM}{operator}

\begin{description}
\item[Syntax:]\mbox{}\\
\texttt{column\_dim($\mathcal{A}$);}\\[2mm]
\begin{tabular}{l l l}
$\mathcal{A}$ &:-& a sparse matrix.
\end{tabular}

\item[Synopsis:]\mbox{}\\
\texttt{spcol\_dim} finds the column dimension of 
                $\mathcal{A}$. \\
\texttt{sprow\_dim} finds the row dimension of $\mathcal{A}$.

\item[Examples:]\mbox{}\\
\texttt{spcol\_dim}($\mathcal{A}$) = 3
\end{description}

\subsubsection{spcompanion}
\label{sparse:spcompanion}
\ttindextype{SPCOMPANION}{operator}

\begin{description}
\item[Syntax:]\mbox{}\\
\texttt{spcompanion(poly,x);}\\[2mm]
\begin{tabular}{l l l}
poly &:-& a monic univariate polynomial in x. \\
x    &:-& the variable.
\end{tabular}

\item[Synopsis:]\mbox{}\\
                \texttt{spcompanion} creates the companion matrix $\mathcal{C}$
                of poly. 

This is the square matrix of dimension $n$, where $n$ is the degree of poly 
w.r.t. $x$.
The entries of $\mathcal{C}$ are: 
                $\mathcal{C}(i,n) = -\texttt{coeffn}(\texttt{poly},x,i-1)$ for $i = 1 
                \ldots n$, $\mathcal{C}(i,i-1) = 1$ for $i = 2 \ldots n$ and 
                the rest are $0$.

\item[Examples:]
\begin{flushleft}  
\begin{math}  
\texttt{spcompanion}(x^4+17*x^3-9*x^2+11,x) = 
\begin{pmatrix}
  0 & 0 & 0 & -11 \\ 1 & 0 & 0 & 0 \\ 0 & 1 & 0 & 9 \\ 0 & 0 & 1 & -17 
\end{pmatrix} 
\end{math}  
\end{flushleft}

\item[Related functions:]\mbox{}\\
\texttt{spfind\_companion}.
\end{description}

\subsubsection{spcopy\_into}
\label{sparse:spcopy_into}
\ttindextype{SPCOPY\_INTO}{operator}

\begin{description}
\item[Syntax:]\mbox{}\\
\texttt{spcopy\_into($\mathcal{A,B}$,r,c);}\\[2mm]
\begin{tabular}{l l l}
$\mathcal{A,B}$ &:-& matrices of type sparse or matrix. \\
r,c          &:-& positive integers. 
\end{tabular}

\item[Synopsis:]\mbox{}\\
 \texttt{spcopy\_into} copies matrix $\mathcal{A}$ into 
                $\mathcal{B}$ with $\mathcal{A}$(1,1) at $\mathcal{B}$(r,c).

\item[Examples:]
\begin{flalign*}
&\mathcal{G} = \begin{pmatrix} 0 & 0 & 0 & 0 \\ 0 & 0 & 0 & 0 \\
0 & 0 & 0 & 0 \\ 0 & 0 & 0 & 0
\end{pmatrix}&&\\[2mm]
&\texttt{spcopy\_into}(\mathcal{A,G},1,2)  = 
\begin{pmatrix} 0 & 1 & 0 & 0 \\ 0 & 0 & 5 & 0 \\ 0 & 0 & 0 
& 9 \\ 0 & 0 & 0 & 0  
\end{pmatrix} &&
\end{flalign*}

\item[Related functions:]\mbox{}\\
\texttt{spaugment\_columns}, \texttt{spextend}, \texttt{spmatrix\_augment}, 
\texttt{spmatrix\_stack}, \texttt{spstack\_rows}, \texttt{spsub\_matrix}.
\end{description}


\subsubsection{spdiagonal}
\label{sparse:spdiagonal}
\ttindextype{SPDIAGONAL}{operator}

\begin{description}
\item[Syntax:]\mbox{}\\
 \texttt{spdiagonal(\{\sparsematlist{}\});}\sparselazyfootnote{}\\[2mm]
\begin{tabular}{l l p{.58\linewidth}}
\sparsematlist &:-& each can be either a scalar 
expr or a square matrix of sparse or matrix type. 
\end{tabular}

\item[Synopsis:]\mbox{}\\
 \texttt{spdiagonal} creates a sparse matrix that contains the 
input on the diagonal.

\item[Examples:]
\begin{flalign*}  
&\mathcal{H} = \begin{pmatrix} 66 & 77 \\ 88 & 99 \end{pmatrix} &&\\[2mm]
&\texttt{spdiagonal}(\{\mathcal{A},x,\mathcal{H}\}) = 
\begin{pmatrix} 1 & 0 & 0 & 0 & 0 & 0 \\ 0 & 5 & 0 & 0 & 0
& 0 \\ 0 & 0 & 9 & 0 & 0 & 0 \\ 0 & 0 & 0 & x & 0 & 0 \\ 0 & 0 & 0 & 0 
& 66 & 77 \\ 0 & 0 & 0 & 0 & 88 & 99 
\end{pmatrix} &&
\end{flalign*}

\item[Related functions:]\mbox{}\\
 \texttt{spjordan\_block}.
\end{description}

\subsubsection{spextend}
\label{sparse:spextend}
\ttindextype{SPEXTEND}{operator}

\begin{description}
\item[Syntax:]\mbox{}\\
 \texttt{spextend($\mathcal{A}$,r,c,expr);}\\[2mm]
\begin{tabular}{l l l}
$\mathcal{A}$ &:-& a sparse matrix. \\
r,c        &:-& positive integers. \\
expr      &:-& algebraic expression or symbol.
\end{tabular}

\item[Synopsis:]\mbox{}\\
                \texttt{spextend} returns a copy of $\mathcal{A}$ that has been 
                extended by r rows and c columns. The new entries are
                made equal to expr.

\item[Examples:]
\begin{flushleft}  
\begin{math}  
\texttt{spextend}(\mathcal{A},1,2,0) =
\begin{pmatrix} 1 & 0 & 0 & 0 & 0 \\ 0 & 5 & 0 & 0 & 0
\\ 0 & 0 & 9 & 0 & 0 \\ 0 & 0 & 0 & 0 & 0 
\end{pmatrix}
\end{math}  
\end{flushleft}

\item[Related functions:]\mbox{}\\
\texttt{spcopy\_into}, \texttt{spmatrix\_augment}, 
\texttt{spmatrix\_stack}, \texttt{spremove\_columns}, \texttt{spremove\_rows}.

\end{description}


\subsubsection{spfind\_companion}
\label{sparse:spfind_companion}
\ttindextype{SPFIND\_COMPANION}{operator}

\begin{description}
\item[Syntax:]\mbox{}\\
 \texttt{spfind\_companion($\mathcal{A}$,x);}\\[2mm]
\begin{tabular}{l l l}
$\mathcal{A}$ &:-& a sparse matrix. \\
x          &:-& the variable.
\end{tabular}

\item[Synopsis:]\mbox{}\\*
  Given a sparse companion matrix, \texttt{spfind\_companion} finds the polynomial 
from which it was made.

\item[Examples:]
\begin{flalign*}
&\mathcal{C} = \begin{pmatrix} 0 & 0 & 0 & -11 \\ 1 & 0 & 0 & 0 
\\ 0 & 1 & 0 & 9 \\ 0 & 0 & 1 & -17 
\end{pmatrix} &&\\[2mm]
& \texttt{spfind\_companion}(\mathcal{C},x) = x^4+17*x^3-9*x^2+11 &&
\end{flalign*}

\item[Related functions:]\mbox{}\\*
 \texttt{spcompanion}.
\end{description}

\subsubsection{spget\_columns, spget\_rows}
\label{sparse:spget_columns}
\ttindextype{SPGET\_COLUMNS}{operator}

\begin{description}
\item[Syntax:]\mbox{}\\*
\texttt{spget\_columns($\mathcal{A}$,column\_list);}\\[2mm]
\begin{tabular}{l l l}
$\mathcal{A}$ &:-& a sparse matrix. \\
c          &:-& either a positive integer or a list of positive 
                integers.
\end{tabular}

\item[Synopsis:]\mbox{}\\*
\texttt{spget\_columns} removes the columns of $\mathcal{A}$ specified in 
                column\_list and returns them as a list of column 
                matrices. 

 \texttt{spget\_rows} performs the same task on the rows of 
                $\mathcal{A}$. 

\item[Examples:]
\begin{flalign*}  
&\texttt{spget\_columns}(\mathcal{A},\{1,3\})  = 
\left\{ 
        \begin{pmatrix} 1 \\ 0 \\ 0 \end{pmatrix},
        \begin{pmatrix} 0 \\ 0 \\ 9 \end{pmatrix}
\right\} && \\[2mm]
&\texttt{spget\_rows}(\mathcal{A},2) & = & 
\left\{ 
        \begin{pmatrix} 0 & 5 & 0 \end{pmatrix}
\right\} &&
\end{flalign*}

\item[Related functions:]\mbox{}\\*
\texttt{spaugment\_columns}, \texttt{spstack\_rows}, 
\texttt{spsub\_matrix}.

\end{description}

\subsubsection{spget\_rows}
\label{sparse:spget_rows}
\ttindextype{SPGET\_ROWS}{operator}
See: \texttt{spget\_columns}.


\subsubsection{spgram\_schmidt}
\label{sparse:spgram_schmidt}
\ttindextype{SPGRAM\_SCHMIDT}{operator}

\begin{description}
\item[Syntax:]\mbox{}\\*
 \texttt{spgram\_schmidt(\{\sparseveclist{}\});}\\[2mm]
\begin{tabular}{l l p{.62\linewidth}}
\sparseveclist &:-& linearly independent vectors.
                             Each vector must be written as a list of
predefined sparse (column) matrices, eg: sparse a(4,1);, a(1,1):=1;
\end{tabular}

\item[Synopsis:]\mbox{}\\*
\texttt{spgram\_schmidt} performs the gram\_schmidt 
                orthonormalisation on the input vectors. 

It returns a list of orthogonal normalised vectors.

\item[Examples:]\mbox{}\\*
Suppose a,b,c,d correspond to sparse matrices representing the following
lists:  \{\{1,0,0,0\},\{1,1,0,0\},\{1,1,1,0\},\{1,1,1,1\}\}.

\texttt{spgram\_schmidt(\{\{a\},\{b\},\{c\},\{d\}\})} = \\[1mm]
 \mbox{}\qquad \{\{1,0,0,0\},\{0,1,0,0\},\{0,0,1,0\},\{0,0,0,1\}\}

\end{description}

\subsubsection{sphermitian\_tp}
\label{sparse:sphermitian_tp}
\ttindextype{SPHERMITIAN\_TP}{operator}

\begin{description}
\item[Syntax:]\mbox{}\\*
\texttt{sphermitian\_tp($\mathcal{A}$);}\\[2mm]
\begin{tabular}{l l l}
$\mathcal{A}$ &:-& a sparse matrix. 
\end{tabular}

\item[Synopsis:]\mbox{}\\*
                \texttt{sphermitian\_tp} computes the hermitian transpose of 
                $\mathcal{A}$. 

\item[Examples:]
\begin{flalign*}  
&\mathcal{J} = \begin{pmatrix} i+1 & i+2 & i+3 \\ 0 & 0 & 0 \\ 0 & i & 0 \end{pmatrix}&&\\[2mm]
&\texttt{sphermitian\_tp}(\mathcal{J}) = 
\begin{pmatrix} -i+1 & 0 & 0 \\ -i+2 & 0 & -i \\-i+3 & 0 & 0\end{pmatrix} 
\end{flalign*}                   

\item[Related functions:]\mbox{}\\*
\texttt{tp}\footnote{standard reduce call for the 
transpose of a matrix - see section \protect\ref{sec:core-matrix-operators}.}.
\end{description}

\subsubsection{sphessian}
\label{sparse:sphessian}
\ttindextype{SPHESSIAN}{operator}

\begin{description}
\item[Syntax:]\mbox{}\\*
 \texttt{sphessian(expr,variable\_list);}\\[2mm]
\begin{tabular}{l l l}
expr           &:-& a scalar expression. \\
variable\_list &:-& either a single variable or a list of variables.
\end{tabular}

\item[Synopsis:]\mbox{}\\*
                \texttt{sphessian} computes the hessian matrix of expr w.r.t.
                the variables in variable\_list. 

\item[Examples:]
\begin{flushleft}  
\hspace*{0.1in}
\begin{math}        
\texttt{sphessian}(x*y*z+x^2,\{w,x,y,z\})  = 
\begin{pmatrix} 0 & 0 & 0 & 0 \\ 0 & 2 & z & y \\ 0 & z & 0 
& x \\ 0 & y & x & 0
\end{pmatrix} 
\end{math}  
\end{flushleft}
\end{description}

\subsubsection{spjacobian}
\label{sparse:spjacobian}
\ttindextype{SPJACOBIAN}{operator}

\begin{description}
\item[Syntax:]\mbox{}\\*
\texttt{spjacobian(expr\_list,variable\_list);}\\[2mm]
\begin{tabular}{l l p{.72\linewidth}}
expr\_list    &:-& either a 
single algebraic expression or a list of algebraic expressions.\\
variable\_list &:-& either a single variable or a list of variables.
\end{tabular}

\item[Synopsis:]\mbox{}\\*
\texttt{spjacobian} computes the jacobian matrix of expr\_list w.r.t. 
variable\_list. 

\item[Examples:]
\begin{flalign*}
&\texttt{spjacobian(\{$x^4,x*y^2,x*y*z^3$\},\{$w,x,y,z$\})} = &&\\[2mm]
&\begin{pmatrix} 0 & 4*x^3 & 0 & 0 \\ 0 & y^2 & 2*x*y & 0 \\ 
0 & y*z^3 & x*z^3 & 3*x*y*z^2 
\end{pmatrix} &&
\end{flalign*}

\item[Related functions:]\mbox{}\\*
\texttt{sphessian}, \texttt{df}\footnote{standard reduce call 
for differentiation - see \protect\ref{sec:DF-operator}.}.
\end{description}

\subsubsection{spjordan\_block}
\label{sparse:spjordan_block}
\ttindextype{SPJORDAN\_BLOCK}{operator}

\begin{description}
\item[Syntax:]\mbox{}\\*
 \texttt{spjordan\_block(expr,square\_size);}\\[2mm]
\begin{tabular}{l l l}
expr        &:-& an algebraic expression or symbol. \\
square\_size &:-& a positive integer.
\end{tabular}

\item[Synopsis:]\mbox{}\\*
\texttt{spjordan\_block} computes the square jordan block matrix $\mathcal{J}$
                of dimension square\_size. 

\item[Examples:]
\begin{flushleft}  
\begin{math}        
\texttt{spjordan\_block(x,5)} = 
\begin{pmatrix} x & 1 & 0 & 0 & 0 \\ 0 & x & 1 & 0 & 0 \\ 0 
& 0 & x & 1 & 0 \\ 0 & 0 & 0 & x & 1 \\ 0 & 0 & 0 & 0 & x
\end{pmatrix} 
\end{math}  
\end{flushleft}

\item[Related functions:]\mbox{}\\*
 \texttt{spdiagonal}, \texttt{spcompanion}.
\end{description}

\subsubsection{splu\_decom}
\label{sparse:splu_decom}
\ttindextype{SPLU\_DECOM}{operator}

\begin{description}
\item[Syntax:]\mbox{}\\*
\texttt{splu\_decom($\mathcal{A}$);}\\[2mm]
\begin{tabular}{l l p{.848\linewidth}}
$\mathcal{A}$ &:-& a sparse matrix containing either 
numeric entries or imaginary entries with numeric coefficients.
\end{tabular}

\item[Synopsis:]\mbox{}\\*
              \texttt{splu\_decom} performs LU decomposition on $\mathcal{A}$,
              ie: it returns \{$\mathcal{L,U}$\} where $\mathcal{L}$
              is a lower diagonal matrix, $\mathcal{U}$ an upper diagonal
              matrix and $\mathcal{A} = \mathcal{LU}$.

\textbf{Caution:}
The algorithm used can swap the rows of $\mathcal{A}$ 
                during the calculation. This means that $\mathcal{LU}$ does
                not equal $\mathcal{A}$ but a row equivalent of it. Due to
                this, \texttt{splu\_decom} returns \{$\mathcal{L,U}$,vec\}. The
                call \texttt{spconvert($\mathcal{A}$,vec)} will return the 
                sparse matrix that has been decomposed, ie: $\mathcal{LU} = $
                \texttt{spconvert($\mathcal{A}$,vec)}.

\item[Examples:]
\begin{flushleft}
\begin{math}  
\mathcal{K} = \begin{pmatrix} 1 & 0 & 0 \\ 0 & 5 & 0 \\ 0 & 0 & 9 \end{pmatrix}
\end{math}  
\end{flushleft}

\begin{align*}  
\texttt{lu} := \texttt{splu\_decom}(\mathcal{K}) = & 
\left\{ 
        \begin{pmatrix} 1 & 0 & 0 \\ 0 & 5 & 0 \\ 0 & 0 & 9 \end{pmatrix}, 
        \begin{pmatrix} 1 & 0 & 0 \\ 0 & 1 & 0 \\ 0 & 0 & 1 \end{pmatrix}, 
	[\; 1 \; 2 \; 3 \; ]
\right\} 
\end{align*}

\begin{flalign*}  
&\texttt{first lu * second lu} =
        \begin{pmatrix} 1 & 0 & 0 \\ 0 & 5 & 0 \\ 0 & 0 & 9 \end{pmatrix} && \\[2mm]
&\texttt{convert($\mathcal{K}$,third lu}) & =  
        \begin{pmatrix} 1 & 0 & 0 \\ 0 & 5 & 0 \\ 0 & 0 & 9 \end{pmatrix} &&
\end{flalign*}

\item[Related functions:]\mbox{}\\*
 \texttt{spcholesky}.
\end{description}

\subsubsection{spmake\_identity}
\label{sparse:spmake_identity}
\ttindextype{SPMAKE\_IDENTITY}{operator}

\begin{description}
\item[Syntax:]\mbox{}\\*
 \texttt{spmake\_identity(square\_size);}\\[2mm]
\begin{tabular}{l l l}
square\_size &:-& a positive integer.
\end{tabular}

\item[Synopsis:]\mbox{}\\*
 \texttt{spmake\_identity} creates the identity matrix of 
                dimension square\_size.

\item[Examples:]
\begin{flushleft}  
\begin{math}  
\texttt{spmake\_identity}(4) = 
        \begin{pmatrix} 1 & 0 & 0 & 0 \\ 0 & 1 & 0 & 0 \\ 0 
& 0 & 1 & 0 \\ 0 & 0 & 0 & 1
 \end{pmatrix}
\end{math}  
\end{flushleft}

\item[Related functions:]\mbox{}\\*
\texttt{spdiagonal}.
\end{description}

\subsubsection{spmatrix\_augment, spmatrix\_stack}
\label{sparse:spmatrix_augment}
\ttindextype{SPMATRIX\_AUGMENT}{operator}

\begin{description}
\item[Syntax:]\mbox{}\\*
 \texttt{spmatrix\_augment(\{\sparsematlist\});}\sparselazyfootnote{}\\[2mm]
\begin{tabular}{l l l}
\sparsematlist &:-& matrices.
\end{tabular}

\item[Synopsis:]\mbox{}\\*
\texttt{spmatrix\_augment} joins the matrices in 
                  matrix\_list together horizontally. 

\texttt{spmatrix\_stack} joins the matrices in matrix\_list 
                together vertically.

\item[Examples:]
\begin{flalign*}  
&\texttt{spmatrix\_augment}(\{\mathcal{A,A}\}) =
        \begin{pmatrix} 1 & 0 & 0 & 1 & 0 & 0 \\ 0 & 5 & 0 
& 0 & 5 & 0 \\ 0 & 0 & 9 & 0 & 0 & 9
 \end{pmatrix} && \\[2mm]
&\texttt{spmatrix\_stack}(\{\mathcal{A,A}\}) & = & 
        \begin{pmatrix} 1 & 0 & 0 \\ 0 & 5 & 0 \\ 0 & 0 & 9 
\\ 1 & 0 & 0 \\ 0 & 5 & 0 \\ 0 & 0 & 9 
 \end{pmatrix}&&
\end{flalign*}

\item[Related functions:]\mbox{}\\*
 \texttt{spaugment\_columns}, \texttt{spstack\_rows}, 
\texttt{spsub\_matrix}.
\end{description}

\subsubsection{matrixp}
\label{sparse:matrixp}
\ttindex{MATRIXP}

\begin{description}
\item[Syntax:]\mbox{}\\*
\texttt{matrixp(test\_input);}\\[2mm]
\begin{tabular}{l l l}
test\_input &:-& anything you like.
\end{tabular}

\item[Synopsis:]\mbox{}\\*
\texttt{matrixp} is a boolean function that returns t if 
                the input is a matrix of type sparse or matrix and nil otherwise.

\item[Examples:]\mbox{}\\*
\texttt{matrixp}($\mathcal{A}$) = t 

\texttt{matrixp}(doodlesackbanana) = nil

\item[Related functions:]\mbox{}\\*
\texttt{squarep}, \texttt{symmetricp}, \texttt{sparsematp}.
\end{description}

\subsubsection{spmatrix\_stack}
\label{sparse:spmatrix_stack}
\ttindextype{SPMATRIX\_STACK}{operator}
See: \texttt{spmatrix\_augment}.


\subsubsection{spminor}
\label{sparse:spminor}
\ttindextype{SPMINOR}{operator}

\begin{description}
\item[Syntax:]\mbox{}\\*
\texttt{spminor($\mathcal{A}$,r,c);}\\[2mm]
\begin{tabular}{l l l} 
$\mathcal{A}$ &:-& a sparse matrix. \\
r,c        &:-& positive integers.
\end{tabular}

\item[Synopsis:]\mbox{}\\*
                \texttt{spminor} computes the (r,c)'th minor of $\mathcal{A}$.
 
\item[Examples:]
\begin{flushleft}  
\begin{math}  
\texttt{spminor}(\mathcal{A},1,3) =
        \begin{pmatrix} 0 & 5 \\ 0 & 0  \end{pmatrix}
\end{math}  
\end{flushleft}

\item[Related functions:]\mbox{}\\*
 \texttt{spremove\_columns}, \texttt{spremove\_rows}.
\end{description}

\subsubsection{spmult\_columns, spmult\_rows}
\label{sparse:spmult_columns}
\ttindextype{SPMULT\_COLUMNS}{operator}

\begin{description}
\item[Syntax:]\mbox{}\\*
\texttt{spmult\_columns($\mathcal{A}$,column\_list,expr);}\\[2mm]
\begin{tabular}{l l l}
$\mathcal{A}$   &:-& a sparse matrix. \\
column\_list &:-& a positive integer or a list of positive integers. \\
expr        &:-& an algebraic expression.
\end{tabular}

\item[Synopsis:]\mbox{}\\*
\texttt{spmult\_columns} returns a copy of $\mathcal{A}$ in which
                the columns specified in column\_list have been 
multiplied by expr. 

\texttt{spmult\_rows} performs the same task on the rows of $\mathcal{A}$.

\item[Examples:]
\begin{flalign*}
&\texttt{spmult\_columns}(\mathcal{A},\{1,3\},x) =
        \begin{pmatrix} x & 0 & 0 \\ 0 & 5 & 0 \\ 0 & 0 & 9*x \end{pmatrix} && \\[2mm]
&\texttt{spmult\_rows}(\mathcal{A},2,10)  = 
        \begin{pmatrix} 1 & 0 & 0 \\ 0 & 50 & 0 \\ 0 & 0 & 9 \end{pmatrix} &&
\end{flalign*}

\item[Related functions:]\mbox{}\\*
\texttt{spadd\_to\_columns}, \texttt{spadd\_to\_rows}.
\end{description}

\subsubsection{spmult\_rows}
\label{sparse:spmult_rows}
\ttindextype{SPMULT\_ROWS}{operator}

See: \texttt{spmult\_columns}.


\subsubsection{sppivot}
\label{sparse:sppivot}
\ttindextype{SPPIVOT}{operator}

\begin{description}
\item[Syntax:]\mbox{}\\*
\texttt{sppivot($\mathcal{A}$,r,c);}\\[2mm]
\begin{tabular}{l l l} 
$\mathcal{A}$ &:-& a sparse matrix. \\
r,c        &:-& positive integers such that $\mathcal{A}$(r,c) neq 0.
\end{tabular}

\item[Synopsis:]\mbox{}\\*
\texttt{sppivot} pivots $\mathcal{A}$ about it's (r,c)'th entry. 
 
To do this, multiples of the r'th row are added to every
     other row in the matrix. 

This means that the c'th column
                will be 0 except for the (r,c)'th entry. 

\item[Related functions:]\mbox{}\\*
\texttt{sprows\_pivot}.
\end{description}


\subsubsection{sppseudo\_inverse}
\label{sparse:sppseudo_inverse}
\ttindextype{SPPSEUDO\_INVERSE}{operator}

\begin{description}
\item[Syntax:]\mbox{}\\*
\texttt{sppseudo\_inverse($\mathcal{A}$);}\\[2mm]
\begin{tabular}{l l l} 
$\mathcal{A}$ &:-& a sparse matrix containing only real numeric entries.
\end{tabular}

\item[Synopsis:]\mbox{}\\*
\texttt{sppseudo\_inverse}, also known as the Moore-Penrose inverse, computes
the pseudo inverse of $\mathcal{A}$. 

Given the singular value decomposition of $\mathcal{A}$, i.e: $\mathcal{A} = 
\mathcal{U} 
\Sigma\mathcal{V}^T$, then the pseudo inverse $\mathcal{A}^{\dagger}$ is defined 
by $\mathcal{A}^{\dagger} = \mathcal{V} \Sigma^{\dagger} \mathcal{U}^{T}$. For the 
diagonal
matrix $\Sigma$, the pseudoinverse $\Sigma^{\dagger}$ is computed by taking the reciprocal
of only the nonzero diagonal elements.

If $\mathcal{A}$ is square and non-singular, then $\mathcal{A}^{\dagger} = \mathcal{A}$.
In general, however,
$\mathcal{A} \mathcal{A}^{\dagger} \mathcal{A} = \mathcal{A}$, and
$\mathcal{A}^{\dagger} \mathcal{A} \mathcal{A}^{\dagger} = \mathcal{A}^{\dagger}$.

Perhaps more importantly, $\mathcal{A}^{\dagger}$ solves the following least-squares
problem: given a rectangular matrix $\mathcal{A}$ and a vector $b$, find the
$x$ minimizing $\|\mathcal{A}x - b\|_2$, 
and which, in addition, has minimum $\ell_{2}$ (euclidean) Norm, $\|x\|_2$.  
This $x$ is $\mathcal{A}^{\dagger} b$.

\item[Examples:]
\begin{flalign*}
&\mathcal{R} = \begin{pmatrix} 0 & 0 & 3 & 0 \\ 9 & 0 & 7 & 0 \end{pmatrix} &&\\[2mm]
&\texttt{sppseudo\_inverse}(\mathcal{R}) =
        \begin{pmatrix} -0.26 & 0.11 \\ 0 & 0 \\ 0.33 & 0 \\ 0.25 & -0.05 \end{pmatrix}&&
\end{flalign*}

\item[Related functions:]\mbox{}\\*
 \texttt{spsvd}.
\end{description}

\subsubsection{spremove\_columns, spremove\_rows}
\label{sparse:spremove_columns}
\ttindextype{REMOVE\_COLUMNS}{operator}

\begin{description}
\item[Syntax:]\mbox{}\\*
\texttt{spremove\_columns($\mathcal{A}$,column\_list);}\\[2mm]
\begin{tabular}{l l l} 
$\mathcal{A}$   &:-& a sparse matrix. \\
column\_list &:-& either a positive integer or a list of 
                  positive integers.
\end{tabular}

\item[Synopsis:]\mbox{}\\*
\texttt{spremove\_columns} removes the columns specified in
                column\_list from $\mathcal{A}$. 

\texttt{spremove\_rows} performs the same task on the rows 
                of $\mathcal{A}$.

\item[Examples:]
\begin{flalign*}  
&\texttt{spremove\_columns}(\mathcal{A},2) =
        \begin{pmatrix} 1 & 0 \\ 0 & 0 \\ 0 & 9  \end{pmatrix} && \\[2mm]
&\texttt{spremove\_rows}(\mathcal{A},\{1,3\}) =
        \begin{pmatrix} 0 & 5 & 0 \end{pmatrix} &&
\end{flalign*}
    

\item[Related functions:]\mbox{}\\*
\texttt{spminor}.
\end{description}

\subsubsection{spremove\_rows}
\label{sparse:spremove_rows}
\ttindextype{SPREMOVE\_ROWS}{operator}

See: \texttt{spremove\_columns}.


\subsubsection{sprow\_dim}
\label{sparse:sprow_dim}
\ttindextype{SPROW\_DIM}{operator}

See: \texttt{spcolumn\_dim}.


\subsubsection{sprows\_pivot}
\label{sparse:sprows_pivot}
\ttindextype{SPROWS\_PIVOT}{operator}

\begin{description}
\item[Syntax:]\mbox{}\\*
\texttt{sprows\_pivot($\mathcal{A}$,r,c,\{row\_list\});}\\[2mm]
\begin{tabular}{l l l} 
$\mathcal{A}$ &:-& a sparse matrix. \\
r,c        &:-& positive integers such that $\mathcal{A}$(r,c) neq 0.\\
row\_list  &:-& positive integer or a list of positive integers.
\end{tabular}

\item[Synopsis:]\mbox{}\\*
\texttt{sprows\_pivot} performs the same task as \texttt{sppivot} but applies 
the pivot only to the rows specified in row\_list.

\item[Related functions:]\mbox{}\\*
\texttt{sppivot}.
\end{description}

\subsubsection{sparsematp}
\label{sparse:sparsematp}
\ttindextype{SPARSEMATP}{predicate}

\begin{description}
\item[Syntax:]\mbox{}\\*
\texttt{sparsematp($\mathcal{A}$);}\\[2mm]
\begin{tabular}{l l l} 
$\mathcal{A}$ &:-& a matrix.
\end{tabular}

\item[Synopsis:]\mbox{}\\*
\texttt{sparsematp} is a boolean function that returns t if 
                the matrix is declared sparse and nil otherwise.

\item[Examples:]
\begin{flalign*}
&\mathcal{L} := \texttt{mat((1,2,3),(4,5,6),(7,8,9));}&&\\[2mm]
&\texttt{sparsematp}(\mathcal{A}) = \texttt{t}  &&\\[2mm]
&\texttt{sparsematp}(\mathcal{L}) = \texttt{nil} &&
\end{flalign*}

\item[Related functions:]\mbox{}\\*
\texttt{matrixp}, \texttt{symmetricp}, \texttt{squarep}.
\end{description}

\subsubsection{squarep}
\label{sparse:squarep}
\ttindextype{SQUAREP}{predicate}

\begin{description}
\item[Syntax:]\mbox{}\\*
\texttt{squarep($\mathcal{A}$);}\\[2mm]
\begin{tabular}{l l l} 
$\mathcal{A}$ &:-& a matrix.
\end{tabular}

\item[Synopsis:]\mbox{}\\*
\texttt{squarep} is a boolean function that returns t if 
                the matrix is square and nil otherwise.

\item[Examples:]
\begin{flalign*}
&\mathcal{L} = \begin{pmatrix} 1 & 3 & 5 \end{pmatrix} && \\[2mm]
& \texttt{squarep}(\mathcal{A}) = \texttt{t}&&\\[2mm]
&  \texttt{squarep}(\mathcal{L}) = \texttt{nil} &&
\end{flalign*}

\item[Related functions:]\mbox{}\\*
\texttt{matrixp}, \texttt{symmetricp}, \texttt{sparsematp}.
\end{description}


\subsubsection{spstack\_rows}
\label{sparse:spstack_rows}
\ttindextype{STACK\_ROWS}{operator}

See: \texttt{spaugment\_columns}.


\subsubsection{spsub\_matrix}
\label{sparse:spsub_matrix}
\ttindextype{SPSUB\_MATRIX}{operator}

\begin{description}
\item[Syntax:]\mbox{}\\*
\texttt{spsub\_matrix($\mathcal{A}$,row\_list,column\_list);}\\[2mm]
\begin{tabular}{l l l} 
$\mathcal{A}$              &:-& a sparse matrix. \\
row\_list, column\_list &:-& \parbox[t]{.605\linewidth}{either a 
positive integer or a list of positive integers.}
\end{tabular}

\item[Synopsis:]\mbox{}\\*
\texttt{spsub\_matrix} produces the matrix consisting of the
              intersection of the rows specified in row\_list and the 
columns specified in column\_list. 

\item[Examples:]
\begin{flushleft}  
\hspace*{0.1in}
\begin{math}  
\texttt{spsub\_matrix}(\mathcal{A},\{1,3\},\{2,3\}) =
        \begin{pmatrix} 5 & 0\\ 0 & 9 \end{pmatrix}
\end{math}  
\end{flushleft}

\item[Related functions:]\mbox{}\\*
\texttt{spaugment\_columns}, \texttt{spstack\_rows}.
\end{description}

\subsubsection{spsvd (singular value decomposition)}
\label{sparse:spsvd} 
\ttindextype{SPSVD}{operator}

\begin{description}
\item[Syntax:]\mbox{}\\*
\texttt{spsvd($\mathcal{A}$);}\\[2mm]
\begin{tabular}{l l l} 
$\mathcal{A}$ &:-& a sparse matrix containing only real numeric entries.
\end{tabular}

\item[Synopsis:]\mbox{}\\*
\texttt{spsvd} computes the singular value decomposition of $\mathcal{A}$. 

If $A$
is an $m\times n$ real matrix of (column) rank $r$, \texttt{svd} returns the
3-element list \{$\mathcal{U},\Sigma,\mathcal{V}$\} where $\mathcal{A} =
\mathcal{U} \Sigma \mathcal{V}^T$. 

Let $k=\min(m,n)$.  Then $U$ is $m\times k$,
$V$ is $n\times k$, and and $\Sigma = \mbox{diag}(\sigma_{1}, \ldots ,\sigma_{k})$,
where $\sigma_{i}\ge 0$ are the singular values of $\mathcal{A}$; only $r$ of
these are non-zero.  The singular values are the non-negative square roots of
the eigenvalues of $\mathcal{A}^T \mathcal{A}$. 

$\mathcal{U}$ and $\mathcal{V}$ are such that $\mathcal{UU}^T = \mathcal{VV}^T = 
\mathcal{V}^T \mathcal{V} = \mathcal{I}_k$.

\textbf{Note:} there are a number of different definitions of SVD in the
literature, in some of which $\Sigma$ is square and $U$ and $V$ rectangular, as
here, but in others $U$ and $V$ are square, and $\Sigma$ is rectangular.

\item[Examples:]
\begin{flalign*}
 & \mathcal{Q} = \begin{pmatrix} 1 & 0 \\ 0 & 3 \end{pmatrix} && \\[2mm]
 & \begin{aligned}
     \mathtt{svd(\mathcal{Q})} = & 
       \left\{ 
         \begin{pmatrix} -1 & 0 \\ 0 & 0 \end{pmatrix}, 
         \begin{pmatrix} 1.0 & 0 \\ 0 & 5.0 \end{pmatrix}, \right. \\
         & \left. \: \; \,
           \begin{pmatrix} -1 & 0 \\ 0 & -1 \end{pmatrix}
       \right\}
    \end{aligned}
  \end{flalign*}
\end{description}  

\subsubsection{spswap\_columns, spswap\_rows}
\label{sparse:spswap_columns}
\ttindextype{SPSWAP\_COLUMNS}{operator}

\begin{description}
\item[Syntax:]\mbox{}\\*
\texttt{spswap\_columns($\mathcal{A}$,c1,c2);}\\[2mm]
\begin{tabular}{l l l}
$\mathcal{A}$ &:-& a sparse matrix. \\
c1,c1      &:-& positive integers. 
\end{tabular}

\item[Synopsis:]\mbox{}\\*
\texttt{spswap\_columns} swaps column c1 of $\mathcal{A}$ with column c2. 

\texttt{spswap\_rows} performs the same task on 2 rows of 
                $\mathcal{A}$.

\item[Examples:]
\begin{flushleft}  
\begin{math}  
\texttt{spswap\_columns}(\mathcal{A},2,3) =
        \begin{pmatrix} 1 & 0 & 0 \\ 0 & 0 & 5 \\ 0 & 9 & 0 \end{pmatrix}
\end{math}  
\end{flushleft}

\item[Related functions:]\mbox{}\\*
\texttt{spswap\_entries}.
\end{description}

\subsubsection{swap\_entries}
\label{sparse:spadd_entries}
\ttindextype{SPSWAP\_ENTRIES}{operator}

\begin{description}
\item[Syntax:]\mbox{}\\*
\texttt{spswap\_entries($\mathcal{A}$,\{r1,c1\},\{r2,c2\});}\\[2mm]
\begin{tabular}{l l l} 
$\mathcal{A}$  &:-& a sparse matrix. \\
r1,c1,r2,c2 &:-& positive integers.
\end{tabular}

\item[Synopsis:]\mbox{}\\*
\texttt{spswap\_entries} swaps $\mathcal{A}$(r1,c1) with 
                $\mathcal{A}$(r2,c2).

\item[Examples:]
\begin{flushleft}  
\begin{math}  
\texttt{spswap\_entries}(\mathcal{A},\{1,1\},\{3,3\}) =
        \begin{pmatrix} 9 & 0 & 0 \\ 0 & 5 & 0 \\ 0 & 0 & 1 \end{pmatrix} 
\end{math}  
\end{flushleft}

\item[Related functions:]\mbox{}\\*
\texttt{spswap\_columns}, \texttt{spswap\_rows}.


\subsubsection{spswap\_rows}
\label{sparse:spswap_rows}
\ttindextype{SPSWAP\_ROWS}{operator}
See: \texttt{spswap\_columns}.
\end{description}

\subsubsection{symmetricp}
\label{sparse:symmetricp}
\ttindextype{SYMMETRICP}{predicate}

\begin{description}
\item[Syntax:]\mbox{}\\*
\texttt{symmetricp($\mathcal{A}$);}\\[2mm]
\begin{tabular}{l l l} 
$\mathcal{A}$ &:-& a matrix. 
\end{tabular}

\item[Synopsis:]\mbox{}\\*
\texttt{symmetricp} is a boolean function that returns t if the 
                matrix is symmetric and nil otherwise.

\item[Examples:]
\begin{flalign*}
&\mathcal{M} = \begin{pmatrix} 1 & 2 \\ 2 & 1 \end{pmatrix} && \\[2mm]
& \texttt{symmetricp}(\mathcal{A}) = \texttt{nil} && \\[2mm] 
& \texttt{symmetricp}(\mathcal{M}) = \texttt{t} &&
\end{flalign*}

\item[Related functions:]\mbox{}\\*
\texttt{matrixp}, \texttt{squarep}, \texttt{sparsematp}.
\end{description}


\subsection{Fast Linear Algebra}
\ttindexswitch[SPARSE]{FAST\_LA}

By turning the \sw{fast\_la} switch on, the speed of the following 
functions will be increased:

\begin{tabular}{l l l l}
spadd\_columns    & spadd\_rows      & spaugment\_columns & spcol\_dim  \\
spcopy\_into      & spmake\_identity & spmatrix\_augment  & spmatrix\_stack\\
spminor           & spmult\_column   &  spmult\_row       & sppivot        \\
spremove\_columns & spremove\_rows   & sprows\_pivot      & squarep      \\
spstack\_rows     & spsub\_matrix    & spswap\_columns    & spswap\_entries\\
spswap\_rows      & symmetricp                                     
\end{tabular}

The increase in speed will be insignificant unless you are making a 
significant number(i.e: thousands) of calls. When using this switch, 
error checking is minimised. This means that illegal input may give
strange error messages. Beware.

\subsection{Acknowledgments}
This package is an extention of the code from the Linear Algebra Package
for \REDUCE{} by Matt Rebbeck (cf. section \ref{LINALG}).

The algorithms for \texttt{spcholesky}, \texttt{splu\_decom}, and \texttt{spsvd} are 
taken from the book Linear Algebra - J.H. Wilkinson \& C. Reinsch[3].

The \texttt{spgram\_schmidt} code comes from Karin Gatermann's Symmetry 
package[4] for {\REDUCE}.
