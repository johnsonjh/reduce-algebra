
\newenvironment{gex}{\left[\begin{array}{c|c}}{\end{array}\right]}
\newcommand{\gc}[1]{\mbox{\boldmath$#1$}}
\newcommand{\true}{\textrm{T}}
\newcommand{\false}{\textrm{F}}
\newcommand{\E}{\textrm{E}}
\newcommand{\GE}{\textrm{GE}}
\newcommand{\gscheme}{\textrm{gscheme}}
\newcommand{\sign}{\textrm{sign}}

\subsection{Introduction}
%% For MathJax, must be part of a paragraph to avoid extra space:
\ifdefined\HCode
\(
\def\gex{\left[\begin{array}{c|c}}
\def\endgex{\end{array}\right]}
\def\gc#1{\boldsymbol{#1}}
\def\true{\textrm{T}}
\def\false{\textrm{F}}
\def\E{\textrm{E}}
\def\GE{\textrm{GE}}
\def\gscheme{\textrm{gscheme}}
\def\sign{\textrm{sign}}
\)%
\fi
It is meanwhile a well-known fact that evaluations obtained with the
interactive use of computer algebra systems (\textsc{cas}) are not
entirely correct in general. Typically, some degenerate cases are
dropped. Consider for instance the evaluation
\[
\frac{x^2}{x}=x,
\]
which is correct only if $x\neq0$.
The problem here is that \textsc{cas} consider variables to be
transcendental elements. The user, in contrast, has in mind variables
in the sense of logic. In other words: The user does not think of
rational functions but of terms.

Next consider the valid expression
\[
\frac{\sqrt{x}+\sqrt{-x}}{x}.
\]
It is meaningless over the reals. \textsc{Cas} often offer no choice than
to interprete surds over the complex numbers even if they distinguish
between a {\em real} and a {\em complex} mode.

Corless and Jeffrey~\cite{CorlessJeffrey:92} have examined the
behavior of a number of \textsc{cas} with such input data. They come to
the conclusion that simultaneous computation of all cases is exemplary
but not feasible due to the combinatorial explosion of cases to be
considered. Therefore, they suggest to ignore the degenerate cases but
to provide the assumptions to the user on request. We claim, in
contrast, that it is in fact feasible to compute all possible cases.

Our setting is as follows: Expressions are evaluated to {\em guarded
expressions} consisting of possibly several conventional expressions
guarded by quantifier-free formulas. For the above examples, we would
obtain
\[
\begin{gex}
\gc{x\neq0}&\gc{x}
\end{gex},\quad
\begin{gex}
\gc{\false}&\gc{\frac{\sqrt{x}+\sqrt{-x}}{x}}
\end{gex}.
\]
As the second example illustrates, we are working in ordered fields,
more precisely in real closed fields. The handling of guarded
expressions as described in this paper can, however, be easily
transferred to other situations.

Our approach can also deal with redundant guarded expressions, such as
\[
\begin{gex}
\gc{\true} & \gc{|x|-x}\\
x\geq0 & 0\\
x<0 & -2x
\end{gex}
\]
which leads to algebraic simplification techniques based on logical
simplification as proposed by Davenport and
Faure \cite{DavenportFaure:94}.

We use {\em formulas} over the language of ordered rings as guards.
This provides powerful tools for heuristically reducing the
combinatorial explosion of cases: equivalent, redundant, tautological,
and contradictive cases can be detected by {\em
simplification}~\cite{Dolzmann:97b} and {\em quantifier
elimination}~\cite{Tarski:48,Collins:75,Weispfenning:88,LoosWeispfenning:93,Weispfenning:96,Weispfenning:94}.
In certain situations, we will allow the formulas also to contain
extra functions such as $\sqrt{\cdot}$ or $|\cdot|$. Then we take care
that there is no quantifier elimination applied.

Simultaneous computation of several cases concerning certain
expressions being zero or not has been extensively investigated as
{\em dynamic
evaluation}~\cite{GomezDiaz:93,DuvalReynaud:94,DuvalReynaud:94a,
BroadberryGomezDiazWatt:95}. It has also been extended to real closed
fields~\cite{DuvalGonzalesVega:93}. The idea behind the development of
these methods is of a more theoretical nature than to overcome the
problems with the interactive usage of \textsc{cas} sketched above: one
wishes to compute in algebraic (or real) extension fields of the
rationals. Guarded expressions occur naturally when solving problems
parametrically. Consider, e.g., the {\em Gr\"obner systems} used
during the computation of {\em comprehensive Gr\"obner bases}
\cite{Weispfenning:92}.

The algorithms described in this paper are implemented in the
\REDUCE package \textsc{guardian}. It is based on the
\REDUCE~\cite{Hearn:91,Melenk:95} package
\textsc{redlog}~\cite{DolzmannSturm:97a,Dolzmann:96a} implementing a
formula data type with corresponding algorithms, in particular
including simplification and quantifier elimination.
% Both \textsc{guardian} and \textsc{redlog} are available on the {\sc
%   www}.\footnote{{\tt http://www.fmi.uni-passau.de/\~{}redlog/}}

\subsection{An outline of our method}
\subsubsection{Guarded expressions}
A {\em guarded expression} is a scheme
\[
\begin{gex}
\gc{\gamma_0} & \gc{t_0}\\
\gamma_1 & t_1\\
\vdots&\vdots\\
\gamma_n& t_n
\end{gex}
\]
where each $\gamma_i$ is a quantifier-free formula, the {\em guard},
and each $t_i$ is an associated {\em conventional expression}. The
idea is that some $t_i$ is a valid interpretation iff $\gamma_i$
holds. Each pair $(\gamma_i,t_i)$ is called a {\em case}.

The first case $(\gamma_0,t_0)$ is the {\em generic} case: $t_0$ is
the expression the system would compute without our package, and
$\gamma_0$ is the corresponding guard.

The guards $\gamma_i$ need neither exclude one another, nor do we
require that they form a complete case distinction. We shall, however,
assume that all cases covered by a guarded expression are already
covered by the generic case; in other words:
\begin{equation}
\bigwedge_{i=1}^n(\gamma_i\longrightarrow\gamma_0).\label{gencoversall}
\end{equation}

Consider the following evaluation of $|x|$ to a guarded expression:
\[
\begin{gex}
\gc{\true} & \gc{|x|}\\
x\geq0 & x\\
x<0 & -x
\end{gex}.
\]
Here the non-generic cases already cover the whole domain. The
generic case is in some way {\em redundant}. It is just present for
keeping track of the system's default behavior. Formally we have
\begin{equation}
\Bigl(\bigvee_{i=1}^n\gamma_i\Bigr)\longleftrightarrow\gamma_0.
\label{formalredund}
\end{equation}
As an example for a non-redundant, i.e., {\em necessary} generic case
we have the evaluation of the reciprocal $\frac{1}{x}$:
\[
\begin{gex}
\gc{x\neq0}& \gc{\frac{1}{x}}
\end{gex}.
\]

In every guarded expression, the generic case is explicitly marked as
either necessary or redundant. The corresponding tag is inherited
during the evaluation process. Unfortunately it can happen that
guarded expressions satisfy~(\ref{formalredund}) without being tagged
redundant, e.g., specialization of
\[
\begin{gex}
\gc{\true}&\gc{\sin x}\\
x=0&0
\end{gex}
\]
to $x=0$ if the system cannot evaluate $\sin(0)$. This does not happen
if one claims for necessary generic cases to have, as the reciprocal
above, no alternative cases at all. Else, in the sequel ``redundant
generic case'' has to be read as ``tagged redundant.''

With guarded expressions, the evaluation splits into two independent
parts: {\em Algebraic evaluation} and a subsequent {\em
simplification} of the guarded expression obtained.
%
\subsubsection{Guarding schemes}
In the introduction we have seen that certain operators introduce case
distinctions. For this, with each operator $f$ there is a {\em
guarding scheme} associated providing information on how to map
$f(t_1,\ldots,t_m)$ to a guarded expression provided that one does not
have to care for the argument expressions $t_1$, \dots,~$t_m$. In the
easiest case, this is a rewrite rule
\[
f(a_1,\ldots,a_m)\to G(a_1,\ldots,a_m).
\]
The actual terms $t_1$, \dots,~$t_m$ are simply substituted for the
formal symbols $a_1$, \dots,~$a_m$ into the generic guarded expression
$G(a_1,\ldots,a_m)$. We give some examples:
\begin{eqnarray}
\frac{a_1}{a_2} & \to &
\begin{gex}
\gc{a_2\neq 0} & \gc{\frac{a_1}{a_2}}
\end{gex}\nonumber\\
\sqrt{a_1} & \to &
\begin{gex}
\gc{a_1\geq 0} & \gc{\sqrt{a_1}}
\end{gex}\nonumber\\
\sign(a_1) & \to &
\begin{gex}
\gc{\true} & \gc{\sign(a_1)}\\
a_1>0 & 1\\
a_1=0 & 0\\
a_1<0 & -1
\end{gex}\nonumber\\
|a_1| & \to &
\begin{gex}
\gc{\true} & \gc{|a_1|}\\
a_1\geq0 & a_1\\
a_1<0 & -a_1
\end{gex}\label{absrewrite}
\end{eqnarray}

For functions of arbitrary arity, e.g., $\min$ or $\max$, we formally
assume infinitely many operators of the same name. Technically, we
associate a procedure parameterized with the number of arguments $m$
that generates the corresponding rewrite rule. As ${\tt
min\_scheme(2)}$ we obtain, e.g.,
\begin{eqnarray}
\min(a_1,a_2) & \to &
\begin{gex}
\gc{\true} & \gc{\min(a_1,a_2)}\\
a_1\leq a_2 & a_1\\
a_2\leq a_1 & a_2
\end{gex}\label{binminscheme},
\end{eqnarray}
while for higher arities there are more case distinctions necessary.

For later complexity analysis, we state the concept of a guarding
scheme formally: a guarding scheme for an $m$-ary operator $f$ is a
map
\[
\gscheme_f: \E^m \to \GE
\]
where $\E$ is the set of expressions, and $\GE$ is the set of guarded
expressions. This allows to split $f(t_1,\ldots,t_m)$ in dependence
on the form of the parameter expressions $t_1$, \dots,~$t_m$.
%
\subsubsection{Algebraic evaluation}\label{algeval}
\subsubsubsection{Evaluating conventional expressions}
The evaluation of conventional expressions into guarded expressions is
performed recursively: Constants $c$ evaluate to
\[
\begin{gex}
\gc{\true} & \gc{c}
\end{gex}.
\]
For the evaluation
of $f(e_1,\ldots,e_m)$ the argument expressions $e_1$, \ldots, $e_m$
are recursively evaluated to guarded expressions
\begin{equation}
e_i'=\begin{gex}
\gc{\gamma_{i0}} & \gc{t_{i0}}\\
\gamma_{i1} & t_{i1}\\
\vdots & \vdots\\
\gamma_{in_i} & t_{in_i}\end{gex}\quad\mbox{for}\quad 1\leq i\leq m.
\label{eprimes}
\end{equation}

Then the operator $f$ is ``moved inside'' the $e_i'$ by combining all
cases, technically a simultaneous Cartesian product computation of
both the sets of guards and the sets of terms:
\begin{equation}
\Gamma=\prod_{i=1}^m\{\gamma_{i0},\ldots,\gamma_{in_i}\},\quad
T=\prod_{i=1}^m\{t_{i0},\ldots,t_{in_i}\}.
\label{cartprod}
\end{equation}
This leads to the intermediate result
\begin{equation}
\begin{gex}
\gc{\gamma_{10}\land\dots\land\gamma_{m0}}&
\gc{f(t_{10},\dots,t_{m0})}\\
\vdots&\vdots\\
\gamma_{1n_1}\land\dots\land\gamma_{m0}&
f(t_{1n_1},\dots,t_{m0})\\
\vdots&\vdots\\
\gamma_{1n_1}\land\dots\land\gamma_{mn_m}&
f(t_{1n_1},\dots,t_{mn_m})
\end{gex}.
\label{intermediate}
\end{equation}
The new generic case is exactly the combination of the generic cases
of the $e_i'$. It is redundant if at least one of these combined cases
is redundant.

Next, all non-generic cases containing at least one {\em redundant}
generic constituent $\gamma_{i0}$ in their guard are deleted. The
reason for this is that generic cases are only used to keep track of
the system default behavior. All other cases get the status of a
non-generic case even if they contain necessary generic constituents
in their guard.

At this point, we apply the guarding scheme of $f$ to all remaining
expressions $f(t_{1i_1},\ldots,t_{mi_m})$ in the
form~(\ref{intermediate}) yielding a nested guarded expression
\begin{equation}
\begin{gex}
\gc{\Gamma_0} &
\begin{gex}
\gc{\delta_{00}} & \gc{u_{00}}\\ \vdots & \vdots \\ \delta_{0k_0} & u_{0k_0}
\end{gex}\\
\vdots & \vdots\\
\Gamma_N &
\begin{gex}
\gc{\delta_{N0}} & \gc{u_{N0}}\\ \vdots & \vdots \\ \delta_{Nk_N} & u_{Nk_N}
\end{gex}
\end{gex},\label{nestedge}
\end{equation}
which can be straightforwardly resolved to a guarded expression
\[
\begin{gex}
\gc{\Gamma_0\land\delta_{00}} & \gc{u_{00}}\\
\vdots & \vdots\\
\Gamma_0\land\delta_{0k_0} & u_{0k_0}\\
\vdots & \vdots\\
\Gamma_N\land\delta_{N0} & u_{N0}\\
\vdots & \vdots\\
\Gamma_N\land\delta_{Nk_N} & u_{Nk_N}
\end{gex}.
\]
This form is treated analogously to the form~(\ref{intermediate}): The
new generic case $(\Gamma_0\land\delta_{00},u_{00})$ is redundant if
at least one of $\bigl(\Gamma_0,f(t_{10},\dots,t_{m0})\bigr)$ and
$(\delta_{00},u_{00})$ is redundant. Among the non-generic cases all
those containing redundant generic constituents in their guard are
deleted, and all those containing necessary generic constituents in
their guard get the status of an ordinary non-generic case.

Finally the standard evaluator of the system---{\tt reval} in the case
of \textsc{reduce}---is applied to all contained expressions, which
completes the algebraic part of the evaluation.
%
\subsubsubsection{Evaluating guarded expressions}
The previous section was concerned with the evaluation of pure
conventional expressions into guarded expressions. Our system
currently combines both conventional and guarded expressions. We are
thus faced with the problem of treating guarded subexpressions during
evaluation.

When there is a {\em guarded} subexpression $e_i$ detected during
evaluation, all contained expressions are recursively evaluated to
guarded expressions yielding a nested guarded expression of the
form~(\ref{nestedge}). This is resolved as described above yielding
the evaluation subresult $e_i'$.

As a special case, this explains how guarded expressions are
(re)evaluated to guarded expressions.
%
\subsubsection{Example}
We describe the evaluation of the expression $\min(x,|x|)$. The first
argument $e_1=x$ evaluates recursively to
\begin{equation}
e_1'=\begin{gex} \gc{\true} & \gc{x} \end{gex}
\label{evalx}
\end{equation}
with a necessary generic case. The nested $x$ inside $e_2=|x|$ evaluates
to the same form~(\ref{evalx}). For obtaining $e_2'$, we apply the
guarding scheme~(\ref{absrewrite}) of the absolute value to the only
term of~(\ref{evalx}) yielding
\[
\begin{gex} \gc{\true} & \begin{gex}
\gc{\true} & \gc{|x|}\\
x\geq 0 & x\\
x<0 & -x
\end{gex}
\end{gex},
\]
where the inner generic case is redundant. This form is resolved to
\[
e_2'=\begin{gex}
\gc{\true\land\true} & \gc{|x|}\\
\true\land x\geq 0 & x\\
\true\land x<0 & -x
\end{gex}
\]
with a redundant generic case. The next step is the combination of
cases by Cartesian product computation. We obtain
\[
\begin{gex}
\gc{\true\land(\true\land\true)} & \gc{\min(x,|x|)}\\
\true\land(\true\land x\geq0) & \min(x,x)\\
\true\land(\true\land x<0) & \min(x,-x)
\end{gex},
\]
which corresponds to~(\ref{intermediate}) above. For the outer $\min$,
we apply the guarding scheme~(\ref{binminscheme}) to all terms
yielding the nested guarded expression
\[
\begin{gex}
\gc{\true\land(\true\land\true)} & \begin{gex}
\gc{\true} & \gc{\min(x,|x|)}\\
x\leq |x| & x\\
|x|\leq x & |x|
\end{gex}\\
\true\land(\true\land x\geq0) & \begin{gex}
\gc{\true} & \gc{\min(x,x)}\\
x\leq x & x\\
x\leq x & x
\end{gex}\\
\true\land(\true\land x<0) & \begin{gex}
\gc{\true} & \gc{\min(x,-x)}\\
x\leq -x & x\\
-x\leq x & -x
\end{gex}
\end{gex},
\]
which is in turn resolved to
\[
\begin{gex}
\gc{(\true\land(\true\land\true))\land\true} & \gc{\min(x,|x|)}\\
(\true\land(\true\land\true))\land x\leq |x| & x\\
(\true\land(\true\land\true))\land |x|\leq x & |x|\\
(\true\land(\true\land x\geq0))\land\true & \min(x,x)\\
(\true\land(\true\land x\geq0))\land x\leq x & x\\
(\true\land(\true\land x\geq0))\land x\leq x & x\\
(\true\land(\true\land x<0))\land\true & \min(x,-x)\\
(\true\land(\true\land x<0))\land x\leq -x & x\\
(\true\land(\true\land x<0))\land -x\leq x & -x
\end{gex}.
\]
From this, we delete the two non-generic cases obtained by combination
with the redundant generic case of the $\min$. The final result of the
algebraic evaluation step is the following:
\begin{equation}
\begin{gex}
\gc{(\true\land(\true\land\true))\land\true} & \gc{\min(x,|x|)}\\
(\true\land(\true\land\true))\land x\leq |x| & x\\
(\true\land(\true\land\true))\land |x|\leq x & |x|\\
(\true\land(\true\land x\geq0))\land x\leq x & x\\
(\true\land(\true\land x\geq0))\land x\leq x & x\\
(\true\land(\true\land x<0))\land x\leq -x & x\\
(\true\land(\true\land x<0))\land -x\leq x & -x
\end{gex}.\label{example}
\end{equation}
\subsubsection{Worst-case complexity}
Our measure of complexity $|G|$ for guarded expressions $G$ is the
number of contained cases:
\[
\left|\begin{gex}
\gc{\gamma_0} & \gc{t_0}\\
\gamma_1 & t_1\\
\vdots&\vdots\\
\gamma_n& t_n
\end{gex}\right|=n+1.
\]

As in Section~\ref{algeval}, consider an $m$-ary operator $f$,
guarded expression arguments $e_1'$, \dots,~$e_m'$ as in
equation~(\ref{eprimes}), and the Cartesian product $T$ as in
equation~(\ref{cartprod}). Then
\[
|f(e_1',\ldots,e_m')| \leq  \sum_{(t_1,\dots,t_m)\in
  T}|\gscheme_f(t_1,\dots,t_m)|
\]
\begin{eqnarray*}
& &{}\leq  \max_{(t_1,\ldots,t_m)\in
T}|\gscheme_f(t_1,\ldots,t_m)|\cdot \#T\\
& & {}=  \max_{(t_1,\ldots,t_m)\in
T}|\gscheme_f(t_1,\ldots,t_m)|\cdot \prod_{j=1}^m|e_j'|\\
& & {}\leq  \max_{(t_1,\ldots,t_m)\in
T}|\gscheme_f(t_1,\ldots,t_m)|\cdot\bigl(\max_{1\leq j\leq m}|e_j'|\bigr)^m.
\end{eqnarray*}

In the important special case that the guarding scheme of $f$ is a
rewrite rule $f(a_1,\ldots,a_m)\to G$, the above complexity
estimation simplifies to
\[
|f(e_1',\ldots,e_m')| \leq  |G|\cdot \prod_{j=1}^m|e_j'| \leq  |G|\cdot
\bigl(\max_{1\leq j\leq m}|e_j'|\bigr)^m.
\]
In other words: $|G|$ plays the role of a factor, which, however,
depends on $f$, and $|f(e_1',\ldots,e_m')|$ is polynomial in the size
of the $e_i$ but exponential in the arity of $f$.

%
\subsubsection{Simplification}
In view of the increasing size of the guarded expressions coming into
existence with subsequent computations, it is indispensable to apply
simplification strategies. There are two different algorithms involved
in the simplification of guarded expressions:
\begin{enumerate}
\item
A {\em formula simplifier} mapping quantifier-free formulas to
equivalent simpler ones.
\item
Effective {\em quantifier elimination} for real closed fields
over the language of ordered rings.
\end{enumerate}

It is not relevant, which simplifier and which quantifier elimination
procedure is actually used. We use the formula simplifier described
in~\cite{Dolzmann:97b}. Our quantifier elimination uses test point
methods developed by
Weispfenning~\cite{Weispfenning:88,LoosWeispfenning:93,Weispfenning:96}.
It is restricted to formulas obeying certain degree restrictions
wrt.~the quantified variables. As an alternative, \textsc{redlog}
provides an interface to Hong's \textsc{qepcad} quantifier elimination
package \cite{Hong:93}. Compared to the simplification, the
quantifier elimination is more time consuming. It can be turned off by
a {\em switch}.

The following simplification steps are applied in the given order:
%
\paragraph{Contraction of cases} This is restricted to the
non-generic cases of the considered guarded expression. We contract
different cases containing the same terms:
\[
\begin{gex}
\gc{\gamma_0}&\gc{t_0}\\
\vdots & \vdots\\
\gamma_i & t_i\\
\vdots & \vdots\\
\gamma_j & t_i\\
\vdots & \vdots
\end{gex}\quad\mbox{becomes}\quad
\begin{gex}
\gc{\gamma_0}&\gc{t_0}\\
\vdots & \vdots\\
\gamma_i\lor\gamma_j & t_i\\
\vdots & \vdots\\
\end{gex}.
\]
\paragraph{Simplification of the guards} The simplifier is applied to
all guards replacing them by simplified equivalents. Since our
simplifier maps $\gamma\lor\gamma$ to $\gamma$, this together with the
contraction of cases takes care for the deletion of duplicate cases.

\paragraph{Keep one tautological case} If the guard of some
non-generic case becomes ``$\true$,'' we delete all other non-generic
cases. Else, if quantifier elimination is turned on, we try to detect
a tautology by eliminating the universal closures
$\underline\forall\gamma$ of the guards $\gamma$. This quantifier
elimination is also applied to the guards of generic cases. These are,
in case of success, simply replaced by ``$\true$'' without deleting
the case.

\paragraph{Remove contradictive cases} A non-generic case is  deleted if
its guard has become ``$\false$.'' If quantifier elimination is turned
on, we try to detect further contradictive cases by eliminating the
existential closure $\underline\exists\gamma$ for each guard $\gamma$.
This quantifier elimination is also applied to generic cases. In case
of success they are not deleted but their guards are replaced by
``$\false$.'' Our assumption (\ref{gencoversall}) allows then to
delete all non-generic cases.

\subsubsection{Example revisited}
We turn back to the form~(\ref{example}) of our example $\min(x,|x|)$.
Contraction of cases with subsequent simplification automatically
yields
\[
\begin{gex}
\gc{\true} & \gc{\min(x,|x|)}\\
\true & x\\
|x|-x\leq 0 & |x|\\
\false & -x
\end{gex},
\]
of which only the tautological non-generic case survives:
\begin{equation}
\begin{gex}
\gc{\true} & \gc{\min(x,|x|)}\\
\true & x
\end{gex}.\label{minabs}
\end{equation}

\subsubsection{Output modes}
An {\em output mode} determines which part of the information
contained in the guarded expressions is provided to the user. {\sc
Guardian} knows the following output modes:

\paragraph{Matrix} Output matrices in the style used throughout this
paper. We have already seen that these can become very large in
general.
\paragraph{Generic case} Output only the generic case.
\paragraph{Generic term} Output only the generic term. Thus the output
is exactly the same as without the guardian package. If the condition
of the generic case becomes ``$\false$,'' a {\em warning} ``{\tt
contradictive situation}'' is given. The computation can, however, be
continued.\bigskip

Note that output modes are restrictions concerning only the output;
internally the system still computes with the complete guarded
expressions.
%
\subsubsection{A smart mode}\label{smartmode}
Consider the evaluation result~(\ref{minabs}) of $\min(x,|x|)$. The
{\em generic term} output mode would output $\min(x,|x|)$, although
more precise information could be given, namely $x$. The problem is
caused by the fact that generic cases are used to keep track of the
system's default behavior. In this section we will describe an
optional {\em smart mode} with a different notion of {\em generic
case}. To begin with, we show why the problem can not be overcome by a
``smart output mode.''

Assume that there is an output mode which outputs $x$
for~(\ref{minabs}). As the next computation involving~(\ref{minabs})
consider division by $y$. This would result in
\[
\begin{gex}
\gc{y\neq0} & \gc{\frac{\min(x,|x|)}{y}}\\
y\neq0 & \frac{x}{y}
\end{gex}.
\]
Again, there are identic conditions for the generic case and some
non-generic case, and, again, the term belonging to the latter is
simpler. Our mode would output $\frac{x}{y}$. Next, we apply the
absolute value once more yielding
\[
\begin{gex}
\gc{y\neq 0} & \gc{\frac{|\min(x,|x|)|}{|y|}}\\
xy\geq0 \land y\neq0 & \frac{x}{y}\\
xy<0 \land y\neq0 & \frac{-x}{y}
\end{gex}.
\]
Here, the condition of the generic case differs from all other
conditions. We thus have to output the generic term. For the user, the
evaluation of $|\frac{x}{y}|$ results in $\frac{|\min(x,|x|)|}{|y|}$.

The smart mode can turn a non-generic case into a necessary generic
one dropping the original generic case and all other non-generic
cases. Consider, e.g.,~(\ref{minabs}), where the conditions are equal,
and the non-generic term is ``simpler.''

In fact, the relevant relationship between the conditions is that the
generic condition {\em implies} the non-generic one. In other words:
Some non-generic condition is not more restrictive than the generic
condition, and thus covers the whole domain of the guarded expression.
Note that from the implication and~(\ref{gencoversall}) we may
conclude that the cases are even equivalent.

Implication is heuristically checked by simplification. If this fails,
quantifier elimination provides a decision procedure. Note that our
test point methods are incomplete in this regard due to the degree
restrictions. Also it cannot be applied straightforwardly to guards
containing operators that do not belong to the language of ordered
rings.

Whenever we happen to detect a relevant implication, we actually turn
the corresponding non-generic case into the generic one. From our
motivation of non-generic cases, we may expect that non-generic
expressions are generally more convenient than generic ones.
%
\subsection{Examples}\label{examples}
We give the results for the following computations as they are printed
in the output mode {\em matrix} providing the full information on the
computation result. The reader can derive himself what the output in
the mode {\em generic case} or {\em generic term} would be.
\begin{itemize}
\item Smart mode or not:
\[
\frac{1}{x^2+2x+1}=\begin{gex}
\gc{x+1\neq0}& \gc{\frac{1}{x^2+2x+1}}
		   \end{gex}.
\]
The simplifier recognizes that the denominator is a square.
\item Smart mode or not:
\[
\frac{1}{x^2+2x+2}=\begin{gex}
\gc{\true}& \gc{\frac{1}{x^2+2x+2}}
		   \end{gex}.
\]
Quantifier elimination recognizes the positive definiteness of the
denominator.
\item Smart mode:
\[
|x|-\sqrt{x}=\begin{gex}
\gc{x\geq 0} & \gc{-\sqrt{x}+x}
	     \end{gex}.
\]
The square root allows to forget about the negative branch of the
absolute value.
\item Smart mode:
\[
|x^2+2x+1|=\begin{gex}
\gc{\true}&\gc{x^2+2x+1}
	   \end{gex}.
\]
The simplifier recognizes the positive semidefiniteness of
the argument. \textsc{Reduce} itself recognizes squares within absolute values
only in very special cases such as $|x^2|$.
\item Smart mode:
\[
\min\bigl(x,\max(x,y)\bigr)=\begin{gex}
\gc{\true}&\gc{x}
		  \end{gex}.
\]
Note that \textsc{reduce} does not know any rules about nested minima and
maxima.
\item
Smart mode:
\[
\min\bigl(\sign(x),-1\bigr)=\begin{gex}
\gc{\true}&\gc{-1}
		  \end{gex}.
\]
\item Smart mode or not:
\[
|x|-x=\begin{gex}
\gc{\true}&\gc{|x|-x}\\
x\geq0&0\\
x<0&-2x
      \end{gex}.
\]
This example is taken from~\cite{DavenportFaure:94}.
\item
Smart mode or not:
\[
\sqrt{1+x^2 y^2 (x^2+y^2-3)}={}
\begin{gex}\gc{\true}&\gc{\sqrt{x^4 y^2 + x^2 y^4 - 3 x^2 y^2 + 1}}\end{gex}
\]
The {\em Motzkin polynomial} is recognized to be positive semidefinite
by quantifier elimination.
\end{itemize}
The evaluation time for the last example is 119\,ms on a \textsc{sun
sparc-4}. This illustrates that efficiency is no problem with such
interactive examples.
\subsection{Outlook}
This section describes possible extensions of the \textsc{guardian}. The
extensions proposed in Section~\ref{simplification} on simplification
of terms and Section~\ref{background} on a background theory are clear
from a theoretical point of view but not yet implemented.
Section~\ref{integration} collects some ideas on the application of
our ideas to the \textsc{reduce} integrator. In this field, there is some
more theoretical work necessary.
%
\subsubsection{Simplification of terms}\label{simplification}
Consider the expression $\sign(x)x-|x|$. It evaluates to the following
guarded expression:
\[
\begin{gex}
\gc{\true} & \gc{-|x|+\sign(x)x}\\
x\neq 0 & 0\\
x=0 & -x
\end{gex}.
\]
This suggests to substitute $-x$ by $0$ in the third case, which would
in turn allow to contract the two non-generic cases yielding
\[
\begin{gex}
\gc{\true} & \gc{-|x|+\sign(x)x}\\
\true & 0\end{gex}.
\]
In smart mode second case would then become the only generic case.

Generally, one would proceed as follows: If the guard is a conjunction
containing as toplevel equations
\[
t_1=0,\quad \dots,\quad t_k=0,
\]
reduce the corresponding expression modulo the set of univariate
linear polynomials among $t_1$, \dots,~$t_k$.

A more general approach would reduce the expression modulo a Gr\"obner
basis of all the $t_1$, \dots,~$t_k$. This leads, however, to larger
expressions in general.

One can also imagine to make use of non-conjunctive guards in the
following way:
\begin{enumerate}
\item Compute a \textsc{dnf} of the guard.
\item Split the case into several cases corresponding to the
conjunctions in the \textsc{dnf}.
\item Simplify the terms.
\item Apply the standard simplification procedure to the resulting
guarded expression. Note that it includes {\em contraction of cases}.
\end{enumerate}
According to experiences with similar ideas in the ``Gr\"obner
simplifier'' described in~\cite{Dolzmann:97b}, this should work
well.
%
\subsubsection{Background theory}\label{background}
In practice one often computes with quantities guaranteed to lie in a
certain range. For instance, when computing an electrical resistance,
one knows in advance that it will not be negative. For such cases one
would like to have some facility to provide external information to
the system. This can then be used to reduce the complexity of the
guarded expressions.

One would provide a function {\tt assert($\varphi$)}, which asserts
the formula {\tt $\varphi$} to hold. Successive applications of {\tt
assert} establish a {\em background theory}, which is a set of
formulas considered conjunctively. The information contained in the
background theory can be used with the guarded expression computation.
The user must, however, not rely on all the background information to
be actually used.

Technically, denote by $\Phi$ the (conjunctive) background theory. For
the {\em simplification of the guards}, we can make use of the fact
that our simplifier is designed to simplify wrt.~a theory,
cf.~\cite{Dolzmann:97b}. For proving that some guard $\gamma$ is
{\em tautological}, we try to prove
\[\underline{\forall}(\Phi\longrightarrow\gamma)\] instead of
$\underline{\forall}\gamma$. Similarly, for proving that $\gamma$ is
{\em contradictive}, we try to disprove
\[\underline{\exists}(\Phi\land\gamma).\] Instead of proving
$\underline{\forall}(\gamma_1\longrightarrow\gamma_2)$ in smart mode, we try to
prove
\[\underline{\forall}\bigl((\Phi\land\gamma_1)\longrightarrow\gamma_2\bigr).\]

Independently, one can imagine to use a background theory for reducing
the {\em output} with the {\em matrix} output mode. For this, one
simplifies each guard wrt.~the theory at the output stage treating
contradictions and tautologies appropriately. Using the theory for
replacing all cases by one at output stage in a smart mode manner
leads once more to the problem of expressions or even guarded
expressions ``mysteriously'' getting more complicated. Applying the
theory only at the output stage makes it possible to implement a
procedure {\tt unassert($\varphi$)} in a reasonable way.
%
\subsubsection{Integration}\label{integration}
\textsc{Cas} integrators make ``mistakes'' similar to those we have
examined. Consider, e.g., the typical result
\[
\int x^a\,dx=\frac{1}{a+1}x^{a+1}.
\]
It does not cover the case $a=-1$, for which one wishes to obtain
\[
\int x^{-1}\,dx=\ln x.
\]
This problem can also be solved by using guarded expressions for
integration results.

Within the framework of this paper, we would have to associate a
guarding scheme to the integrator {\tt int}. It is not hard to see
that this cannot be done in a reasonable way without putting as much
knowledge into the scheme as into the integrator itself. Thus for
treating integration, one has to modify the integrator to provide
guarded expressions.

Next, we have to clarify what the guarded expression for the above
integral would look like. Since we know that the integral is defined
for all interpretations of the variables, our
assumption~(\ref{gencoversall}) implies that the generic condition be
``$\true$.'' We obtain the guarded expression
\[
\begin{gex}
\gc{\true}& \gc{\int x^a\,dx}\\
a\neq-1& \frac{1}{a+1}x^{a+1}\\
a=-1 & \ln x
\end{gex}.
\]
Note that the redundant generic case does not model the system's
current behavior.
%
\subsubsection{Combining algebra with logic}
Our method, in the described form, uses an already implemented
algebraic evaluator. In the previous section, we have seen that this
point of view is not sufficient for treating integration
appropriately.

Also our approach runs into trouble with built-in knowledge such as
\begin{eqnarray}
\sqrt{x^2}&=&|x|\label{sqrtrule},\\
\sign(|x|)&=&1\label{signrule}.
\end{eqnarray}
Equation~(\ref{sqrtrule}) introduces an absolute value operator within
a non-generic term without making a case distinction.
Equation~(\ref{signrule}) is wrong when not considering $x$
transcendental. In contrast to the situation with reciprocals, our
technique cannot be used to avoid this ``mistake.'' We obtain
\[
\sign(|x|)=\begin{gex}
\gc{\true} & \gc{1}\\
x\neq0 & 1\\
x=0 & 0
	   \end{gex}
\]
yielding two different answers for $x=0$.

We have already seen in the example Section~\ref{examples} that the
implementation of knowledge such as~(\ref{sqrtrule})
and~(\ref{signrule}) is usually quite {\it ad hoc}, and can be mostly
covered by using guarded expressions. This obesrvation gives rise to
the following question: When designing a new \textsc{cas} based on guarded
expressions, how should the knowledge be distributed between the
algebraic side and the logic side?
%
\subsection{Conclusions}
Guarded expressions can be used to overcome well-known problems with
interpreting expressions as terms. We have explained in detail how to
compute with guarded expressions including several simplification
techniques. Moreover we gain algebraic simplification power from the
logical simplifications. Numerous examples illustrate the power of our
simplification methods. The largest part of our ideas is efficiently
implemented, and the software is published. The outlook on background
theories and on the treatment of integration by guarded expressions
points on interesting future extensions.
\nocite{Bradford:92}
