
%\expandafter
%\let\expandafter\thebibliography\csname cb@thebibliography\endcsname
%\savedsectionbib{\chapter}{chapter}

\begin{thebibliography}{999}

\bibitem{Fillebrown:92}
Sandra Fillebrown.
\newblock Faster computation of Bernoulli numbers.
\newblock \emph{Journal of Algorithms}, 13:431--445, 1992.

%\bibitem{Koepf:92} Wolfram Koepf,
%\emph{Power Series in Computer Algebra},
%J.\ Symbolic Computation 13 (1992)

% bibitems from applysym.tex

\bibitem{Hereman:95} W.\,Hereman, Chapter 13 in vol 3 of the CRC Handbook of
Lie Group Analysis of Differential Equations, Ed.: N.H.\,Ibragimov,
CRC Press, Boca Raton, Florida (1995).
Systems described in this paper are among others:  \\
DELiA (Alexei Bocharov et.al.) Pascal \\
DIFFGROB2 (Liz Mansfield) Maple \\
DIMSYM (James Sherring and Geoff Prince) REDUCE \\
HSYM (Vladimir Gerdt) Reduce \\
LIE (V. Eliseev, R.N. Fedorova and V.V. Kornyak) Reduce \\
LIE (Alan Head) muMath \\
Lie (Gerd Baumann) Mathematica \\
LIEDF/INFSYM (Peter Gragert and Paul Kersten) Reduce \\
Liesymm (John Carminati, John Devitt and Greg Fee) Maple \\
MathSym (Scott Herod) Mathematica \\
NUSY (Clara Nucci) Reduce \\
PDELIE (Peter Vafeades) Macsyma \\
SPDE (Fritz Schwarz) Reduce and Axiom \\
SYM\_DE (Stanly Steinberg) Macsyma \\
Symmgroup.c (Dominique Berube and Marc de Montigny) Mathematica \\
STANDARD FORM (Gregory Reid and Alan Wittkopf) Maple \\
SYMCAL (Gregory Reid) Macsyma and Maple \\
SYMMGRP.MAX (Benoit Champagne, Willy Hereman and Pavel Winternitz) Macsyma \\
LIE package (Khai Vu) Maple \\
Toolbox for symmetries (Mark Hickman) Maple \\
Lie symmetries (Jeffrey Ondich and Nick Coult) Mathematica.

\bibitem{Lie:1880} S.\, Lie, Sophus Lie's 1880 Transformation Group Paper,
Translated by M.\, Ackerman, comments by R.\, Hermann, Mathematical Sciences 
Press, Brookline, (1975).

\bibitem{Lie:1967} S.\,Lie, Differentialgleichungen, Chelsea Publishing Company,
New York, (1967).

\bibitem{Wolf:93} T.\,Wolf, An efficiency improved program \texttt{LIEPDE}
for determining Lie - symmetries of PDEs, Proceedings of the workshop on
Modern group theory methods in Acireale (Sicily) Nov.\,(1992)

\bibitem{Riquier:1910} C.\,Riquier, Les syst\`{e}mes d'\'{e}quations 
aux d\'{e}riv\'{e}es partielles, Gauthier--Villars, Paris (1910).

\bibitem{Thomas:1937} J.\,Thomas, Differential Systems, AMS, Colloquium
publications, v.\,21, N.Y.\,(1937).

\bibitem{Janet:1929} M.\,Janet, Le\c{c}ons sur les syst\`{e}mes d'\'{e}quations aux
d\'{e}riv\'{e}es, Gauthier--Villars, Paris (1929).

\bibitem{Topunov:89} V.L.\,Topunov, Reducing Systems of Linear Differential
Equations to a Passive Form, Acta Appl.\,Math.\,16 (1989) 191--206.

\bibitem{Bocharov:89} A.V.\,Bocharov and M.L.\,Bronstein, Efficiently
Implementing Two Methods of the Geometrical Theory of Differential
Equations: An Experience in Algorithm and Software Design, Acta.\,Appl.
Math.\,16 (1989) 143--166.

\bibitem{Olver:89} P.J. Olver, Applications of Lie Groups to Differential
Equations, Springer-Verlag New York (1986).

\bibitem{Reid:90} G.J.\,Reid, A triangularization algorithm which
determines the Lie symmetry algebra of any system of PDEs, J.Phys.\,A:
Math.\,Gen.\,23 (1990) L853-L859.

\bibitem{FS} F.\,Schwarz, Automatically Determining Symmetries of Partial
Differential Equations, Computing 34, (1985) 91-106.

\bibitem{Fushchich:89} W.I.\,Fushchich and V.V.\,Kornyak, Computer Algebra
Application for Determining Lie and Lie--B\"{a}cklund Symmetries of
Differential Equations, J.\,Symb.\,Comp.\,7 (1989) 611--619.

\bibitem{Kamke:Vol1} E.\,Kamke, Differentialgleichungen, L\"{o}sungsmethoden
und L\"{o}sungen, Band 1, Gew\"{o}hnliche Differentialgleichungen,
Chelsea Publishing Company, New York, 1959.

\bibitem{Kamke:Vol2} E.\,Kamke, Differentialgleichungen, L\"{o}sungsmethoden
und L\"{o}sungen, Band 2, Partielle Differentialgleichungen, 6.Aufl.,
Teubner, Stuttgart:Teubner, 1979.

\bibitem{Wolf:85} T.\,Wolf, An Analytic Algorithm for Decoupling and Integrating
systems of Nonlinear Partial Differential Equations, J.\,Comp.\,Phys.,
no.\,3, 60 (1985) 437-446 and, Zur analytischen Untersuchung und exakten
L\"{o}sung von Differentialgleichungen mit Computeralgebrasystemen,
Dissertation B, Jena (1989).

\bibitem{Brand:92} T.\,Wolf, A. Brand, The Computer Algebra Package \texttt{CRACK}
      for Investigating PDEs, Manual for the package \texttt{CRACK} in the REDUCE
      network library and in Proceedings of ERCIM School on Partial 
      Differential Equations and Group Theory, April 1992 in Bonn, GMD Bonn.

\bibitem{MacCallum:91} M.A.H.\,MacCallum, F.J.\,Wright, Algebraic Computing with REDUCE,
Clarendon Press, Oxford (1991).

\bibitem{MacCallum:88} M.A.H.\, MacCallum, An Ordinary Differential Equation
Solver for REDUCE, Proc.\, ISAAC'88, Springer Lect.\, Notes in Comp Sci.
358, 196--205.

\bibitem{Stephani:89} H.\,Stephani, Differential equations, Their solution using
symmetries, Cambridge University Press (1989).

\bibitem{Karpman:89} V.I.\,Karpman, Phys.\,Lett.\,A 136, 216 (1989)

\bibitem{Champagne:91} B.\,Champagne, W.\,Hereman and P.\,Winternitz, The computer
      calculation of Lie point symmetries of large systems of differential
      equations, Comp.\,Phys.\,Comm.\,66, 319-340 (1991)

\bibitem{Markus} M.\,Kubitza, private communication

% bibitems from arnum.tex

\bibitem{Bradford:86}
R.~J. Bradford, A.~C. Hearn, J.~A. Padget, and E.~Schr{\"u}fer.
\newblock Enlarging the {REDUCE} domain of computation.
\newblock In \emph{Proceedings of {SYMSAC} '86}, pages 100--106, 1986.

\bibitem{Davenport:81}
James~Harold Davenport.
\newblock On the integration of algebraic functions.
\newblock In \emph{Lecture Notes in Computer Science}, volume 102. Springer
  Verlag, 1981.

\bibitem{Trager:76}
B.~M. Trager.
\newblock Algebraic factoring and rational function integration.
\newblock In \emph{Proceedings of {SYMSAC} '76}, pages 196--208, 1976.

% bibitems from bibasis.tex

\bibitem{Gerdt:98} V.P.Gerdt and Yu.A.Blinkov. \emph{Involutive Bases of Polynomial Ideals}. 
Mathematics and Computers in Simulation, 45, 519--542, 1998; \emph{Minimal Involutive Bases}, ibid. 543--560.
 
\bibitem{Seiler:2010} W.M.Seiler. \emph{Involution: The Formal Theory of Differential Equations and its Applications 
in Computer Algebra}. Algorithms and Computation in Mathematics, 24, Springer, 2010.  arXiv:math.AC/0501111

\bibitem{Gerdt:2005} Vladimir P. Gerdt. \emph{Involutive Algorithms for Computing Gr\"obner Bases}.
Computational Commutative and Non-Commutative Algebraic Geometry. IOS Press, Amsterdam, 2005, pp.199--225.

\bibitem{Faugere:2003}
J.-C.Faug\`{e}re and A.Joux. Algebraic Cryptanalysis of Hidden Field Equations
(HFE) Using Gr\"obner Bases. \emph{LNCS} 2729, Springer-Verlag, 2003, pp.44--60.

\bibitem{Gerdt:2008}
V.P.Gerdt and M.V.Zinin. A Pommaret Division Algorithm for Computing Gr\"obner Bases in Boolean Rings.
\emph{Proceedings of ISSAC 2008}, ACM Press, 2008, pp.95--102.

\bibitem{Gerdt:2008a}
V.P.Gerdt and M.V.Zinin. Involutive Method for Computing Gr\"obner Bases over $F_2$.
\emph{Programming and Computer Software}, Vol.34, No. 4, 2008, 191--203.

\bibitem{Gerdt:2010}
Vladimir Gerdt, Mikhail Zinin and Yuri Blinkov. On computation of Boolean involutive bases,
Proceedings of International Conference Polynomial Computer Algebra 2009, pp. 17-24
(International Euler Institute, April 7-12, 2009, St. Peterburg, Russia)

% bibitems from cali.tex

\bibitem{BayerStillman:92} D. Bayer, M. Stillman: Computation of Hilbert
functions. \textit{J. Symb. Comp. \textbf{14}} (1992), 31 - 50.

\bibitem{Becker:93} T. Becker, H. Kredel, V. Weispfenning: Gr\"obner bases. A
computational approach to commutative algebra. Grad. Texts in Math.
141, Springer, New York 1993.

\bibitem{Bigatti:93} A. M. Bigatti, P. Conti, L. Robbiano, C. Traverso: A
``divide and conquer'' algorithm for Hilbert-Poincare series,
multiplicity and dimension of monomial ideals. In: Proc. AAECC-10,
LNCS 673 (1993), 76 - 88.

\bibitem{Boege:86} W. Boege, R. Gebauer, H. Kredel: Some examples for
solving systems of algebraic equations by calculating Gr\"obner bases. {\it
J. Symb. Comp. \textbf{2}} (1986), 83 - 98.

\bibitem{Buchberger:85} B. Buchberger: Gr\"obner bases: An algorithmic method in
polynomial ideal theory. In: Recent trends in multidimensional
system theory (N.~K.~Bose ed), Reidel, Dortrecht 1985, 184 - 232.

\bibitem{Buchberger:88} B. Buchberger: Applications of Gr\"obner bases in non-linear
computational geometry. LNCS 296 (1988), 52 - 80.

\bibitem{Cox:92} D. Cox, J. Little, D. O'Shea: Ideals, varieties, and
algorithms.  Undergraduate Texts in Math., Springer, New York 1992.

\bibitem{Eisenbud} D. Eisenbud: Commutative algebra with a view toward
algebraic geometry. Springer, 1995. 

\bibitem{Faugere:93} Faugere, Gianni, Lazard, Mora: Efficient computations
of zerodimensional Gr\"obner bases by change of ordering. {\it
J. Symb. Comp. \textbf{16}} (1993), 329 - 344. 

\bibitem{Gianni:88} P. Gianni, B. Trager, G. Zacharias: Gr\"obner bases and
primary decomposition of polynomial ideals. \textit{J. Symb. Comp. \textbf{6}}
 (1988), 149 - 167.

\bibitem{Giovini:91} A. Giovini, T. Mora, G. Niesi, L. Robbiano, C.
Traverso: "One sugar cube, please" or Selection strategies in the
Buchberger algorithm. In: Proceedings of the ISSAC'91, ACM Press
1991, 49 - 54.

\bibitem{Graebe:93a} H.-G. Gr\"abe: Two remarks on independent sets.
\textit{J. Alg. Comb. \textbf{2}} (1993), 137 - 145. 

\bibitem{Graebe:94} H.-G. Gr\"abe: The tangent cone algorithm and
homogenization. \textit{J. Pure Applied Alg.\textbf{97}} (1994), 303 - 312.

\bibitem{Graebe:95a} H.-G. Gr\"abe: Algorithms in local algebra. To appear

\bibitem{Graebe:94a} H.-G. Gr\"abe: On factorized Gr\"obner bases. Report Nr. 6
(1994), Inst. f. Informatik, Univ. Leipzig.

To appear in: Proc. ``Computer Algebra in Science and Engineering'',
Bielefeld 1994.

\bibitem{Graebe:95b} H.-G. Gr\"abe: Triangular systems and factorized Gr\"obner
bases. Report Nr. 7 (1995), Inst. f. Informatik, Univ. Leipzig.

\bibitem{Graebe:97} H.-G. Gr\"abe: Factorized Gr\"obner bases and primary
decomposition. To appear. 

\bibitem{Kredel:87} H. Kredel: Primary ideal decomposition. In: Proc.
EUROCAL'87, Lecture Notes in Comp. Sci. 378 (1986), 270 - 281.

\bibitem{Kredel:88a} H. Kredel, V. Weispfenning: Computing dimension and
independent sets for polynomial ideals. \textit{J. Symb. Comp. \textbf{6}} 
(1988), 231 - 247.

\bibitem{Marinari:91} M. Marinari, H.-M. M\"oller, T. Mora: Gr\"obner bases of
ideals given by dual bases. In: Proc. ISSAC'91, ACM Press 1991, 55 -
63.

\bibitem{Mishra:93} B. Mishra: Algorithmic Algebra. Springer, New York
1993.

\bibitem{MoellerMora:86} H.-M. M\"oller, F. Mora: New constructive methods in
classical ideal theory. \textit{J. Alg. \textbf{100}} (1986), 138 -178.

\bibitem{Moeller:93} H.-M. M\"oller: On decomposing systems of polynomial
equations with finitely many solutions. \emph{J. AAECC \textbf{4}} (1993),
217 - 230.  

\bibitem{Mora:88} T. Mora, L. Robbiano: The Gr\"obner fan of an ideal.
\textit{J. Symb. Comp. \textbf{6}} (1988), 183 - 208.

\bibitem{Mora:88a} T. Mora: Seven variations on standard bases. 
Preprint, Univ. Genova, 1988.

\bibitem{MoraPfisterTraverso:92} T. Mora, G. Pfister, C. Traverso: An introduction to
the tangent cone algorithm. In: \emph{Issues in non-linear geometry and 
robotics, C.M. Hoffman ed.}, JAI Press.

\bibitem{Robbiano:89} L. Robbiano: Computer algebra and commutative algebra.
LNCS 357 (1989), 31 - 44.

\bibitem{Rutman:92} E. W. Rutman: Gr\"obner bases and primary decomposition of
modules. \textit{J. Symb. Comp. \textbf{14}} (1992), 483 - 503.

% bibitems from camal.tex


\bibitem{Padget90}
A.~Barnes and J.~A. Padget.
\newblock Univariate power series expansions in {Reduce}.
\newblock In S.~Watanabe and M.~Nagata, editors, \emph{Proceedings of ISSAC'90},
  pages 82--7. ACM, Addison-Wesley, 1990.

\bibitem{Barton67a}
D.~Barton.
\newblock \emph{Astronomical Journal}, 72:1281--7, 1967.

\bibitem{Barton67b}
D.~Barton.
\newblock A scheme for manipulative algebra on a computer.
\newblock \emph{Computer Journal}, 9:340--4, 1967.

\bibitem{Barton72}
D.~Barton and J.~P. Fitch.
\newblock The application of symbolic algebra system to physics.
\newblock \emph{Reports on Progress in Physics}, 35:235--314, 1972.

\bibitem{Bourne:72}
Stephen~R. Bourne.
\newblock Literal expressions for the co-ordinates of the moon. {I}. the first
  degree terms.
\newblock \emph{Celestial Mechanics}, 6:167--186, 1972.

\bibitem{Brown:1896}
E.~W. Brown.
\newblock \emph{An Introductory Treatise on the Lunar Theory}.
\newblock Cambridge University Press, 1896.

\bibitem{Delaunay:1860}
C.~Delaunay.
\newblock \emph{Th\'eorie du Mouvement de la Lune}.
\newblock (Extraits des M\'em. Acad. Sci.). Mallet-Bachelier, Paris, 1860.

\bibitem{Fateman:74a}
Richard~J. Fateman.
\newblock On the multiplication of poisson series.
\newblock \emph{Celestial Mechanics}, 10(2):243--249, October 1974.

\bibitem{Fitch:LN75}
J.~P. Fitch.
\newblock Syllabus for algebraic manipulation lectures in cambridge.
\newblock \emph{SIGSAM Bulletin}, 32:15, 1975.

\bibitem{CAMALF:83}
J.~P. Fitch.
\newblock \emph{{CAMAL} {User's} {Manual}}.
\newblock University of Cambridge Computer Laboratory, 2nd edition, 1983.

\bibitem{Jefferys:70}
W.~H. Jeffereys.
\newblock \emph{Celestial Mechanics}, 2:474--80, 1970.

% bibitems from cde.tex

\bibitem{gdeq} Geometry of Differential Equations web site:
  \url{https://gdeq.org}.

%\bibitem{Many} \textsc{A. V. Bocharov, V. N. Chetverikov, S. V.  Duzhin, N.  G.
%    Khor{\cprime}kova, I. S.  Krasil{\cprime}shchik, A.  V.  Samokhin, Yu.\ N.
%    Torkhov, A. M. Verbovetsky and A. M.  Vinogradov}: Symmetries and
%  Conservation Laws for Differential Equations of Mathematical Physics, I.  S.
%  Krasil{\cprime}shchik and A. M.  Vinogradov eds., Translations of Math.
%  Monographs \textbf{182}, Amer.\ Math.\ Soc. (1999).

\bibitem{BaldwinHereman:2010} \textsc{D. Baldwin, W. Hereman}, \emph{A symbolic algorithm
    for computing recursion operators of nonlinear partial differential
    equations}, International Journal of Computer Mathematics, vol. 87 (5),
  pp. 1094-1119 (2010).

\bibitem{Dubrovin:96} \textsc{B.A. Dubrovin}, \newblock Geometry of 2D topological
field theories, Lecture Notes in Math. 1620, Springer-Verlag (1996) 120--348.

\bibitem{DubrovinNovikov:83} \textsc{B.A. Dubrovin and S.P. Novikov}, \newblock Hamiltonian
  formalism of one-dimensional systems of hydrodynamic type and the
  Bogolyubov-Whitham averaging method, Soviet Math. Dokl. \textbf{27} No. 3
  (1983) 665--669.

\bibitem{DubrovinNovikov:84} \textsc{B.A. Dubrovin and S.P. Novikov}, Poisson brackets of
  hydrodynamic type, Soviet Math. Dokl. \textbf{30} No. 3 (1984), 651--2654.

\bibitem{Ferapontov:97} \textsc{E.V. Ferapontov, C.A.P. Galvao, O. Mokhov, Y. Nutku},
  Bi-Hamiltonian structure of equations of associativity in 2-d
  topological field theory, Comm. Math. Phys. \textbf{186 }(1997) 649-669.

\bibitem{Ferapontov:2014} \textsc{E.V. Ferapontov, M.V. Pavlov, R.F. Vitolo},
\emph{Projective-geometric aspects of homogeneous third-order Hamiltonian
operators}, J. Geom.\ Phys.\ \textbf{85} (2014) 16-28, DOI:
10.1016/j.geomphys.2014.05.027.

\bibitem{Ferapontov:2016} \textsc{E.V. Ferapontov, M.V. Pavlov, R.F. Vitolo},
\emph{Towards the classification of homogeneous third-order Hamiltonian
  operators}, \url{https://arxiv.org/abs/1508.02752}

\bibitem{Getzler:2002} \emph{E. Getzler}, A Darboux theorem for Hamiltonian
operators in the formal calculus of variations, Duke J. Math. \textbf{111}
(2002), 535-560.

\bibitem{IgoninVerbovetskyVitolo:2004}\textsc{S. Igonin, A. Verbovetsky, R. Vitolo:}
  \emph{Variational Multivectors and Brackets in the Geometry of Jet Spaces}, V
  Int. Conf. on on Symmetry in Nonlinear Mathematical Physics, Kyiv 2003; Part
  3 of Volume 50 of Proceedings of Institute of Mathematics of NAS of Ukraine,
  Editors A.G.  Nikitin, V.M. Boyko, R.O. Popovych and I.A. Yehorchenko (2004),
  1335--1342; \url{https://www.imath.kiev.ua/~snmp2003/Proceedings/vitolo.pdf}.

\bibitem{KerstenKrasilshchikVerboretsky:2004} \textsc{P.H.M. Kersten, I.S. Krasil'shchik, A.M. Verbovetsky,}
  \emph{Hamiltonian operators and $\ell^*$-covering}, Journal of Geometry and
  Physics \textbf{50} (2004), 273--302.

\bibitem{KerstenKrasilshchikVerboretsky:2006} \textsc{P.H.M. Kersten, I.S. Krasil'shchik, A.M. Verbovetsky,}
  \emph{A geometric study of the dispersionless Boussinesq equation}, Acta
  Appl.\ Math. \textbf{90} (2006), 143--178.

\bibitem{KerstenKrasilshchikVerbovetskyVitolo:HSGP} \textsc{P.~Kersten,
  I.~Krasil{\cprime}shchik, A.~Verbovetsky, and R.~Vitolo}, \emph{Hamiltonian
    structures for general {PDE}s}, Differential equations: Geometry,
  Symmetries and Integrability. The Abel Symposium 2008 (B.~Kruglikov,
  V.~V. Lychagin, and E.~Straume, eds.), Springer-Verlag, 2009, pp.~187--198,
  \eprint{0812.4895}.

\bibitem{KrasilshchikVerbovetsky:JGP:2011} \textsc{I.~Krasil{\cprime}shchik and A.~Verbovetsky},
  \emph{Geometry of jet spaces and integrable systems}, J.\ Geom.\
  Phys. (2011) doi:10.1016/j.geomphys.2010.10.012, \url{arXiv:1002.0077}.

\bibitem{KrasilshchikVerbovetskyVitolo:SPT:2012} \textsc{I.~Krasil{\cprime}shchik, A.~Verbovetsky,
    R. Vitolo},  \emph{A unified approach to computation of integrable
    structures}, Acta Appl.\ Math.\ (2012).

\bibitem{KrasilshchikVerbovetskyVitolo:2017} \textsc{I.~Krasil{\cprime}shchik, A.~Verbovetsky,
    R. Vitolo},  \emph{The symbolic computation of integrability structures for
    partial differential equations}, book, Springer series
``Texts and monographs in symbolic computations'' ISBN 978-3-319-71655-8 (2018).

\bibitem{Kupershmidt:94} \textsc{B. Kuperschmidt}:
  \emph{Geometric Hamiltonian forms for the
    Kadomtsev--Petviashvili and Zabolotskaya--Khokhlov equations}, in Geometry
  in Partial Differential Equations, A. Prastaro, Th.M. Rassias eds., World
  Scientific (1994), 155--172.

\bibitem{Marvan:2009} \textsc{M. Marvan}, \emph{Sufficient set of integrability
    conditions of an orthonomic system}.  Foundations of Computational
    Mathematics \textbf{9} (2009), 651--674.

\bibitem{NeyziNutkuSheftel:JPA:2005} \textsc{F. Neyzi, Y. Nutku, and
      M.B. Sheftel}, \emph{Multi-Hamiltonian structure of Plebanski's second
      heavenly equation} J. Phys.\ A: Math.\ Gen.\ \textbf{38} (2005),
    8473. \eprint{nlin/0505030v2}.

\bibitem{NormanVitolo:InsideReduce} \textsc{A.C. Norman, R. Vitolo}, \emph{Inside Reduce}, part
    of the official \REDUCE documentation included in the source code, see
    below.

\bibitem{Nucci:92} \textsc{M.C. Nucci}, \emph{Interactive REDUCE
      programs for calculating classical, non-classical, and approximate
      symmetries of differential equations}, in Computational and Applied
    Mathematics II. Differential Equations, W.F. Ames, and P.J. Van der Houwen,
    Eds., Elsevier, Amsterdam (1992) pp. 345--350.

\bibitem{Nucci:96} \textsc{M.C. Nucci}, \emph{Interactive REDUCE programs for
    calculating Lie point, non-classical, Lie-B\"{a}cklund, and approximate
    symmetries of differential equations: manual and floppy disk}, in
CRC Handbook of Lie Group Analysis of Differential Equations. Vol. 3
N.H. Ibragimov, Ed., CRC Press, Boca Raton (1996) pp. 415--481.

\bibitem{Oliveri:ReLie} \textsc{F. Oliveri}, \textsc{ReLie}, \REDUCE software and user
  guide, \url{http://mat521.unime.it/oliveri/}.

\bibitem{Olver:93} \textsc{P. Olver}, Applications of Lie Groups to Partial
  Differential Equations, 2nd ed, GTM Springer, 1992.

\bibitem{PavlovVitolo:2015} \textsc{M.V. Pavlov, R.F. Vitolo}: \emph{On the bi-Hamiltonian
    geometry of the WDVV equations}, \texttt{https://arxiv.org/abs/1409.7647}

\bibitem{SaccomandiVitolo:2014} \textsc{G. Saccomandi, R. Vitolo}: \emph{On the Mathematical
    and Geometrical Structure of the Determining Equations for Shear Waves in
    Nonlinear Isotropic Incompressible Elastodynamics}, J. Math.\ Phys.\
  \textbf{55} (2014), 081502.

\bibitem{Wolf:99d} \textsc{T. Wolf}, \emph{A comparison of four approaches to the
    calculation of conservation laws}, Euro.\ Jnl of Applied Mathematics 13
  part 2 (2002) 129-152.

\bibitem{Wolf:APPLYSYM} \textsc{T. Wolf}, \emph{APPLYSYM - a package for the
    application of Lie-symmetries}, software distributed together with the
  computer algebra system REDUCE, (1995).

\bibitem{Brand:95} \textsc{T. Wolf, A. Brand}, \emph{Investigating DEs with CRACK
    and Related Programs}, SIGSAM Bulletin, Special Issue, (June 1995), p 1-8.

\bibitem{WolfBrand:CRACK} \textsc{T. Wolf, A. Brand}: CRACK, user guide, examples and
    documentation \url{http://lie.math.brocku.ca/Crack_demo.html}. For
    applications, see also the publications of T. Wolf.

% bibitems from cdiff.tex

\bibitem{Reduce:Obtaining} Obtaining \REDUCE: \url{https://reduce-algebra.sourceforge.io/}.

%\bibitem{gdeq} Geometry of Differential Equations web site:
%    \url{https://gdeq.org}.

\bibitem{noteppp} \texttt{notepad++}:
    \url{https://notepad-plus.sourceforge.net/}

\bibitem{ed} List of text editors:
    \url{https://en.wikipedia.org/wiki/List_of_text_editors}

\bibitem{emacswin} How to install \texttt{emacs} in Windows:
    \url{http://www.cmc.edu/math/alee/emacs/emacs.html}. See also
    \url{http://www.gnu.org/software/emacs/windows/ntemacs.html}

\bibitem{reducewin} How to install \REDUCE in Windows:
    \url{https://reduce-algebra.sourceforge.io/obtaining.php#windetails}

\bibitem{Roelofs:92} \textsc{G.H.M. Roelofs}, The SUPER VECTORFIELD package for
    REDUCE. Version 1.0, Memorandum 1099, Dept. Appl. Math., University of
    Twente, 1992. Available at \url{https://gdeq.org}.

\bibitem{Roelofs:92a} \textsc{G.H.M. Roelofs}, The INTEGRATOR package for
    REDUCE. Version 1.0, Memorandum 1100, Dept. Appl. Math., University of
    Twente, 1992. Available at \url{https://gdeq.org}.

\bibitem{Post:96} \textsc{G.F. Post}, A manual for the package TOOLS 2.1,
    Memorandum 1331, Dept. Appl. Math., University of Twente, 1996. Available
    at \url{https://gdeq.org}.

\bibitem{redide} \REDUCE IDE for \texttt{emacs}:
    \url{https://reduce-algebra.sourceforge.io/reduce-ide/}

\bibitem{Krasilshchik:99} \textsc{A. V. Bocharov, V. N. Chetverikov, S. V.  Duzhin, N.
      G.  Khor{\cprime}kova, I. S.  Krasil{\cprime}shchik, A.  V.  Samokhin,
      Yu.\ N.  Torkhov, A. M. Verbovetsky and A. M.  Vinogradov}: Symmetries
    and Conservation Laws for Differential Equations of Mathematical Physics,
    I.  S.  Krasil{\cprime}shchik and A. M.  Vinogradov eds., Translations of
    Math.  Monographs \textbf{182}, Amer.\ Math.\ Soc. (1999).

%\bibitem{KKV} \textsc{P.H.M. Kersten, I.S. Krasil'shchik, A.M. Verbovetsky,}
%    \emph{Hamiltonian operators and $\ell^*$-covering}, Journal of Geometry and
%    Physics \textbf{50} (2004), 273--302.

%\bibitem{Mar} \textsc{M. Marvan}, \emph{Sufficient set of integrability
%    conditions of an orthonomic system}.  Foundations of Computational
%    Mathematics \textbf{9} (2009) 651--674.

\bibitem{Vitolo:CDE} \textsc{R. Vitolo}, \emph{CDE: a reduce package for computations
  in geometry of Differential Equations}. Available at \url{https://gdeq.org}.

% bibitems from cgb.tex


\bibitem[DS97a]{DolzmannSturm:97a}
Andreas Dolzmann and Thomas Sturm.
\newblock Redlog: Computer algebra meets computer logic.
\newblock \emph{ACM SIGSAM Bulletin}, 31(2):2--9, June 1997.

\bibitem[DS97b]{Dolzmann:97b}
Andreas Dolzmann and Thomas Sturm.
\newblock Simplification of quantifier-free formulae over ordered fields.
\newblock \emph{Journal of Symbolic Computation}, 24(2):209--231, August 1997.

\bibitem[DS99]{Dolzmann:99}
Andreas Dolzmann and Thomas Sturm.
\newblock \emph{Redlog User Manual}.
\newblock FMI, Universit\"at Passau, D-94030 Passau, Germany, April 1999.
\newblock Edition 2.0 for Version 2.0.

\bibitem[Wei92]{Weispfenning:92}
Volker Weispfenning.
\newblock Comprehensive {G}r\"obner bases.
\newblock \emph{Journal of Symbolic Computation}, 14:1--29, July 1992.

% bibitems from defint.tex

\bibitem{Prudnikov:90c} A.P. Prudnikov, Yu.A. Brychkov and O.I. Marichev,
\emph{Integrals and Series, Volume 3: More Special Functions} Gordon 
and Breach Science Publishers (1990)

\bibitem{Adamchik90} V.S. Adamchik and O.I. Marichev, \emph{The 
Algorithm for Calculating Integrals of Hypergeometric Type Functions 
and its Realization in Reduce System} from \emph{ISSAC 90:Symbolic and 
Algebraic Computation} Addison-Wesley Publishing Company (1990) 

\bibitem{Luke:69} Yudell L. Luke, \emph{The Special Functions and their
Approximations, Volume 1} Academic Press (1969).

% bibitems from fps.tex

\bibitem{Hansen:75}
E.\ R. Hansen, {\em A table of series and products.}
Prentice-Hall, Englewood Cliffs, NJ, 1975.

%\bibitem{Koepf:92} Wolfram Koepf,
%{\em Power Series in Computer Algebra},
%J.\ Symbolic Computation 13 (1992)

\bibitem{Koepf:93a} Wolfram Koepf,
{\em Examples for the Algorithmic Calculation of Formal
Puiseux, Laurent and Power series},
SIGSAM Bulletin 27, 1993, 20-32.

\bibitem{Koepf:93b} Wolfram Koepf,
{\em Algorithmic development of power series.} In:
Artificial intelligence and symbolic mathematical computing,
ed.\ by J.\ Calmet and J.\ A.\ Campbell,
International Conference AISMC-1, Karlsruhe, Germany, August 1992, Proceedings,
Lecture Notes in Computer Science \textbf{737}, Springer-Verlag,
Berlin--Heidelberg, 1993, 195--213.

\bibitem{Koepf:94c} Wolfram Koepf,
{\em Algorithmic work with orthogonal polynomials and special functions.}
Konrad-Zuse-Zentrum Berlin (ZIB), Preprint SC 94-5, 1994.

% bibitems from groebner.tex

\bibitem{AmrheinGloor:98}
Beatrice Amrhein and Oliver Gloor.
\newblock The fractal walk.
\newblock In Bruno~Buchberger an~Franz~Winkler, editor, {\em Gr\"obner Bases
  and Applications}, volume 251 of {\em LMS}, pages 305 --322. Cambridge
  University Press, February 1998.

\bibitem{AmrheinGloorKuechlin:96a}
Beatrice Amrhein, Oliver Gloor, and Wolfgang Kuechlin.
\newblock How fast does the walk run?
\newblock In Alain Carriere and Louis~Remy Oudin, editors, {\em 5th Rhine
  Workshop on Computer Algebra}, volume PR 801/96, pages 8.1 -- 8.9. Institut
  Franco--Allemand de Recherches de Saint--Louis, January 1996.

\bibitem{AmrheinGloorKuechlin:96b}
Beatrice Amrhein, Oliver Gloor, and Wolfgang Kuechlin.
\newblock Walking faster.
\newblock In J.~Calmet and C.~Limongelli, editors, {\em Design and
  Implementation of Symbolic Computation Systems}, volume 1128 of {\em Lecture
  Notes in Computer Science}, pages 150 --161. Springer, 1996.

\bibitem{Becker:93}
Thomas Becker and Volker Weispfenning.
\newblock {\em {G}r{\"o}bner Bases}.
\newblock Springer, 1993.

\bibitem{Boege:86}
W.~Boege, R.~Gebauer, and H.~Kredel.
\newblock Some examples for solving systems of algebraic equations by
  calculating {G}r{\"o}bner bases.
\newblock {\em J. Symbolic Computation}, 2(1):83--98, March 1986.

%\bibitem{Buchberger:85}
%Bruno Buchberger.
%\newblock {G}r{\"o}bner bases: An algorithmic method in polynomial ideal
%  theory.
%\newblock In N.~K. Bose, editor, {\em Progress, directions and open problems in
%  multidimensional systems theory}, pages 184--232. Dordrecht: Reidel, 1985.

%\bibitem{Buchberger:88}
%Bruno Buchberger.
%\newblock Applications of {G}r{\"o}bner bases in non-linear computational
%  geometry.
%\newblock In R.~Janssen, editor, {\em Trends in Computer Algebra}, pages
%  52--80. Berlin, Heidelberg, 1988.

\bibitem{Collart:97}
S.~Collart, M.~Kalkbrener, and D.~Mall.
\newblock Converting bases with the {G}r\"obner walk.
\newblock {\em J. Symbolic Computation}, 24:465 -- 469, 1997.

\bibitem{Davenport:88a}
James~H. Davenport, Yves Siret, and Evelyne Tournier.
\newblock {\em Computer Algebra, Systems and Algorithms for Algebraic
  Computation}.
\newblock Academic Press, 1989.

\bibitem{Faugere:93}
J.~C. Faug{\`e}re, P.~Gianni, D.~Lazard, and T.~Mora.
\newblock Efficient computation of zero-dimensional {G}r\"obner bases by change
  of ordering.
\newblock Technical report, 1989.

\bibitem{Gebauer:88}
R{\"u}diger Gebauer and H.~Michael M{\"o}ller.
\newblock On an installation of {B}uchberger's algorithm.
\newblock {\em J. Symbolic Computation}, 6(2 and 3):275--286, 1988.

\bibitem{Giovini:91}
A.~Giovini, T.~Mora, G.~Niesi, L.~Robbiano, and C.~Traverso.
\newblock One sugar cube, please or selection strategies in the {B}uchberger
  algorithm.
\newblock In {\em Proc. of {ISSAC} '91}, pages 49--55, 1991.

\bibitem{Hillebrand:99}
Dietmar Hillebrand.
\newblock Triangulierung nulldimensionaler {I}deale - {I}mplementierung und
  {V}ergleich zweier {A}lgorithmen - in {G}erman . {D}iplomarbeit im
  {S}tudiengang {M}athematik der {U}niversit{\"a}t {D}ortmund. {B}etreuer:
  Prof. {D}r. {H}. {M}. {M}{\"o}ller.
\newblock Technical report, 1999.

\bibitem{Kredel:88}
Heinz Kredel.
\newblock Admissible termorderings used in computer algebra systems.
\newblock {\em {SIGSAM} Bulletin}, 22(1):28--31, January 1988.

\bibitem{Kredel:88a}
Heinz Kredel and Volker Weispfenning.
\newblock Computing dimension and independent sets for polynomial ideals.
\newblock {\em J. Symbolic Computation}, 6(1):231--247, November 1988.

\bibitem{Melenk:88}
Herbert Melenk, H.~Michael M{\"o}ller, and Winfried Neun.
\newblock On {G}r{\"o}bner bases computation on a supercomputer using {REDUCE}.
\newblock Preprint SC 88-2, Konrad-Zuse-Zentrum f{\"u}r Informationstechnik
  Berlin, January 1988.

% bibitems from guardian.tex

\bibitem{Bradford:92}
{Bradford, R.}
\newblock Algebraic simplification of multiple valued functions.
\newblock In {\em Design and Implementation of Symbolic Computation Systems\/}
  (1992), J.~Fitch, Ed., vol.~721 of {\em Lecture Notes in Computer Science},
  Springer-Verlag, pp.~13--21.
\newblock Proceedings of the DISCO 92.

\bibitem{BroadberryGomezDiazWatt:95}
{Broadberry, P., G\'omez-D\'{\i}az, T., and Watt, S.}
\newblock On the implementation of dynamic evaluation.
\newblock In {\em Proceedings of the International Symposium on Symbolic and
  Algebraic Manipulation (ISSAC 95)\/} (New York, N.Y., 1995), A.~Levelt, Ed.,
  ACM Press, pp.~77--89.

\bibitem{Collins:75}
{Collins, G.~E.}
\newblock Quantifier elimination for the elementary theory of real closed
  fields by cylindrical algebraic decomposition.
\newblock In {\em Automata Theory and Formal Languages. 2nd GI Conference\/}
  (Berlin, Heidelberg, New York, May 1975), H.~Brakhage, Ed., vol.~33 of {\em
  Lecture Notes in Computer Science}, Gesellschaft f\"ur Informatik,
  Springer-Verlag, pp.~134--183.

\bibitem{CorlessJeffrey:92}
{Corless, R.~M., and Jeffrey, D.~J.}
\newblock Well \dots it isn't quite that simple.
\newblock {\em ACM SIGSAM Bulletin 26}, 3 (Aug. 1992), 2--6.
\newblock Feature.

\bibitem{DavenportFaure:94}
{Davenport, J.~H., and Faure, C.}
\newblock The ``unknown'' in computer algebra.
\newblock {\em Programmirovanie 1}, 1 (1994).

\bibitem{DolzmannSturm:97b}
{Dolzmann, A., and Sturm, T.}
\newblock Simplification of quan\-ti\-fier-free formulas over ordered fields.
\newblock Technical Report MIP-9517, FMI, Universit\"at Passau, D-94030 Passau,
  Germany, Oct. 1995.
\newblock To appear in the Journal of Symbolic Computation.

\bibitem{DolzmannSturm:96}
{Dolzmann, A., and Sturm, T.}
\newblock Redlog---computer algebra meets computer logic.
\newblock Technical Report MIP-9603, FMI, Universit\"at Passau, D-94030 Passau,
  Germany, Feb. 1996.

\bibitem{Dolzmann:96a}
{Dolzmann, A., and Sturm, T.}
\newblock Redlog user manual.
\newblock Technical Report MIP-9616, FMI, Universit\"at Passau, D-94030 Passau,
  Germany, Oct. 1996.
\newblock Edition 1.0 for Version 1.0.

\bibitem{DuvalGonzalesVega:93}
{Duval, D., and Gonz\'ales-Vega, L.}
\newblock Dynamic evaluation and real closure.
\newblock In {\em Proceedings of the IMACS Symposium on Symbolic Computation\/}
  (1993).

\bibitem{DuvalReynaud:94}
{Duval, D., and Reynaud, J.-C.}
\newblock Sketches and computation {I}: Basic definitions and static
  evaluation.
\newblock {\em Mathematical Structures in Computer Science 4}, 2 (1994),
  185--238.

\bibitem{DuvalReynaud:94a}
{Duval, D., and Reynaud, J.-C.}
\newblock Sketches and computation {II}: Dynamic evaluation and applications.
\newblock {\em Mathematical Structures in Computer Science 4}, 2 (1994),
  239--271.

\bibitem{GomezDiaz:93}
{G\'omez-D\'{\i}az, T.}
\newblock Examples of using dynamic constructible closure.
\newblock In {\em Proceedings of the IMACS Symposium on Symbolic Computation\/}
  (1993).

\bibitem{Hearn:91}
{Hearn, A.~C., and Fitch, J.~P.}
\newblock {\em Reduce User's Manual for Version~3.6}.
\newblock RAND, Santa Monica, CA 90407-2138, July 1995.
\newblock RAND Publication CP78.

\bibitem{Hong:93}
{Hong, H., Collins, G.~E., Johnson, J.~R., and Encarnacion, M.~J.}
\newblock {QEPCAD} interactive version 12.
\newblock Kindly communicated to us by Hoon Hong, Sept. 1993.

\bibitem{LoosWeispfenning:93}
{Loos, R., and Weispfenning, V.}
\newblock Applying linear quantifier elimination.
\newblock {\em The Computer Journal 36}, 5 (1993), 450--462.
\newblock Special issue on computational quantifier elimination.

\bibitem{Melenk:95}
{Melenk, H.}
\newblock Reduce symbolic mode primer.
\newblock In {\em REDUCE 3.6 User's Guide for UNIX}. Konrad-Zuse-Institut,
  Berlin, 1995.

\bibitem{Tarski:48}
{Tarski, A.}
\newblock A decision method for elementary algebra and geometry.
\newblock Tech. rep., University of California, 1948.
\newblock Second edn., rev. 1951.

\bibitem{Weispfenning:88}
{Weispfenning, V.}
\newblock The complexity of linear problems in fields.
\newblock {\em Journal of Symbolic Computation 5}, 1 (Feb. 1988), 3--27.

%\bibitem{Weispfenning:92}
%{Weispfenning, V.}
%\newblock Comprehensive {G}r\"obner bases.
%\newblock {\em Journal of Symbolic Computation 14\/} (July 1992), 1--29.

\bibitem{Weispfenning:94}
{Weispfenning, V.}
\newblock Quantifier elimination for real algebra---the cubic case.
\newblock In {\em Proceedings of the International Symposium on Symbolic and
  Algebraic Computation in Oxford\/} (New York, July 1994), ACM Press,
  pp.~258--263.

\bibitem{Weispfenning:96}
{Weispfenning, V.}
\newblock Quantifier elimination for real algebra---the quadratic case and
  beyond.
\newblock To appear in AAECC.

% bibitems from invbase.tex

\bibitem{Lille}
Zharkov A.Yu., Blinkov Yu.A. Involution Approach to Solving Systems
of Algebraic Equations. Proceedings of the IMACS '93, 1993, 11-16.

%\bibitem{Buch}
%Buchberger B. Gr\"obner bases: an Algorithmic Method in Polynomial
%Ideal Theory. In: (Bose N.K., ed.) Recent Trends in Multidimensional
%System Theory, Reidel, 1985.

\bibitem{Lazard:83}
Lazard D. Gr\"obner Bases, Gaussian Elimination and Resolution of
Systems of Algebraic Equations. Proceedings of EUROCAL '83.
Lecture Notes in Computer Science 162, Springer 1983, 146-157.

% bibitems from lie.tex

\bibitem{MacCallum:99}
M.A.H. MacCallum.
\newblock On the classification of the real four-dimensional lie algebras.
\newblock 1979.

\bibitem{Schoebel:93}
C.~Schoebel.
\newblock Classification of real n-dimensional lie algebras with a
  low-dimensional derived algebra.
\newblock In {\em Proc. {Symposium on Mathematical Physics} '92}, 1993.

\bibitem{Schoebel:92}
F.~Schoebel.
\newblock The symbolic classification of real four-dimensional lie algebras.
\newblock 1992.

% bibitems from linalg.tex

\bibitem{Rebbeck:93} Matt Rebbeck: NORMFORM: A {\REDUCE} package for the 
computation of various matrix normal forms. ZIB, Berlin. (1993)

%\bibitem{Reduce} Anthony C. Hearn: {\REDUCE} User's Manual 3.6.
%	RAND (1995)

\bibitem{Maple} Bruce W. Char\ldots [et al.]: Maple (Computer 
        Program). Springer-Verlag (1991)

\bibitem{linalg} Linalg - a linear algebra package for Maple[3].

\bibitem{WiRe} J. H. Wilkinson \& C. Reinsch: Linear Algebra 
(volume II). Springer-Verlag (1971)

\bibitem{gat} Karin Gatermann: Symmetry: A {\REDUCE} package for the 
computation of linear representations of groups. ZIB, Berlin. (1992)

% bibitems from mrvlimit.tex

\bibitem[Grn96]{Gruntz:96} Gruntz, Dominik,
\textit{On Computing Limits in a Symbolik Manipulation System}, \\
PhD Thesis, ETH Z\"urich

\bibitem[Red36]{Red36} Hearn, Anthony C. and Fitch, John F.
\textit{REDUCE User's Manual 3.6}, \\ RAND Corporation, 1995

% bibitems from normform.tex

\bibitem{MulLev} T.M.L.Mulders and A.H.M. Levelt: {\it The Maple 
        normform and Normform packages.} (1993)

\bibitem{Mulders} T.M.L.Mulders: {\it Algoritmen in De Algebra, A 
        Seminar on Algebraic Algorithms, Nigmegen.} (1993)

\bibitem{HornJohnson:86} Roger A. Horn and Charles A. Johnson: {\it Matrix 
        Analysis.} Cambridge University Press (1986)

%\bibitem{Maple} Bruce W. Chat\ldots [et al.]: {\it Maple (Computer 
%        Program)}. Springer-Verlag (1991)

%\bibitem{Reduce} Anthony C. Hearn: {\REDUCE} {\it User's Manual 3.6.}
%	RAND (1995)

% bibitems from odesolve.tex

\bibitem{CATHODE} CATHODE (Computer Algebra Tools for Handling
Ordinary Differential Equations)
\href{http://www-lmc.imag.fr/CATHODE/}%
{\texttt{http://www-lmc.imag.fr/CATHODE/}},
\href{http://www-lmc.imag.fr/CATHODE2/}%
{\texttt{http://www-lmc.imag.fr/CATHODE2/}}.

%\bibitem{Hearn-manual} A. C. Hearn and J. P. Fitch (ed.),
%\textit{REDUCE User's Manual 3.6}, RAND Publication CP78 (Rev. 7/95),
%RAND, Santa Monica, CA 90407-2138, USA (1995).

\bibitem{MacCallum:88} M. A. H. MacCallum, An Ordinary Differential
Equation Solver for REDUCE, \textit{Proc.\ ISSAC~'88, ed.\ P. Gianni,
Lecture Notes in Computer Science} \textbf{358}, Springer-Verlag
(1989), 196--205.

\bibitem{MacCallum:ODESOLVE} M. A. H. MacCallum, ODESOLVE, \LaTeX{} file
\texttt{reduce/doc/odesolve.tex} distributed with \REDUCE~3.6.  The
first part of this document is included in the printed REDUCE User's
Manual 3.6 \cite{Hearn:91}, 345--346.

\bibitem{Man:94a} Y.-K. Man, \textit{Algorithmic Solution of ODEs and
Symbolic Summation using Computer Algebra}, PhD Thesis, School of
Mathematical Sciences, Queen Mary and Westfield College, University of
London (July 1994).

\bibitem{Man:97} Y.-K. Man and M. A. H. MacCallum, A Rational
Approach to the Prelle-Singer Algorithm, \textit{J. Symbolic
Computation}, \textbf{24} (1997), 31--43.

\bibitem{PostelZimmermann:96} F. Postel and P. Zimmermann, A Review of the ODE
Solvers of \textsc{Axiom}, \textsc{Derive}, \textsc{Maple},
\textsc{Mathematica}, \textsc{Macsyma}, \textsc{MuPAD} and
\textsc{Reduce}, \textit{Proceedings of the 5th Rhine Workshop on
Computer Algebra, April 1-3, 1996, Saint-Louis, France.}
Specific references are to the version dated April 11, 1996.
The latest version of this review, together with log files for each of
the systems, is available from
\href{https://www.loria.fr/~zimmerma/ComputerAlgebra/}%
{\texttt{https://www.loria.fr/\textasciitilde zimmerma/ComputerAlgebra/}}.

\bibitem{PrelleSinger:83} M. J. Prelle and M. F. Singer, Elementary
First Integrals of Differential Equations, \textit{Trans.\ AMS}
\textbf{279} (1983), 215--229.

\bibitem{Brand:92} T. Wolf and A. Brand, The Computer Algebra Package
\texttt{CRACK} for Investigating PDEs, \LaTeX{} file
\texttt{reduce/doc/crack.tex} distributed with \REDUCE~3.6.  A shorter
document is included in the printed REDUCE User's Manual 3.6
\cite{Hearn:91}, 241--244.

\bibitem{Wright:97} F. J. Wright, An Enhanced ODE Solver for REDUCE.
\textit{Programmirovanie} No 3 (1997), 5--22, in Russian, and
\textit{Programming and Computer Software} No 3 (1997), in English.

\bibitem{Wright:99} F. J. Wright, Design and Implementation of
\ODESolve{1+} : An Enhanced REDUCE ODE Solver.  CATHODE Workshop
Report, Marseilles, May 1999, CATHODE (1999). \\
\href{http://centaur.maths.qmw.ac.uk/Papers/Marseilles/}%
{\texttt{http://centaur.maths.qmw.ac.uk/Papers/Marseilles/}}.

\bibitem{Zwillinger:92} D. Zwillinger, \textit{Handbook of Differential
Equations}, Academic Press.  (Second edition 1992.)

% bibitems from orthovec.tex

\bibitem{Eastwood:87}
James~W. Eastwood.
\newblock Orthovec: A {REDUCE} program for {3-D} vector analysis in orthogonal
  curvilinear coordinates.
\newblock {\em Comp. Phys. Commun.}, 47(1):139--147, October 1987.

\bibitem{Eastwood:91}
James~W. Eastwood.
\newblock {ORTHOVEC:} version 2 of the {REDUCE} program for {3-D} vector
  analysis in orthogonal curvilinear coordinates.
\newblock {\em Comp. Phys. Commun.}, 64(1):121--122, April 1991.

\bibitem{Spiegel:59}
M~. Spiegel.
\newblock {\em Vector Analysis}.
\newblock Schaum Publishing Co., 1959.

% bibitems from qsum.tex

\bibitem{AskeyWilson:85}
% check
Askey R.\ and Wilson, J.:
{\sl Some Basic Hypergeometric Orthogonal Polynomials that Generalize Jacobi
Polynomials}. Memoirs Amer.\ Math.\ Soc.\ 319, Providence, RI, 1985.

\bibitem{Gasper:95}
Gasper, G.:
{\sl Lecture Notes for an Introductory Minicourse on $q$-Series}.
1995. To obtain from  
ftp://unvie6.un.or.at/siam/opsf\_new/\linebreak 00index\_by\_author.html.

\bibitem{GasperRahman:90}
Gasper, G.\ and Rahman, M.:
{\sl Basic Hypergeometric Series},
Encyclopedia of Mathematics and its Applications, 
\textbf{35}, (G.-C.\ Rota, ed.), Cambridge University Press,
London and New York, 1990.

\bibitem{Gosper:78}
Gosper Jr., R.\ W.:
Decision procedure for indefinite hypergeometric
summation. Proc.\ Natl.\ Acad.\ Sci.\ USA \textbf{75}, 1978, 40--42.

\bibitem{Knuth:TAoCP2}
Knuth, D.\ E.:
{\sl The Art of Computer Programming, Seminumerical Algorithms}.
2nd ed., 1981, Addison-Wesley Publishing Company.

\bibitem{Koepf:95e}
Koepf, W.:
Algorithms for $m$-fold hypergeometric summation.
Journal of Symbolic Computation \textbf{20}, 1995, 399--417.
%(Zbl.\ Math.\ 851.68049, summary).

\bibitem{Koepf:95}
Koepf, W.:
REDUCE package for indefinite and definite summation.
% Konrad-Zuse-Zentrum Berlin (ZIB), Technical Report TR 94-9, 1994.
SIGSAM Bulletin \textbf{29}, 1995, 14--30.

\bibitem{Koornwinder:93}
Koornwinder, T.\ H.:
On Zeilberger's algorithm and its $q$-analogue: a rigorous description.
J.\ of Comput.\ and Appl.\ Math.\ \textbf{48}, 1993, 91--111.

\bibitem{KoekoekSwarttouw:94}
Koekoek, R.\ und Swarttouw, R.F.:
{\sl The Askey-scheme of Hypergeometric Orthogonal
Polynomials and its $q$-analogue}. Report 94--05, Tech\-nische Universiteit
Delft, Faculty of Technical Mathematics and Informatics, Delft, 1994.


\bibitem{PauleRiese:95}
Paule, P.\ und Riese, A.:
A Mathematica \textsl{q}-analogue of Zeilberger's\linebreak[4]
algorithm based on an
algebraically motivated approach to \textsl{q}-hyper\-geometric telescoping.
Fields Proceedings of the Workshop `Special Functions, \textsl{q}-Series
and Related Topics', organized by the Fields Institute for Research in
Mathematical Sciences at Univerisity College,
12-23 June 1995, Toronto, Ontario,179--210.

\bibitem{Zeilberger:90}
Zeilberger, D.:
A fast algorithm for proving terminating hypergeometric identities.
Discrete Math.\ \textbf{80}, 1990, 207--211.

\bibitem{Zeilberger:91}
Zeilberger, D.:
The method of creative telescoping.
J.\ Symbolic Computation \textbf{11}, 1991, 195--204.

% bibitems from rataprx.tex

\bibitem{JonesThron:80} Jones, W B.; Thron, W.J., \\
\textit{Continued fractions. Analytic Theory and Applications},
 (Encyclopedia of Mathematics and its Applications, Vol 11),
Addison-Wesley Publishing Company, Reading, Massachusetts, 1980.

\bibitem{Euler:1748} L Euler L., \textit{Introductio in analysin infinitorum, Vol 1,
 Chapter 18}, 1748.

\bibitem{BakerGravesMorris:81} Baker(Jr.), G.A. and Graves-Morris, P.,\\
\textit{Pad\'{e} Approximants, Part I: Basic Theory},
(Encyclopedia of Mathematics and its Applications, Vol 13),
Addison-Wesley Publishing Company, Reading, Massachusetts, 1981.

% bibitems from ratint.tex

\bibitem[Bron97]{Bronstein:97} Bronstein, Manuel,
{\it Symbolic Integration I: Transendental Functions},
Springer-Verlag, Heidelberg, 1997.

\bibitem[Dav88]{Davenport:88a} Davenport, James H. et al,
{\it Computer Algebra- Systems and Algorithms for Algebraic Computation},
Academic Press, 1988.

\bibitem[Ged92]{Geddes:92} Geddes, K.O. et al,
{\it Algorithms for Computer Algebra}, Klewer Academic \mbox{Publishers}, 1992.

%\bibitem[Red36]{Red36} Hearn, Anthony C. and Fitch, John F.
%{\it REDUCE User's Manual 3.6}, RAND Corporation, 1995

% bibitems from rlfi.tex

\bibitem{Antweiler:89}
Werner Antweiler, Andreas Strotmann, and Volker Winkelmann.
\newblock A {\TeX-{reduce}-interface}.
\newblock {\em SIGSAM Bulletin}, 23:26--33, February 1989.

\bibitem{Drska:90}
Ladislav Drska, Richard Liska, and Milan Sinor.
\newblock Two practical packages for computational physics - \protect{GCPM,
  RLFI}.
\newblock {\em Comp. Phys. Comm.}, 61:225--230, 1990.

\bibitem{Fateman:87}
Richard~J. Fateman.
\newblock \protect{\TeX\ } output from macsyma-like systems.
\newblock {\em ACM SIGSAM Bulletin}, 21(4):1--5, 1987.
\newblock Issue \#82.

\bibitem{Hearn:91}
Anthony~C. Hearn.
\newblock \protect{REDUCE} user's manual, version 3.6.
\newblock Technical Report CP 78 (Rev. 7/95), The RAND Corporation, Santa
  Monica, 1995.

\bibitem{Knuth:84}
Donald~E. Knuth.
\newblock {\em The \TeX\ book}.
\newblock Addison-Wesley, Reading, 1984.

\bibitem{Lamport:86}
Leslie Lamport.
\newblock {\em \LaTeX\ - A Document Preparation System}.
\newblock Addison-Wesley, Reading, 1986.

% bibitems from sparse.tex

\bibitem{matt-linalg} Matt Rebbeck: A Linear Algebra Package for {\REDUCE}, ZIB
, Berlin. (1994)

%\bibitem{Reduce} Anthony C. Hearn: {\REDUCE} User's Manual 3.6.
%	RAND (1995)

%\bibitem{WiRe} J. H. Wilkinson \& C. Reinsch: Linear Algebra 
%(volume II). Springer-Verlag (1971)

%\bibitem{gat} Karin Gatermann: Symmetry: A {\REDUCE} package for the 
%computation of linear representations of groups. ZIB, Berlin. (1992)

% bibitems from specfn.tex

\bibitem{AbramowitzStegun:72}
Milton Abramowitz and Irene~A. Stegun, editors.
\newblock {\em Handbook of Mathematical Functions}.
\newblock Dover Publications, New York, 1972.

\bibitem{BenderOrszag:78}
Carl~M. Bender and Steven~A. Orszag.
\newblock {\em Advanced Mathematical Methods for Scientists and Engineers}.
\newblock McGraw-Hill, 1978.

\bibitem{Edmonds:57}
A.~R. Edmonds.
\newblock {\em Angular Momentum in Quantum Mechanics}.
\newblock Princeton University Press, 1957.

%\bibitem{Fillebrown:92}
%Sandra Fillebrown.
%\newblock Faster computation of bernoulli numbers.
%\newblock {\em Journal of Algorithms}, 13:431--445, 1992.

\bibitem{Khinchin:64}
Aleksandr~J. Khinchin.
\newblock {\em Continued Fractions}.
\newblock University of Chicago Press, 1964.

\bibitem{Koepf:92}
Wolfram Koepf.
\newblock Power series in computer algebra.
\newblock {\em Journal of Symbolic Computation}, 13:581--603, 1992.

\bibitem{LandoltBoernstein:68}
Landolt-Boernstein.
\newblock {\em Zahlenwerte und Funktionen aus Naturwissenschaften und Technik}.
\newblock Springer, 1968.

\bibitem{Lawden:89}
Derek F.~Lawden.
\newblock {\em Elliptic Functions and Applications}.
\newblock Springer-Verlag, 1989.

\bibitem{WhittakerWatson:69}
E.T.~Whittaker \& G.N.~Watson.
\newblock {\em A Course in Modern Analysis}.
\newblock Cambridge University Press, 1969.

\bibitem{HareCorless:92}
D.E.G.~Hare R.M.~Corless, G.H.~Gonnet and D.J. Jeffrey.
\newblock {\em On Lambert's W Function}.
\newblock Preprint, University of Waterloo, 1992.

% bibitems from specfn2.tex

%\bibitem{Prudnikov:90c} A.~P.~Prudnikov, Yu.~A.~Brychkov, O.~I.~Marichev,
%{\em Integrals and Series, Volume 3: More special functions},
%Gordon and Breach Science Publishers (1990).

\bibitem{GrahamKnuthPatashnik:89} R.~L.~Graham, D.~E.~Knuth, O.~Patashnik,
{\em Concrete Mathematics}, Addison-Wesley Publishing Company (1989).

% bibitems from symmetry.tex

\bibitem{JamesKerber:84} G.\ James, A.\ Kerber: {\it Representation Theory
of the Symmetric Group.} Addison, Wesley (1981).

\bibitem{LudwigFalter:88} W.\ Ludwig, C.\ Falter: {\it Symmetries in Physics.}
Springer, Berlin, Heidelberg, New York (1988).

\bibitem{Serre:77} J.--P.\ Serre, {\it Linear Representations of Finite
Groups}. Springer, New~York (1977).

\bibitem{StiefelFaessler:79} E.\  Stiefel, A.\  F{\"a}ssler, {\it Gruppentheoretische
Methoden und ihre Anwendung}. Teubner, Stuttgart (1979).
(English translation to appear by Birkh\"auser (1992)).

% bibitems from trigint.tex

\bibitem[Jeff]{JeffreyRich:94} Jeffery, D.J. and Rich, A.D.
\textit{The Evaluation of Trigonometric Integrals avoiding Spurious Discontinuities
}, ACM Trans. Math Software 20, 1, 1994, pages 124--135, DOI: 10.1145/174603.174409.


% bibitems from trigismp.tex

\bibitem{Roach:Talk}
Roach, Kelly: Difficulties with Trigonometrics. Notes of a talk.

\bibitem{Hearn:COMPACT}
Hearn, A.C.: COMPACT User Manual.

% bibitems from user.tex

\bibitem{Hussin:2000}
Hussin, V., Kiselev, A.V., Krutov, A.O., Wolf, T.: N=2 Supersymmetric a=4 -
Korteweg-de Vries hierarchy derived via Gardner's deformation of
Kaup-Bousinesq equation, J. Math. Phys. 51, 083507 (2010);
doi:10.1063/1.3447731 (19 pages) link, pdf file

\bibitem{KisWolf:05a}
Kiselev, A. and Wolf, T.: Supersymmetric Representations and Integrable
Super-Extensions of the Burgers and Bussinesq Equations, SIGMA, Vol. 2 (2006),
Paper 030, 19 pages (arXiv math-ph/0511071). ps file, pdf file

\bibitem{KisWolf:05b}
Kiselev, A. and Wolf, T.: On weakly non-local, nilpotent, and super-recursion
operators for N=1 homogeneous super-equations, Proc. Int. Workshop
``Supersymmetries and Quantum Symmetries'' (SQS'05), Dubna, July 24--31, 2005,
JINR, p. 231--237. (arXiv nlin.SI/0511056)
http://theor.jinr.ru/~sqs05/SQS05.pdf dvi file, ps file,

\bibitem{KisWolf:06}
Kiselev, A. and Wolf, T.: Classification of integrable super-equations by the
SsTools environment, Comp Phys Comm, Vol. 177, no. 3 (2007) p 315-328 (and on
arXiv: nlin.SI/0609065). dvi file, ps file, pdf file

\bibitem{KrasilshchikKersten:2000}
I. S. Krasil'shchik, P. H. M. Kersten, Symmetries and recursion operators for
classical and supersymmetric differential equations, Kluwer Acad. Publ.,
Dordrecht etc, (2000) 

% bibitems from xideal.tex

%\bibitem{Buchberger}
%	B.~Buchberger, {\em
%	Gr{\"o}bner Bases: an algorithmic method in polynomial ideal theory,}
%	in {\em Multidimensional Systems Theory\/} ed.~N.K.~Bose
%	(Reidel, Dordrecht, 1985) chapter 6.

\bibitem{HartleyTuckey:93}
	D.~Hartley and P.A.~Tuckey, {\em
	A Direct Characterisation of Gr{\"o}bner Bases in Clifford and
	Grassmann Algebras,}
	Preprint MPI-Ph/93--96 (1993).

\bibitem{Apel:92}
	J.~Apel, {\em A relationship between Gr{\"o}bner bases of ideals and
	vector modules of G-algebras,}
	Contemporary Math.~\textbf{131}(1992)195--204.

\bibitem{Stokes:90}
	T.~Stokes, {\em
	Gr{\"o}bner bases in exterior algebra,}
	J.~Automated Reasoning \textbf{6}(1990)233--250.

\bibitem{Schruefer:85}
	E.~Schr{\"u}fer, {\em
	EXCALC, a system for doing calculations in the calculus of modern
	differential geometry, User's manual,}
	(The Rand Corporation, Santa Monica, 1986).

% bibitems from zeilberg.tex

%\bibitem{Gosper:78}
%Gosper Jr., R.\ W.:
%Decision procedure for indefinite hypergeometric
%summation. Proc.\ Natl.\ Acad.\ Sci.\ USA \textbf{75}, 1978, 40--42.

\bibitem{Koepf:94b}
Koepf, W.:
Algorithms for the indefinite and definite summation.
Konrad-Zuse-Zentrum Berlin (ZIB), Preprint SC 94-33, 1994.

%\bibitem{Koornwinder:93}
%Koornwinder, T.\ H.:
%On Zeilberger's algorithm and its $q$-analogue: a rigorous description.
%J.\ of Comput.\ and Appl.\ Math.\ \textbf{48}, 1993, 91--111.

\bibitem{NikiforovUvarovSuslov:91}
Nikiforov, A.\ F., Suslov, S.\ K,\ and Uvarov, V.\ B.: {\sl Classical
orthogonal polynomials of a discrete variable.} Springer-Verlag,
Berlin--Heidelberg--New York, 1991.

\bibitem{PauleSchorn:95}
Paule, P.\ and Schorn, M.: A \textsc{Mathematica} version of Zeilberger's
algorithm for proving binomial coefficient identities. J.\ Symbolic
Computation, 1994, to appear.

\bibitem{OverhauserKim:94}
Problem 94--2, SIAM Review \textbf{36}, March 1994.

\bibitem{Strehl:93}
Strehl, V.:
Binomial sums and identities. Maple Technical Newsletter \textbf{10}, 1993, 37--49.

\bibitem{Wilf:90}
Wilf, H.\ S.:
{\sl Generatingfunctionology}. Academic Press, Boston, 1990.

\bibitem{Wilf:93}
Wilf, H.\ S.:
Identities and their computer proofs. ``SPICE'' Lecture Notes,
August 31--September 2, 1993.
Anonymous ftp file \texttt{pub/wilf/lecnotes.ps} on
the server \texttt{ftp.cis.upenn.edu}.

%\bibitem{Zeilberger:90}
%Zeilberger, D.:
%A fast algorithm for proving terminating hypergeometric identities.
%Discrete Math.\ \textbf{80}, 1990, 207--211.

%\bibitem{Zeilberger:91}
%Zeilberger, D.:
%The method of creative telescoping.
%J.\ Symbolic Computation \textbf{11}, 1991, 195--204.

% bibitems from ztrans.tex

\bibitem{Bronstein:1981} Bronstein, I.N. and Semedjajew, K.A.,
{\it Taschenbuch der Mathematik},
Verlag Harri Deutsch, Thun und Frankfurt(Main),
 1981.\\ISBN 3 87144 492 8.

\end{thebibliography}
