

\begin{center}
Differential linear homogenous Equation Solutions in the \\
neighbourhood of Irregular and Regular singular points \\[\baselineskip]
Version 3.1~ -~ Septembre 89 \\[2\baselineskip]
Groupe de Calcul Formel de Grenoble \\
laboratoire TIM3 \\[\baselineskip]
(C. Dicrescenzo, F. Richard-Jung, E. Tournier) \\[\baselineskip]
E-mail: dicresc@afp.imag.fr
\end{center}
\vspace*{3\baselineskip}

\subsection{INTRODUCTION}

This software enables the basis of formal solutions to be computed for an
ordinary homogeneous differential equation with polynomial coefficients
over Q of any order, in the neighbourhood of zero ( regular or irregular
singular point, or ordinary point ). \\
Tools have been added to deal with equations with a polynomial right-hand
side, parameters and a singular point not to be found at zero.

This software can be used in two ways : 
\begin{itemize}
\item direct ( DELIRE procedure )
\item interactive ( DESIR procedure)
\end{itemize}

The basic procedure is the DELIRE procedure which enables the solutions of
a linear homogeneous differential equation to be computed in the neigh-
bourhood of zero.

The DESIR procedure is a procedure without argument whereby DELIRE can be
called without preliminary treatment to the data, that is to say, in an
interactive autonomous way. This procedure also proposes some transfor-
mations on the initial equation. This allows one to start comfortably
with an equation which has a non zero singular point, a polynomial
right-hand side and parameters. \\
This document is a succint user manual. For more details on the underlying
mathematics and the algorithms used, the reader can refer to : 
\begin{center}
\begin{description}
\item[E. Tournier] : Solutions formelles d'equations differentielles - Le
logiciel de calcul formel DESIR. \\
These d'Etat de l'Universite Joseph Fourier (Grenoble - avril 87).
\end{description}
\end{center}

He will find more precision on use of parameters in :
\begin{center}
\begin{description}
\item[F. Richard-Jung] : Representation graphique de solutions d'equations
differentielles dans le champ complexe. \\
These de l'Universite Louis Pasteur (Strasbourg - septembre 88).
\end{description}
\end{center}

\subsection{FORMS OF SOLUTIONS}
We have tried to represent solutions in the simplest form possible. For
that, we have had to choose different forms according to the complexity
of the equation (parameters) and the later use we shall have of these
solutions.

\char`\"{}{\bf general solution}\char`\"{}~ =~ \{......, \{ split\_sol , cond \},....\}

\begin{center}
\begin{tabular}{lcp{9cm}}

cond & = & list of conditions or empty list (if there is no condition)
	     that parameters have to verify such that split\_sol is in the
	     basis of solutions. In fact, if there are parameters, basis of
	     solutions can have different expressions according to the values
	     of parameters. ( Note : if cond=\{\}, the list \char`\"{}general
	     solution\char`\"{} has one element only.) \\
split\_sol &  = & \{ $q , ram , polysol , r$ \} \\
	   &    & ( \char`\"{} split solution \char`\"{} enables precise information
	           on the solution to be obtained immediately ) 
\end{tabular}
\end{center}

The variable in the differential operator being x, solutions are expressed in
respect to a new variable xt, which is a fractional power of x, in the
following way :

\begin{center}
\begin{tabular}{lcl}
$q$ & : & polynomial in $1/xt$ with complex coefficients \\
$ram$ & : & $xt$ = ${x}^{ram}$ ($1/ram$ is an integer) \\
$polysol$ & : & $polynomial$ in $log(xt)$ with formal series in $xt$ coefficients \\
$r$ & : & root of a complex coefficient polynomial (\char`\"{}indicial 
          equation\char`\"{}).
\end{tabular}
\end{center}
\vspace*{\baselineskip}
\char`\"{}{\bf standard solution}\char`\"{}~ = $e^{qx} x^{r*ram} polysolx$

$qx$ and $polysolx$ are $q$ and $polysol$ expressions in which $xt$ has been
replaced by $x^{ram}$

N.B. : the form of these solutions is simplified according to the nature of
the point zero.
\begin{itemize}
\item[-] if 0 is a regular singular point : the series appearing in $polysol$ are
         convergent, $ram$ = 1 and $q$ = 0.
\item[-] if 0 is a regular point, we also have : $polysol$ is constant in $log(xt)$
         (no logarithmic terms).
\end{itemize}

\subsection{INTERACTIVE USE}

\begin{tabular}{lcl}
To call the procedure & : & desir(); \\
                      &   & solution:=desir();
\end{tabular}

The DESIR procedure computes formal solutions of a linear homogeneous
differential equation in an interactive way. \\
In this equation the variable \emph{must be x}. 

The procedure requires the order and the coefficients of the equation, the
names of parameters if there are any, then if the user wants to transform
this equation and how ( for example to bring back a singular point to zero
see procedures changehom, changevar, changefonc - ). \par
This procedure DISPLAYS the solutions and RETURNS a list of general term
\{ lcoeff, \{....,\{ general\_solution \},....\}\}. The number of elements in
this list is linked to the number of transformations requested : \\
\begin{tabular}{rl}
{*} & lcoeff : list of coefficients of the differential equation \\
{*} & general\_solution : solution written in the general form 
\end{tabular}

\subsection{DIRECT USE}

\textbf{procedure delire($x,k,grille,lcoeff,param$);}\\
This procedure computes formal solutions of a linear homogeneous differential 
equation with polynomial coefficients over Q and of any order, in the
neighborhood of zero, regular or irregular singular point. In fact it
initializes the call of the NEWTON procedure that is a recursive procedure
(algorithm of NEWTON-RAMIS-MALGRANGE)

\begin{center}
\begin{tabular}{lcp{11cm}}
$x$ & : & variable \\
$k$ & : & \char`\"{}number of desired terms\char`\"{}. \\ 
  &   & For each formal series in $xt$ appearing in $polysol$, \\
  &   & $a_0+a_1 xt+a_2 xt^2+...+a_n xt^n+\ldots$, we compute the
        $k+1$ first coefficients $a_0, a_1,\ldots,a_k.$ \\
$grille$ & : & the coefficients of the differential operator are polynomial in
	     $x^{grille}$ (in general $grille=1$) \\
$lcoeff$ & : & list of coefficients of the differential operator (in increasing
	     order of differentiation) \\
$param$ & : & list of parameters
\end{tabular}
\end{center}

This procedure RETURNS the list of general solutions.

\subsection{USEFUL FUNCTIONS}

\subsubsection{Reading of equation coefficients}

\textbf{procedure lectabcoef( );} \\
This procedure is called by DESIR to read the coefficients of an equation,
in \emph{increasing order of differentiation}, but can be used independently.

reading of n : order of the equation. \\
reading of parameters (only if a variable other than $x$ appears in the
coefficients) \\
this procedure returns the list \{ $lcoeff , param$ \} made up of the list
of coefficients and the list of parameters (which can be empty).

\subsubsection{Verification of results}

\textbf{\bf procedure solvalide($solutions,solk,k$);} \\
This procedure enables the validity of the solution number solk in the list
\char`\"{}solutions\char`\"{} to be verified. \\
$solutions$ = \{$lcoeff$,\{....,\{$general\_solution$\},....\}\} is any element
of the list returned by DESIR or is \{$lcoeff,sol$\} where $sol$ is the list 
returned by DELIRE. 

If we carry over the solution $e^{qx} x^{r*ram} polysolx$ in the equation, the
result has the form $e^{qx} x^{r*ram} reste$, where $reste$ is a polynomial in
$log(xt)$, with polynomial coefficients in $xt$. This procedure computes the
minimal valuation V of $reste$ as polynomial in $xt$, using $k$ \char`\"{}number
of desired terms\char`\"{} asked for at the call of DESIR or DELIRE, and DISPLAYS
the \char`\"{}theoretical\char`\"{} size order of the regular part of the result
: $x^{ram*(r+v)}$. 

On the other hand, this procedure carries over the solution in the equation
and DISPLAYS the significative term of the result. This is of the form :

\[e^{qx} x^a polynomial(log(xt)), \qquad \mbox{with} \quad a>=ram*(r+v).\]

Finally this procedure RETURNS the complete result of the carry over of the
solution in the equation. \par
This procedure cannot be used if the solution number solk is linked to a
condition.

\subsubsection{Writing of different forms of results}

\textbf{procedure standsol(solutions);} 

This procedure enables the simplified form of each solution to be obtained
from the list \char`\"{}solutions\char`\"{}, \{$lcoeff$,\{...,\{$general\_solution$\},....\}\}
which is one of the elements of the list returned by DESIR, or \{$lcoeff,sol$\} where
$sol$ is the list returned by DELIRE. \par
This procedure RETURNS a list of 3 elements : \{ $lcoeff, solstand, solcond$ \} \\
\begin{tabular}{lcp{9cm}}
$lcoef$ & = & list of differential equation coefficients \\
$solstand$ & = & list of solutions written in standard form \\
$solcond$ & = & list of conditional solutions that have not been written
              in standard form. This solutions remain in general form.
\end{tabular}

This procedure has no meaning for \char`\"{}conditional\char`\"{} solutions.
In case, a value has to be given to~ the parameters, that can be done either by 
calling the procedure SORPARAM that displays and returns these solutions in the
standard form, either by calling the procedure SOLPARAM which returns
these solutions in general form.

\textbf{procedure sorsol(sol);} 

This procedure is called by DESIR to write the solution $sol$, given in
general form, in standard form with enumeration of different conditions (if
there are any). \\
It can be used independently.

\subsubsection{Writing of solutions after the choice of parameters}

\textbf{procedure sorparam($solutions,param$);}

This is an interactive procedure which displays the solutions evaluated :
the value of parameters is requested. \\
%
\begin{tabular}{lcl}
$solutions$ & : & \{$lcoeff$,\{....,\{$general\_solution$\},....\}\} \\
$param$ & : & list of parameters.
\end{tabular} \\
%
It returns the list formed of 2 elements :
\begin{itemize}
\item list of evaluated coefficients of the equation
\item list of standard solutions evaluated for the value of parameters.
\end{itemize}

\textbf{procedure solparam($solutions,param,valparam$);} \\
%
This procedure evaluates the general solutions for the value of parameters
given by valparam and returns these solutions in general form. \\
%
\begin{tabular}{lcl}
$solutions$ & : & \{$lcoeff$,\{....,\{$general\_solution$\},....\}\} \\
$param$ & : & list of parameters \\
$valparam$ & : & list of parameters values
\end{tabular} \\
It returns the list formed of 2 elements :
\begin{itemize}
\item list of evaluated coefficients of the equation 
\item list of solutions in general form, evaluated for the value of parameters.
\end{itemize}

\subsubsection{Transformations}

\textbf{procedure changehom($lcoeff,x,secmember,id$);} \\
%
Differentiation of an equation with right-hand side. \\
\begin{tabular}{lcl}
$lcoeff$ & : & list of coefficients of the equation \\
$x$ & : & variable \\
$secmember$ & : & right-hand side \\
$id$ & : & order of the differentiation.
\end{tabular} \\
%
It returns the list of coefficients of the differentiated equation.
It enables an equation with polynomial right-hand side to be transformed
into a homogeneous equation by differentiating id times,
$id$ = degre($secmember$) + 1. \\
%
\textbf{procedure changevar($lcoeff,x,v,fct$);} \\
%
Changing of variable in the homogeneous equation defined by the list,lcoeff
of its coefficients : the old variable $x$ and the new variable $v$ are linked
by the relation $x = fct(v)$. \par

It returns the list of coefficients in respect to the variable $v$ of the new
equation. 

examples of use : 

\begin{itemize}
\item[-] translation enabling a rational singularity to be brought back to zero.
\item[-] $x = 1/v$ brings the infinity to 0.
\end{itemize}

\textbf{procedure changefonc($lcoeff,x,q,fct$);} \\
%
Changing of unknown function in the homogeneous equation defined by the
list lcoeff of its coefficients : \\
%
\begin{tabular}{lcl}
$lcoeff$ & : & list of coefficients of the initial equation \\
$x$      & : & variable \\
$q$ & : & new unknown function \\
$fct$ & : & $y$ being the unknown function~ $y = fct(q)$ 
\end{tabular} \\
%
It returns the list of coefficients of the new equation. 

Example of use :

this procedure enables the computation,in the neighbourhood of an irregular
singularity, of the \char`\"{}reduced\char`\"{} equation associated to one
of the slopes (the Newton polygon having a null slope of no null length). 
This equation gives much informations on the associated divergent series.

\subsubsection{Optional writing of intermediary results}

{\bf switch trdesir} : when it is ON, at each step of the Newton algorithm, a
description of the Newton polygon is displayed (it is possible to follow
the break of slopes), and at each call of the FROBENIUS procedure ( case of
a null slope ) the corresponding indicial equation is displayed. \par

By default, this switch is OFF.

\subsection{LIMITATIONS}

\begin{enumerate}
\item This DESIR version is limited to differential equations leading
to indicial equations of degree $<=$ 3. To pass beyond this limit, a further
version written in the D5 environment of the computation with algebraic
numbers has to be used.
\item The computation of a basis of solutions for an equation depending on
parameters is assured only when the indicial equations are of degree $<=$ 2.
\end{enumerate}








