%
% List of documentation/packages missing:
%
% crack/conlaw.tex
% crack/crack.tex
% crack/v3tools.tex
% eds/eds.tex
% gentran/gentran.tex -- too long!
% mathml/*.tex
% misc/reset.tex
% mrvlimit/mrvlimit.tex
% occal/ocpaper.tex
% ratint/ratint.tex
% rtrace/rdebug.tex -- PSL only
% scope/scope.tex -- too long!
% susy2/susy2.tex                            
% trigint/trigint.tex
% tri/*.tex -- too long!



\chapter{User Contributed Packages} \index{User packages}
\label{chap-user}
The complete {\REDUCE} system includes a number of packages contributed by
users that are provided as a service to the user community.  Questions
regarding these packages should be directed to their individual authors.

All such packages have been precompiled as part of the installation process.
However, many must be specifically loaded before they can be used. (Those
that are loaded automatically are so noted in their description.) You should
also consult the user notes for your particular implementation for further
information on whether this is necessary.  If it is, the relevant command is
\f{LOAD\_PACKAGE},\ttindex{LOAD\_PACKAGE} which takes a list of one or
more package names as argument, for example:

\begin{verbatim}
        load_package algint;
\end{verbatim}
although this syntax may vary from implementation to implementation.

Nearly all these packages come with separate documentation and test files
(except those noted here that have no additional documentation), which is
included, along with the source of the package, in the {\REDUCE} system
distribution.  These items should be studied for any additional details on
the use of a particular package.

\let\origsectionmark=\sectionmark
%\def\sectionmark#1{%
%  \def\xyzzy##1:##2\relax{\origsectionmark{##1}}%
%  \xyzzy#1\relax}
\def\sectionmark#1{}


The packages available in the current release of {\REDUCE} are as follows:

\newpage

\section{ALGINT: Integration of square roots} 
\indexpackage{ALGINT}
\label{ALGINT}

This package, which is an extension of the basic integration package
distributed with {\REDUCE}, will analytically integrate a wide range of
expressions involving square roots where the answer exists in that class
of functions. It is an implementation of the work described in J.H.
Davenport, ``On the Integration of Algebraic Functions", LNCS 102,
Springer Verlag, 1981.  Both this and the source code should be consulted
for a more detailed description of this work.

\hypertarget{switch:ALGINT}{}
The \texttt{ALGINT} package is loaded automatically when the switch \sw{ALGINT}
is turned on.  
One enters an expression for integration, as with the regular integrator,
for example:
\begin{verbatim}
        int(sqrt(x+sqrt(x**2+1))/x,x);
\end{verbatim}
If one later wishes to integrate expressions without using the facilities of
this package, the switch \sw{ALGINT} \ttindexswitch[ALGINT]{ALGINT} should be turned
off. 

\hypertarget{switch:TRA}{}
The switches supported by the standard integrator (e.g., \sw{TRINT})
\ttindexswitch[ALGINT]{TRINT} are also supported by this package.  In addition, the
switch \sw{TRA}, \ttindexswitch[ALGINT]{TRA} if on, will give further tracing
information about the specific functioning of the algebraic integrator.

There is no additional documentation for this package.

Author: James H. Davenport.

\newpage

\section{APPLYSYM: Infinitesimal symmetries of differential equations}
\indexpackage{APPLYSYM}

This package provides programs APPLYSYM, QUASILINPDE and DETRAFO for
applying infinitesimal symmetries of differential equations, the
generalization of special solutions and the calculation of symmetry and
similarity variables.

Author: Thomas Wolf.

\cbinput{applysym}

\newpage

\section{ARNUM: An algebraic number package} 
\label{sec:package-arnum}
\indexpackage{ARNUM}


This package provides facilities for handling algebraic numbers as
polynomial coefficients in {\REDUCE} calculations. It includes facilities for
introducing indeterminates to represent algebraic numbers, for calculating
splitting fields, and for factoring and finding greatest common divisors
in such domains.

Author: Eberhard Schr\"ufer.

\cbinput{arnum}

\newpage

\section{ASSERT: Dynamic Verification of Assertions on Function Types}
\indexpackage{ASSERT}
\label{ASSERT}

ASSERT admits to add to symbolic mode RLISP code assertions (partly)          
specifying \emph{types} of the arguments and results of RLISP expr
procedures. These types can be associated with functions testing the
validity of the respective arguments during runtime.

Author: Thomas Sturm.

\cbinput{assert}

\newpage

\section{ASSIST: Useful utilities for various applications} 
\indexpackage{ASSIST}
\label{ASSIST}\hypertarget{ASSIST}{}

ASSIST contains a large number of additional general purpose functions
that allow a user to better adapt \REDUCE\ to various calculational
strategies and to make the programming task more straightforward and more
efficient.

Author: Hubert Caprasse.

\cbinput{assist}

\newpage

\section{AVECTOR: A vector algebra and calculus package} 
\indexpackage{AVECTOR}

This package provides REDUCE with the ability to perform vector algebra
using the same notation as scalar algebra.  The basic algebraic operations
are supported, as are differentiation and integration of vectors with
respect to scalar variables, cross product and dot product, component
manipulation and application of scalar functions (e.g. cosine) to a vector
to yield a vector result.

Author: David Harper.

\cbinput{avector}

\newpage

\section{BIBASIS: A Package for Calculating Boolean Involutive Bases}
\indexpackage{BIBASIS} \label{BIBASIS}

Authors: Yuri A. Blinkov and Mikhail V. Zinin

\cbinput{bibasis}

\newpage

\section{BOOLEAN: A package for boolean algebra} 
\indexpackage{BOOLEAN}

This package supports the computation with boolean expressions in the
propositional calculus.  The data objects are composed from algebraic
expressions connected by the infix boolean operators {\bf and}, {\bf or},
{\bf implies}, {\bf equiv}, and the unary prefix operator {\bf not}.
{\bf Boolean} allows you to simplify expressions built from these
operators, and to test properties like equivalence, subset property etc.

Author: Herbert Melenk.

\cbinput{boolean}

\newpage

\section{CALI: A package for computational commutative algebra}
\indexpackage{CALI}

This package contains algorithms for computations in commutative algebra
closely related to the Gr\"obner algorithm for ideals and modules.  Its
heart is a new implementation of the Gr\"obner algorithm that also allows
for the computation of syzygies.  This implementation is also applicable to
submodules of free modules with generators represented as rows of a matrix.

Author: Hans-Gert Gr\"abe.

%\cbinput{cali}

\newpage

\section{CAMAL: Calculations in celestial mechanics}
\indexpackage{CAMAL}
\label{CAMAL}

This packages implements in REDUCE the Fourier transform procedures of the
CAMAL package for celestial mechanics.

Author: John P. Fitch.

\cbinput{camal}

\newpage

\section{CANTENS: A Package for Manipulations 
and Simplifications of Indexed Objects}

\indexpackage{CANTENS}

This package creates an environment which allows the user to
manipulate and simplify expressions containing various indexed objects
like tensors, spinors, fields and quantum fields.

Author: Hubert Caprasse.

\cbinput{cantens}

\newpage

\section{CDIFF: A package for computations in geometry
  of Differential Equations}
\indexpackage{CDIFF}
\label{CDIFF}


Authors: P. Gragert, P.H.M. Kersten, G. Post and G. Roelofs.

\cbinput{cdiff}

\newpage

% \section{CHANGEVR: Change of Independent Variable(s) in DEs}
% \index{CHANGEVR package} \index{Package ! CHANGEVR}

% This package provides facilities for changing the independent variables in
% a differential equation. It is basically the application of the chain rule.

% Author: G. \"{U}\c{c}oluk.

% \cbinput{changevar}

% \newpage

\section{CGB: Computing Comprehensive Gr\"obner Bases}
\indexpackage{CGB}

Authors: Andreas Dolzmann, Thomas Sturm, and Winfried Neun

\cbinput{cgb}

\newpage

\section{COMPACT: Package for compacting expressions} 
\indexpackage{Compact}

\ttindextype{COMPACT}{operator}
COMPACT is a package of functions for the reduction of a polynomial in the
presence of side relations.  COMPACT applies the side relations to the
polynomial so that an equivalent expression results with as few terms as
possible.  For example, the evaluation of
\begin{verbatim}
     compact(s*(1-sin x^2)+c*(1-cos x^2)+sin x^2+cos x^2,
             {cos x^2+sin x^2=1});
\end{verbatim}
yields the result\pagebreak[1]
\begin{verbatim}
              2           2
        SIN(X) *C + COS(X) *S + 1 .
\end{verbatim}
The switch \sw{TRCOMPACT} can be used to trace the operation.
\ttindexswitch[COMPACT]{TRCOMPACT}

Author:  Anthony C. Hearn.


\newpage

\section{CRACK: Solving overdetermined systems of PDEs or ODEs}
\indexpackage{CRACK}

CRACK is a package for solving overdetermined systems of partial or
ordinary differential equations (PDEs, ODEs).  Examples of programs which
make use of CRACK (finding symmetries of ODEs/PDEs, first integrals, an
equivalent Lagrangian or a "differential factorization" of ODEs) are
included.  The application of symmetries is also possible by using the
APPLYSYM package.

Authors: Andreas Brand, Thomas Wolf.

\newpage

\section{CVIT: Fast calculation of Dirac gamma matrix traces}
\indexpackage{CVIT}
\label{CVIT}

This package provides an alternative method for computing traces of Dirac
gamma matrices, based on an algorithm by Cvitanovich that treats gamma
matrices as 3-j symbols.

Authors: V.Ilyin, A.Kryukov, A.Rodionov, A.Taranov.

\cbinput{cvit}

\newpage

\section{DEFINT: A definite integration interface}
\indexpackage{DEFINT}
\label{DEFINT}

This package finds the definite integral of an expression in a stated
interval.  It uses several techniques, including an innovative approach
based on the Meijer G-function, and contour integration.

Authors: Kerry Gaskell, Stanley M. Kameny, Winfried Neun.

\cbinput{defint}

\newpage

\section{DESIR: Differential linear homogeneous equation solutions in the
              neighborhood of irregular and regular singular points}
\indexpackage{DESIR}

This package enables the basis of formal solutions to be computed for an
ordinary homogeneous differential equation with polynomial coefficients
over Q of any order, in the neighborhood of zero (regular or irregular
singular point, or ordinary point).

Authors: C. Dicrescenzo, F. Richard-Jung, E. Tournier.

\cbinput{desir}

\newpage

\section{DFPART: Derivatives of generic functions}
\indexpackage{DFPART}

This package supports computations with total and partial derivatives of
formal function objects.  Such computations can be useful in the context
of differential equations or power series expansions.

Author: Herbert Melenk.

\cbinput{dfpart}

\newpage

\section{DUMMY: Canonical form of expressions with dummy variables}
\indexpackage{DUMMY}

This package allows a user to find the canonical form of expressions
involving dummy variables. In that way, the simplification of
polynomial expressions can be fully done. The indeterminates are general
operator objects endowed with as few properties as possible. In that way
the package may be used in a large spectrum of applications.

Author: Alain Dresse.

\cbinput{dummy}

\newpage

\section{EXCALC: A differential geometry package} 
\indexpackage{EXCALC}

EXCALC is designed for easy use by all who are familiar with the calculus
of Modern Differential Geometry. The program is currently able to handle
scalar-valued exterior forms, vectors and operations between them, as well
as non-scalar valued forms (indexed forms). It is thus an ideal tool for
studying differential equations, doing calculations in general relativity
and field theories, or doing simple things such as calculating the
Laplacian of a tensor field for an arbitrary given frame.

Author: Eberhard Schr\"ufer.

\cbinput{excalc}

\newpage

\section{FIDE: Finite difference method for partial differential equations}
\indexpackage{FIDE}

This package performs  automation of  the process of numerically
solving  partial  differential  equations  systems  (PDES)  by  means of
computer algebra.  For PDES solving, the finite difference method is applied.
The  computer  algebra  system  REDUCE  and  the  numerical  programming
language FORTRAN  are used in the presented methodology. The main aim of
this methodology is to speed up the process of preparing numerical
programs for  solving PDES.  This process is quite often, especially for
complicated systems, a tedious and time consuming task.

Documentation for this package is in plain text.

Author: Richard Liska.

\cbinput{fide}

\newpage

\section{FPS: Automatic calculation of formal power series}
\indexpackage{FPS}

This package can expand a specific class of functions into their
corresponding Laurent-Puiseux series.

Authors: Wolfram Koepf and Winfried Neun.

\cbinput{fps}

\newpage


\section{GCREF: A Graph Cross Referencer}
\indexpackage{GCREF}\label{GCREF}

This package reuses the code of the RCREF package to create a graph displaying
the interdependency of procedures in a Reduce source code file. 

Authors: A. Dolzmann, T. Sturm.

\cbinput{gcref}

\newpage

\section{GENTRAN: A code generation package} 
\indexpackage{GENTRAN}
\label{GENTRAN}

GENTRAN is an automatic code GENerator and TRANslator. It constructs
complete numerical programs based on sets of algorithmic specifications
and symbolic expressions. Formatted FORTRAN, RATFOR, PASCAL or C code can be
generated through a series of interactive commands or under the control of
a template processing routine. Large expressions can be automatically
segmented into subexpressions of manageable size, and a special
file-handling mechanism maintains stacks of open I/O channels to allow
output to be sent to any number of files simultaneously and to facilitate
recursive invocation of the whole code generation process.

Author: Barbara L. Gates.

%\cbinput{gentran}

\newpage

\section{GNUPLOT: Display of functions and surfaces}
\indexpackage{PLOT}\indexpackage{GNUPLOT}

This package is an interface to the popular GNUPLOT package.
It allows you to display functions in 2D and surfaces in 3D
on a variety of output devices including X terminals, PC monitors, and
postscript and Latex printer files.

NOTE: The GNUPLOT package may not be included in all versions of REDUCE.

Author: Herbert Melenk.

\cbinput{gnuplot}

\newpage

\section{GROEBNER: A Gr\"obner basis package} 
\indexpackage{GROEBNER}
\label{GROEBNER}

GROEBNER is a package for the computation of Gr\"obner
Bases using the Buchberger algorithm and related methods
for polynomial ideals and modules.  It can be used over a variety of
different coefficient domains, and for different variable and term
orderings.

Gr\"obner Bases can be used for various purposes in commutative
algebra, e.g. for elimination of variables,\index{Variable elimination}
converting surd expressions to implicit polynomial form,
computation of dimensions, solution of polynomial equation systems 
\index{Polynomial equations} etc. 
The package is also used internally by the \f{SOLVE} \ttindex{SOLVE} 
operator.

Authors: Herbert Melenk, H.M. M\"oller and Winfried Neun.

\cbinput{groebner}

\newpage

\section{GUARDIAN: Guarded Expressions in Practice}
\indexpackage{GUARDIAN}\label{GUARDIAN}

Computer algebra systems typically drop some degenerate cases when
evaluating expressions, e.g., $x/x$ becomes $1$ dropping the case
$x=0$. We claim that it is feasible in practice to compute also the
degenerate cases yielding {\em guarded expressions}. We work over real
closed fields but our ideas about handling guarded expression can be
easily transferred to other situations. Using formulas as guards
provides a powerful tool for heuristically reducing the combinatorial
explosion of cases: equivalent, redundant, tautological, and
contradictive cases can be detected by simplification and quantifier
elimination. Our approach allows to simplify the expressions on the
basis of simplification knowledge on the logical side. The method
described in this paper is implemented in the {\sc reduce} package
{\sc guardian}.

Authors: Andreas Dolzmann and Thomas Sturm.

\cbinput{guardian}

\newpage

\section{IDEALS: Arithmetic for polynomial ideals} 
\indexpackage{IDEALS}

This package implements the basic arithmetic for polynomial ideals by
exploiting the Gr\"obner bases package of REDUCE.  In order to save
computing time all intermediate Gr\"obner bases are stored internally such
that time consuming repetitions are inhibited.

Author: Herbert Melenk.

\cbinput{ideals}

\newpage

\section{INEQ: Support for solving inequalities} 
\indexpackage{INEQ}

This package supports the operator {\bf ineq\_solve} that 
tries to solves single inequalities and sets of coupled inequalities.

Author: Herbert Melenk.

\cbinput{ineq}

\newpage

\section{INVBASE: A package for computing involutive bases} 
\indexpackage{INVBASE}

Involutive bases are a new tool for solving problems in connection with
multivariate polynomials, such as solving systems of polynomial equations
and analyzing polynomial ideals.  An involutive basis of polynomial ideal
is nothing but a special form of a redundant Gr\"obner basis.  The
construction of involutive bases reduces the problem of solving polynomial
systems to simple linear algebra.

Authors: A.Yu. Zharkov and Yu.A. Blinkov.

\cbinput{invbase}

\newpage

\section{LALR: A parser generator}
\indexpackage{LALR}

Author: Arthur Norman

\cbinput{lalr}

\newpage

\section{LAPLACE: Laplace transforms} 
\indexpackage{LAPLACE}

This package can calculate ordinary and inverse Laplace transforms of
expressions.  Documentation is in plain text.

Authors: C. Kazasov, M. Spiridonova, V. Tomov.

\cbinput{laplace}

\newpage

\section{LIE: Functions for the classification of real n-dimensional Lie
algebras}
\indexpackage{LIE}

{\bf LIE} is a package of functions for the classification of real
n-dimensional Lie algebras.  It consists of two modules: {\bf liendmc1}
and {\bf lie1234}.  With the help of the functions in the {\bf liendmcl}
module, real n-dimensional Lie algebras $L$ with a derived algebra
$L^{(1)}$ of dimension 1 can be classified.

Authors: Carsten and Franziska Sch\"obel.

\cbinput{lie}

\newpage

\section{LIMITS: A package for finding limits} 
\indexpackage{LIMITS}


This package loads automatically.

Author: Stanley L. Kameny.

\cbinput{limits}

\newpage

\section{LINALG: Linear algebra package} 
\indexpackage{LINALG}
\label{LINALG}

This package provides a selection of functions that are useful 
in the world of linear algebra.

Author: Matt Rebbeck.

\cbinput{linalg}

\newpage

\section{LPDO: Linear Partial Differential Operators}
\indexpackage{LPDO}
\label{LPDO}

Author: Thomas Sturm

\cbinput{lpdo}

\newpage

\section{MODSR: Modular solve and roots} 
\indexpackage{MODSR}
\ttindextype{M\_SOLVE}{operator}\ttindextype{M\_ROOTS}{operator}

This package supports solve (M\_SOLVE) and roots (M\_ROOTS) operators for
modular polynomials and modular polynomial systems.  The moduli need not
be primes. M\_SOLVE requires a modulus to be set.  M\_ROOTS takes the
modulus as a second argument. For example:

\begin{verbatim}
on modular; setmod 8;
m_solve(2x=4);            ->  {{X=2},{X=6}}
m_solve({x^2-y^3=3});
    ->  {{X=0,Y=5}, {X=2,Y=1}, {X=4,Y=5}, {X=6,Y=1}}
m_solve({x=2,x^2-y^3=3}); ->  {{X=2,Y=1}}
off modular;
m_roots(x^2-1,8);         ->  {1,3,5,7}
m_roots(x^3-x,7);         ->  {0,1,6}
\end{verbatim}

\ttindex{LEGENDRE\_SYMBOL}
The operator \f{legendre\_symbol}(a,p) denotes the Legendre symbol
\begin{displaymath}
  \left(\frac{a}{p}\right) \equiv a^{\frac{p-1}{2}} \pmod{p}
\end{displaymath}
which, by its very definition can only have one of the values $\{-1,0,1\}$.

There is no further documentation for this package.

Author: Herbert Melenk.

\newpage

\section{NCPOLY: Non--commutative polynomial ideals}
\indexpackage{NCPOLY}

This package allows the user to set up automatically a consistent
environment for computing in an algebra where the non--commutativity is
defined by Lie-bracket commutators.  The package uses the {REDUCE} {\bf
noncom} mechanism for elementary polynomial arithmetic; the commutator
rules are automatically computed from the Lie brackets.

Authors: Herbert Melenk and Joachim Apel.

\cbinput{ncpoly}

\newpage

\section{NORMFORM: Computation of matrix normal forms} 
\indexpackage{NORMFORM}
\label{NORMFORM}

This package contains routines for computing the following
normal forms of matrices:
\begin{itemize}
\item smithex\_int
\item smithex
\item frobenius
\item ratjordan
\item jordansymbolic
\item jordan.
\end{itemize}

Author: Matt Rebbeck.

\cbinput{normform}

\newpage

\section{NUMERIC: Solving numerical problems}
\indexpackage{NUMERIC}
\ttindex{NUM\_SOLVE}\index{Newton's method}\ttindex{NUM\_ODESOLVE}
\ttindex{BOUNDS}\index{Chebyshev fit}
\ttindex{NUM\_MIN}\index{Minimum}\ttindex{NUM\_INT}\index{Quadrature}
This package implements basic algorithms of numerical analysis.
These include:
\begin{itemize}
\item solution of algebraic equations by Newton's method
\begin{verbatim}
    num_solve({sin x=cos y, x + y = 1},{x=1,y=2})
\end{verbatim}
\item solution of ordinary differential equations
\begin{verbatim}
    num_odesolve(df(y,x)=y,y=1,x=(0 .. 1), iterations=5)
\end{verbatim}
\item bounds of a function over an interval
\begin{verbatim}
    bounds(sin x+x,x=(1 .. 2));
\end{verbatim}
\item minimizing a function (Fletcher Reeves steepest descent)
\begin{verbatim}
    num_min(sin(x)+x/5, x);
\end{verbatim}
\item Chebyshev curve fitting
\begin{verbatim}
    chebyshev_fit(sin x/x,x=(1 .. 3),5);
\end{verbatim}
\item numerical quadrature
\begin{verbatim}
    num_int(sin x,x=(0 .. pi));
\end{verbatim}
\end{itemize}

Author: Herbert Melenk.

\cbinput{numeric}

\newpage

\section[ODESOLVE: Ordinary differential equations solver]%
        {ODESOLVE: \protect\\ Ordinary differential equations solver}

\indexpackage{ODESOLVE}

The ODESOLVE package is a solver for ordinary differential equations.  At
the present time it has very limited capabilities.  It can handle only a
single scalar equation presented as an algebraic expression or equation,
and it can solve only first-order equations of simple types, linear
equations with constant coefficients and Euler equations.  These solvable
types are exactly those for which Lie symmetry techniques give no useful
information.  For example, the evaluation of
\begin{verbatim}
        depend(y,x);
        odesolve(df(y,x)=x**2+e**x,y,x);
\end{verbatim}
yields the result
\begin{verbatim}
               X                    3
            3*E  + 3*ARBCONST(1) + X
        {Y=---------------------------}
                        3
\end{verbatim}

Main Author: Malcolm A.H. MacCallum.

Other contributors: Francis Wright, Alan Barnes.

\cbinput{odesolve}

\newpage


\section{ORTHOVEC: Manipulation of scalars and vectors}

\indexpackage{Orthovec}

ORTHOVEC is a collection of REDUCE procedures and operations which
provide a simple-to-use environment for the manipulation of scalars and
vectors.  Operations include addition, subtraction, dot and cross
products, division, modulus, div, grad, curl, laplacian, differentiation,
integration, and Taylor expansion.

Author: James W. Eastwood.

\cbinput{orthovec}

\newpage

\section{PHYSOP: Operator calculus in quantum theory}

\indexpackage{PHYSOP}

This package has been designed to meet the requirements of theoretical
physicists looking for a computer algebra tool to perform complicated
calculations in quantum theory with expressions containing operators.
These operations consist mainly of the calculation of commutators between
operator expressions and in the evaluations of operator matrix elements in
some abstract space.

Author: Mathias Warns.

\cbinput{physop}

\newpage

\section{PM: A REDUCE pattern matcher} 
\indexpackage{PM}

PM is a general pattern matcher similar in style to those found in systems
such as SMP and Mathematica, and is based on the pattern matcher described
in Kevin McIsaac, ``Pattern Matching Algebraic Identities'', SIGSAM Bulletin,
19 (1985), 4-13.

Author: Kevin McIsaac.

\cbinput{pm}

\newpage

\section{POLYDIV: Enhanced Polynomial Division}

\indexpackage{POLYDIV}
This package provides better access to the standard internal
polynomial division facilities of REDUCE and implements polynomial
pseudo-division.  It provides optional local control over the main
variable used for division.

Author: Francis J. Wright

\cbinput{polydiv}

\newpage

\section{QSUM: Indefinite and Definite Summation
of \textsl{q}-hypergeometric Terms}
\indexpackage{QSUM}

Authors: Harald B�ing and Wolfram Koepf

\cbinput{qsum}

\newpage

\section{RANDPOLY: A random polynomial generator} 
\indexpackage{RANDPOLY}

This package is based on a port of the Maple random polynomial
generator together with some support facilities for the generation
of random numbers and anonymous procedures.

Author: Francis J. Wright.

\cbinput{randpoly}

\newpage

\section{RATAPRX: Rational Approximations Package for REDUCE}
\indexpackage{RATAPRX}

Authors: Lisa Temme and Wolfram Koepf

\cbinput{rataprx}

\newpage

\section{REACTEQN: Support for chemical reaction equation systems}

\indexpackage{REACTEQN}

This package allows a user to transform chemical reaction systems into
ordinary differential equation systems (ODE) corresponding to the laws of
pure mass action.

%Documentation for this package is in plain text.

Author: Herbert Melenk.

\cbinput{reacteqn}

\newpage

\section{REDLOG: Extend \REDUCE{} to a computer logic system}

\indexpackage{REDLOG}

The name REDLOG stand for REDuce LOGic system. Redlog implements
symbolic algorithms on first-order formulas with respect to
user-chosen first-order languages and theories. The available domains
include real numbers, integers, complex numbers, p-adic numbers,
quantified propositional calculus, term algebras.

Documentation for this package can be found \href{http://redlog.eu/}{online}.

Authors: Andreas Dolzmann and Thomas Sturm

\section{RESET: Code to reset REDUCE to its initial state} 
\indexpackage{RESET}

This package defines a command RESETREDUCE that works through the
history of previous commands, and clears any values which have been
assigned, plus any rules, arrays and the like.  It also sets the various
switches to their initial values.  It is not complete, but does work for
most things that cause a gradual loss of space.  It would be relatively
easy to make it interactive, so allowing for selective resetting.

There is no further documentation on this package.

Author: John Fitch.

\newpage

\section{RESIDUE: A residue package} 
\indexpackage{RESIDUE}

This package supports the calculation of residues of arbitrary
expressions.

Author: Wolfram Koepf.

\cbinput{residue}


\newpage

\section{RLFI: REDUCE \LaTeX{} formula interface} 
\indexpackage{RLFI}

This package adds \LaTeX{} syntax to REDUCE.  Text generated by REDUCE in
this mode can be directly used in \LaTeX{} source documents.  Various
mathematical constructions are supported by the interface including
subscripts, superscripts, font changing, Greek letters, divide-bars,
integral and sum signs, derivatives, and so on.

Author: Richard Liska.

\cbinput{rlfi}

\newpage

\section{ROOTS: A REDUCE root finding package} 
\indexpackage{ROOTS}

This root finding package can be used to find some or all of the roots of a
univariate polynomial with real or complex coefficients, to the accuracy
specified by the user.

It is designed so that it can be used as an independent package, or it may
be called from {\tt SOLVE} if {\tt ROUNDED} is on. For example,
the evaluation of
\begin{verbatim}
        on rounded,complex;
        solve(x**3+x+5,x);
\end{verbatim}
yields the result
\begin{verbatim}
    {X= - 1.51598,X=0.75799 + 1.65035*I,X=0.75799 - 1.65035*I}
\end{verbatim}

This package loads automatically.

Author: Stanley L. Kameny.

\cbinput{roots}

\newpage

\section[RSOLVE: Rational/integer polynomial solvers]%
        {RSOLVE: \protect\\ Rational/integer polynomial solvers}

\indexpackage{RSOLVE}

This package provides operators that compute the exact rational zeros
of a single univariate polynomial using fast modular methods.  The
algorithm used is that described by R. Loos (1983): Computing rational
zeros of integral polynomials by $p$-adic expansion, \textit{SIAM J.
Computing}, \textbf{12}, 286--293.

Author: Francis J. Wright.

\cbinput{rsolve}

\newpage

\section{RTRACE: Tracing in \REDUCE}
\indexpackage{RTRACE}

Authors: Herbert Melenk and Francis J. Wright

\cbinput{rtrace}

\newpage

\section{SCOPE: REDUCE source code optimization package} 
\indexpackage{SCOPE}
\label{SCOPE}

SCOPE is a package for the production of an optimized form of a set of
expressions.  It applies an heuristic search for common (sub)expressions
to almost any set of proper REDUCE assignment statements.  The
output is obtained as a sequence of assignment statements.  GENTRAN is
used to facilitate expression output.

Author:  J.A. van Hulzen.

\newpage

\section{SETS: A basic set theory package} 
\indexpackage{SETS}

%The SETS package provides algebraic-mode support for set operations on
%lists regarded as sets (or representing explicit sets) and on implicit
%sets represented by identifiers.

Author: Francis J. Wright.

\cbinput{sets}

\newpage

\section{SPARSE: Sparse Matrix Calculations}
\indexpackage{SPARSE}

Author: Stephen Scowcroft

\cbinput{sparse}

\newpage

\section{SPDE: Finding symmetry groups of {PDE}'s}

\indexpackage{SPDE}

The package SPDE provides a set of functions which may be used to
determine the symmetry group of Lie- or point-symmetries of a given system
of partial differential equations. In many cases the determining system is
solved completely automatically. In other cases the user has to provide
additional input information for the solution algorithm to terminate.

Author: Fritz Schwarz.

\cbinput{spde}

\newpage

\section{SPECFN: Package for special functions} 
\indexpackage{SPECFN}

\index{Gamma function}       \ttindex{GAMMA}
\index{Digamma function}     \ttindex{Digamma}
\index{Polygamma functions}  \ttindex{POLYGAMMA}
\index{Pochhammer's symbol}  \ttindex{POCHHAMMER}
\index{Euler numbers}        \ttindex{EULER}
\index{Bernoulli numbers}    \ttindex{BERNOULLI}
\index{Zeta function (Riemann's)}  \ttindex{ZETA}
\index{Bessel functions} \ttindex{BesselJ} \ttindex{BesselY}
                         \ttindex{BesselK} \ttindex{BesselI}
\index{Hankel functions} \ttindex{Hankel1} \ttindex{Hankel2}
\index{Kummer functions} \ttindex{KummerM} \ttindex{KummerU}
\index{Struve functions} \ttindex{StruveH} \ttindex{StruveL}
\index{Lommel functions} \ttindex{Lommel1} \ttindex{Lommel2}
\index{Beta function}       \ttindex{BETA}
\index{Whittaker functions} \ttindex{WhittakerM}
                            \ttindex{WhittakerW}
\index{Dilogarithm function}   \ttindex{DILOG}
\index{Psi function}           \ttindex{PSI}
\index{Orthogonal polynomials} 
\index{Hermite polynomials}    \ttindex{HermiteP}
\index{Jacobi's polynomials}   \ttindex{JacobiP}
\index{Legendre polynomials}   \ttindex{LegendreP}
\index{Laguerre polynomials}   \ttindex{LaguerreP}
\index{Chebyshev polynomials}  \ttindex{ChebyshevT} \ttindex{ChebyshevU}
\index{Gegenbauer polynomials} \ttindex{GegenbauerP}
\index{Euler polynomials}      \ttindex{EulerP}
\index{Binomial coefficients}  \ttindex{Binomial}
\index{Stirling numbers} \ttindex{Stirling1} \ttindex{Stirling2}
\index{Spherical and Solid Harmonics} \ttindex{SphericalHarmonicY}
\ttindex{SolidHarmonicY}
\index{Jacobi Elliptic Functions and Integrals}
\ttindex{Jacobiamplitude} \ttindex{Jacobisn} \ttindex{Jacobidn}
\ttindex{Jacobicn} \ttindex{EllipticF} \ttindex{EllipticE}
\ttindex{EllipticTheta} \ttindex{JacobiZeta}
\index{Airy functions} \ttindex{Airy\_Ai} \ttindex{Airy\_Bi}
\ttindex{Airy\_Aiprime} \ttindex{Airy\_Biprime}
\index{3j and 6j symbols} \index{Clebsch Gordan coefficients}
\ttindex{ThreejSymbol} \ttindex{SixjSymbol} \ttindex{Clebsch\_Gordan}

This special function package is separated into two portions to make it
easier to handle.  The packages are called SPECFN and SPECFN2.  The first
one is more general in nature, whereas the second is devoted to special
special functions.  Documentation for the first package can be found in
the file specfn.tex in the ``doc'' directory, and examples in specfn.tst
and specfmor.tst in the examples directory.

The package SPECFN is designed to provide algebraic and numerical
manipulations of several common special functions, namely:

\begin{itemize}
\item Bernoulli Numbers and Euler Numbers;
\item Stirling Numbers;
\item Binomial Coefficients;
\item Pochhammer notation;
\item The Gamma function;
\item The Psi function and its derivatives;
\item The Riemann Zeta function;
\item The Bessel functions J and Y of the first and second kind;
\item The modified Bessel functions I and K;
\item The Hankel functions H1 and H2;
\item The Kummer hypergeometric functions M and U;
\item The Beta function, and Struve, Lommel and Whittaker functions;
\item The Airy functions;
\item The Exponential Integral, the Sine and Cosine Integrals;
\item The Hyperbolic Sine and Cosine Integrals;
\item The Fresnel Integrals and the Error function;
\item The Dilog function;
\item Hermite Polynomials;
\item Jacobi Polynomials;
\item Legendre Polynomials;
\item Spherical and Solid Harmonics;
\item Laguerre Polynomials;
\item Chebyshev Polynomials;
\item Gegenbauer Polynomials;
\item Euler  Polynomials;
\item Bernoulli Polynomials.
\item Jacobi Elliptic Functions and Integrals;
\item 3j symbols, 6j symbols and Clebsch Gordan coefficients;
\end{itemize}

Author:  Chris Cannam, with contributions from Winfried Neun, Herbert
Melenk, Victor Adamchik, Francis Wright and several others.

\section{SPECFN2: Package for special special functions} 
\indexpackage{SPECFN2}

\index{Generalized Hypergeometric functions} 
\index{Meijer's G function}

This package provides algebraic manipulations of generalized
hypergeometric functions and Meijer's G function.  Generalized
hypergeometric functions are simplified towards special functions and
Meijer's G function is simplified towards special functions or generalized
hypergeometric functions.

Author: Victor Adamchik, with major updates by Winfried Neun.

\cbinput{specfn2}

\newpage

\section{SUM: A package for series summation} 
\indexpackage{SUM}
\hypertarget{operator:SUM}{}
\hypertarget{operator:PROD}{}

This package implements the Gosper algorithm for the summation of series.
It defines operators {\tt SUM} and {\tt PROD}.  The operator {\tt SUM}
returns the indefinite or definite summation of a given expression, and
{\tt PROD} returns the product of the given expression.

This package loads automatically.

Author: Fujio Kako.

\cbinput{sum}

\newpage

\section{SYMMETRY: Operations on symmetric matrices} 
\indexpackage{SYMMETRY}

This package computes symmetry-adapted bases and block diagonal forms of
matrices which have the symmetry of a group.  The package is the
implementation of the theory of linear representations for small finite
groups such as the dihedral groups.

Author: Karin Gatermann.

\cbinput{symmetry}

\newpage

\section{TAYLOR: Manipulation of Taylor series}

\indexpackage{TAYLOR}
\index{Taylor series}
\index{Laurent series} \index{Puiseux series}

This package carries out the Taylor expansion of an expression in one or
more variables and efficient manipulation of the resulting Taylor series.
Capabilities include basic operations (addition, subtraction,
multiplication and division) and also application of certain algebraic and
transcendental functions.

Author: Rainer Sch\"opf.

\cbinput{taylor}

\newpage

\section{TPS: A truncated power series package} 
\indexpackage{TPS} 
\ttindex{PS}

This package implements formal Laurent series expansions in one variable
using the domain mechanism of REDUCE.  This means that power series
objects can be added, multiplied, differentiated etc.,  like other first
class objects in the system.  A lazy evaluation scheme is used and thus
terms of the series are not evaluated until they are required for printing
or for use in calculating terms in other power series.  The series are
extendible giving the user the impression that the full infinite series is
being manipulated.  The errors that can sometimes occur using series that
are truncated at some fixed depth (for example when a term in the required
series depends on terms of an intermediate series beyond the truncation
depth) are thus avoided.

Authors:  Alan Barnes and Julian Padget.

\cbinput{tps}

\newpage

\section{TRI: TeX REDUCE interface} 
\indexpackage{TRI}

This package provides facilities written in REDUCE-Lisp for typesetting
REDUCE formulas using \TeX.  The \TeX-REDUCE-Interface incorporates three
levels of \TeX output: without line breaking, with line breaking, and
with line breaking plus indentation.

Author: Werner Antweiler.

\newpage

\section{TRIGSIMP: Simplification and factorization of trigonometric
and hyperbolic functions} 
\indexpackage{TRIGSIMP}

\label{TRIGSIMP}

\iffalse
TRIGSIMP is a useful tool for all kinds of trigonometric and hyperbolic
simplification and factorization.  There are three procedures included in
TRIGSIMP: trigsimp, trigfactorize and triggcd.  The first is for finding
simplifications of trigonometric or hyperbolic expressions with many
options, the second for factorizing them and the third for finding the
greatest common divisor of two trigonometric or hyperbolic polynomials.
\fi
Author: Wolfram Koepf.

\cbinput{trigsimp}

\newpage

\section{TURTLE: Turtle Graphics Interface for REDUCE}

\indexpackage{TURTLE}
Author: Caroline Cotter

\cbinput{turtle}


\newpage

\section{WU: Wu algorithm for polynomial systems} 
\indexpackage{WU}

This is a simple implementation of the Wu algorithm implemented in REDUCE
working directly from ``A Zero Structure Theorem for
Polynomial-Equations-Solving,'' Wu Wen-tsun, Institute of Systems Science,
Academia Sinica, Beijing.

Author: Russell Bradford.

\cbinput{wu}

\newpage

\section{XCOLOR: Color factor in some field theories}

\indexpackage{XCOLOR}

This package calculates the color factor in non-abelian gauge field
theories using an algorithm due to Cvitanovich.

Documentation for this package is in plain text.

Author: A. Kryukov.

\cbinput{xcolor}

\newpage

\section{XIDEAL: Gr\"obner Bases for exterior algebra} 
\indexpackage{XIDEAL}

XIDEAL constructs Gr\"obner bases for solving the left ideal membership
problem: Gr\"obner left ideal bases or GLIBs. For graded ideals, where each
form is homogeneous in degree, the distinction between left and right
ideals vanishes. Furthermore, if the generating forms are all homogeneous,
then the Gr\"obner bases for the non-graded and graded ideals are
identical. In this case, XIDEAL is able to save time by truncating the
Gr\"obner basis at some maximum degree if desired.

Author: David Hartley.

\cbinput{xideal}

\newpage

\section{ZEILBERG: Indefinite and definite summation}

\indexpackage{ZEILBERG}

This package is a careful implementation of the Gosper and Zeilberger
algorithms for indefinite and definite summation of hypergeometric terms,
respectively.  Extensions of these algorithms are also included that are
valid for ratios of products of powers, factorials, $\Gamma$ function
terms, binomial coefficients, and shifted factorials that are
rational-linear in their arguments.

Authors: Gregor St\"olting and Wolfram Koepf.

\cbinput{zeilberg}

\newpage

\section{ZTRANS: \texorpdfstring{$Z$}{\textit{Z}}-transform package}

\indexpackage{ZTRANS}

This package is an implementation of the $Z$-transform of a sequence.
This is the discrete analogue of the Laplace Transform.

Authors: Wolfram Koepf and Lisa Temme.

\cbinput{ztrans}

\let\sectionmark=\origsectionmark
