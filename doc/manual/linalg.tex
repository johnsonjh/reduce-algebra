

\newcommand\exprlist  {expr$_{1}$,expr$_{2}$, \ldots ,expr$_{{\tt n}}$}
\newcommand\lineqlist {lin\_eqn$_{1}$,lin\_eqn$_{2}$, \ldots ,lin\_eqn$_{n}$}
\newcommand\matlist   {mat$_{1}$,mat$_{2}$, \ldots ,mat$_{n}$}
\newcommand\veclist   {vec$_{1}$,vec$_{2}$, \ldots ,vec$_{n}$}

\newcommand\lazyfootnote{\footnote{If you're feeling lazy then the \{\}'s can 
                  be omitted.}}

%\renewcommand{\thefootnote}{\fnsymbol{footnote}}

\setcounter{secnumdepth}{3}

\index{Linear Algebra package}

\subsection{Introduction}

This package provides a selection of functions that are useful in the world of
linear algebra. These functions are described alphabetically in subsection
\ref{linalg:subsec3} and are labelled \ref{linalg:add_columns} to
\ref{linalg:kronecker_product}.  They can be classified into four sections(n.b:
the numbers after the dots signify the function label in section
\ref{linalg:subsec3}).

Contributions to this package have been made by Walter Tietze (ZIB).

\subsubsection{Basic matrix handling}

\begin{center}
  \begin{tabular}{l l l l l l}
    add\_columns     & \ldots & \ref{linalg:add_columns}  & 
    add\_rows        & \ldots & \ref{linalg:add_rows}  \\
    add\_to\_columns & \ldots & \ref{linalg:add_to_columns}  &
    add\_to\_rows    & \ldots & \ref{linalg:add_to_rows}  \\
    augment\_columns & \ldots & \ref{linalg:augment_columns}  &
    char\_poly       & \ldots & \ref{linalg:char_poly}  \\
    column\_dim      & \ldots & \ref{linalg:column_dim}  &
    copy\_into       & \ldots & \ref{linalg:copy_into} \\
    diagonal         & \ldots & \ref{linalg:diagonal} &
    extend           & \ldots & \ref{linalg:extend} \\
    find\_companion  & \ldots & \ref{linalg:find_companion}  &
    get\_columns     & \ldots & \ref{linalg:get_columns} \\
    get\_rows        & \ldots & \ref{linalg:get_rows} &
    hermitian\_tp    & \ldots & \ref{linalg:hermitian_tp} \\
    matrix\_augment  & \ldots & \ref{linalg:matrix_augment} &
    matrix\_stack    & \ldots & \ref{linalg:matrix_stack} \\
    minor            & \ldots & \ref{linalg:minor} &
    mult\_columns    & \ldots & \ref{linalg:mult_columns} \\ 
    mult\_rows       & \ldots & \ref{linalg:mult_rows} &
    pivot            & \ldots & \ref{linalg:pivot} \\
    remove\_columns  & \ldots & \ref{linalg:remove_columns} &
    remove\_rows     & \ldots & \ref{linalg:remove_rows} \\
    row\_dim         & \ldots & \ref{linalg:row_dim} &
    rows\_pivot      & \ldots & \ref{linalg:rows_pivot} \\
    stack\_rows      & \ldots & \ref{linalg:stack_rows} &
    sub\_matrix      & \ldots & \ref{linalg:sub_matrix} \\
    swap\_columns    & \ldots & \ref{linalg:swap_columns} &
    swap\_entries    & \ldots & \ref{linalg:swap_entries} \\
    swap\_rows       & \ldots & \ref{linalg:swap_rows} &
  \end{tabular}
\end{center}


\subsubsection{Constructors}

Functions that create matrices.

\begin{center}
\begin{tabular}{l l l l l l}
band\_matrix       & \ldots &\ref{linalg:band_matrix} &
block\_matrix      & \ldots &\ref{linalg:block_matrix} \\
char\_matrix       & \ldots &\ref{linalg:char_matrix} & 
coeff\_matrix      & \ldots &\ref{linalg:coeff_matrix} \\ 
companion          & \ldots &\ref{linalg:companion} & 
hessian            & \ldots &\ref{linalg:hessian} \\
hilbert            & \ldots &\ref{linalg:hilbert} & 
mat\_jacobian      & \ldots &\ref{linalg:mat_jacobian} \\
jordan\_block      & \ldots &\ref{linalg:jordan_block} & 
make\_identity     & \ldots &\ref{linalg:make_identity} \\
random\_matrix     & \ldots &\ref{linalg:random_matrix} & 
toeplitz           & \ldots &\ref{linalg:toeplitz} \\
Vandermonde        & \ldots &\ref{linalg:Vandermonde} &
Kronecker\_Product & \ldots &\ref{linalg:kronecker_product}  
\end{tabular}
\end{center}

\subsubsection{High level algorithms}

\begin{center}
\begin{tabular}{l l l l l l}
char\_poly       & \ldots &\ref{linalg:char_poly} & 
cholesky         & \ldots &\ref{linalg:cholesky} \\ 
gram\_schmidt    & \ldots &\ref{linalg:gram_schmidt} & 
lu\_decom        & \ldots &\ref{linalg:lu_decom} \\
pseudo\_inverse  & \ldots &\ref{linalg:pseudo_inverse} & 
simplex          & \ldots &\ref{linalg:simplex} \\
svd              & \ldots &\ref{linalg:svd} & 
triang\_adjoint  & \ldots &\ref{linalg:triang_adjoint}
\end{tabular}
\end{center}

There is a separate {\small NORMFORM}[1] package for computing 
the following matrix normal forms in \REDUCE:
\begin{center}
  smithex, smithex\_int, frobenius, ratjordan, jordansymbolic, jordan.
\end{center}


\subsubsection{Predicates}

\begin{center}
  \begin{tabular}{l l l l l l}
    matrixp     & \ldots &\ref{linalg:matrixp} & 
    squarep     & \ldots &\ref{linalg:squarep} \\
    symmetricp  & \ldots &\ref{linalg:symmetricp} & 
  \end{tabular}
\end{center}


\subsubsection*{Note on examples:} 

In the examples the matrix $\mathcal{A}$ will be 
\begin{displaymath}
\mathcal{A} = \begin{pmatrix} 1 & 2 & 3 \\ 4 & 5 & 6 \\ 7 & 8 & 9 \end{pmatrix}
\end{displaymath}


\subsubsection*{Notation}

Throughout $\mathcal{I}$ is used to indicate the identity matrix and 
$\mathcal{A}^T$ to indicate the transpose of the matrix $\mathcal{A}$.

\subsection{Getting started}

If you have not used matrices within {\REDUCE} before then the following may be
helpful.

\subsubsection*{Creating matrices}

Initialisation of matrices takes the following syntax:

{\tt mat1 := mat((a,b,c),(d,e,f),(g,h,i));}

will produce 

\begin{flushleft}
\begin{math}
mat1 := \begin{pmatrix} a & b & c \\ d & e & f \\ g & h & i \end{pmatrix}
\end{math}
\end{flushleft}

\subsubsection*{Getting at the entries}

The $(i,j)$th entry can be accessed by:

\texttt{mat1(i,j);}

\subsubsection*{Loading the linear\_algebra package}

The package is loaded by:

\texttt{load\_package linalg;}


\subsection{What's available}

\label{linalg:subsec3}

\subsubsection{add\_columns, add\_rows}
\label{linalg:add_columns}
\ttindextype{ADD\_COLUMNS}{operator}

\begin{description}
\item[Syntax:]\mbox{}\\
\texttt{add\_columns($\mathcal{A}$,c1,c2,expr);} 

\begin{tabular}{l l l}
$\mathcal{A}$ & :- & a matrix. \\
$c1,c2$      & :- & positive integers. \\
expr       & :- & a scalar expression. 
\end{tabular}

\item[Synopsis:]\mbox{}\\
\texttt{add\_columns} replaces column $c$2 of $\mathcal{A}$ by\\
$\texttt{expr} * \texttt{column($\mathcal{A}$,c1)} + \texttt{column($\mathcal{A}$,c2)}$.\\
\texttt{add\_rows} performs the equivalent task on the rows of $\mathcal{A}$.

\item[Examples:]
\begin{flalign*}
&\texttt{add\_columns}(\mathcal{A},1,2,x)  = 
  \begin{pmatrix} 1 & x+2 & 3 \\ 4 & 4*x+5 & 6 \\ 7 & 7*x+8 & 9 \end{pmatrix} && \\[2mm]
&\texttt{add\_rows}(\mathcal{A},2,3,5)  = 
  \begin{pmatrix} 1 & 2 & 3 \\ 4 & 5 & 6 \\ 27 & 33 & 39 \end{pmatrix}  &&
\end{flalign*}

\item[Related functions:]\mbox{}\\
\texttt{add\_to\_columns}, \texttt{add\_to\_rows}, 
\texttt{mult\_columns}, \texttt{mult\_rows}.
\end{description}

\subsubsection{add\_rows}
\label{linalg:add_rows}
\ttindextype{ADD\_ROWS}{operator}

See: \texttt{add\_columns}.


\subsubsection{add\_to\_columns, add\_to\_rows}
\label{linalg:add_to_columns}
\ttindextype{ADD\_TO\_COLUMNS}{operator}

\begin{description}
  \item[Syntax:]\mbox{}\\*
\texttt{add\_to\_columns($\mathcal{A}$,column\_list,expr);}\\[2mm]
\begin{tabular}{l l l}
$\mathcal{A}$   &:-& a matrix. \\
column\_list &:-& a positive integer or a list of positive integers. \\
expr        &:-& a scalar expression.
\end{tabular}

\item[Synopsis:]\mbox{}\\*
\texttt{add\_to\_columns} adds expr to each column specified in 
column\_list of $\mathcal{A}$.  

\texttt{add\_to\_rows} performs the equivalent task on the rows of 
$\mathcal{A}$.

\item[Examples:]
\begin{flalign*}  
&\texttt{add\_to\_columns}(\mathcal{A},\{1,2\},10)  =  
\begin{pmatrix} 11 & 12 & 3 \\ 14 & 15 & 6 \\ 17 & 18 & 9 \end{pmatrix} && \\[2mm]
&\texttt{add\_to\_rows}(\mathcal{A},2,-x)  =  
\begin{pmatrix} 1 & 2 & 3 \\ -x+4 & -x+5 & -x+6 \\ 7 & 8 & 9 \end{pmatrix} &&
\end{flalign*}

\item[Related functions:]\mbox{}\\
\texttt{add\_columns}, \texttt{add\_rows}, \texttt{mult\_rows}, 
\texttt{mult\_columns}.
\end{description}

\subsubsection{add\_to\_rows}
\label{linalg:add_to_rows}
\ttindextype{ADD\_TO\_ROWS}{operator}

See: \texttt{add\_to\_columns}.


\subsubsection{augment\_columns, stack\_rows}
\label{linalg:augment_columns}
\ttindextype{AUGMENT\_COLUMNS}{operator}

\begin{description}
\item[Syntax:]\mbox{}\\
 \texttt{augment\_columns($\mathcal{A}$,column\_list);}\\[2mm]
\begin{tabular}{l l l}
$\mathcal{A}$  &:-& a matrix. \\
column\_list &:-&  either a positive integer or a list of positive 
                   integers. 
\end{tabular}

\item[Synopsis:]\mbox{}\\
\texttt{augment\_columns} gets hold of the columns of $\mathcal{A}$ specified 
in column\_list and sticks them together. \\
\texttt{stack\_rows} performs the same task on rows of 
                $\mathcal{A}$.

\item[Examples:]
\begin{flalign*}  
&\texttt{augment\_columns}(\mathcal{A},\{1,2\})  = 
 \begin{pmatrix}{cc} 1 & 2 \\ 4 & 5 \\ 7 & 8  \end{pmatrix} && \\[2mm]
&\texttt{stack\_rows}(\mathcal{A},\{1,3\})  = 
 \begin{pmatrix} 1 & 2 & 3 \\ 7 & 8 & 9 \end{pmatrix} &&
\end{flalign*}

\item[Related functions:]\mbox{}\\
\texttt{get\_columns}, \texttt{get\_rows}, 
\texttt{sub\_matrix}.
\end{description}


\subsubsection{band\_matrix}
\label{linalg:band_matrix}
\ttindextype{BAND\_MATRIX}{operator}

\begin{description}
\item[Syntax:]\mbox{}\\
\texttt{band\_matrix(expr\_list,square\_size);}\\[2mm]
\begin{tabular}{l l p{.72\linewidth}}
expr\_list  &:-& 
either a single scalar expression or a list of an odd number of scalar
expressions. \\
square\_size &:-& a positive integer.
\end{tabular}


\item[Synopsis:]\mbox{}\\
                \texttt{band\_matrix} creates a square matrix of
                dimension \texttt{square\_size}. The diagonal consists of the
                middle expr of the \texttt{expr\_list}. The expressions to the
                left of this fill the required number of sub-diagonals and the
                expressions to the right the super-diagonals.

\item[Examples:]
\begin{flushleft}  
\begin{math}  
\texttt{band\_matrix}(\{x,y,z\},6) = 
 \begin{pmatrix} y & z & 0 & 0 & 0 & 0 \\
                 x & y & z & 0 & 0 & 0 \\ 
                 0 & x & y & z & 0 & 0 \\
                 0 & 0 & x & y & z & 0 \\
                 0 & 0 & 0 & x & y & z \\
                 0 & 0 & 0 & 0 & x & y 
 \end{pmatrix}
\end{math}  
\end{flushleft}

\item[Related functions:]\mbox{}\\
 \texttt{diagonal}.
\end{description}


\subsubsection{block\_matrix}
\label{linalg:block_matrix}
\ttindextype{BLOCK\_MATRIX}{operator}

\begin{description}

\item[Syntax:]\mbox{}\\
\texttt{block\_matrix(r,c,matrix\_list);}\\[2mm]
\begin{tabular}{l l l}
$r,c$          &:-& positive integers. \\
\texttt{matrix\_list} &:-& a list of matrices. 
\end{tabular}

\item[Synopsis:]\mbox{}\\
\texttt{block\_matrix} creates a matrix that consists of $r\times c$ matrices 
filled from the \texttt{matrix\_list} row-wise.

\item[Examples:]
\begin{flalign*}  
&\mathcal{B} = \begin{pmatrix} 1 & 0 \\ 0 & 1 \end{pmatrix}, \,\,
 \mathcal{C} = \begin{pmatrix} 5 \\ 5 \end{pmatrix}, \,\,
 \mathcal{D} = \begin{pmatrix} 22 & 33 \\ 44 & 55 \end{pmatrix} && \\[2mm]
&\texttt{block\_matrix}(2,3,\{\mathcal{B,C,D,D,C,B}\})  = 
 \begin{pmatrix} 1 & 0 & 5 & 22 & 33 \\ 0 & 1 & 5 & 44 & 55 \\
     22 & 33 & 5 & 1 & 0 \\ 44 & 55 & 5 & 0 & 1 \end{pmatrix} &&
\end{flalign*}
\end{description}


\subsubsection{char\_matrix}
\label{linalg:char_matrix}
\ttindextype{CHAR\_MATRIX}{operator}

\begin{description}
\item[Syntax:]\mbox{}\\
\texttt{char\_matrix($\mathcal{A},\lambda$);}\\[2mm]
\begin{tabular}{l l l}
$\mathcal{A}$ &:-& a square matrix. \\
$\lambda$  &:-& a symbol or algebraic expression. 
\end{tabular}

\item[Synopsis:]\mbox{}\\
\texttt{char\_matrix} creates the characteristic matrix $\mathcal{C}$ of 
$\mathcal{A}$.  This is $\mathcal{C} = \lambda \mathcal{I} - \mathcal{A}$. 

\item[Examples:]
\begin{flushleft}  
\begin{math}  
\texttt{char\_matrix}(\mathcal{A},x) = 
\begin{pmatrix} x-1 & -2 & -3 \\ -4 & x-5 & -6 \\ -7 & -8 & x-9 \end{pmatrix}  
\end{math}  
\end{flushleft}

\item[Related functions:]\mbox{}\\
\texttt{char\_poly}. 
\end{description}


\subsubsection{char\_poly}
\label{linalg:char_poly}
\ttindextype{CHAR\_POLY}{operator}

\begin{description}
\item[Syntax:]\mbox{}\\
\texttt{char\_poly($\mathcal{A},\lambda$);}\\[2mm]
\begin{tabular}{l l l}
$\mathcal{A}$ &:-& a square matrix. \\
$\lambda$ &:-& a symbol or algebraic expression.
\end{tabular}

\item[Synopsis:]\mbox{}\\
\texttt{char\_poly} finds the characteristic polynomial of
                $\mathcal{A}$.  

This is the determinant of $\lambda \mathcal{I} - \mathcal{A}$.

\item[Examples:]\mbox{}\\
\texttt{char\_poly($\mathcal{A},x$) $= x^3-15*x^2-18*x$} 

\item[Related functions:]\mbox{}\\
 \texttt{char\_matrix}. 
\end{description}


\subsubsection{cholesky}
\label{linalg:cholesky}
\ttindextype{CHOLESKY}{operator}

\begin{description}
\item[Syntax:]\mbox{}\\
\texttt{cholesky($\mathcal{A}$);}\\[2mm]
\begin{tabular}{l l l}
$\mathcal{A}$ &:-& a positive definite matrix containing numeric entries.
\end{tabular}

\item[Synopsis:]\mbox{}\\
\texttt{cholesky} computes the cholesky decomposition of $\mathcal{A}$.

It returns \{$\mathcal{L},\mathcal{U}$\} where $\mathcal{L}$
is a lower matrix, $\mathcal{U}$ is an upper matrix, \\ $\mathcal{A} = 
\mathcal{LU}$, and $\mathcal{U} = \mathcal{L}^T$.

\item[Examples:]
\begin{flalign*}  
&\mathcal{F} = \begin{pmatrix} 1 & 1 & 0 \\ 1 & 3 & 1 \\ 0 & 1 & 1 \end{pmatrix} &&\\[2mm]
&\f{cholesky}(\mathcal{F})  = 
 \left\{ \begin{pmatrix} 1 & 0 & 0 \\ 1 & \sqrt{2} & 0 \\ 
0 & \frac{1}{\sqrt{2}} & \frac{1}{\sqrt{2}} \end{pmatrix},
  \begin{pmatrix} 1 & 1 & 0 \\ 0 & \sqrt{2} & \frac{1}{\sqrt{2}} \\ 0 
& 0 & \frac{1}{\sqrt{2}} \end{pmatrix} 
\right\} &&
\end{flalign*}

\item[Related functions:]\mbox{}\\
\texttt{lu\_decom}.
\end{description}


\subsubsection{coeff\_matrix}
\label{linalg:coeff_matrix} 
\ttindextype{COEFF\_MATRIX}{operator}

\begin{description}
\item[Syntax:]\mbox{}\\
\texttt{coeff\_matrix(\{\lineqlist{}\});} 
\lazyfootnote{}\\[2mm]
\begin{tabular}{l l p{.435\linewidth}}
\lineqlist  &:-& linear equations. Can be 
of the form \textit{equation $=$ number} or just \textit{equation} which is
equivalent to \textit{equation $=$ 0}.
\end{tabular}

\item[Synopsis:]\mbox{}\\
\texttt{coeff\_matrix} creates the coefficient matrix $\mathcal{C}$ of the linear equations. 
It returns \{$\mathcal{C,X,B}$\} such that $\mathcal{CX} = \mathcal{B}$.

\item[Examples:]
\begin{flalign*}
 & \texttt{coeff\_matrix}(\{x+y+4*z=10,y+x-z=20,x+y+4\}) =  && \\[3mm]
 & \left\{
      \begin{pmatrix} 4 & 1 & 1 \\ -1 & 1 & 1 \\ 0 & 1 & 1 \end{pmatrix},
      \begin{pmatrix} z \\ y \\ x \end{pmatrix},
      \begin{pmatrix} 10 \\ 20 \\ -4 \end{pmatrix}
  \right\} &&
\end{flalign*}
\end{description}

\subsubsection{column\_dim, row\_dim}
\label{linalg:column_dim}
\ttindextype{COLUMN\_DIM}{operator}

\begin{description}
\item[Syntax:]\mbox{}\\
\texttt{column\_dim($\mathcal{A}$);}\\[2mm]
\begin{tabular}{l l l}
$\mathcal{A}$ &:-& a matrix.
\end{tabular}

\item[Synopsis:]\mbox{}\\
\texttt{column\_dim} finds the column dimension of 
                $\mathcal{A}$. \\
\texttt{row\_dim} finds the row dimension of $\mathcal{A}$.

\item[Examples:]\mbox{}\\
\texttt{column\_dim}($\mathcal{A}$) = 3
\end{description}

\subsubsection{companion}
\label{linalg:companion}
\ttindextype{COMPANION}{operator}

\begin{description}
\item[Syntax:]\mbox{}\\
\texttt{companion(poly,x);}\\[2mm]
\begin{tabular}{l l l}
\texttt{poly} &:-& a monic univariate polynomial in $x$. \\
$x$    &:-& the variable.
\end{tabular}

\item[Synopsis:]\mbox{}\\
                \texttt{companion} creates the companion matrix $\mathcal{C}$
                of \texttt{poly}. 

This is the square matrix of dimension $n$, where $n$ is the degree
of \texttt{poly} w.r.t. $x$. The entries of $\mathcal{C}$ are: $\mathcal{C}(i,n)
= -\texttt{coeffn}(\texttt{poly},x,i-1)$ for $i = 1,\ldots, n$,
$\mathcal{C}(i,i-1) = 1$ for $i=2,\ldots,n$ and the rest are $0$.

\item[Examples:]
\begin{flushleft}  
\begin{math}  
\texttt{companion}(x^4+17*x^3-9*x^2+11,x) = 
 \begin{pmatrix} 0 & 0 & 0 & -11 \\ 1 & 0 & 0 & 0 \\ 
0 & 1 & 0 & 9 \\ 0 & 0 & 1 & -17 \end{pmatrix}
\end{math}  
\end{flushleft}

\item[Related functions:]\mbox{}\\
\texttt{find\_companion}.
\end{description}


\subsubsection{copy\_into}
\label{linalg:copy_into}
\ttindextype{COPY\_INTO}{operator}

\begin{description}
\item[Syntax:]\mbox{}\\
\texttt{copy\_into($\mathcal{A,B}$,r,c);}\\[2mm]
\begin{tabular}{l l l}
$\mathcal{A,B}$ &:-& matrices. \\
$r,c$          &:-& positive integers. 
\end{tabular}

\item[Synopsis:]\mbox{}\\
\texttt{copy\_into} copies matrix $\mathcal{A}$ into 
                $\mathcal{B}$ with $\mathcal{A}(1,1)$ at $\mathcal{B}(r,c)$.

\item[Examples:]
\begin{flalign*}
&\mathcal{G} = \begin{pmatrix} 0 & 0 & 0 & 0 \\ 0 & 0 & 0 & 0 \\
0 & 0 & 0 & 0 \\ 0 & 0 & 0 & 0 \end{pmatrix} && \\[2mm]
&\texttt{copy\_into}(\mathcal{A,G},1,2)  = 
 \begin{pmatrix} 0 & 1 & 2 & 3 \\ 0 & 4 & 5 & 6 \\ 0 & 7 & 8 
& 9 \\ 0 & 0 & 0 & 0  \end{pmatrix}&&
\end{flalign*}

\item[Related functions:]\mbox{}\\
\texttt{augment\_columns}, \texttt{extend}, \texttt{matrix\_augment}, 
\texttt{matrix\_stack}, \texttt{stack\_rows}, \texttt{sub\_matrix}.

\end{description}


\subsubsection{diagonal}
\label{linalg:diagonal}
\ttindextype{DIAGONAL}{operator}

\begin{description}
\item[Syntax:]\mbox{}\\
\texttt{diagonal(\{\matlist{}\});}\lazyfootnote{}\\[2mm]
\begin{tabular}{l l p{.58\linewidth}}
\matlist &:-& each can be either a scalar expr or a square matrix. 
\end{tabular}

\item[Synopsis:]\mbox{}\\
\texttt{diagonal} creates a matrix that contains the 
input on the diagonal.

\item[Examples:]
\begin{flalign*}  
& \mathcal{H} = \begin{pmatrix} 66 & 77 \\ 88 & 99 \end{pmatrix} && \\[2mm]
& \texttt{diagonal}(\{\mathcal{A},x,\mathcal{H}\}) = 
 \begin{pmatrix} 1 & 2 & 3 & 0 & 0 & 0 \\ 4 & 5 & 6 & 0 & 0
& 0 \\ 7 & 8 & 9 & 0 & 0 & 0 \\ 0 & 0 & 0 & x & 0 & 0 \\ 0 & 0 & 0 & 0 
& 66 & 77 \\ 0 & 0 & 0 & 0 & 88 & 99 
\end{pmatrix} &&
\end{flalign*}

\item[Related functions:]\mbox{}\\
\texttt{jordan\_block}.
\end{description}


\subsubsection{extend}
\label{linalg:extend}
\ttindextype{EXTEND}{operator}

\begin{description}
\item[Syntax:]\mbox{}\\
\texttt{extend($\mathcal{A}$,r,c,expr);}\\
\begin{tabular}{l l l}
$\mathcal{A}$ &:-& a matrix. \\
$r,c$              &:-& positive integers. \\
\texttt{expr}      &:-& algebraic expression or symbol.
\end{tabular}

\item[Synopsis:]\mbox{}\\
                \texttt{extend} returns a copy of $\mathcal{A}$ that has been 
                extended by $r$ rows and $c$ columns. The new entries are
                made equal to \texttt{expr}.

\item[Examples:]
\begin{flushleft}  
\begin{math}  
\texttt{extend}(\mathcal{A},1,2,x) =
\begin{pmatrix} 1 & 2 & 3 & x & x \\ 4 & 5 & 6 & x & x
\\ 7 & 8 & 9 & x & x \\ x & x & x & x & x 
\end{pmatrix}
\end{math}  
\end{flushleft}

\item[Related functions:]\mbox{}\\
\texttt{copy\_into}, \texttt{matrix\_augment}, 
\texttt{matrix\_stack}, \texttt{remove\_columns}, \texttt{remove\_rows}.

\end{description}


\subsubsection{find\_companion}
\label{linalg:find_companion}
\ttindextype{FIND\_COMPANION}{operator}

\begin{description}
\item[Syntax:]\mbox{}\\
\texttt{find\_companion($\mathcal{A}$,x);}\\[2mm]
\begin{tabular}{l l l}
$\mathcal{A}$ &:-& a matrix. \\
$x$          &:-& the variable.
\end{tabular}

\item[Synopsis:]\mbox{}\\*
  Given a companion matrix, \texttt{find\_companion} finds the polynomial 
from which it was made.

\item[Examples:]
\begin{flalign*}  
& \mathcal{C} = \begin{pmatrix} 0 & 0 & 0 & -11 \\ 1 & 0 & 0 & 0 
\\ 0 & 1 & 0 & 9 \\ 0 & 0 & 1 & -17 \end{pmatrix} && \\[2mm]
& \texttt{find\_companion}(\mathcal{C},x) = x^4+17*x^3-9*x^2+11 &&
\end{flalign*}

\item[Related functions:]\mbox{}\\*
\texttt{companion}.
\end{description}

\subsubsection{get\_columns, get\_rows}
\label{linalg:get_columns}
\ttindextype{GET\_COLUMNS}{operator}

\begin{description}
\item[Syntax:]\mbox{}\\*
\texttt{get\_columns($\mathcal{A}$,column\_list);}\\[2mm]
\begin{tabular}{l l l}
$\mathcal{A}$ &:-& a matrix. \\
$c$          &:-& either a positive integer or a list of positive 
                integers.
\end{tabular}

\item[Synopsis:]\mbox{}\\*
\texttt{get\_columns} removes the columns of $\mathcal{A}$ specified in 
                \texttt{column\_list} and returns them as a list of column 
                matrices. 

 \texttt{get\_rows} performs the same task on the rows of 
                $\mathcal{A}$. 

\item[Examples:]
\begin{flalign*}  
&\texttt{get\_columns}(\mathcal{A},\{1,3\})  =  
\left\{ 
        \begin{pmatrix} 1 \\ 4 \\ 7 \end{pmatrix},
        \begin{pmatrix} 3 \\ 6 \\ 9 \end{pmatrix} 
\right\} && \\[2mm]
&\texttt{get\_rows}(\mathcal{A},2)  = 
\left\{ 
        \begin{pmatrix} 4 & 5 & 6 \end{pmatrix}
\right\} &&
\end{flalign*}

\item[Related functions:]\mbox{}\\*
\texttt{augment\_columns}, \texttt{stack\_rows}, \texttt{sub\_matrix}.
\end{description}


\subsubsection{get\_rows}
\label{linalg:get_rows}
\ttindextype{GET\_ROWS}{operator}
See: \texttt{get\_columns}.


\subsubsection{gram\_schmidt}
\label{linalg:gram_schmidt}
\ttindextype{GRAM\_SCHMIDT}{operator}

\begin{description}
\item[Syntax:]\mbox{}\\*
\texttt{gram\_schmidt(\{\veclist{}\});} \lazyfootnote{}\\[2mm]
\begin{tabular}{l l p{.62\linewidth}}
\veclist &:-& linearly-independent vectors.
                             Each vector must be written as a list, 
                             eg:\{1,0,0\}. 
\end{tabular}

\item[Synopsis:]\mbox{}\\*
\texttt{gram\_schmidt} performs the Gram-Schmidt orthonormalisation on the input vectors. 
It returns a list of orthogonal normalised vectors.

\item[Examples:]\mbox{}\\*
\texttt{gram\_schmidt(\{\{1,0,0\},\{1,1,0\},\{1,1,1\}\})} = 
\{\{1,0,0\},\{0,1,0\},\{0,0,1\}\}\\
\texttt{gram\_schmidt(\{\{1,2\},\{3,4\}\})} $\displaystyle= 
\{\{ \frac{1}{{\sqrt{5}}} , \frac{2}{\sqrt{5}} \},
\{ \frac{2*\sqrt{5}}{5} , \frac{-\sqrt{5}}{5} \}\}$
\end{description}

\subsubsection{hermitian\_tp}
\label{linalg:hermitian_tp}
\ttindextype{HERMITIAN\_TP}{operator}

\begin{description}
\item[Syntax:]\mbox{}\\*
\texttt{hermitian\_tp($\mathcal{A}$);}\\[2mm]
\begin{tabular}{l l l}
$\mathcal{A}$ &:-& a matrix. 
\end{tabular}

\item[Synopsis:]\mbox{}\\*
         \texttt{hermitian\_tp} computes the hermitian transpose of $\mathcal{A}$. 

This is a matrix in which the $(i,j)$th entry is the conjugate of the $(j,i)$th
entry of $\mathcal{A}$.

\item[Examples:]
\begin{flalign*}  
&\mathcal{J} = \begin{pmatrix} i+1 & i+2 & i+3 \\ 4 & 5 & 2 \\ 1 & i & 0 \end{pmatrix}&&\\[2mm]
&\texttt{hermitian\_tp}(\mathcal{J})  = 
\begin{pmatrix} -i+1 & 4 & 1 \\ -i+2 & 5 & -i \\-i+3 & 2 & 0 \end{pmatrix}
\end{flalign*}                   

\item[Related functions:]\mbox{}\\*
\texttt{tp}\footnote{standard reduce call for the 
transpose of a matrix - see section \protect\ref{sec:core-matrix-operators}.}.
\end{description}


\subsubsection{hessian}
\label{linalg:hessian}
\ttindextype{HESSIAN}{operator}

\begin{description}
\item[Syntax:]\mbox{}\\*
\texttt{hessian(expr,variable\_list);}\\[2mm]
\begin{tabular}{l l l}
expr           &:-& a scalar expression. \\
variable\_list &:-& either a single variable or a list of variables.
\end{tabular}

\item[Synopsis:]\mbox{}\\*
                \texttt{hessian} computes the hessian matrix of expr w.r.t.
                the varibles in \texttt{variable\_list}. 

This is an $n\times n$ matrix where $n$ is the number of variables and the
$(i,j)$th entry is \texttt{df(expr,variable\_list(i),variable\_list(j))}.

\item[Examples:]
\begin{flushleft}  
\begin{math}        
\texttt{hessian}(x*y*z+x^2,\{w,x,y,z\}) =
 \begin{pmatrix} 0 & 0 & 0 & 0 \\ 0 & 2 & z & y \\ 0 & z & 0 & x \\ 0 & y & x & 0
  \end{pmatrix}
\end{math}  
\end{flushleft}

\item[Related functions:]\mbox{}\\*
\texttt{df}\footnote{standard reduce call for 
differentiation - see section \protect\ref{sec:DF-operator}.}.
\end{description}


\subsubsection{hilbert}
\label{linalg:hilbert}
\ttindextype{HILBERT}{operator}

\begin{description}
\item[Syntax:]\mbox{}\\*
\texttt{hilbert(square\_size,expr);}\\[2mm]
\begin{tabular}{l l l}
square\_size &:-& a positive integer. \\
expr         &:-& an algebraic expression.
\end{tabular}

\item[Synopsis:]\mbox{}\\*
\texttt{hilbert} computes the square hilbert matrix of 
                dimension \texttt{square\_size}. 

This is the symmetric matrix in which the $(i,j)$th entry is
$1/(i+j-\texttt{expr})$.

\item[Examples:]
\begin{flushleft}  
\begin{math}
\texttt{hilbert}(3,y+x)  =
\begin{pmatrix} \frac{-1}{x+y-2} & \frac{-1}{x+y-3} 
& \frac{-1}{x+y-4} \\ \frac{-1}{x+y-3} & \frac{-1}{x+y-4} & 
\frac{-1}{x+y-5} \\ \frac{-1}{x+y-4} & \frac{-1}{x+y-5} & 
\frac{-1}{x+y-6} 
\end{pmatrix}
\end{math}  
\end{flushleft}
\end{description}


\subsubsection{jacobian}
\label{linalg:mat_jacobian}
\ttindextype{JACOBIAN}{operator}

\begin{description}
\item[Syntax:]\mbox{}\\*
\texttt{mat\_jacobian(expr\_list,variable\_list);}\\[2mm]
\begin{tabular}{l l p{.72\linewidth}}
expr\_list  &:-& either a 
single algebraic expression or a list of algebraic expressions.\\
variable\_list &:-& either a single variable or a list of variables.
\end{tabular}

\item[Synopsis:]\mbox{}\\*
\texttt{mat\_jacobian} computes the jacobian matrix of \texttt{expr\_list} w.r.t. 
\texttt{variable\_list}. 

This is a matrix whose $(i,j)$th entry
is \texttt{df(expr\_list(i),variable\_list(j))}.  The matrix is $n\times m$
where $n$ is the number of variables and $m$ the number of expressions.

\item[Examples:]
\begin{flalign*}
&\texttt{mat\_jacobian(\{$x^4,x*y^2,x*y*z^3$\},\{$w,x,y,z$\})} = && \\[2mm]
& \begin{pmatrix} 0 & 4*x^3 & 0 & 0 \\ 0 & y^2 & 2*x*y & 0 \\ 
0 & y*z^3 & x*z^3 & 3*x*y*z^2 
\end{pmatrix} &&
\end{flalign*}

\item[Related functions:]\mbox{}\\*
\texttt{hessian}, \texttt{df}\footnote{standard reduce call 
for differentiation - see {\REDUCE} User's Manual[2].}.

NOTE: The function \texttt{mat\_jacobian} used to be called just "\texttt{jacobian}"
however us of that name was in conflict with another Reduce package.
\end{description}


\subsubsection{jordan\_block}
\label{linalg:jordan_block}
\ttindextype{JORDAN\_BLOCK}{operator}

\begin{description}
\item[Syntax:]\mbox{}\\*
\texttt{jordan\_block(expr,square\_size);}\\[2mm]
\begin{tabular}{l l l}
expr        &:-& an algebraic expression or symbol. \\
square\_size &:-& a positive integer.
\end{tabular}

\item[Synopsis:]\mbox{}\\*
\texttt{jordan\_block} computes the square jordan block matrix $\mathcal{J}$
                of dimension \texttt{square\_size}.

The entries of $\mathcal{J}$ are: $\mathcal{J}(i,i) = \texttt{expr}$ for
                $i=1,\ldots,n$, $\mathcal{J}(i,i+1) = 1$ for $i=1,\ldots,n-1$,
                and all other entries are 0.

\item[Examples:]\nopagebreak
\begin{flushleft}  
\begin{math}        
\texttt{jordan\_block(x,5)}  = 
 \begin{pmatrix} x & 1 & 0 & 0 & 0 \\ 0 & x & 1 & 0 & 0 \\ 0 
& 0 & x & 1 & 0 \\ 0 & 0 & 0 & x & 1 \\ 0 & 0 & 0 & 0 & x
 \end{pmatrix}
\end{math}  
\end{flushleft}

\item[Related functions:]\mbox{}\\*
\texttt{diagonal}, \texttt{companion}.
\end{description}


\subsubsection{lu\_decom}
\label{linalg:lu_decom}
\ttindextype{LU\_DECOM}{operator}

\begin{description}
\item[Syntax:]\mbox{}\\*
\texttt{lu\_decom($\mathcal{A}$);}\\[2mm]
\begin{tabular}{l l l}
$\mathcal{A}$ &:-& \parbox[t]{.848\linewidth}{a matrix containing either 
numeric entries or imaginary entries with numeric coefficients.}
\end{tabular}

\item[Synopsis:]\mbox{}\\*
  \texttt{lu\_decom} performs LU decomposition on $\mathcal{A}$, ie: it returns
  \{$\mathcal{L,U}$\} where $\mathcal{L}$ is a lower diagonal matrix, $\mathcal{U}$ an
  upper diagonal matrix and $\mathcal{A} = \mathcal{LU}$.

\textbf{Caution:}
  The algorithm used can swap the rows of $\mathcal{A}$ during the calculation. This
  means that $\mathcal{LU}$ does not equal $\mathcal{A}$ but a row equivalent of it. Due
  to this, \texttt{lu\_decom} returns \{$\mathcal{L,U}$,vec\}. The call {\tt
  convert($\mathcal{A}$,vec)} will return the matrix that has been decomposed, ie:
  $\mathcal{LU} = $ \texttt{convert($\mathcal{A}$,vec)}.

\item[Examples:]
\begin{flushleft}
\begin{math}  
\mathcal{K} = \begin{pmatrix} 1 & 3 & 5 \\ -4 & 3 & 7 \\ 8 & 6 & 4 \end{pmatrix}
\end{math}  
\end{flushleft}

\begin{align*}  
\texttt{lu} := \texttt{lu\_decom}(\mathcal{K}) = &
\left\{ 
        \begin{pmatrix} 8 & 0 & 0 \\ -4 & 6 & 0 \\ 1 & 2.25 & 1.125 1 \end{pmatrix}, 
        \begin{pmatrix} 1 & 0.75 & 0.5 \\ 0 & 1 & 1.5 \\ 0 & 0 & 1 \end{pmatrix}, 
	[\; 3 \; 2 \; 3 \; ]
\right\} 
\end{align*}  

\begin{flalign*}
\texttt{first lu * second lu} & = 
        \begin{pmatrix} 8 & 6 & 4 \\ -4 & 3 & 7 \\ 1 & 3 & 5 \end{pmatrix} && \\[2mm]
\texttt{convert($\mathcal{K}$,third lu)} & = 
        \begin{pmatrix} 8 & 6 & 4 \\ -4 & 3 & 7 \\ 1 & 3 & 5 \end{pmatrix} &&
\end{flalign*}  

\begin{flushleft}
\begin{math}  
\mathcal{P} = \begin{pmatrix} i+1 & i+2 & i+3 \\ 4 & 5 & 2 \\ 1 & i & 0 \end{pmatrix}
\end{math}  
\end{flushleft}

\begin{align*}
\texttt{lu} := \texttt{lu\_decom}(\mathcal{P}) = &
\left\{ 
        \begin{pmatrix} 1 & 0 & 0 \\ 4 & -4*i+5 & 0 \\ i+1 & 
3 & 0.41463*i+2.26829 \end{pmatrix}, \right. \nonumber \\ & 
\left. \: \; \,  \begin{pmatrix} 1 & i & 0 \\ 0 & 1 & 
0.19512*i+0.24390 \\ 0 & 0 & 1 \end{pmatrix}, \,\, 
[\; 3 \; 2 \; 3 \;] \,\,
\right\} \nonumber
\end{align*}

\begin{flalign*}  
\texttt{first lu * second lu} & = 
        \begin{pmatrix} 1 & i & 0 \\ 4 & 5 & 2 \\ i+1 & i+2 & i+3 \end{pmatrix} && \\[2mm]
\mathtt{convert(\mathcal{P},third lu}) & = 
        \begin{pmatrix} 1 & i & 0 \\ 4 & 5 & 2 \\ i+1 & i+2 & i+3  \end{pmatrix} &&
\end{flalign*}

\item[Related functions:]\mbox{}\\
\texttt{cholesky}.
\end{description}


\subsubsection{make\_identity}
\label{linalg:make_identity}
\ttindextype{MAKE\_IDENTITY}{operator}

\begin{description}
\item[Syntax:]\mbox{}\\*
\texttt{make\_identity(square\_size);}\\[2mm]
\begin{tabular}{l l l}
square\_size &:-& a positive integer.
\end{tabular}

\item[Synopsis:]\mbox{}\\*
\texttt{make\_identity} creates the identity matrix of 
                dimension square\_size.

\item[Examples:]
\begin{flushleft}  
\begin{math}  
\texttt{make\_identity}(4)  =  
        \begin{pmatrix} 1 & 0 & 0 & 0 \\ 0 & 1 & 0 & 0 \\
                        0 & 0 & 1 & 0 \\ 0 & 0 & 0 & 1
 \end{pmatrix}
\end{math}  
\end{flushleft}

\item[Related functions:]\mbox{}\\*
\texttt{diagonal}.
\end{description}


\subsubsection{matrix\_augment, matrix\_stack}
\label{linalg:matrix_augment}
\ttindextype{MATRIX\_AUGMENT}{operator}

\begin{description}
\item[Syntax:]\mbox{}\\*
\texttt{matrix\_augment(\{\matlist\});}\lazyfootnote{}\\[2mm]
\begin{tabular}{l l l}
\matlist &:-& matrices.
\end{tabular}

\item[Synopsis:]\mbox{}\\*
\texttt{matrix\_augment} sticks the matrices in 
                  \texttt{matrix\_list} together horizontally. 

\texttt{matrix\_stack} sticks the matrices in \texttt{matrix\_list}
                together vertically.

\item[Examples:]
\begin{flalign*}  
&\texttt{matrix\_augment}(\{\mathcal{A,A}\})  = 
        \begin{pmatrix} 1 & 2 & 3 & 1 & 2 & 3 \\ 4 & 4 & 6 
& 4 & 5 & 6 \\ 7 & 8 & 9 & 7 & 8 & 9
 \end{pmatrix} &&\\[2mm]
&\texttt{matrix\_stack}(\{\mathcal{A,A}\})  = 
        \begin{pmatrix} 1 & 2 & 3 \\ 4 & 5 & 6 \\ 7 & 8 & 9 
\\ 1 & 2 & 3 \\ 4 & 5 & 6 \\ 7 & 8 & 9 
 \end{pmatrix}  &&
\end{flalign*}

\item[Related functions:]\mbox{}\\
\texttt{augment\_columns}, \texttt{stack\_rows}, 
\texttt{sub\_matrix}.
\end{description}


\subsubsection{matrixp}
\label{linalg:matrixp}
\ttindex{MATRIXP}

\begin{description}
\item[Syntax:]\mbox{}\\*
\texttt{matrixp(test\_input);}\\[2mm]
\begin{tabular}{l l l}
test\_input &:-& anything you like.
\end{tabular}

\item[Synopsis:]\mbox{}\\*
\texttt{matrixp} is a boolean function that returns t if 
                the input is a matrix and nil otherwise.

\item[Examples:]\mbox{}\\*
\texttt{matrixp}($\mathcal{A}$) = t 

\texttt{matrixp}(doodlesackbanana) = nil

\item[Related functions:]\mbox{}\\*
\texttt{squarep}, \texttt{symmetricp}.
\end{description}


\subsubsection{matrix\_stack}
\label{linalg:matrix_stack}
\ttindextype{MATRIX\_STACK}{operator}
See: \texttt{matrix\_augment}.


\subsubsection{minor}
\label{linalg:minor}
\ttindextype{MINOR}{operator}

\begin{description}
\item[Syntax:]\mbox{}\\*
\texttt{minor($\mathcal{A}$,r,c);}\\[2mm]
\begin{tabular}{l l l} 
$\mathcal{A}$ &:-& a matrix. \\
$r,c$        &:-& positive integers.
\end{tabular}

\item[Synopsis:]\mbox{}\\*
                \texttt{minor} computes the $(r,c)$th minor of $\mathcal{A}$.
 
                This is created by removing the $r$th row and the $c$th 
                column from $\mathcal{A}$.
                
\item[Examples:]
\begin{flushleft}  
\begin{math}  
\texttt{minor}(\mathcal{A},1,3)  = 
        \begin{pmatrix} 4 & 5 \\ 7 & 8 \end{pmatrix}
\end{math}  
\end{flushleft}

\item[Related functions:]\mbox{}\\*
\texttt{remove\_columns}, \texttt{remove\_rows}.
\end{description}


\subsubsection{mult\_columns, mult\_rows}
\label{linalg:mult_columns}
\ttindextype{MULT\_COLUMNS}{operator}

\begin{description}
\item[Syntax:]\mbox{}\\*
\texttt{mult\_columns($\mathcal{A}$,column\_list,expr);}\\[2mm]
\begin{tabular}{l l l}
$\mathcal{A}$   &:-& a matrix. \\
column\_list &:-& a positive integer or a list of positive integers. \\
expr        &:-& an algebraic expression.
\end{tabular}

\item[Synopsis:]\mbox{}\\*
\texttt{mult\_columns} returns a copy of $\mathcal{A}$ in which
                the columns specified in column\_list have been 
multiplied by expr. 

\texttt{mult\_rows} performs the same task on the rows of $\mathcal{A}$.

\item[Examples:]
\begin{flalign*}  
&\texttt{mult\_columns}(\mathcal{A},\{1,3\},x)  = 
       \begin{pmatrix} x & 2 & 3*x \\ 4*x & 5 & 6*x \\ 7*x & 8 & 9*x  \end{pmatrix} && \\[2mm]
&\texttt{mult\_rows}(\mathcal{A},2,10)  = 
        \begin{pmatrix} 1 & 2 & 3 \\ 40 & 50 & 60 \\ 7 & 8 & 9 \end{pmatrix} &&
\end{flalign*}

\item[Related functions:]\mbox{}\\*
\texttt{add\_to\_columns}, \texttt{add\_to\_rows}.
\end{description}


\subsubsection{\texttt{mult\_rows}}
\label{linalg:mult_rows}
\ttindextype{MULT\_ROWS}{operator}
See: \texttt{mult\_columns}.


\subsubsection{pivot}
\label{linalg:pivot}
\ttindextype{PIVOT}{operator}

\begin{description}
\item[Syntax:]\mbox{}\\*
\texttt{pivot($\mathcal{A}$,r,c);}\\[2mm]
\begin{tabular}{l l l} 
$\mathcal{A}$ &:-& a matrix. \\
$r,c$        &:-& positive integers such that $\mathcal{A}(r,c) \neq 0$.
\end{tabular}

\item[Synopsis:]\mbox{}\\*
\texttt{pivot} pivots $\mathcal{A}$ about its $(r,c)$th entry.
 
To do this, multiples of the r'th row are added to every
     other row in the matrix. 

This means that the c'th column
                will be 0 except for the (r,c)'th entry. 

\item[Examples:]
\begin{flushleft}  
\begin{math}  
\texttt{pivot}(\mathcal{A},2,3) = 
        \begin{pmatrix} -1 & -0.5 & 0 \\ 4 & 5 & 6 \\ 1 & 0.5 & 0 \end{pmatrix}
\end{math}  
\end{flushleft}

\item[Related functions:]\mbox{}\\
\texttt{rows\_pivot}.
\end{description}


\subsubsection{pseudo\_inverse}
\label{linalg:pseudo_inverse}
\ttindextype{PSEUDO\_INVERSE}{operator}

\begin{description}
\item[Syntax:]\mbox{}\\*
\texttt{pseudo\_inverse($\mathcal{A}$);}\\[2mm]
\begin{tabular}{l l l} 
$\mathcal{A}$ &:-& a matrix containing only real numeric entries.
\end{tabular}

\item[Synopsis:]\mbox{}\\*
\texttt{pseudo\_inverse}, also known as the Moore-Penrose inverse, computes
the pseudo inverse of $\mathcal{A}$. 

Given the singular value decomposition of $\mathcal{A}$, i.e: $\mathcal{A} = 
\mathcal{U} 
\Sigma\mathcal{V}^T$, then the pseudo inverse $\mathcal{A}^{\dagger}$ is defined 
by $\mathcal{A}^{\dagger} = \mathcal{V} \Sigma^{\dagger} \mathcal{U}^{T}$. For the 
diagonal
matrix $\Sigma$, the pseudoinverse $\Sigma^{\dagger}$ is computed by taking the reciprocal
of only the nonzero diagonal elements.

If $\mathcal{A}$ is square and non-singular, then $\mathcal{A}^{\dagger} = \mathcal{A}$.
In general, however,
$\mathcal{A} \mathcal{A}^{\dagger} \mathcal{A} = \mathcal{A}$, and
$\mathcal{A}^{\dagger} \mathcal{A} \mathcal{A}^{\dagger} = \mathcal{A}^{\dagger}$.

Perhaps more importantly, $\mathcal{A}^{\dagger}$ solves the following least-squares
problem: given a rectangular matrix $\mathcal{A}$ and a vector $b$, find the
$x$ minimizing $\|\mathcal{A}x - b\|_2$, 
and which, in addition, has minimum $\ell_{2}$ (euclidean) Norm, $\|x\|_2$.  
This $x$ is $\mathcal{A}^{\dagger} b$.

\item[Examples:]
\begin{flushleft}
\begin{displaymath}
\mathcal{R} = \begin{pmatrix} 1 & 2 & 3 & 4 \\ 9 & 8 & 7 & 6 \end{pmatrix},
\quad
\text{pseudo\_inverse}(\mathcal{R}) =
        \begin{pmatrix} -0.2 & 0.1 \\ -0.05 & 0.05 \\ 0.1 & 0 \\ 0.25 & -0.05 \end{pmatrix} 
\end{displaymath}  
\end{flushleft}

\item[Related functions:]\mbox{}\\*
\texttt{svd}.
\end{description}

\subsubsection{random\_matrix}
\label{linalg:random_matrix}
\ttindextype{RANDOM\_MATRIX}{operator}

\begin{description}
\item[Syntax:]\mbox{}\\*
\texttt{random\_matrix(r,c,limit);}\\[2mm]
\begin{tabular}{l l l} 
$r,c$, limit &:-& positive integers. \\
\end{tabular}

\item[Synopsis:]\mbox{}\\*
\texttt{random\_matrix} creates an $r\times c$ matrix with random entries in the
range $-\text{limit} < \text{entry} < \text{limit}$.

\hypertarget{switch:IMAGINARY}{}
\hypertarget{switch:NOT_NEGATIVE}{}
\hypertarget{switch:ONLY_INTEGER}{}
\hypertarget{switch:SYMMETRIC}{}
\hypertarget{switch:LOWER_MATRIX}{}
\hypertarget{switch:UPPER_MATRIX}{}
\ttindexswitch[LINALG]{IMAGINARY}
\ttindexswitch[LINALG]{NOT\_NEGATIVE}
\ttindexswitch[LINALG]{ONLY\_INTEGER}
\ttindexswitch[LINALG]{SYMMETRIC}
\ttindexswitch[LINALG]{LOWER\_MATRIX}
\ttindexswitch[LINALG]{UPPER\_MATRIX}
\item[Switches:]\mbox{}\\*
\begin{tabular}{l l p{0.685\linewidth}}
\texttt{imaginary}  &:-& if 
on, then matrix entries are $x+iy$ where $-\text{limit} < x,y < \text{limit}$.\\
\texttt{not\_negative} &:-& if on then $0 < 
\text{entry} < \text{limit}$. In the imaginary case we have $0<x,y<\text{limit}$.\\
\texttt{only\_integer} &:-& if on then each 
entry is an integer. In the imaginary case $x,y$ are integers.\\
\texttt{symmetric} &:-& if on then the matrix is symmetric. \\
\texttt{upper\_matrix} &:-& if on then the 
matrix is upper triangular. \\
\texttt{lower\_matrix} &:-& if on then the matrix is lower triangular.
\end{tabular}

\item[Examples:]
\begin{flalign*}  
& \texttt{random\_matrix}(3,3,10)  = 
        \begin{pmatrix} -4.729721 & 6.987047 & 7.521383 \\
- 5.224177 & 5.797709 & - 4.321952 \\
- 9.418455 & - 9.94318 & - 0.730980
 \end{pmatrix} && \\
\intertext{\texttt{on only\_integer, not\_negative, upper\_matrix, imaginary;}}
&\texttt{random\_matrix}(4,4,10) = 
 \begin{pmatrix} 2*i+5 & 3*i+7 & 7*i+3 & 6 \\ 0 & 2*i+5 & 
5*i+1 & 2*i+1 \\ 0 & 0 & 8 & i \\ 0 & 0 & 0& 5*i+9 
\end{pmatrix} &&
\end{flalign*}
\end{description}


\subsubsection{remove\_columns, remove\_rows}
\label{linalg:remove_columns}
\ttindextype{REMOVE\_COLUMNS}{operator}

\begin{description}
\item[Syntax:]\mbox{}\\*
\texttt{remove\_columns($\mathcal{A}$,column\_list);}\\[2mm]
\begin{tabular}{l l l} 
$\mathcal{A}$   &:-& a matrix. \\
column\_list &:-& either a positive integer or a list of 
                  positive integers.
\end{tabular}

\item[Synopsis:]\mbox{}\\*
\texttt{remove\_columns} removes the columns specified in
                column\_list from $\mathcal{A}$. 

\texttt{remove\_rows} performs the same task on the rows 
                of $\mathcal{A}$.

\item[Examples:]
\begin{flalign*}  
\texttt{remove\_columns}(\mathcal{A},2) & =  
        \begin{pmatrix} 1 & 3 \\ 4 & 6 \\ 7 & 9  \end{pmatrix} && \\[2mm]
\texttt{remove\_rows}(\mathcal{A},\{1,3\}) & = 
        \begin{pmatrix} 4 & 5 & 6 \end{pmatrix} &&
\end{flalign*}


\item[Related functions:]\mbox{}\\*
\texttt{minor}.
\end{description}


\subsubsection{remove\_rows}
\label{linalg:remove_rows}
\ttindextype{REMOVE\_ROWS}{operator}
See: \texttt{remove\_columns}.


\subsubsection{row\_dim}
\label{linalg:row_dim}
\ttindextype{ROW\_DIM}{operator}

See: \texttt{column\_dim}.


\subsubsection{rows\_pivot}
\label{linalg:rows_pivot}
\ttindextype{ROWS\_PIVOT}{operator}

\begin{description}
\item[Syntax:]\mbox{}\\*
\texttt{rows\_pivot($\mathcal{A}$,r,c,\{row\_list\});}\\[2mm]
\begin{tabular}{l l l} 
$\mathcal{A}$ &:-& a matrix. \\
r,c        &:-& positive integers such that $\mathcal{A}$(r,c) neq 0.\\
row\_list  &:-& positive integer or a list of positive integers.
\end{tabular}

\item[Synopsis:]\mbox{}\\*
\texttt{rows\_pivot} performs the same task as \texttt{pivot} but applies 
the pivot only to the rows specified in row\_list.

\item[Examples:]
\begin{flalign*}
 &\mathcal{N} = \begin{pmatrix} 1 & 2 & 3 \\ 4 & 5 & 6 \\ 7 & 8 & 
                 9 \\1 & 2 & 3 \\ 4 & 5 & 6 \end{pmatrix} && \\[2mm]
& \texttt{rows\_pivot}(\mathcal{N},2,3,\{4,5\}) & = & \begin{pmatrix}
 1 & 2 & 3 \\ 4 & 5 & 6 \\ 7 & 8 & 9 \\ -0.75 & 0 & 0.75 \\ 
-0.375 & 0 & 0.375 
 \end{pmatrix} &&
\end{flalign*}

\item[Related functions:]\mbox{}\\
\texttt{pivot}.
\end{description}


\subsubsection{simplex}
\label{linalg:simplex}
\ttindextype{SIMPLEX}{operator}

\begin{description}
\item[Syntax:]\mbox{}\\*
\texttt{simplex(max/min,objective function,\{linear 
inequalities\},[\{bounds\}]);}\\[2mm]
\begin{tabular}{l l p{.63\linewidth}} 
  max/min             & :- & either max or min 
    (signifying maximise and minimise). \\
  objective function  & :- & the function you are maximising or 
  minimising. \\
  linear inequalities & :- & the constraint 
    inequalities. Each one must be of the form 
    \textit{sum of variables ($<=,=,>=$) number}.\\
  bounds & :- & bounds on the variables as
    specified for the LP file format. Each bound is of one of the forms $l\leq
    v$, $v\leq u$, or $l\leq v\leq u$, where $v$ is a variable and $l$, $u$ are
    numbers or \texttt{infinity} or \texttt{-infinity}
\end{tabular}

\item[Synopsis:]\mbox{}\\*
\texttt{simplex} applies the revised simplex algorithm to find the 
optimal(either maximum or minimum) value of the objective function 
under the linear inequality constraints.

It returns \{optimal value,\{ values of variables at this optimal\}\}.

\ttindexswitch[LINALG]{FASTSIMPLEX}
The \texttt{\{bounds\}} argument is optional and admissible only when the switch
\texttt{fastsimplex} is on, which is the default.

Without a \texttt{\{bounds\}} argument, the algorithm implies that all the
variables are non-negative.

\item[Examples:]
\begin{verbatim}
simplex(max,x+y,{x>=10,y>=20,x+y<=25});

***** Error in simplex: Problem has no feasible solution.

simplex(max,10x+5y+5.5z,{5x+3z<=200,x+0.1y+0.5z<=12,
        0.1x+0.2y+0.3z<=9, 30x+10y+50z<=1500});

{525.0,{x=40.0,y=25.0,z=0}}
\end{verbatim}
\end{description}


\subsubsection{squarep}
\label{linalg:squarep}
\ttindextype{SQUAREP}{predicate}

\begin{description}
\item[Syntax:]\mbox{}\\*
\texttt{squarep($\mathcal{A}$);}\\[2mm]
\begin{tabular}{l l l} 
$\mathcal{A}$ &:-& a matrix.
\end{tabular}

\item[Synopsis:]\mbox{}\\*
\texttt{squarep} is a boolean function that returns t if 
                the matrix is square and nil otherwise.

\item[Examples:]
\begin{flalign*}
&\mathcal{L} = \begin{pmatrix} 1 & 3 & 5 \end{pmatrix} && \\[2mm]
& \texttt{squarep}(\mathcal{A}) = \texttt{t}&&\\[2mm]
&  \texttt{squarep}(\mathcal{L}) = \texttt{nil} &&
\end{flalign*}

\item[Related functions:]\mbox{}\\*
\texttt{matrixp}, \texttt{symmetricp}.
\end{description}


\subsubsection{stack\_rows}
\label{linalg:stack_rows}
\ttindextype{STACK\_ROWS}{operator}

See: \texttt{augment\_columns}.


\subsubsection{sub\_matrix}
\label{linalg:sub_matrix}
\ttindextype{SUB\_MATRIX}{operator}

\begin{description}
\item[Syntax:]\mbox{}\\*
\texttt{sub\_matrix($\mathcal{A}$,row\_list,column\_list);}\\[2mm]
\begin{tabular}{l l p{.605\linewidth}} 
$\mathcal{A}$              &:-& a matrix. \\
row\_list, column\_list &:-& either a 
positive integer or a list of positive integers.
\end{tabular}

\item[Synopsis:]\mbox{}\\*
\texttt{sub\_matrix} produces the matrix consisting of the
              intersection of the rows specified in row\_list and the 
columns specified in column\_list. 

\item[Examples:]
\begin{flushleft}  
\begin{math}  
\texttt{sub\_matrix}(\mathcal{A},\{1,3\},\{2,3\}) =
        \begin{pmatrix} 2 & 3 \\ 8 & 9 \end{pmatrix}
\end{math}  
\end{flushleft}

\item[Related functions:]\mbox{}\\*
\texttt{augment\_columns}, \texttt{stack\_rows}.
\end{description}


\subsubsection{svd (singular value decomposition)}
\label{linalg:svd}
\ttindextype{SVD}{operator}

\begin{description}
\item[Syntax:]\mbox{}\\*
\texttt{svd($\mathcal{A}$);}\\[2mm]
\begin{tabular}{l l l} 
$\mathcal{A}$ &:-& a matrix containing only real numeric entries.
\end{tabular}

\item[Synopsis:]\mbox{}\\*
\texttt{svd} computes the singular value decomposition of $\mathcal{A}$. 
If $A$
is an $m\times n$ real matrix of (column) rank $r$, \texttt{svd} returns the
3-element list \{$\mathcal{U},\Sigma,\mathcal{V}$\} where $\mathcal{A} =
\mathcal{U} \Sigma \mathcal{V}^T$. 

Let $k=\min(m,n)$.  Then $U$ is $m\times k$,
$V$ is $n\times k$, and and $\Sigma = \mbox{diag}(\sigma_{1}, \ldots ,\sigma_{k})$,
where $\sigma_{i}\ge 0$ are the singular values of $\mathcal{A}$; only $r$ of
these are non-zero.  The singular values are the non-negative square roots of
the eigenvalues of $\mathcal{A}^T \mathcal{A}$. 

$\mathcal{U}$ and $\mathcal{V}$ are such that $\mathcal{UU}^T = \mathcal{VV}^T = 
\mathcal{V}^T \mathcal{V} = \mathcal{I}_k$.

\textbf{Note:} there are a number of different definitions of SVD in the
literature, in some of which $\Sigma$ is square and $U$ and $V$ rectangular, as
here, but in others $U$ and $V$ are square, and $\Sigma$ is rectangular.

\item[Examples:]
\begin{flalign*}
 & \mathcal{Q} = \begin{pmatrix} 1 & 3 \\ -4 & 3 \\ 3 & 6 \end{pmatrix} && \\[2mm]
 & \begin{aligned}
     \mathtt{svd(\mathcal{Q})} = &
       \left\{ 
         \begin{pmatrix} 0.0236042 & 0.419897 \\ -0.969049 & 0.232684 \\ 0.245739 & 0.877237 \end{pmatrix},
         \begin{pmatrix}  4.83288 & 0 \\ 0 & 7.52618 \end{pmatrix}, \right.  \\
         & \left. \: \; \,
           \begin{pmatrix} 0.959473 & 0.281799 \\ - 0.281799 & 0.959473 \end{pmatrix}
       \right\} \\[2mm]
      \mathtt{svd(TP(\mathcal{Q}))} = &
       \left\{ 
         \begin{pmatrix} 0.959473 & 0.281799 \\ - 0.281799 & 0.959473 \end{pmatrix},
         \begin{pmatrix}  4.83288 & 0 \\ 0 & 7.52618 \end{pmatrix}, \right.  \\
         & \left. \: \; \,
         \begin{pmatrix} 0.0236042 & 0.419897 \\ -0.969049 & 0.232684 \\ 0.245739 & 0.877237 \end{pmatrix}
       \right\} 
    \end{aligned}
\end{flalign*}
\end{description}


\subsubsection{swap\_columns, swap\_rows}
\label{linalg:swap_columns} 
\ttindextype{SWAP\_COLUMNS}{operator}

\begin{description}
\item[Syntax:]\mbox{}\\*
\texttt{swap\_columns($\mathcal{A}$,c1,c2);}\\[2mm]
\begin{tabular}{l l l}
$\mathcal{A}$ &:-& a matrix. \\
c1,c1      &:-& positive integers. 
\end{tabular}

\item[Synopsis:]\mbox{}\\*
\texttt{swap\_columns} swaps column c1 of $\mathcal{A}$ with column c2. 

\texttt{swap\_rows} performs the same task on 2 rows of $\mathcal{A}$.

\item[Examples:]
\begin{flushleft}  
\begin{math}  
\texttt{swap\_columns}(\mathcal{A},2,3) = 
        \begin{pmatrix} 1 & 3 & 2 \\ 4 & 6 & 5 \\ 7 & 9 & 8 \end{pmatrix}
\end{math}  
\end{flushleft}

\item[Related functions:]\mbox{}\\*
\texttt{swap\_entries}.
\end{description}


\subsubsection{swap\_entries}
\label{linalg:swap_entries}
\ttindextype{SWAP\_ENTRIES}{operator}

\begin{description}
\item[Syntax:]\mbox{}\\*
\texttt{swap\_entries($\mathcal{A}$,\{r1,c1\},\{r2,c2\});}\\[2mm]
\begin{tabular}{l l l} 
$\mathcal{A}$  &:-& a matrix. \\
r1,c1,r2,c2 &:-& positive integers.
\end{tabular}

\item[Synopsis:]\mbox{}\\*
\texttt{swap\_entries} swaps $\mathcal{A}$(r1,c1) with 
                $\mathcal{A}$(r2,c2).

\item[Examples:]
\begin{flushleft}  
\begin{math}  
\texttt{swap\_entries}(\mathcal{A},\{1,1\},\{3,3\}) =
        \begin{pmatrix} 9 & 2 & 3 \\ 4 & 5 & 6 \\ 7 & 8 & 1 \end{pmatrix}
\end{math}  
\end{flushleft}

\item[Related functions:]\mbox{}\\*
\texttt{swap\_columns}, \texttt{swap\_rows}.
\end{description}


\subsubsection{swap\_rows}
\label{linalg:swap_rows}
\ttindextype{SWAP\_ROWS}{operator}
See: \texttt{swap\_columns}.


\subsubsection{symmetricp}
\label{linalg:symmetricp}
\ttindextype{SYMMETRICP}{predicate}

\begin{description}
\item[Syntax:]\mbox{}\\*
\texttt{symmetricp($\mathcal{A}$);}\\[2mm]
\begin{tabular}{l l l} 
$\mathcal{A}$ &:-& a matrix. 
\end{tabular}

\item[Synopsis:]\mbox{}\\*
\texttt{symmetricp} is a boolean function that returns t if the 
                matrix is symmetric and nil otherwise.

\item[Examples:]
\begin{flalign*}
&\mathcal{M} = \begin{pmatrix} 1 & 2 \\ 2 & 1 \end{pmatrix} && \\[2mm]
& \texttt{symmetricp}(\mathcal{A}) = \texttt{nil} && \\[2mm] 
& \texttt{symmetricp}(\mathcal{M}) = \texttt{t} &&
\end{flalign*}

\item[Related functions:]\mbox{}\\*
\texttt{matrixp}, \texttt{squarep}.
\end{description}


\subsubsection{toeplitz}
\label{linalg:toeplitz}
\ttindextype{TOEPLITZ}{operator}

\begin{description}
\item[Syntax:]\mbox{}\\*
\texttt{toeplitz(\{\exprlist{}\});} \lazyfootnote{}\\[2mm]
\begin{tabular}{l l l}
\exprlist{} &:-& algebraic expressions.
\end{tabular}

\item[Synopsis:]\mbox{}\\*
\texttt{toeplitz} creates the toeplitz matrix from the 
                expression list. 

This is a square symmetric matrix in 
                which the first expression is placed on the diagonal 
                and the i'th expression is placed on the (i-1)'th sub 
                and super diagonals.

It has dimension n where n is the 
                number of expressions.

\item[Examples:]
\begin{flushleft}  
\begin{math}  
\texttt{toeplitz}(\{w,x,y,z\}) =
        \begin{pmatrix} w & x & y & z \\ x & w & x & y \\
      y & x & w & x \\ z & y & x & w
\end{pmatrix}
\end{math}  
\end{flushleft}
\end{description}

\subsubsection{triang\_adjoint}
\label{linalg:triang_adjoint}
\ttindextype{TRIANG\_ADJOINT}{operator}

\begin{description}
\item[Syntax:]\mbox{}\\*
\texttt{triang\_adjoint($\mathcal{A}$);}\\[2mm]
\begin{tabular}{l l l}
$\mathcal{A}$  &:-& a matrix.
\end{tabular}

\item[Synopsis:]

\texttt{triang\_adjoint} computes the triangularizing adjoint $\mathcal{F}$ of
matrix $\mathcal{A}$ due to the algorithm of Arne Storjohann. $\mathcal{F}$ is
lower triangular matrix and the resulting matrix $\mathcal{T}$ of
$\mathcal{F * A = T}$ is upper triangular with the property that the $i$-th
entry in the diagonal of $\mathcal{T}$ is the determinant of the principal
$i$-th submatrix of the matrix $\mathcal{A}$.

\item[Examples:]
\begin{flalign*}
&\texttt{triang\_adjoint}(\mathcal{A}) = 
\begin{pmatrix} 1 & 0 & 0 \\ -4 & 1 & 0 \\ -3 & 6 & -3 \end{pmatrix} && \\[2mm]
&\mathcal{F} * \mathcal{A} = 
\begin{pmatrix} 1 & 2 & 3 \\ 0 & -3 & -6 \\ 0 & 0 & 0 \end{pmatrix} &&
\end{flalign*}
\end{description}

\subsubsection{Vandermonde}
\label{linalg:Vandermonde}
\ttindextype{VANDERMONDE}{operator}

\begin{description}
\item[Syntax:]\mbox{}\\*
\texttt{vandermonde}(\{\exprlist{}\});
%\addtocounter{footnote}{-1}\footnotemark
\lazyfootnote{}\\[2mm]
\begin{tabular}{l l l}
\exprlist{} &:-& algebraic expressions.
\end{tabular}

\item[Synopsis:]\mbox{}\\*
  \texttt{Vandermonde} creates the Vandermonde matrix from the expression list.
  This is the square matrix in which the $(i,j)$th entry is $\text{expr}_i^{(j-1)}$.
  It has dimension $n$, where $n$ is the number of expressions.

\item[Examples:]
\begin{flushleft}  
\begin{math}  
  \texttt{vandermonde}(\{x,2*y,3*z\}) =
  \begin{pmatrix} 1 & x & x^2 \\ 1 & 2*y & 4*y^2 \\ 1 & 3*z & 9*z^2 \end{pmatrix}
\end{math}  
\end{flushleft}
\end{description}

\subsubsection{kronecker\_product}
\ttindextype{KRONECKER\_PRODUCT}{operator}
\label{linalg:kronecker_product}
\begin{description}
\item[Syntax:]\mbox{}\\*
\texttt{kronecker\_product}($M_1,M_2$)\\[2mm]
\begin{tabular}{l l l}
$M_1,M_2$ &:-& Matrices
\end{tabular}

\item[Synopsis:]\mbox{}\\*
\texttt{kronecker\_product} creates a matrix containing the Kronecker product 
(also called \texttt{direct product} or \texttt{tensor product}) of its arguments.

\item[Examples:]
\begin{verbatim}
a1 := mat((1,2),(3,4),(5,6))$
a2 := mat((1,1,1),(2,z,2),(3,3,3))$
kronecker_product(a1,a2);
\end{verbatim}
\begin{flushleft}
\begin{math}
\begin{pmatrix} 1 & 1 & 1 & 2 & 2 & 2 \\
2 &  z & 2 & 4  &2*z &4 \\
3 &  3 & 3 & 6  & 6  &6 \\
3 &  3 & 3 & 4  & 4  &4 \\
6 & 3*z& 6 & 8  &4*z &8 \\
9 &  9 & 9 & 12 &12  &12\\
5 &  5 & 5 & 6  & 6  &6 \\
10 &5*z& 10& 12 &6*z &12 \\ 
15 &15 & 15& 18 &18  &18
\end{pmatrix}
\end{math}
\end{flushleft}
\end{description}

\subsection{Fast Linear Algebra}
\ttindexswitch[LINALG]{FAST\_LA}

By turning the \sw{fast\_la} switch on, the speed of the following 
functions will be increased:

\begin{tabular}{l l l l}
   add\_columns    & add\_rows      & augment\_columns & column\_dim  \\
   copy\_into      & make\_identity & matrix\_augment  & matrix\_stack\\
   minor           & mult\_column   &  mult\_row       & pivot        \\
   remove\_columns & remove\_rows   & rows\_pivot      & squarep      \\
   stack\_rows     & sub\_matrix    & swap\_columns    & swap\_entries\\
   swap\_rows      & symmetricp                                     
\end{tabular}

The increase in speed will be insignificant unless you are making a 
significant number(i.e: thousands) of calls. When using this switch, 
error checking is minimised. This means that illegal input may give
strange error messages. Beware.

%\newpage

\subsection{Acknowledgments}

Many of the ideas for this package came from the Maple[3] Linalg package
[4].

The algorithms for \texttt{cholesky}, \texttt{lu\_decom}, and \texttt{svd} are 
taken from the book Linear Algebra - J.H. Wilkinson \& C. Reinsch[5].

The \texttt{gram\_schmidt} code comes from Karin Gatermann's Symmetry 
package[6] for {\REDUCE}.


\begin{thebibliography}{9}
\bibitem{matt} Matt Rebbeck: NORMFORM: A {\REDUCE} package for the 
computation of various matrix normal forms. ZIB, Berlin. (1993)
\bibitem{Reduce} Anthony C. Hearn: {\REDUCE} User's Manual 3.6.
	RAND (1995)
\bibitem{Maple} Bruce W. Char\ldots [et al.]: Maple (Computer 
        Program). Springer-Verlag (1991)
\bibitem{linalg} Linalg - a linear algebra package for Maple[3].
\bibitem{WiRe} J. H. Wilkinson \& C. Reinsch: Linear Algebra 
(volume II). Springer-Verlag (1971)
\bibitem{gat} Karin Gatermann: Symmetry: A {\REDUCE} package for the 
computation of linear representations of groups. ZIB, Berlin. (1992)
\end{thebibliography}
