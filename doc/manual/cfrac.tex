\section{CONTINUED\_FRACTION Operator}
\index{approximation}\index{rational number}

The operator CONTINUED\_FRACTION approximates the real number 
( \nameref{rational} number, \nameref{rounded} number)
into a continued fraction. CONTINUE_FRACTION has one or
two arguments, the number to be converted and an optional
precision:
\begin{verbatim}
    continued\_fraction(<num>)
 or
    continued\_fraction(<num>,<size>)
\end{verbatim}
The result is a list of two elements: the
first one is the rational value of the approximation, the second one
is the list of terms of the continued fraction which represents the
same value according to the definition \verb&t0 +1/(t1 + 1/(t2 + ...))&.
Precision: the second optional parameter \meta{size} is an upper bound
for the absolute value of the result denominator. If omitted, the
approximation is performed up to the current system precision.

{\tt Examples:}
\begin{verbatim}
continued_fraction pi;

                  ->

  1146408
{---------,{3,7,15,1,292,1,1,1,2,1}}
  364913

continued_fraction(pi,100);

                  ->

  22
{----,{3,7}}
  7


\end{verbatim}
