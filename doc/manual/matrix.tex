\chapter{Matrix Calculations} \index{Matrix calculations}
A very powerful feature of {\REDUCE} is the ease with which matrix
calculations can be performed. To extend our syntax to this class of
calculations we need to add another prefix operator, {\tt MAT},
\ttindex{MAT} and a further
variable and expression type as follows:

\section{MAT Operator}\ttindex{MAT}
\hypertarget{operator:MAT}{}
This prefix operator is used to represent $n\times m$ matrices. {\tt
MAT} has {\em n} arguments interpreted as rows of the matrix, each of
which is a list of {\em m} expressions representing elements in that row.
For example, the matrix
\[ \left( \begin{array}{lcr} a & b & c \\ d & e & f \end{array} \right) \]
would be written as {\tt mat((a,b,c),(d,e,f))}.

Note that the single column matrix
\[ \left( \begin{array}{c} x \\ y \end{array} \right) \]
becomes {\tt mat((x),(y))}.  The inside parentheses are required to
distinguish it from the single row matrix
\[ \left( \begin{array}{lr} x & y \end{array} \right) \]
that would be written as {\tt mat((x,y))}.

\section{Matrix Variables}
\hypertarget{command:MATRIX}{}

An identifier may be declared a matrix variable by the declaration {\tt
MATRIX}.\ttindex{MATRIX}
The size of the matrix may be declared explicitly in the matrix
declaration, or by default in assigning such a variable to a matrix
expression. For example,
\begin{verbatim}
        matrix x(2,1),y(3,4),z;
\end{verbatim}
declares {\tt X} to be a 2 x 1 (column) matrix, {\tt Y} to be a 3 x 4
matrix and {\tt Z} a matrix whose size is to be declared later.

Matrix declarations can appear anywhere in a program. Once a symbol is
declared to name a matrix, it can not also be used to name an array,
operator or a procedure, or used as an ordinary variable. It can however
be redeclared to be a matrix, and its size may be changed at that time.
Note however that matrices once declared are {\em global\/} in scope, and so
can then be referenced anywhere in the program.  In other words, a
declaration within a block (or a procedure) does not limit the scope of
the matrix to that block, nor does the matrix go away on exiting the block
(use {\tt CLEAR} instead for this purpose).  An element of a matrix is
referred to in the expected manner; thus {\tt x(1,1)} gives the first
element of the matrix {\tt X} defined above.  References to elements of a
matrix whose size has not yet been declared leads to an error.  All
elements of a matrix whose size is declared are initialized to 0.  As a
result, a matrix element has an {\em instant evaluation\/}\index{Instant
evaluation} property and cannot stand for itself.  If this is required,
then an operator should be used to name the matrix elements as in:
\begin{verbatim}
        matrix m; operator x;  m := mat((x(1,1),x(1,2));
\end{verbatim}

\section{Matrix Expressions}

These follow the normal rules of matrix algebra as defined by the
following syntax:\ttindex{MAT}
\begin{verbatim}
        <matrix expression> ::=
                  MAT<matrix description>|<matrix variable>|
                  <scalar expression>*<matrix expression>|
                  <matrix expression>*<matrix expression>
                  <matrix expression>+<matrix expression>|
                  <matrix expression>^<integer>|
                  <matrix expression>/<matrix expression>
\end{verbatim}
Sums and products of matrix expressions must be of compatible size;
otherwise an error will result during their evaluation.  Similarly, only
square matrices may be raised to a power.  A negative power is computed as
the inverse of the matrix raised to the corresponding positive power.
{\tt a/b} is interpreted as {\tt a*b\verb|^|(-1)}.

{\it Examples:}

Assuming {\tt X} and {\tt Y} have been declared as matrices, the following
are matrix expressions
\begin{verbatim}
        y
        y^2*x-3*y^(-2)*x
        y + mat((1,a),(b,c))/2
\end{verbatim}
The computation of the quotient of two matrices normally uses a two-step
elimination method due to Bareiss.  An alternative method using Cramer's
method is also available.  This is usually less efficient than the Bareiss
method unless the matrices are large and dense, although we have no solid
statistics on this as yet.  To use Cramer's method instead, the switch
{\tt CRAMER}\ttindex{CRAMER} should be turned on.


\section{Operators with Matrix Arguments}

The operator {\tt LENGTH}\ttindex{LENGTH} applied to a matrix returns a
list of the number of rows and columns in the matrix.  Other operators
useful in matrix calculations are defined in the following subsections.
Attention is also drawn to the LINALG
\extendedmanual{(chapter~\ref{LINALG})} and NORMFORM
\extendedmanual{(chapter~\ref{NORMFORM})} packages.

\subsection{DET Operator}\ttindex{DET}
\hypertarget{operator:DET}{}
Syntax:
\begin{verbatim}
        DET(EXPRN:matrix_expression):algebraic.
\end{verbatim}

The operator {\tt DET} is used to represent the determinant of a square
matrix expression.  E.g.,
\begin{verbatim}
        det(y^2)
\end{verbatim}
is a scalar expression whose value is the determinant of the square of the
matrix {\tt Y}, and
\begin{verbatim}
        det mat((a,b,c),(d,e,f),(g,h,j));
\end{verbatim}
is a scalar expression whose value is the determinant of the matrix
\[ \left( \begin{array}{lcr} a & b & c \\ d & e & f \\ g & h & j
\end{array} \right) \]

Determinant expressions have the {\em instant evaluation\/} property.
\index{Instant evaluation}  In other words, the statement
\begin{verbatim}
        let det mat((a,b),(c,d)) = 2;
\end{verbatim}
sets the {\em value\/} of the determinant to 2, and does not set up a rule
for the determinant itself.

\subsection{MATEIGEN Operator}\ttindex{MATEIGEN}
\hypertarget{operator:MATEIGEN}{}
Syntax:
\begin{verbatim}
        MATEIGEN(EXPRN:matrix_expression,ID):list.
\end{verbatim}

{\tt MATEIGEN} calculates the eigenvalue equation and the corresponding
eigenvectors of a matrix, using the variable {\tt ID} to denote the
eigenvalue.  A square free decomposition of the characteristic polynomial
is carried out.  The result is a list of lists of 3 elements, where the
first element is a square free factor of the characteristic polynomial,
the second its multiplicity and the third the corresponding eigenvector
(as an {\em n} by 1 matrix).  If the square free decomposition was
successful, the product of the first elements in the lists is the minimal
polynomial.  In the case of degeneracy, several eigenvectors can exist for
the same eigenvalue, which manifests itself in the appearance of more than
one arbitrary variable in the eigenvector.  To extract the various parts
of the result use the operations defined on lists.

{\it Example:}
 The command
\begin{verbatim}
        mateigen(mat((2,-1,1),(0,1,1),(-1,1,1)),eta);
\end{verbatim}
gives the output
\begin{verbatim}
        {{ETA - 1,2,

          [ARBCOMPLEX(1)]
          [             ]
          [ARBCOMPLEX(1)]
          [             ]
          [      0      ]

          },

         {ETA - 2,1,

          [      0      ]
          [             ]
          [ARBCOMPLEX(2)]
          [             ]
          [ARBCOMPLEX(2)]

          }}
\end{verbatim}

\subsection{TP Operator}\ttindex{TP}
\hypertarget{operator:TP}
Syntax:
\begin{verbatim}
        TP(EXPRN:matrix_expression):matrix.
\end{verbatim}

This operator takes a single matrix argument and returns its transpose.

\subsection{Trace Operator}\ttindex{TRACE}
\hypertarget{operator:TRACE}
Syntax:
\begin{verbatim}
        TRACE(EXPRN:matrix_expression):algebraic.
\end{verbatim}
The operator {\tt TRACE} is used to represent the trace of a square matrix.

\subsection{Matrix Cofactors}\ttindex{COFACTOR}
\hypertarget{operator:COFACTOR}{}
Syntax:
\begin{verbatim}
  COFACTOR(EXPRN:matrix_expression,ROW:integer,COLUMN:integer):
           algebraic
\end{verbatim}

The operator {\tt COFACTOR} returns the cofactor of the element in row
{\tt ROW} and column {\tt COLUMN} of the matrix {\tt MATRIX}.  Errors occur
if {\tt ROW} or {\tt COLUMN} do not simplify to integer expressions or if
{\tt MATRIX} is not square.

\subsection{NULLSPACE Operator}\ttindex{NULLSPACE}
\hypertarget{operator:NULLSPACE}{}
Syntax:
\begin{verbatim}
        NULLSPACE(EXPRN:matrix_expression):list
\end{verbatim}
{\tt NULLSPACE} calculates for a matrix {\tt A} a list of linear
independent vectors (a basis) whose linear combinations satisfy the
equation $A x = 0$.  The basis is provided in a form such that as many
upper components as possible are isolated.

Note that with {\tt b := nullspace a} the expression {\tt length b} is the
{\em nullity\/} of A, and that {\tt second length a - length b} calculates the
{\em rank\/} of A.  The rank of a matrix expression can also be found more
directly by the {\tt RANK} operator described below.

{\it Example:} The command
\begin{verbatim}
        nullspace mat((1,2,3,4),(5,6,7,8));
\end{verbatim}
   gives the output
 
\begin{verbatim}
        {
         [ 1  ]
         [    ]
         [ 0  ]
         [    ]
         [ - 3]
         [    ]
         [ 2  ]
         ,
         [ 0  ]
         [    ]
         [ 1  ]
         [    ]
         [ - 2]
         [    ]
         [ 1  ]
         }
\end{verbatim}
 
In addition to the {\REDUCE} matrix form, {\tt NULLSPACE} accepts as input a
matrix given as a list of lists, that is interpreted as a row matrix.  If
that form of input is chosen, the vectors in the result will be
represented by lists as well.  This additional input syntax facilitates
the use of {\tt NULLSPACE} in applications different from classical linear
algebra.

\subsection{RANK Operator}\ttindex{RANK}
\hypertarget{operator:RANK}{}
 
Syntax:
\begin{verbatim}
        RANK(EXPRN:matrix_expression):integer
\end{verbatim}
{\tt RANK} calculates the rank of its argument, that, like {\tt NULLSPACE}
can either be a standard matrix expression, or a list of lists, that can
be interpreted either as a row matrix or a set of equations.

{\tt Example:}

\begin{verbatim}
        rank mat((a,b,c),(d,e,f));
\end{verbatim}
returns the value 2.

\section{Matrix Assignments} \index{Matrix assignment}

Matrix expressions may appear in the right-hand side of assignment
statements. If the left-hand side of the assignment, which must be a
variable, has not already been declared a matrix, it is declared by default
to the size of the right-hand side. The variable is then set to the value
of the right-hand side.

Such an assignment may be used very conveniently to find the solution of a
set of linear equations. For example, to find the solution of the
following set of equations
\begin{verbatim}
        a11*x(1) + a12*x(2) = y1
        a21*x(1) + a22*x(2) = y2
\end{verbatim}
we simply write
\begin{verbatim}
        x := 1/mat((a11,a12),(a21,a22))*mat((y1),(y2));
\end{verbatim}

\section{Evaluating Matrix Elements}

Once an element of a matrix has been assigned, it may be referred to in
standard array element notation.  Thus {\tt y(2,1)} refers to the element
in the second row and first column of the matrix {\tt Y}.

