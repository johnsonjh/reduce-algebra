\index{NORMFORM package}

\subsection{Introduction}
%% For MathJax, must be part of a paragraph to avoid extra space:
\ifdefined\VerbMath{\makehashletter
\(
\newcommand{\rank}{\mathop{\mathrm{rank}}}
\newcommand{\f}[1]{\texttt{#1}}
\)%
}\else
\newcommand{\rank}{\mathop{\mathrm{rank}}}
\fi
When are two given matrices similar? Similar matrices have the same
trace, determinant, characteristic polynomial,
and eigenvalues, but the matrices
\[
 \mathcal{U} = \begin{pmatrix} 0 & 1 \\ 0 & 0 \end{pmatrix}
  \quad\text{ and } \quad
 \mathcal{V} = \begin{pmatrix} 0 & 0 \\ 0 & 0 \end{pmatrix}
\]
are the same in all four of the above but are not similar. Otherwise
there could exist a nonsingular $\mathcal{ N} {\in} M_{2}$ (the set of
all $2 \times 2$ matrices) such that $\mathcal{U} = \mathcal{N} \mathcal{V}
\mathcal{N}^{-1} = \mathcal{N} \, \mathit{0} \, \mathcal{N}^{-1} = \mathit{0}$,
which is a contradiction since $\mathcal{U} \neq \mathit{0}$.

Two matrices can look very different but still be similar. One
approach to determining whether two given matrices are similar is to
compute the normal form of them. If both matrices reduce to the same
normal form they must be similar.

{\small NORMFORM} is a package for computing the following normal
forms of matrices:

\begin{verbatim}
 - smithex
 - smithex_int
 - frobenius
 - ratjordan
 - jordansymbolic
 - jordan
\end{verbatim}

The package is loaded by \texttt{load\_package normform;}

By default all calculations are carried out in $\mathbf{Q}$ (the rational
numbers). For \f{smithex}, \f{frobenius}, \f{ratjordan},
\f{jordansymbolic}, and \f{jordan}, this field can be extended.
Details are given in the respective sections.

The \f{frobenius}, \f{ratjordan}, and \f{jordansymbolic} normal
forms can also be computed in a modular base. Again, details are given
in the respective sections.

The algorithms for each routine are contained in the source code.

{\small NORMFORM} has been converted from the normform and Normform
packages written by T.M.L. Mulders and A.H.M. Levelt. These have been
implemented in Maple [4].


\subsection{Smith normal form}
\ttindextype{SMITHEX}{operator}

\begin{description}
\item[Function]\mbox{}\\*
\f{smithex}($\mathcal{A},\, x$) computes the Smith normal form $\mathcal{ S}$
of the matrix $\mathcal{ A}$.

It returns \{$\mathcal{ S}, \mathcal{ P}, \mathcal{ P}^{-1}$\} where $\mathcal{ S},
\mathcal{ P}$, and $\mathcal{ P}^{-1}$ are such that $\mathcal{ P S P}^{-1} =
\mathcal{ A}$.

$\mathcal{ A}$ is a rectangular matrix of univariate polynomials in $x$.

$x$ is the variable name.

\item[Field extensions]\mbox{}\\*
Calculations are performed in $\mathbf{Q}$. To extend this field the
{\small ARNUM} package can be used. For details see subsection \ref{sec:normform-arnum}.

\item[Synopsis:]
\begin{itemize}
\item The Smith normal form $\mathcal{S}$ of an n by m matrix $\mathcal{A}$
with univariate polynomial entries in $x$ over a field $\mathbf{F}$ is
computed. That is, the polynomials are then regarded as elements of the
{\it E}uclidean domain $\mathbf{F}(x)$.

\item The Smith normal form is a diagonal matrix $\mathcal{S}$ where:

  \begin{itemize}
  \item $\rank(\mathcal{A})$ = number of nonzero rows (columns) of
        $\mathcal{S}$.
  \item $\mathcal{S}(i,i)$ is a monic polynomial for $0 < i \leq \rank(\mathcal{A})$.
  \item $\mathcal{S}(i,i)$ divides $\mathcal{S}(i+1,i+1)$ for $0 < i < \rank(\mathcal{A})$.
  \item $\mathcal{ S}(i,i)$ is the greatest common divisor of all $i$ by
        $i$ minors of $\mathcal{A}$.
  \end{itemize}

      Hence, if we have the case that $n = m$, as well as
      $\rank(\mathcal{A}) = n$, then
      \[
      \prod_{i=1}^{n} \mathcal{ S}(i,i) =
      \frac{\det(\mathcal{A})} { \mathop{\mathrm{lcoeff}}(\det(\mathcal{A}),x)}.
      \]

\item The Smith normal form is obtained by doing elementary row and
      column operations. This includes interchanging rows (columns),
      multiplying through a row (column) by $-1$, and adding integral
      multiples of one row (column) to another.

\item Although the rank and determinant can be easily obtained from
      $\mathcal{S}$, this is not an efficient method for computing these
      quantities except that this may yield a partial factorization of
      $\det(\mathcal{A})$ without doing any explicit factorizations.

\end{itemize}

\item[Example:]\mbox{}\\*
\texttt{load\_package normform;}
\[
 \mathcal{A} = \begin{pmatrix} x & x+1 \\ 0 & 3*x^2 \end{pmatrix}
\]
\[
 \mathtt{smithex}(\mathcal{A}, x)  =
  \left\{
    \begin{pmatrix} 1 & 0 \\ 0 & x^3 \end{pmatrix},
    \begin{pmatrix} 1 & 0 \\ 3*x^2 & 1 \end{pmatrix},
    \begin{pmatrix} x & x+1 \\ -3 & -3 \end{pmatrix}
  \right\}
\]
\end{description}

\subsection{smithex\_int}
\ttindextype{SMITHEX\_INT}{operator}

\begin{description}
\item[Function]\mbox{}\\*
Given an $n$ by $m$ rectangular matrix $\mathcal{A}$ that contains
\emph{only} integer entries, \f{smithex\_int}($\mathcal{A}$) computes the
Smith normal form $\mathcal{S}$ of $\mathcal{A}$.

It returns $\{\mathcal{S}, \mathcal{P}, \mathcal{ P}^{-1}\}$ where $\mathcal{S}$,
$\mathcal{P}$, and $\mathcal{ P}^{-1}$ are such that $\mathcal{P S P}^{-1} =
\mathcal{A}$.


\item[Synopsis]

\begin{itemize}
\item The Smith normal form $\mathcal{ S}$ of an $n$ by $m$ matrix
$\mathcal{ A}$ with integer entries is computed.

\item The Smith normal form is a diagonal matrix $\mathcal{ S}$ where:

  \begin{itemize}
  \item $\rank(\mathcal{A})$ = number of nonzero rows (columns) of
        $\mathcal{S}$.
  \item $\mathop{\mathrm{sign}}(\mathcal{S}(i,i)) = 1$ for $0 < i \leq \rank(\mathcal{A})$.
  \item $\mathcal{S}(i,i)$ divides $\mathcal{S}(i+1,i+1)$ for $0< i < \rank(\mathcal{A})$.
  \item $\mathcal{S}(i,i)$ is the greatest common divisor of all $i$ by
        $i$ minors of $\mathcal{A}$.
  \end{itemize}

      Hence, if we have the case that $n = m$, as well as
      rank($\mathcal{A}$) $= n$, then
      \[
        \left|\det(\mathcal{A})\right| = \prod_{i=1}^{n} \mathcal{S}(i,i) .
      \]

\item The Smith normal form is obtained by doing elementary row and
      column operations. This includes interchanging rows (columns),
      multiplying through a row (column) by $-1$, and adding integral
      multiples of one row (column) to another.
\end{itemize}

\item[Example]\mbox{}\\*
\texttt{load\_package normform;}
\[
 \mathcal{A} =
    \begin{pmatrix} 9 & -36 & 30 \\ -36 & 192 & -180 \\ 30 & -180 & 180 \end{pmatrix}
\]
\[
\mathtt{smithex\_int}(\mathcal{A}) =
\left\{ \begin{pmatrix} 3 & 0 & 0 \\ 0 & 12 & 0 \\ 0 & 0 & 60 \end{pmatrix},
        \begin{pmatrix} -17 & -5 & -4 \\ 64 & 19 & 15 \\ -50 & -15 & -12 \end{pmatrix},
        \begin{pmatrix} 1 & -24 & 30 \\ -1 & 25 & -30 \\ 0 & -1 & 1 \end{pmatrix} \right\}
\]
\end{description}

\subsection{frobenius}
\ttindextype{FROBENIUS}{operator}

\begin{description}
\item[Function]\mbox{}\\*
$\mathtt{frobenius}(\mathcal{A})$ computes the Frobenius normal form
$\mathcal{F}$ of the matrix $\mathcal{A}$.

It returns $\{\mathcal{F}, \mathcal{P}, \mathcal{P}^{-1}\}$ where $\mathcal{F}$,
$\mathcal{P}$, and $\mathcal{P}^{-1}$ are such that $\mathcal{P F P}^{-1} =
\mathcal{A}$.

$\mathcal{A}$ is a square matrix.

\item[Field extensions]\mbox{}\\*
Calculations are performed in $\mathcal{Q}$. To extend this field the
{\small ARNUM} package can be used. For details see subsection \ref{sec:normform-arnum}

\item[Modular arithmetic]\mbox{}\\*
%
\f{frobenius} can be calculated in a modular base. For details see
subsection \ref{sec:normform-modular}.

\item[Synopsis]

\begin{itemize}
\item $\mathcal{F}$ has the following structure:
      \[
      \mathcal{F} = \begin{pmatrix} \mathcal{C}{p_{1}} &  &  &
      \\  & \mathcal{C}{p_{2}} &  &  \\  &  & \ddots &  \\  &  &  &
      \mathcal{C}{p_{k}} \end{pmatrix}
      \]
      where the $\mathcal{C}({p_{i}})$'s are companion matrices
      associated with polynomials $p_{1}, p_{2},\ldots, p_{k}$,
      with the property that $p_{i}$ divides
      $p_{i+1}$ for $i =1\ldots k-1$. All unmarked entries are
      zero.

\item The Frobenius normal form defined in this way is unique (ie: if
      we require that $p_{i}$ divides $p_{i+1}$ as above).
\end{itemize}

\item[Example]\mbox{}\\*
\texttt{load\_package normform;}
\[
\mathcal{A} = \begin{pmatrix} \frac{-x^2+y^2+y}{y} &
\frac{-x^2+x+y^2-y}{y} \\[1mm] \frac{-x^2-x+y^2+y}{y} & \frac{-x^2+x+y^2-y}
{y} \end{pmatrix}
\]
\(
\f{frobenius}(\mathcal{A}) =
\)
\[
\left\{
   \begin{pmatrix} 0 & \frac{x*(x^2-x-y^2+y)}{y} \\[1mm]
                   1 & \frac{-2*x^2+x+2*y^2}{y} \end{pmatrix},
   \begin{pmatrix}
     1 & \frac{-x^2+y^2+y}{y} \\[1mm] 0 & \frac{-x^2-x+y^2+y}{y}
   \end{pmatrix},
   \begin{pmatrix} 1 & \frac{-x^2+y^2+y}{x^2+x-y^2-y} \\[1mm]
      0 & \frac{-y}{x^2+x-y^2-y} \end{pmatrix}
 \right\}
\]
\end{description}

\subsection{ratjordan}
\ttindextype{RATJORDAN}{operator}

\begin{description}
\item[Function]\mbox{}\\*
\f{ratjordan}($\mathcal{A}$) computes the rational Jordan normal form
$\mathcal{R}$ of the matrix $\mathcal{A}$.

It returns $\{\mathcal{R}, \mathcal{P}, \mathcal{P}^{-1}\}$ where $\mathcal{R}$,
$\mathcal{P}$, and $\mathcal{P}^{-1}$ are such that $\mathcal{P R P}^{-1} =
\mathcal{A}$.

$\mathcal{A}$ is a square matrix.

\item[Field extensions]\mbox{}\\*
Calculations are performed in $\mathcal{Q}$. To extend this field the
{\small ARNUM} package can be used. For details see subsection \ref{sec:normform-arnum}.

\item[Modular arithmetic]\mbox{}\\*
\f{ratjordan} can be calculated in a modular base. For details see
subsection \ref{sec:normform-modular}.

\item[Synopsis]

\begin{itemize}
\item $\mathcal{R}$ has the following structure:
      \[
      \mathcal{R} = \begin{pmatrix} r_{11} \\  &
      r_{12} \\  &  & \ddots \\  &  &  & r_{21}  \\ &  &
      &  &  r_{22} \\ &  &  &  &  & \ddots \end{pmatrix}
      \]

      The $r_{ij}$'s have the following shape:
      \[
      r_{ij} = \begin{pmatrix} \mathcal{C}(p) &
      \mathcal{I}  &  &  & \\  &  \mathcal{C}(p) & \mathcal{I}  & & \\ &
      & \ddots & \ddots & \\ &  &  &  \mathcal{C}(p) & \mathcal{I} \\ &
      &  &  & \mathcal{C}(p) \end{pmatrix}
      \]

      where there are e${ij}$ times $\mathcal{C}(p)$ blocks
      along the diagonal and $\mathcal{C}(p)$ is the companion
      matrix  associated with the irreducible polynomial $p$. All
      unmarked entries are zero.
\end{itemize}

\item[Example]\mbox{}\\*
\texttt{load\_package normform;}
\[
\mathcal{A} = \begin{pmatrix} x+y & 5 \\ y & x^2 \end{pmatrix}
\]
\(
\f{ratjordan}(\mathcal{A}) =
\)
\[
\left\{
    \begin{pmatrix} 0 & -x^3-x^2*y+5*y \\[1mm] 1 & x^2+x+y \end{pmatrix},
    \begin{pmatrix} 1 & x+y \\[1mm] 0 & y \end{pmatrix},
    \begin{pmatrix} 1 & \frac{-(x+y)}{y} \\[1mm] 0 &  \frac{1}{y} \end{pmatrix}
  \right\}
\]
\end{description}

\subsection{jordansymbolic}
\ttindextype{JORDANSYMBOLIC}{operator}

\begin{description}
\item[Function]\mbox{}\\*
\f{jordansymbolic}$(\mathcal{A})$ computes the Jordan
normal form $\mathcal{J}$ of the matrix $\mathcal{A}$.

It returns $\{\mathcal{J}, \mathcal{L}, \mathcal{P}, \mathcal{P}^{-1}\}$, where
$\mathcal{J}$, $\mathcal{P}$, and $\mathcal{P}^{-1}$ are such that $\mathcal{P J P}^
{-1} = \mathcal{A}$. $\mathcal{L} = \{ ll , \xi \}$, where $\xi$ is
a name and $ll$ is a list of irreducible factors of $p(\xi)$.

$\mathcal{A}$ is a square matrix.

\item[Field extensions]\mbox{}\\*
Calculations are performed in $\mathcal{Q}$. To extend this field the
{\small ARNUM} package can be used. For details see subsection \ref{sec:normform-arnum}.

\item[Modular arithmetic]\mbox{}\\*
{\tt jordansymbolic} can be calculated in a modular base. For details
see subsection \ref{sec:normform-modular}.

\item[Extras]\mbox{}\\*
\hypertarget{switch:LOOKING_GOOD}{}
\ttindexswitch[NORMFORM]{LOOKING\_GOOD}
If using \texttt{xr}, the X interface for \REDUCE, the appearance of the
output can be improved by setting the switch \sw{looking\_good} to on. This
converts all lambda to $\xi$ and improves the indexing, e.g., \texttt{lambda12}
$\Rightarrow \xi_{12}$. The example below shows the
output when this switch is on.

\item[Synopsis]

\begin{itemize}
\item A \emph{Jordan block} ${\jmath}_{k}(\lambda)$ is a $k$ by $k$
      upper triangular matrix of the form:
%
      \[
      {\jmath}_{k}(\lambda) = \begin{pmatrix} \lambda & 1
      &  &  & \\  &  \lambda & 1  & & \\ &
      & \ddots & \ddots & \\ &  &  &  \lambda & 1 \\ &
      &  &  & \lambda \end{pmatrix}
      \]
%
      There are $k-1$ terms ``$+1$'' in the superdiagonal; the scalar
      $\lambda$ appears $k$ times on the main diagonal. All other
      matrix entries are zero, and ${\jmath}_{1}(\lambda) = (\lambda)$.

\item A Jordan matrix $\mathcal{J} \in M_{n}$ (the set of all $n$ by $n$
      matrices) is a direct sum of \emph{jordan blocks}
%
      \[
      \mathcal{J} = \begin{pmatrix} \jmath_{n_1}(\lambda_{1})
      \\  & \jmath_{n_2}(\lambda_{2}) \\ & & \ddots \\ & & &
      \jmath_{n_k}(\lambda_{k}) \end{pmatrix} ,
      n_{1}+n_{2}+\cdots + n_{k} = n
      \]
%
      in which the orders $n_{i}$ may not be distinct and the
      values ${\lambda_{i}}$ need not be distinct.

\item Here ${\lambda}$ is a zero of the characteristic polynomial
      $p$ of $\mathcal{A}$. If $p$ does not split completely,
      symbolic names are chosen for the missing zeroes of $p$.
      If, by some means, one knows such missing zeroes, they can be
      substituted for the symbolic names. For this,
      \f{jordansymbolic} actually returns $\{ \mathcal{J,L,P,P}^{-1} \}$.
      $\mathcal{J}$ is the Jordan normal form of $\mathcal{A}$ (using
      symbolic names if necessary). $\mathcal{L} = \{ {\it ll}, \xi \}$,
      where $\xi$ is a name and ${\it ll}$ is a list of irreducible
      factors of $p(\xi)$. If symbolic names are used then
      ${\xi}_{ij}$ is a zero of $ll_{i}$. $\mathcal{P}$ and
      $\mathcal{P}^{-1}$ are as above.
\end{itemize}

\item[Example]\mbox{}\\*
%
\texttt{load\_package normform;}\\
\texttt{on looking\_good;}
\[
\mathcal{A} = \begin{pmatrix} 1 & y \\ y^2 & 3 \end{pmatrix}
\]
\(
\f{jordansymbolic}(\mathcal{A}) =
\)
\[
 \left\{
     \begin{pmatrix} \xi_{11} & 0 \\ 0 & \xi_{12} \end{pmatrix} ,
     \left\{ \left\{ -y^3+\xi^2-4*\xi+3 \right\}, \xi \right\}, \right.
\]
\[
 \hspace{0.1in} \left.
       \begin{pmatrix} \xi_{11} -3 & \xi_{12} -3 \\[1mm] y^2 & y^2 \end{pmatrix},
       \begin{pmatrix} \frac{\xi_{11} -2} {2*(y^3-1)}
                 & \frac{\xi_{11} + y^3 -1}{2*y^2*(y^3+1)} \\[1mm]
                \frac{\xi_{12} -2}{2*(y^3-1)}
                 & \frac{\xi_{12}+y^3-1}{2*y^2*(y^3+1)}
       \end{pmatrix}
  \right\}
\]
\texttt{solve(-y\^{}3+xi\^{}2-4*xi+3,xi);}
\[
\{ \xi = \sqrt{y^3+1} + 2,\, \xi = -\sqrt{y^3+1}+2 \} \\
\]
\texttt{J = sub(\{xi(1,1)=sqrt(y\^{}3+1)+2, xi(1,2)=-sqrt(y\^{}3+1)+2\},
   \mbox{}\ \ \ \ \ \ \ \ first  jordansymbolic (A))}
%
\[
\mathcal{J} = \begin{pmatrix} \sqrt{y^3+1} + 2 & 0 \\ 0 &
-\sqrt{y^3+1} + 2 \end{pmatrix}
\]
For a similar example ot this in standard {\REDUCE} (ie: not using
\texttt{xr}), see the \texttt{normform.rlg} file.
\end{description}

\subsection{jordan}
\ttindextype{JORDAN}{operator}

\begin{description}
\item[Function]\mbox{}\\*
%
\f{jordan}($\mathcal{A}$) computes the Jordan normal form
$\mathcal{J}$ of the matrix $\mathcal{A}$.

It returns $\{\mathcal{J}, \mathcal{P}, \mathcal{P}^{-1}\}$, where
$\mathcal{J}$, $\mathcal{P}$, and $\mathcal{P}^{-1}$ are such that $\mathcal{P J P}^
{-1} = \mathcal{A}$.

$\mathcal{A}$ is a square matrix.

\item[Field extensions]\mbox{}\\*
%
Calculations are performed in $\mathcal{Q}$. To extend this field the
{\small ARNUM} package can be used. For details see subsection \ref{sec:normform-arnum}.

\item[Note]\mbox{}\\*
\ttindextype{FULLROOTS}{switch}
In certain polynomial cases the switch \sw{fullroots} is turned on to compute the
zeroes. This can lead to the calculation taking a long time, as well as
the output being very large. In this case a message \\
\verb|***** WARNING: fullroots turned on. May take a while.|\\
will be printed. It may be
better to kill the calculation and compute \f{jordansymbolic} instead.

\item[Synopsis]

\begin{itemize}
\item The Jordan normal form $\mathcal{J}$ with entries in an algebraic
      extension of $\mathcal{Q}$ is computed.

\item A \emph{Jordan block} ${\jmath}_{k}(\lambda)$ is a $k$ by $k$
      upper triangular matrix of the form:
%
      \[
      {\jmath}_{k}(\lambda) = \begin{pmatrix} \lambda & 1
      &  &  & \\  &  \lambda & 1  & & \\ &
      & \ddots & \ddots & \\ &  &  &  \lambda & 1 \\ &
      &  &  & \lambda \end{pmatrix}
      \]
%
      There are $k-1$ terms ``$+1$'' in the superdiagonal; the scalar
      $\lambda$ appears $k$ times on the main diagonal. All other
      matrix entries are zero, and ${\jmath}_{1}(\lambda) = (\lambda)$.

\item A Jordan matrix $\mathcal{J} \in M_{n}$ (the set of all $n$ by $n$
      matrices) is a direct sum of {\it jordan blocks}.
%
      \[
      \mathcal{J} = \begin{pmatrix} \jmath_{n_1}(\lambda_{1})
      \\  & \jmath_{n_2}(\lambda_{2}) \\ & & \ddots \\ & & &
      \jmath_{n_k}(\lambda_{k}) \end{pmatrix},
       n_{1}+ n_{2}+\cdots +n_{k} = n
      \]
%
      in which the orders $n_{i}$ may not be distinct and the
      values ${\lambda_{i}}$ need not be distinct.

\item Here ${\lambda}$ is a zero of the characteristic polynomial
      $p$ of $\mathcal{A}$. The zeroes of the characteristic
      polynomial are computed exactly, if possible. Otherwise they are
      approximated by floating point numbers.
\end{itemize}

\item[Example]\mbox{}\\*
%
\texttt{load\_package normform;}
\[
\mathcal{A} = \begin{pmatrix} -9 & -21 & -15 & 4 & 2 & 0 \\
-10 & 21 & -14 & 4 & 2 & 0 \\ -8 & 16 & -11 & 4 & 2 & 0 \\ -6 & 12 & -9
& 3 & 3 & 0 \\ -4 & 8 & -6 & 0 & 5 & 0 \\ -2 & 4 & -3 & 0 & 1 & 3
\end{pmatrix}
\]
\texttt{J = first jordan(A);}
\[
\mathcal{J} = \begin{pmatrix} 3 & 0 & 0 & 0 & 0 & 0 \\ 0 & 3
& 0 & 0 & 0 & 0 \\ 0 & 0 & 1 & 1 & 0 & 0 \\ 0 & 0 & 0 & 1 & 0 & 0 \\
 0 & 0 & 0 & 0 & i+2 & 0 \\ 0 & 0 & 0 & 0 & 0 & -i+2
\end{pmatrix}
\]
\end{description}


\subsection{Algebraic extensions: Using the \textsc{arnum} package}
\label{sec:normform-arnum}

The package is loaded by the command \texttt{load\_package arnum;}. The algebraic
field $\mathbf{Q}$ can now be extended. For example,
\texttt{defpoly sqrt2**2-2;} will extend it to include ${\sqrt{2}}$ (defined here by
\texttt{sqrt2}). The {\small ARNUM} package was written by Eberhard
Schr\"ufer and is described in section \ref{sec:package-arnum}.

\subsubsection{Example}

\texttt{load\_package normform;} \\
\texttt{load\_package arnum;} \\
\texttt{defpoly sqrt2**2-2;} \\
(sqrt2 now changed to ${\sqrt{2}}$ for looks!)
\[
\mathcal{A} = \begin{pmatrix} 4*{\sqrt{2}}-6 & -4*{\sqrt{2}}+7 &
-3*{\sqrt{2}}+6 \\ 3*{\sqrt{2}}-6 & -3*{\sqrt{2}}+7 & -3*{\sqrt{2}}+6
\\ 3*{\sqrt{2}} & 1-3*{\sqrt{2}} & -2*{\sqrt{2}} \end{pmatrix}
\]
\(
\displaystyle
\f{ratjordan}(\mathcal{A}) =
\left\{ \begin{pmatrix} {\sqrt{2}} & 0 & 0 \\ 0 & {\sqrt{2}}
& 0 \\ 0 & 0 & -3*{\sqrt{2}}+1 \end{pmatrix}, \right.
\)
\[
\hspace{0.1in} \left. \begin{pmatrix}
 7*{\sqrt{2}}-6 & \frac{2*{\sqrt{2}}-49}{31} & \frac{-21*{\sqrt{2}}+18}{31} \\[1mm]
 3*{\sqrt{2}}-6 & \frac{21*{\sqrt{2}}-18}{31} & \frac{-21*{\sqrt{2}}+18}{31} \\[1mm]
 3*{\sqrt{2}}+1 & \frac{-3*{\sqrt{2}}+24}{31} & \frac{3*{\sqrt{2}}-24}{31}
 \end{pmatrix}, \right.
\]
\[
\hspace{0.1in} \left. \begin{pmatrix} 0 & {\sqrt{2}}+1 &
1 \\ -1 & 4*{\sqrt{2}}+9 & 4*{\sqrt{2}} \\ -1 & -\frac{1}{6}*{\sqrt{2}}
+1 & 1 \end{pmatrix} \right\}
\]

%\newpage


\subsection{Modular arithmetic}
\label{sec:normform-modular}

\ttindexswitch{MODULAR}
\ttindexswitch{BALANCED\_MOD}
Calculations can be performed in a modular base by setting the switch \sw{modular}
to on. The base can then be set by \texttt{setmod p;} (p a prime). The
normal form will then have entries in $\mathcal{Z}/$p$\mathcal{Z}$.

By also switching on \sw{balanced\_mod} the output will be shown using
a symmetric modular representation.

Information on this modular manipulation can be found in chapter \ref{sec:core-polyrat}.

\subsubsection{Example}

\texttt{load\_package normform;} \\
\texttt{on modular;} \\
\texttt{setmod 23;}
\[
\mathcal{A} = \begin{pmatrix} 10 & 18 \\ 17 & 20 \end{pmatrix}
\]
\(
\f{jordansymbolic}(\mathcal{A}) =
\)
\[
\left\{
 \begin{pmatrix} 18 & 0 \\ 0 & 12 \end{pmatrix},
 \left\{ \left\{ \lambda + 5, \lambda + 11  \right\}, \lambda \right\},
         \begin{pmatrix} 15 & 9 \\ 22 & 1 \end{pmatrix},
         \begin{pmatrix} 1 & 14 \\ 1 & 15 \end{pmatrix} \right\}
\]
\texttt{on balanced\_mod;}

\(
\f{jordansymbolic}(\mathcal{A}) =
\)
\[
\left\{
  \begin{pmatrix} -5 & 0 \\ 0 & -11 \end{pmatrix},
  \left\{ \left\{ \lambda + 5, \lambda + 11  \right\}, \lambda \right\},
  \begin{pmatrix} -8 & 9 \\ -1 & 1 \end{pmatrix},
  \begin{pmatrix} 1 & -9 \\ 1 & -8 \end{pmatrix} \right\}
\]
