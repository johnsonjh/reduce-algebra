\subsection{Introduction}
%% For MathJax, must be part of a paragraph to avoid extra space:
\ifdefined\VerbMath
\(
\newcommand{\MeijerGparams}[2]{\genfrac{}{}{0pt}{}{#1}{#2}}
\)%
\else
\newcommand{\MeijerGparams}[2]{\genfrac{}{}{0pt}{}{#1}{#2}}
\fi
This documentation describes part of \REDUCE's definite
integration package that is able to calculate the definite integrals of
many functions, including several special functions.  There are other
parts of this package, such as Stan Kameny's code for contour integration,
that are not included here.  The integration process described here is not
the more normal approach of initially calculating the indefinite integral,
but is instead the rather unusual idea of representing each function as a
Meijer G-function (a formal definition of the Meijer G-function can be
found in \cite {Prudnikov:90c}), and then calculating the integral by using
the following Meijer G integration formula.

\begin{equation}
\int_{0}^{\infty} x^{\alpha-1} G^{s t}_{u v}
\left( \sigma x \  \Bigg\vert \  \MeijerGparams{( c_u)}{(d_v)} \right)
G^{m n}_{p q} \left( \omega x^{l/k} \  \Bigg\vert \ \MeijerGparams{(a_p)}{(b_q)}
\right) dx = k G^{i j}_{k l} \left( \xi \ \Bigg\vert \
\MeijerGparams{(g_k)}{(h_l)} \right)  \hspace{5mm} \label{defint:eq1}
\end{equation}

The resulting Meijer G-function is then retransformed, either directly
or via a hypergeometric function simplification, to give
the answer. A more detailed account of this theory can be found in
\cite {Adamchik90}.

\subsection{Integration between zero and infinity}

As an example, if one wishes to calculate the following integral

\[
\int_{0}^{\infty} x^{-1} e^{-x} \sin(x) \, dx
\]

then initially the  correct Meijer G-functions are found, via a
pattern matching
process, and are substituted into eq.~\ref{defint:eq1} to give

\[
\sqrt{\pi} \int_{0}^{\infty} x^{-1} G^{1 0}_{0 1} \left(x
\ \Bigg\vert \
\MeijerGparams{.}{0}\right) G^{1 0}_{0 2}\left(\frac{x^{2}}{4}
\ \Bigg\vert \ \MeijerGparams{. \; . }{\frac{1}{2} \; 0} \right) dx
\]

The cases for validity of the integral are then checked. If these
are found to be satisfactory then the formula is calculated and we
obtain the following Meijer G-function

\[
G^{1 2}_{2 2} \left(1 \ \Bigg\vert \ \MeijerGparams{\frac{1}{2} \; 1}{\frac{1}{2} \; 0} \right)
\]

This is reduced to the following hypergeometric function

\[
_2F_1 (\frac{1}{2},1;\frac{3}{2};-1)
\]

which is then calculated to give the correct answer of

\[
\frac{\pi}{4}
\]

The above formula (1) is also true for the integration of a single
Meijer G-function by replacing the second Meijer G-function
with a trivial Meijer G-function.

A list of numerous particular Meijer G-functions is available in
\cite {Prudnikov:90c}.

\subsection{Integration over other ranges}

Although the description so far has been limited to the computation of
definite integrals between 0 and infinity, it can also be extended to
calculate integrals between 0 and some specific upper bound, and
by further extension, integrals between any two bounds.  One approach is
to use the Heaviside function, i.e.

\[
\int_{0}^{\infty} x^{2} e^{-x} H(1-x)\,dx = \int_{0}^{1} x^{2} e^{-x}dx
\]

Another approach, again not involving the normal indefinite integration
process, again uses Meijer G-functions, this time by means of the
following formula

\begin{equation}
\int_{0}^{y} x^{\alpha-1} G^{m n}_{p q}
\left( \sigma x \  \Bigg\vert \ \MeijerGparams{(a_u)}{(b_v)} \right) dx=%
y^{\alpha}\,G^{m \; n+1}_{p+1 \; q+1} \left( \sigma y \  \Bigg\vert \
\MeijerGparams{( a_1..a_n,1-\alpha,a_{n+1}..a_p)}{(b_1..b_m,-\alpha,b_{m+1}..b_q)} \right) \label{defint:eq2}
\end{equation}

For a more detailed look at the theory behind this see
\cite{Adamchik90}.

For example, if one wishes to calculate the following integral

\[
\int_{0}^{y} \sin(2 \sqrt{x}) \, dx
\]

then initially the correct Meijer G-function is found, by a pattern
matching process, and is substituted
into eq.~\ref{defint:eq2} to give

\[
\int_{0}^{y} G^{1 0}_{0 2}\left(x
\ \Bigg\vert \ \MeijerGparams{. \; .}{\frac{1}{2} \; 0} \right) dx
\]

which then in turn gives

\[
y \; G^{1 1}_{1 3}\left(y \ \Bigg\vert \ \MeijerGparams{0}{\frac{1}{2} -\!1 \; 0} \right) dx
\]

and returns the result

\[
\frac{\sqrt{\pi} \, J_{3/2}(2 \, \sqrt{\,y}) \, y}{y^{1/4}}
\]

\subsection{Using the definite integration package}
To use this package, you must first load it by the command
\begin{verbatim}
load_package defint;
\end{verbatim}
Definite integration is then possible using the \verb+int+
command with the syntax:
\begin{verbatim}
   INT(EXPRN:algebraic,VAR:kernel,LOW:algebraic,UP:algebraic)
       :algebraic.
\end{verbatim}
where LOW and UP are the lower and upper bounds respectively for
the definite integration of EXPRN with respect to VAR.

\subsubsection{Examples}

\[
\int_{0}^{\infty} e^{-x} dx
\]


\begin{verbatim}
 int(e^(-x),x,0,infinity);

 1
\end{verbatim}

\[
\int_{0}^{\infty} x \sin(1/x) \, dx
\]

\begin{verbatim}
 int(x*sin(1/x),x,0,infinity);

           1
INT(X*SIN(---),X,0,INFINITY)
           X
\end{verbatim}

\[
\int_{0}^{\infty} x^2 \cos(x) \, e^{-2x} dx
\]

\begin{verbatim}
 int(x^2*cos(x)*e^(-2*x),x,0,infinity);

   4
 -----
  125
\end{verbatim}

\[
\int_{0}^{\infty} x e^{-1/2x} H(1-x) \,dx = \int_{0}^{1} x e^{-1/2x} dx
\]

\begin{verbatim}
 int(x*e^(-1/2x)*Heaviside(1-x),x,0,infinity);

  2*(2*SQRT(E) - 3)
 -------------------
       SQRT(E)
\end{verbatim}

\[
\int_{0}^{1} x \log(1+x) \,dx
\]

\begin{verbatim}
 int(x*log(1+x),x,0,1);

  1
 ---
  4
\end{verbatim}

\[
\int_{0}^{y} \cos(2x) \,dx
\]

\begin{verbatim}
 int(cos(2x),x,y,2y);

 SIN(4*Y) - SIN(2*Y)
---------------------
          2
\end{verbatim}


\subsection{Integral Transforms}

A useful application of the definite integration package is in the
calculation of various integral transforms. The transforms
available are as follows:

\begin{itemize}
\item Laplace transform
\item Hankel transform
\item Y-transform
\item K-transform
\item StruveH transform
\item Fourier sine transform
\item Fourier cosine transform
\end{itemize}

\subsubsection{Laplace transform}

The Laplace transform

\[
f(s) = \mathcal{L}\{\mathrm{F}(t)\} = \int_{0}^{\infty} e^{-st}F(t)\,dt
\]

can be calculated by using the \verb+laplace_transform+ command.

This requires as parameters

\begin{itemize}
\item the function to be integrated
\item the integration variable.
\end{itemize}

For example

\[
\mathcal{L}\{e^{-at}\}
\]

is entered as

\begin{verbatim}
        laplace_transform(e^(-a*x),x);
\end{verbatim}

and returns the result

\[
\frac{1}{s+a}
\]

\subsubsection{Hankel transform}

The Hankel transform

\[
f(\omega) = \int_{0}^{\infty} F(t) \,J_{\nu}(2\sqrt{\omega t}) \,dt
\]

can be calculated by using the \verb+hankel_transform+ command e.g.

\begin{verbatim}
        hankel_transform(f(x),x);
\end{verbatim}

This is used in the same way as the \verb+laplace_transform+ command.

\subsubsection{Y-transform}

The Y-transform

\[
f(\omega) = \int_{0}^{\infty} F(t) \,Y_{\nu}(2\sqrt{\omega t}) \,dt
\]

can be calculated by using the \verb+Y_transform+ command e.g.

\begin{verbatim}
        Y_transform(f(x),x);
\end{verbatim}

This is used in the same way as the \verb+laplace_transform+ command.

\subsubsection{K-transform}

The K-transform

\[
f(\omega) = \int_{0}^{\infty} F(t) \,K_{\nu}(2\sqrt{\omega t}) \,dt
\]

can be calculated by using the \verb+K_transform+ command e.g.

\begin{verbatim}
        K_transform(f(x),x);
\end{verbatim}

This is used in the same way as the \verb+laplace_transform+ command.

\subsubsection{StruveH transform}

The StruveH transform

\[
f(\omega) = \int_{0}^{\infty} F(t) \,StruveH(\nu,2\sqrt{\omega t}) \,dt
\]

can be calculated by using the \verb+struveh_transform+ command e.g.

\begin{verbatim}
        struveh_transform(f(x),x);
\end{verbatim}

This is used in the same way as the \verb+laplace_transform+ command.

\subsubsection{Fourier sine transform}

The Fourier sine transform

\[
f(s) = \int_{0}^{\infty} F(t) \sin(st) \,dt
\]

can be calculated by using the \verb+fourier_sin+ command e.g.
\begin{verbatim}
        fourier_sin(f(x),x);
\end{verbatim}

This is used in the same way as the \verb+laplace_transform+ command.

\subsubsection{Fourier cosine transform}

The Fourier cosine transform

\[
f(s) = \int_{0}^{\infty} F(t) \cos(st) \,dt
\]

can be calculated by using the \verb+fourier_cos+ command e.g.
\begin{verbatim}
        fourier_cos(f(x),x);
\end{verbatim}

This is used in the same way as the \verb+laplace_transform+ command.

\subsection{Additional Meijer G-function Definitions}

The relevant Meijer G representation for any function is found by a
pattern-matching process which is carried out on a list of Meijer
G-function definitions. This list, although extensive, can never hope
to be complete and therefore the user may wish to add more definitions.
Definitions can be added by adding the following lines:

\begin{verbatim}
  defint_choose(f(~x),~var => f1(n,x);

  symbolic putv(mellin!-transforms!*,n,'
                (() (m n p q) (ai) (bj) (C) (var)));

\end{verbatim} 
     where f(x) is the new function, i = 1..p, j=1..q, C = a constant,
%where i = 1..p, j=1..q, C = a constant,
     var = variable, n = an indexing number.

For example when considering $\cos(x)$ we have

{\it Meijer G representation --}
\[
\sqrt{\pi} \,G^{1 0}_{0 2}\left(\frac{x^{2}}{4} \ \Bigg\vert
\ \MeijerGparams{ . \; . }{ 0 \; \frac{1}{2}} \right) dx
\]

{\it Internal definite integration package representation --}
\begin{verbatim}
  defint_choose(cos(~x),~var)   => f1(3,x);
\end{verbatim}

where 3 is the indexing number corresponding to the 3
in the following formula

\begin{verbatim}
  symbolic putv(mellin!-transforms!*,3,'
                (() (1 0 0 2) () (nil (quotient 1 2))
                (sqrt pi) (quotient (expt x 2) 4)));
\end{verbatim} 

or the more interesting example of $J_{n}(x)$:

{\it Meijer G representation --}
\[
G^{1 0}_{0 2} \left(\frac{x^{2}}{4} \ \Bigg\vert
\ \MeijerGparams{. \; .}{\frac{n}{2} \; {\frac{-n}{2}}} \right) dx
\]

{\it Internal definite integration package representation --}

\begin{verbatim}
  defint_choose(besselj(~n,~x),~var) => f1(50,x,n);

  symbolic putv(mellin!-transforms!*,50,'
                ((n) (1 0 0 2) () ((quotient n 2)
                                   (minus quotient n 2)) 1
                                   (quotient (expt x 2) 4)));
\end{verbatim} 

\subsection{The print\_conditions function}

The required conditions for the validity of the transform integrals
can be viewed using the following command:

\begin{verbatim}
        print_conditions().
\end{verbatim}

For example after calculating the following laplace transform

\begin{verbatim}
        laplace_transform(x^k,x);
\end{verbatim}

using the \verb+print_conditions+ command would produce

\begin{verbatim}
repart(sum(ai) - sum(bj)) + 1/2 (q + 1 - p)>(q - p) repart(s)

 and ( - min(repart(bj))<repart(s))<1 - max(repart(ai))

 or mod(arg(eta))=pi*delta

 or ( - min(repart(bj))<repart(s))<1 - max(repart(ai))

 or mod(arg(eta))<pi*delta
\end{verbatim}

where
\[
\begin{array}{l}
delta = s+t-\frac{u-v}{2}\\
eta = 1-\alpha(v-u)-\mu-\rho\\
\mu = \sum_{j=1}^{q} b_{j} - \sum_{i=1}^{p} a_{i} + \frac{p-q}{2} + 1\\
\rho = \sum_{j=1}^{v} d{j} - \sum_{i=1}^{u} c_{i} + \frac{u-v}{2} + 1\\
s,t,u,v,p,q,\alpha \; {\rm as \; in \; (1)}
\end{array}
\]


\subsection{Tracing}
\hypertarget{switch:TRDEFINT}{}

A new switch \sw{TRDEFINT} can be set to ON to print information about intermediate steps
of the calculation.

\subsection{Acknowledgements}
I would like to thank Victor Adamchik whose implementation of the
definite integration package for {\REDUCE} is vital to this
interface.
