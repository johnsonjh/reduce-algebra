

%\newcommand{\xr}{\texttt{XR}}

%\setlength{\oddsidemargin}{5mm}
%\setlength{\evensidemargin}{-5mm}
%\setlength{\textwidth}{159.2mm}
%\setlength{\textheight}{235mm}
%\addtolength{\topmargin}{-18mm}

\iffalse
\newpage
  Gnuplot
\newpage
\title{GNUPLOT Interface for REDUCE\\Version 4}
\author{Herbert Melenk \\ 
Konrad--Zuse--Zentrum f\"ur Informationstechnik Berlin \\
E--mail: Melenk@zib.de}
\maketitle

\fi

\index{GNUPLOT package}
%\markboth{APPENDIX}{GNUPLOT Interface for REDUCE}

%\section{APPENDIX: GNUPLOT Interface for REDUCE}

\subsection{Introduction}

The GNUPLOT system provides easy to use graphics output 
for curves or surfaces which are defined by  
formulas and/or data sets. GNUPLOT supports 
a variety of output devices such as
\verb+VGA screen+, \verb+postscript+, \verb+pic+ \TeX,
\verb+MS Windows+.
The {\REDUCE} GNUPLOT package lets one use the GNUPLOT
graphical output directly from inside {\REDUCE}, either for
the interactive display of curves/surfaces or for the production
of pictures on paper. 

{\REDUCE} supports GNUPLOT 3.4 (or higher). 
For  DOS, Windows, Windows 95, Windows NT, OS/2
and Unix versions of {\REDUCE}
GNUPLOT binaries are delivered together with {\REDUCE}
\footnote{The GNUPLOT developers have agreed that GNUPLOT
binaries can be distributed together with {\REDUCE}.
As GNUPLOT is a package distributed without cost,
the GNUPLOT support of {\REDUCE} also is an
add-on to the {\REDUCE} kernel system without charge.
We recommend fetching the full GNUPLOT system 
by anonymous FTP from a file server.
} % end of footnote
. However, this is a basic set only. 
If you intend to use more facilities of the GNUPLOT
system you should pick up the full GNUPLOT file tree
from a server, e.g.

\begin{itemize}
\item dartmouth.edu (129.170.16.4)
\item monu1.cc.monash.edu.au (130.194.1.101)
\item irisa.irisa.fr (131.254.2.3)
\end{itemize}


\subsection{Command PLOT}
\ttindex{PLOT}

Under {\REDUCE} GNUPLOT is used as graphical output
server, invoked by the command \verb+PLOT(...)+.
This command can have a variable number of
parameters:

\begin{itemize}
\item A functions to plot; a function can be
  \begin{itemize}
    \item an expression with one unknown, e.g. $u*sin(u)**2$

    \item a list of expressions with one (identical) unknown, 
          e.g. $\{sin(u),cos(u)\}$

    \item an expression with two unknowns, e.g. 
          $u*sin(u)**2+sqrt(v)$

    \item a parametic expression  of the form $point(<u>,<v>)$ or
          $point(<u>,<v>,<w>)$ where $<u>$, $<v>$ and $<w>$ are
          expressions which depend of one or two parameters;
          if there is one parameter, the object describes a curve
          in the plane (only $<u>$ and $<v>$) or in the 3D space;
          if there are two parameters, the object describes a
          surface in 3D. The parameters are treated as independent
          variables. Example:\\
                $ point(sin t,cos t,t/10)$

    \item an equation with a symbol on the left-hand side
         and an expression with one or two unknowns on the
         right-hand side, e.g.\\ $dome=1/(x**2+y**2)$

    \item an equation with an expression on the 
         left--hand side and a zero on right--hand side
         describing implicitly a one dimensional 
         variety in the plane (implicitly given curve), e.g.
            \\ $x{\verb+^+}3 + x*y{\verb+^+}2 -9x = 0$, or a two dimensional
         surface in the 3 dimensional Euclidean space,

    \item an equation with an expression in two variables on the 
         left--hand side and a list of numbers on the 
         right--hand side; the contour lines corresponding
         to the given values are drawn, e.g.
            \\ $x{\verb+^+}3 - y{\verb+^+}2 + x*y= \{-2,-1,0,1,2\}$,

    \item a list of points in 2 or 3 dimensions
         e.g. \\ $\{\{0,0\},\{0,1\},\{1,1\}\}$ representing
         a curve,

    \item a list of lists of points in 2 or 3 dimensions
         e.g.\\ $\{\{\{0,0\},\{0,1\},\{1,1\}\},
                 \{\{0,0\},\{0,1\},\{1,1\}\}\}$
         representing a family of curves.
  \end{itemize}

\hypertarget{reserved:intervalop}
\item A range for a variable; this has the form\\
    $variable=(lower\_bound\, . . \, upper\_bound)$ where
  $lower\_bound$ and $upper\_bound$ must be expressions which
  evaluate to numbers. If no range is specified the
  default ranges for independent variables are $(-10\,\,..\,\,10)$
  and the range for the dependent variable is set to 
  maximum number of the GNUPLOT executable (using double
  floats on most IEEE machines and single floats under DOS).
  Additionally the number of interval subdivisions can be
  assigned as a formal quotient\\
    $variable=(lower\_bound\, . . \, upper\_bound)/<it>$
  where $<it>$ is a positive integer. E.g.
    $(1 .. 5)/30$ means the interval from $1$ to $5$
  subdivided into $30$ pieces of equal size. A subdivision
  parameter overrides the value of the variable $points$
  for this variable.
  


\item A plot option, either as fixed keyword,
  e.g. \verb$hidden3d$ or as equation e.g. \verb$term=pictex$;
  free texts such as titles and labels should be enclosed in
  string quotes.
\end{itemize}
Please note that a blank has to be inserted between a number
and a dot - otherwise the REDUCE translator will be mislead.
 
If a function is given as an equation the left-hand side
  is mainly used as a label for the axis of the dependent variable.

In two dimensions,  \verb+PLOT+ can be called with
more than one explicit function; all curves 
are drawn in one picture. However,
all these must use the same independent variable name.
One of the functions can be a point set or a point set list.
Normally all functions and point sets are plotted by
lines. A point set is drawn by points only if functions and 
the point set are drawn in one picture.

In three dimensions only one surface can be shown per call.
Also an implicilty given curve  must be the sole object for one
picture.

The functional expressions are evaluated in \verb$rounded$ mode.
This is done automatically - it is not necessary to turn
on rounded mode explicitly. 

\newpage

Examples:
\begin{verbatim}
plot(cos x);
plot(s=sin phi,phi=(-3 .. 3));
plot(sin phi,cos phi,phi=(-3 .. 3));
plot (cos sqrt(x**2 + y**2),x=(-3 .. 3),y=(-3 .. 3),hidden3d);
plot {{0,0},{0,1},{1,1},{0,0},{1,0},{0,1},{0.5,1.5},{1,1},{1,0}};

 % parametric: screw

on rounded;
w:=for j:=1:200 collect {1/j*sin j,1/j*cos j,j/200}$
plot w;

 % parametric: globe
dd:=pi/15$
w:=for u:=dd step dd until pi-dd collect
    for v:=0 step dd until 2pi collect
      {sin(u)*cos(v), sin(u)*sin(v), cos(u)}$
plot w;

 % implicit: superposition of polynomials
plot((x^2+y^2-9)*x*y=0);
\end{verbatim}
 
Piecewise defined functions: 
A composed graph can be defined by a rule--based operator.
In that case each rule must contain a clause which restricts
the rule application to numeric arguments, e.g.
\begin{verbatim}
   operator my_step1;
   let {my_step1(~x) => -1 when numberp x and x<-pi/2, 
        my_step1(~x) =>  1 when numberp x and x>pi/2,
        my_step1(~x) => sin x
            when numberp x and -pi/2<=x and x<=pi/2};
   plot(my_step2(x));
\end{verbatim}
Of course, such a rule may call a procedure:
\begin{verbatim}
   procedure my_step3(x);
      if x<-1 then -1 else if x>1 then 1 else x;
   operator my_step2;
   let my_step2(~x) => my_step3(x) when numberp x; 
   plot(my_step2(x));
\end{verbatim}
The direct use of a produre with a numeric $if$ clause
is impossible. 

Plot options:
The following plot options are supported in the \verb+PLOT+ command:

\begin{itemize}
   \item $points=<integer>$: the number of unconditionally computed
       data points; for a grid $points^2$ grid points are used.
     The default value is 20. The value of $points$ is used
     only for variables for which no individual interval
     subdivision has been specified in the range specification.
   \item $refine=<integer>$: the maximum depth of adaptive 
       interval intersections. The default is 8. A value 0 switches
       any refinement off. Note that a high value may increase the
       computing time significantly.
\end{itemize}

The following additional GNUPLOT options are supported in the \verb+PLOT+ command:

\begin{itemize}
\item $title=name$: the title (string) is put on top
     of the picture. 

\item axes labels: $xlabel="text1"$, $ylabel="text2"$, and for
  surfaces $zlabel="text3"$. If omitted the axes are labeled
  by the independent and dependent variable names from the
  expression. Note that the axes names $x$label, $y$label and
  $z$label here are used in the usual sense, $x$ for the 
  horizontal and $y$ for the vertical axis under 2-d and
  $z$ for the perpendicular axis under 3-d -- these names
  do not refer to the variable names used in the expressions.

\begin{verbatim}
          plot(1,X,(4*X**2-1)/2,(X*(12*X**2-5))/3,
               x=(-1 .. 1), ylabel="L(x,n)",
               title="Legendre Polynomials");
\end{verbatim}

\item $terminal=name$: prepare output for device type $name$.
     Every installation uses a default terminal as output
     device; some installations support additional
     devices such as printers; consult the original
     GNUPLOT documentation or the GNUPLOT Help for details.
\item $output="filename"$: redirect the output to a file.

\item $size="s_x,s_y"$: rescale the graph (not the
      window) where $s_x$ and $s_y$ are scaling
     factors for the size in x or y
     direction. Defaults are $s_x=1,x_z=1$.
     Note that scaling factors greater than one
     will often cause the picture to be too big for
     the window.
\begin{verbatim}
      plot(1/(x**2+y**2),x=(0.1 .. 5),
              y=(0.1 .. 5), size="0.7,1");
\end{verbatim}
\item $view="r_x,r_z"$: set viewpoint for 3 dimensions 
     by turning the object around the x or z axis;
     the values are degrees (integers).
Defaults are $r_x=60,r_z=30$.
\begin{verbatim}
      plot(1/(x**2+y**2),x=(0.1 .. 5),
              y=(0.1 .. 5), view="30,130");
\end{verbatim}
\item  $contour$ resp $nocontour$: in 3 dimensions an 
       additional contour map is drawn (default: $nocontour$)
       Note that $contour$ is an option
       which is executed by GNUPLOT by interpolating the precomputed
       function values. If you want to draw contour lines of a 
       delicate formula, you better use the contour form of the
       REDUCE PLOT command.
\item $surface$ resp $nosurface$: in 3 dimensions the
       surface is drawn resp suppressed (default: $surface$).
\item $hidden3d$: hidden line removal in 3 dimensions.
\end{itemize}

\subsection{Paper output}
The following example works for a  postscript printer.
If your printer uses a different communication, please find
the correct setting for the $terminal$ variable in the $Gnuplot$
documentation.

For a postscript printer, add the options $terminal=postscript$
and $output="filename"$
to your plot command, e.g.
\begin{verbatim}
    plot(sin x,x=(0 .. 10),terminal=postscript,output="sin.ps");
\end{verbatim}

\subsection{Mesh generation for implicit curves}

The basic mesh for finding an implicitly given curve,
the $x,y$ plane is subdivided into an initial set of triangles.
Those triangles which have an explicit zero point or which have
two points with different signs are refined by subdivision.
A further refinement is performed for triangles which do not have
exactly two zero neighbours because such places may represent crossings,
bifurcations, turning points or other difficulties.
The initial subdivision and the refinements are controlled by
the option \verb+points+ which is initially set to 20:
the initial grid is refined unconditionally until approximately
 \verb+points+ *  \verb+points+ equally distributed
points in the $x,y$ plane have been generated.

The final mesh can be visualized in the picture by setting
\begin{verbatim}
    on show_grid;
\end{verbatim}

\subsection{Mesh generation for surfaces}

By default the functions are computed  at predefined
mesh points: the ranges are divided by the number
associated with the option \verb$points$ in both 
directions.

For two dimensions the given mesh is adaptively
smoothed when the curves are too coarse, especially 
if singularities are present. On the other hand
refinement can be rather time consuming if used with
complicated expressions. You can control it with the option 
\verb+refine+. At singularities the graph is
interrupted.

In three dimensions no refinement is possible as 
GNUPLOT supports surfaces
only with a fixed regular grid. In the case
of a singularity the near neighborhood is
tested; if a point there allows a function evaluation, its 
clipped value is used instead, otherwise a zero is inserted.

When plotting surfaces in three dimensions you have the
option of hidden line removal. Because of an error in
Gnuplot 3.2 the axes cannot be labeled
correctly when hidden3d is used ; therefore they aren't labelled at all. Hidden line
removal is not available with point lists.


\subsection{GNUPLOT operation}
\ttindex{PLOTRESET}

The command \verb+PLOTRESET;+ deletes the current GNUPLOT output
window. The next call to \verb+PLOT+ will then open a new one.

If GNUPLOT is invoked directly by an output pipe (UNIX and Windows),
an eventual error in the GNUPLOT data transmission might cause GNUPLOT to
quit. As {\REDUCE} is unable to detect the broken pipe, you
have to reset the plot system by calling the 
command \verb+PLOTRESET;+ explicitly. Afterwards new graphic output
can be produced. 

Under Windows 3.1 and Windows NT, GNUPLOT has a text and a graph window.
If you don't want to see the text window, iconify it and
activate the option \verb+update wgnuplot.ini+ from the
graph window system menu - then the present screen layout
(including the graph window size) will be saved and the text
windows will come up iconified in future. You can also select 
some more features there and so tailor the graphic output.
Before you terminate {\REDUCE} you should terminate the
graphic window by calling \verb+PLOTRESET;+.
If you terminate {\REDUCE} without deleting the
GNUPLOT windows, use the command button from the
GNUPLOT text window - it offers an exit function.

\subsection{Saving GNUPLOT command sequence}
\ttindex{TRPLOT}\ttindex{PLOTKEEP}

If you want to use the internal GNUPLOT command sequence
more than once (e.g. for producing a picture for a publication),
you may set 
\begin{verbatim}
ON TRPLOT,PLOTKEEP;
\end{verbatim}
TRPLOT causes all GNUPLOT commands
to be written additionally to the actual
{\REDUCE} output. Normally the data files are
erased after calling GNUPLOT, however with PLOTKEEP on
the files are not erased.

\subsection{Direct Call of GNUPLOT}
\ttindextype{GNUPLOT}{command}

GNUPLOT has a lot of facilities which are not accessed by
the operators and parameters described above. Therefore
genuine GNUPLOT commands can be sent by {\REDUCE}.
Please consult the GNUPLOT manual for the available
commands and parameters. The general syntax for a GNUPLOT call
inside {\REDUCE} is

    $gnuplot(<cmd>,<p_1>,<p_2> \cdots)$

where $<cmd>$ is a command name and $<p_1>,<p_2> \cdots$
are the parameters, inside {\REDUCE} separated by
commas. The parameters are evaluated by
{\REDUCE} and then transmitted to GNUPLOT in
GNUPLOT syntax. Usually a drawing is built by a
sequence of commands which are buffered 
by {\REDUCE} or the operating
system. For terminating and activating them use the {\REDUCE}
command \verb+plotshow+. Example:
\begin{verbatim}
     gnuplot(set,polar);
     gnuplot(set,noparametric);
     gnuplot(plot,x*sin x);
     plotshow;
\end{verbatim}
In this example the function expression is transferred literally
to GNUPLOT, while {\REDUCE}
is responsible for computing the function values when \verb+PLOT+ is called.
Note that GNUPLOT restrictions with respect to variable
and function names have to be taken into account when
using this type of operation. {\bf Important}: String quotes are
not transferred to the GNUPLOT executable; if the GNUPLOT syntax
needs string quotes, you must add doubled stringquotes  {\bf inside}
the argument string, e.g.
\begin{verbatim}
     gnuplot(plot,"""mydata""","using 2:1");
\end{verbatim}

\subsection{Examples}

The following examples pictures are taken from a collection of sample
plot (gnuplot.tst) and a set of tests for plotting special functions
(will be published in the {\REDUCE} electronic library).

The pictures are taken directly form the X screen by xv or 
produced by the $Gnuplot$ postscript driver directly.

A simple plot for sin(1/x) :
\begin{verbatim}
plot(sin(1/x),x=(-1 .. 1),y=(-3 .. 3));
\end{verbatim}

\unitlength=1cm
\begin{picture}(12,8)(0,0)
\put(0,0){\Includegraphics[bb=128 93 430 315]{gnuplotex1}}
\end{picture}

Some implicitly defined curves.
\begin{verbatim}
plot(x^3+y^3 -3*x*y ={0,1,2,3},x=(-2.5 .. 2),y=(-5 .. 5));
\end{verbatim}
\unitlength=1cm
\begin{picture}(10,8)(0,0)
\put(-1,-1){\Includegraphics[bb=0 0 360 270]{bild1}}
\end{picture}

\newpage
A test for hidden surfaces:
\begin{verbatim}
plot (cos sqrt(x**2 + y**2),x=(-3 .. 3),y=(-3 .. 3),hidden3d);
\end{verbatim}

\begin{picture}(12,8)(0,0)
\put(0,0){\Includegraphics[bb=50 0 350 220]{gnuplotex2}}
\end{picture}

This may be slow on some machines because of a delicate evaluation context.
\begin{verbatim}
plot(sinh(x*y)/sinh(2*x*y),hidden3d);
\end{verbatim}

\begin{picture}(12,8)(0,0)
\put(0,0){\Includegraphics[bb=128 93 430 315]{gnuplotex3}}
\end{picture}
\newpage
\begin{verbatim}
on rounded;
w:= {for j:=1 step 0.1 until 20 collect {1/j*sin j,1/j*cos j,j},
     for j:=1 step 0.1 until 20 collect
	{(0.1+1/j)*sin j,(0.1+1/j)*cos j,j} }$
plot w;
\end{verbatim}
\begin{picture}(12,8)(0,0)
\put(0,0){\Includegraphics[bb=127 93 429 314]{gnuplotex4}}
\end{picture}

An example taken from: Cox, Little, O'Shea:  Ideals, Varieties and Algorithms
\begin{verbatim}
plot(point(3u+3u*v^2-u^3,3v+3u^2*v-v^3,3u^2-3v^2),hidden3d,title="Enneper Surface");
\end{verbatim}

\begin{picture}(10,7.8)(-1,0.5)
\put(1,1){\Includegraphics[bb=69 69 319 246]{bild2}}
\end{picture}

The following examples use the specfn package to draw a 
collection of Chebyshev's T polynomials and Bessel Y Functions.
The special function package has to be loaded explicitely
to make the operator ChebyshevT and BesselY available.

\newpage
\begin{verbatim}
load_package specfn;
plot(chebyshevt(1,x),chebyshevt(2,x),chebyshevt(3,x),chebyshevt(4,x),
    chebyshevt(5,x),x=(-1 .. 1),title="Chebyshev t Polynomials");
\end{verbatim}

\begin{picture}(12,8)(0,0)
\put(0,0){\Includegraphics[bb=128 93 430 315]{gnuplotex5}}
\end{picture}

\begin{verbatim}
plot(bessely(0,x),bessely(1,x),bessely(2,x),x=(0.1 .. 10)
     ,y=(-1 .. 1), title="Bessel functions of 2nd kind");
\end{verbatim}

\begin{picture}(12,8)(0,0)
\put(0,0){\Includegraphics[bb=128 93 430 315]{gnuplotex6}}
\end{picture}

