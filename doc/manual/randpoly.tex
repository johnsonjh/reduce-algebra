

%\begin{abstract}
  This package is based on a port of the Maple random polynomial
  generator together with some support facilities for the generation
  of random numbers and anonymous procedures.
%\end{abstract}


\subsection{Introduction}

The operator {\tt randpoly} is based on a port of the Maple random
polynomial generator.  In fact, although by default it generates a
univariate or multivariate polynomial, in its most general form it
generates a sum of products of arbitrary integer powers of the
variables multiplied by arbitrary coefficient expressions, in which
the variable powers and coefficient expressions are the results of
calling user-supplied functions (with no arguments).  Moreover, the
``variables'' can be arbitrary expressions, which are composed with
the underlying polynomial-like function.

The user interface, code structure and algorithms used are essentially
identical to those in the Maple version.  The package also provides an
analogue of the Maple {\tt rand} random-number-generator generator,
primarily for use by {\tt randpoly}.  There are principally two
reasons for translating these facilities rather than designing
comparable facilites anew: (1) the Maple design seems satisfactory and
has already been ``proven'' within Maple, so there is no good reason
to repeat the design effort; (2) the main use for these facilities is
in testing the performance of other algebraic code, and there is an
advantage in having essentially the same test data generator
implemented in both Maple and REDUCE\@.  Moreover, it is interesting
to see the extent to which a facility can be translated without change
between two systems.  (This aspect will be described elsewhere.)

Sections \ref{randpolysec:Basic} and \ref{randpolysec:Advanced} describe respectively
basic and more advanced use of {\tt randpoly}; \S\ref{randpolysec:Subsidiary}
describes subsidiary functions provided to support advanced use of
{\tt randpoly}; \S\ref{randpolysec:Examples} gives examples; an appendix gives
some details of the only non-trivial algorithm, that used to compute
random sparse polynomials.  Additional examples of the use of {\tt
randpoly} are given in the test and demonstration file {\tt
randpoly.tst}.


\subsection{Basic use of {\tt randpoly}}
\label{randpolysec:Basic}

The operator {\tt randpoly} requires at least one argument
corresponding to the polynomial variable or variables, which must be
either a single expression or a list of expressions.%
\footnote{If it is a single expression then the univariate code is
invoked; if it is a list then the multivariate code is invoked, and in
the special case of a list of one element the multivariate code is
invoked to generate a univariate polynomial, but the result should be
indistinguishable from that resulting from specifying a single
expression not in a list.} %
In effect, {\tt randpoly} replaces each input expression by an
internal variable and then substitutes the input expression for the
internal variable in the generated polynomial (and by default expands
the result as usual), although in fact if the input expression is a
REDUCE kernel then it is used directly.  The rest of this document
uses the term ``variable'' to refer to a general input expression or
the internal variable used to represent it, and all references to the
polynomial structure, such as its degree, are with respect to these
internal variables.  The actual degree of a generated polynomial might
be different from its degree in the internal variables.

By default, the polynomial generated has degree 5 and contains 6
terms.  Therefore, if it is univariate it is dense whereas if it is
multivariate it is sparse.


\subsubsection{Optional arguments}

Other arguments can optionally be specified, in any order, after the
first compulsory variable argument.  All arguments receive full
algebraic evaluation, subject to the current switch settings etc.  The
arguments are processed in the order given, so that if more than one
argument relates to the same property then the last one specified
takes effect.  Optional arguments are either keywords or equations
with keywords on the left.

In general, the polynomial is sparse by default, unless the keyword
{\tt dense} is specified as an optional argument.  (The keyword {\tt
sparse} is also accepted, but is the default.)  The default degree can
be changed by specifying an optional argument of the form
\begin{center}
  {\tt degree = {\it natural number}}.
\end{center}
In the multivariate case this is the total degree, i.e.\ the sum of
the degrees with respect to the individual variables.  The keywords
{\tt deg} and {\tt maxdeg} can also be used in place of {\tt degree}.
More complicated monomial degree bounds can be constructed by using
the coefficient function described below to return a monomial or
polynomial coefficient expression.  Moreover, {\tt randpoly} respects
internally the REDUCE ``asymptotic'' commands {\tt let}, {\tt weight}
etc.\ described in \S10.4 of the \REDUCE{} 3.6 manual, which can be used
to exercise additional control over the polynomial generated.

In the sparse case (only), the default maximum number of terms
generated can be changed by specifying an optional argument of the
form
\begin{center}
  {\tt terms = {\it natural number}}.
\end{center}
The actual number of terms generated will be the minimum of the value
of {\tt terms} and the number of terms in a dense polynomial of the
specified degree, number of variables, etc.


\subsection{Advanced use of {\tt randpoly}}
\label{randpolysec:Advanced}

The default order (or minimum or trailing degree) can be changed by
specifying an optional argument of the form
\begin{center}
  {\tt ord = {\it natural number}}.
\end{center}
The keyword is {\tt ord} rather than {\tt order} because {\tt order}
is a reserved command name in REDUCE\@.  The keyword {\tt mindeg} can
also be used in place of {\tt ord}.  In the multivariate case this is
the total degree, i.e.\ the sum of the degrees with respect to the
individual variables.  The order normally defaults to 0.

However, the input expressions to {\tt randpoly} can also be
equations, in which case the order defaults to 1 rather than 0.  Input
equations are converted to the difference of their two sides before
being substituted into the generated polynomial.  The purpose of this
facility is to easily generate polynomials with a specified zero -- for
example
\begin{center}\tt
  randpoly(x = a);
\end{center}
generates a polynomial that is guaranteed to vanish at $x = a$, but is
otherwise random.

Order specification and equation input are extensions of the current
Maple version of {\tt randpoly}.

The operator {\tt randpoly} accepts two further optional arguments in
the form of equations with the keywords {\tt coeffs} and {\tt expons}
on the left.  The right sides of each of these equations must evaluate
to objects that can be applied as functions of no variables.  These
functions should be normal algebraic procedures (or something
equivalent); the {\tt coeffs} procedure may return any algebraic
expression, but the {\tt expons} procedure must return an integer
(otherwise {\tt randpoly} reports an error).  The values returned by
the functions should normally be random, because it is the randomness
of the coefficients and, in the sparse case, of the exponents that
makes the constructed polynomial random.

A convenient special case is to use the function {\tt rand} on the
right of one or both of these equations; when called with a single
argument {\tt rand} returns an anonymous function of no variables that
generates a random integer.  The single argument of {\tt rand} should
normally be an integer range in the form $a~..~b$, where $a$, $b$ are
integers such that $a < b$.  The spaces around (or at least before)
the infix operator ``..'' are necessary in some cases in REDUCE and
generally recommended.  For example, the {\tt expons} argument might
take the form
\begin{center}\tt
  expons = rand(0~..~n)
\end{center}
where {\tt n} will be the maximum degree with respect to each variable
{\em independently}.  In the case of {\tt coeffs} the lower limit will
often be the negative of the upper limit to give a balanced
coefficient range, so that the {\tt coeffs} argument might take the
form
\begin{center}\tt
  coeffs = rand(-n~..~n)
\end{center}
which will generate random integer coefficients in the range $[-n,n]$.


\subsection{Subsidiary functions: rand, proc, random}
\label{randpolysec:Subsidiary}

\subsubsection{Rand: a random-number-generator generator}

The first argument of {\tt rand} must be either an integer range in
the form $a~..~b$, where $a$, $b$ are integers such that $a < b$, or a
positive integer $n$ which is equivalent to the range $0~..~n-1$.  The
operator {\tt rand} constructs a function of no arguments that calls
the REDUCE random number generator function {\tt random} to return a
random integer in the range specified; in the case that the first
argument of {\tt rand} is a single positive integer $n$ the function
constructed just calls {\tt random($n$)}, otherwise the call of {\tt
random} is scaled and shifted.

As an additional convenience, if {\tt rand} is called with a second
argument that is an identifier then the call of {\tt rand} acts
exactly like a procedure definition with the identifier as the
procedure name.  The procedure generated can then be called with an
empty argument list by the algebraic processor.

[Note that {\tt rand()} with no argument is an error in REDUCE and
does not return directly a random number in a default range as it does
in Maple -- use instead the REDUCE function {\tt random} (see below).]


\subsubsection{Proc: an anonymous procedure generator}

The operator {\tt proc} provides a generalization of {\tt rand}, and
is primarily intended to be used with expressions involving the {\tt
random} function (see below).  Essentially, it provides a mechanism to
prevent functions such as {\tt random} being evaluated when the
arguments to {\tt randpoly} are evaluated, which is too early.  {\tt
Proc} accepts a single argument which is converted into the body of an
anonymous procedure, which is returned as the value of {\tt proc}.
(If a named procedure is required then the normal REDUCE {\tt
procedure} statement should be used instead.)  Examples are given in
the following sections, and in the file {\tt randpoly.tst}.


\subsubsection{Random: a generalized interface}

As an additional convenience, this package extends the interface to
the standard REDUCE {\tt random} function so that it will directly
accept either a natural number or an integer range as its argument,
exactly as for the first argument of {\tt rand}.  Hence effectively
\begin{center}\tt
  rand(X) = proc random(X)
\end{center}
although {\tt rand} is marginally more efficient.  However, {\tt proc}
and the generalized {\tt random} interface allow expressions such as
the following anonymous random fraction generator to be easily
constructed:
\begin{center}\tt
  proc(random(-99~..~99)/random(1~..~99))
\end{center}


\subsubsection{Further support for procs}

{\tt Rand} is a special case of {\tt proc}, and (for either) if the
switch {\tt comp} is {\tt on} (and the compiler is available) then the
generated procedure body is compiled.

{\tt Rand} with a single argument and {\tt proc} both return as their
values anonymous procedures, which if they are not compiled are Lisp
lambda expressions.  However, if compilation is in effect then they
return only an identifier that has no external significance%
\footnote{It is not interned on the oblist.} %
but which can be applied as a function in the same way as a lambda
expression.

It is primarily intended that such ``proc expressions'' will be used
immediately as input to {\tt randpoly}.  The algebraic processor is
not intended to handle lambda expressions.  However, they can be
output or assigned to variables in algebraic mode, although the output
form looks a little strange and is probably best not displayed.  But
beware that lambda expressions cannot be evaluated by the algebraic
processor (at least, not without declaring some internal Lisp
functions to be algebraic operators).  Therefore, for testing purposes
or curious users, this package provides the operators {\tt showproc}
and {\tt evalproc} respectively to display and evaluate ``proc
expressions'' output by {\tt rand} or {\tt proc} (or in fact any
lambda expression), in the case of {\tt showproc} provided they are
not compiled.


\subsection{Examples}
\label{randpolysec:Examples}

The file {\tt randpoly.tst} gives a set of test and demonstration
examples.

The following additional examples were taken from the Maple {\tt
randpoly} help file and converted to REDUCE syntax by replacing [~] by
\{~\} and making the other changes shown explicitly:
\begin{verbatim}
randpoly(x);

       5       4       3       2
 - 54*x  - 92*x  - 30*x  + 73*x  - 69*x - 67


randpoly({x, y}, terms = 20);

    5       4         4       3  2       3         3
31*x  - 17*x *y - 48*x  - 15*x *y  + 80*x *y + 92*x

       2  3      2         2         4         3         2
 + 86*x *y  + 2*x *y - 44*x  + 83*x*y  + 85*x*y  + 55*x*y

                       5       4      3       2
 - 27*x*y + 33*x - 98*y  + 51*y  - 2*y  + 70*y  - 60*y - 10


randpoly({x, sin(x), cos(x)});

                   4            3              3
sin(x)*( - 4*cos(x)  - 85*cos(x) *x + 50*sin(x)

                    2
         - 20*sin(x) *x + 76*sin(x)*x + 96*sin(x))


% randpoly(z, expons = rand(-5..5));  % Maple
% A generalized random "polynomial"!
% Note that spaces are needed around .. in REDUCE.
on div; off allfac;
randpoly(z, expons = rand(-5 .. 5));

       4       3       -3       -4      -5
 - 39*z  + 14*z  - 77*z   - 37*z   - 8*z

off div; on allfac;
% randpoly([x], coeffs = proc() randpoly(y) end);  % Maple
randpoly({x}, coeffs = proc randpoly(y));

    5  5       5  4       5  3       5  2       5         5
95*x *y  - 53*x *y  - 78*x *y  + 69*x *y  + 58*x *y - 58*x

       4  5       4  4       4  3       4  2       4
 + 64*x *y  + 93*x *y  - 21*x *y  + 24*x *y  - 13*x *y

       4       3  5       3  4       3  3       3  2
 - 28*x  - 57*x *y  - 78*x *y  - 44*x *y  + 37*x *y

       3         3       2  5       2  4    2  3       2  2
 - 64*x *y - 95*x  - 71*x *y  - 69*x *y  - x *y  - 49*x *y

       2         2         5         4         3         2
 + 77*x *y + 48*x  + 38*x*y  + 93*x*y  - 65*x*y  - 83*x*y

                       5       4       3       2
 + 25*x*y + 51*x + 35*y  - 18*y  - 59*y  + 73*y  - y + 31


% A more conventional alternative is ...
%   procedure r; randpoly(y)$  randpoly({x}, coeffs = r);
% or, in fact, equivalently ...
%   randpoly({x}, coeffs = procedure r; randpoly(y));

randpoly({x, y}, dense);

    5       4         4       3  2       3         3
85*x  + 43*x *y + 68*x  + 87*x *y  - 93*x *y - 20*x

       2  2       2        2         4         3         2
 - 74*x *y  - 29*x *y + 7*x  + 10*x*y  + 62*x*y  - 86*x*y

                       5       4       3       2
 + 15*x*y - 97*x - 53*y  + 71*y  - 46*y  - 28*y  + 79*y + 44
\end{verbatim}


%\appendix


\newtheorem{prop}{Proposition}

\newenvironment{randpproof}%
   {\par\addvspace\baselineskip\noindent{\bf Proof~}}%
   {\hspace*{\fill}$\Box$\par\addvspace\baselineskip}

\subsection{Appendix: Algorithmic background}

The only part of this package that involves any mathematics that is
not completely trivial is the procedure to generate a sparse set of
monomials of specified maximum and minimum total degrees in a
specified set of variables.  This involves some combinatorics, and the
Maple implementation calls some procedures from the Maple
Combinatorial Functions Package {\tt combinat} (of which I have
implemented restricted versions in REDUCE).

Given the maximum possible number $N$ of terms (in a dense
polynomial), the required number of terms (in the sparse polynomial)
is selected as a random subset of the natural numbers up to $N$, where
each number indexes a term.  In the univariate case these indices are
used directly as monomial exponents, but in the multivariate case they
are converted to monomial exponent vectors using a lexicographic
ordering.


\subsubsection{Numbers of polynomial terms}

By explicitly enumerating cases with 1, 2, etc.\ variables, as
indicated by the inductive proof below, one deduces that:

\begin{prop}
  In $n$ variables, the number of distinct monomials having total
  degree precisely $r$ is $^{r+n-1}C_{n-1}$, and the maximum number of
  distinct monomials in a polynomial of maximum total degree $d$ is
  $^{d+n}C_n$.
\end{prop}

\begin{randpproof}
  Suppose the first part of the proposition is true, namely that there
  are at most
  \[
    N_h(n,r) = {}^{r+n-1}C_{n-1}
  \]
  distinct monomials in an $n$-variable {\em homogeneous\/}
  polynomial of total degree $r$.  Then there are at most
  \[
    N(d,r) = \sum_{r=0}^d {}^{r+n-1}C_{n-1} = {}^{d+n}C_n
  \]
  distinct monomials in an $n$-variable polynomial of maximum total
  degree $d$.

  The sum follows from the fact that
  \[
    {}^{r+n}C_n = \frac{(r+n)^{\underline n}}{n!}
  \]
  where $x^{\underline n} = x(x-1)(x-2)\cdots(x-n+1)$ denotes a
  falling factorial, and
  \[
    \sum_{a \leq x < b} x^{\underline n} =
    \left. \frac{x^{\underline{n+1}}}{n+1} \right|_a^b.
  \]
  (See, for example, D. H. Greene \& D. E. Knuth, {\it Mathematics
  for the Analysis of Algorithms}, Birkh\"auser, Second Edn.\ 1982,
  equation (1.37)).  Hence the second part of the proposition follows
  from the first.

  The proposition holds for 1 variable ($n = 1$), because there is
  clearly 1 distinct monomial of each degree precisely $r$ and hence
  at most $d+1$ distinct monomials in a polynomial of maximum degree
  $d$.

  Suppose that the proposition holds for $n$ variables, which are
  represented by the vector $X$.  Then a homogeneous polynomial of
  degree $r$ in the $n+1$ variables $X$ together with the single
  variable $x$ has the form
  \[
    x^r P_0(X) + x^{r-1} P_1(X) + \cdots + x^0 P_r(X)
  \]
  where $P_s(X)$ represents a polynomial of maximum total degree $s$
  in the $n$ variables $X$, which therefore contains at most
  $^{s+n}C_n$ distinct monomials.  The homogeneous polynomial of
  degree $r$ in $n+1$ terms therefore contains at most
  \[
    \sum_{s=0}^r {}^{s+n}C_n = {}^{r+n+1}C_{n+1}
  \]
  distinct monomials.
  Hence the proposition holds for $n+1$ variables, and therefore by
  induction it holds for all $n$.
\end{randpproof}


\subsubsection{Mapping indices to exponent vectors}

The previous proposition is also the basis of the algorithm to map
term indices $m \in \mathbb{N}$ to exponent vectors $v \in \mathbb{N}^n$,
where $n$ is the number of variables.

Define a norm $\|\cdot\|$ on exponent vectors by $\|v\| = \sum_{i=1}^n
v_i$, which corresponds to the total degree of the monomial.  Then,
from the previous proposition, the number of exponent vectors of
length $n$ with norm $\|v\| \leq d$ is $N(n,d) = {}^{d+n}C_n$.  The
elements of the $m^{\text{th}}$ exponent vector are constructed recursively by
applying the algorithm to successive tail vectors, so let a subscript
denote the length of the vector to which a symbol refers.

The aim is to compute the vector of length $n$ with index $m = m_n$.
If this vector has norm $d_n$ then the index and norm must satisfy
\[
  N(n,d_n-1) \leq m_n < N(n,d_n),
\]
which can be used (as explained below) to compute $d_n$ given $n$ and
$m_n$.  Since there are $N(n,d_n-1)$ vectors with norm less than
$d_n$, the index of the $(n-1)$-element tail vector must be given by
$m_{n-1} = m_n - N(n,d_n-1)$, which can be used recursively to compute
the norm $d_{n-1}$ of the tail vector.  From this, the first element
of the exponent vector is given by $v_1 = d_n - d_{n-1}$.

The algorithm therefore has a natural recursive structure that
computes the norm of each tail subvector as the recursion stack is
built up, but can only compute the first term of each tail subvector
as the recursion stack is unwound.  Hence, it constructs the exponent
vector from right to left, whilst being applied to the elements from
left to right.  The recursion is terminated by the observation that
$v_1 = d_1 = m_1$ for an exponent vector of length $n = 1$.

The main sub-procedure, given the required length $n$ and index $m_n$
of an exponent vector, must return its norm $d_n$ and the index of its
tail subvector of length $n-1$.  Within this procedure, $N(n,d)$ can
be efficiently computed for values of $d$ increasing from 0, for which
$N(n,0) = {}^nC_n = 1$, until $N(n,d) > m$ by using the observation
that
\[
   N(n,d) = {}^{d+n}C_n = \frac{(d+n)(d-1+n)\cdots(1+n)}{d!}.
\]
