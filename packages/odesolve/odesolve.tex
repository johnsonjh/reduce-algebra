\documentclass[a4paper]{article} % LaTeX2e

\usepackage{hyperref}

\newcommand{\ODESolve}[1]{\texttt{ODESolve\,#1}}
\newcommand{\odesolve}{\texttt{odesolve}}
\newcommand{\REDUCE}{\textsc{Reduce}}

\title{\ODESolve{1.065} : \\ An Enhanced \REDUCE{} ODE Solver}

\author{Francis J. Wright \\
School of Mathematical Sciences \\
Queen Mary, University of London \\
Mile End Road, London E1 4NS, UK. \\
\texttt{F.J.Wright@qmw.ac.uk} \\
\href{http://centaur.maths.qmw.ac.uk/}
{\texttt{http://centaur.maths.qmw.ac.uk/}}}

\date{14 August 2001}

\begin{document}
\maketitle

\tableofcontents

\section{Introduction}

\ODESolve{1+} is an experimental project to update and enhance the
ordinary differential equation (ODE) solver (\odesolve{}) that has
been distributed as a standard component of \REDUCE{}
\cite{Hearn-manual,MacCallum-doc,MacCallum-ISSAC} for about 10 years.
\ODESolve{1+} is intended to provide a strict superset of the
facilities provided by \odesolve{}.  This document describes a
substantial re-implementation of previous versions of \ODESolve{1+}
that now includes almost none of the original \odesolve{} code.  This
version is targeted at \REDUCE~3.7 or later, and will not run in
earlier versions.  This project is being conducted partly under the
auspices of the European CATHODE project \cite{CATHODE}.  Various test
files, including three versions based on a published review of ODE
solvers \cite{Zimmermann}, are included in the \ODESolve{1+}
distribution.  For further background see \cite{FJW1}, which describes
version 1.03.  See also \cite{FJW2}.

\ODESolve{1+} is intended to implement some solution techniques itself
(i.e.\ most of the simple and well known techniques \cite{Zwillinger})
and to provide an automatic interface to other more sophisticated
solvers, such as PSODE \cite{Man,Man-MacCallum,Prelle-Singer} and
CRACK \cite{CRACK-doc}, to handle cases where simple techniques fail.
It is also intended to provide a unified interface to other special
solvers, such as Laplace transforms, series solutions and numerical
methods, under user request.  Although none of these extensions is
explicitly implemented yet, a general extension interface is
implemented (see \S\ref{OEI}).

The main motivation behind \ODESolve{1+} is pragmatic.  It is intended
to meet user expectations, to have an easy user interface that
normally does the right thing automatically, and to return solutions
in the form that the user wants and expects.  Quite a lot of
development effort has been directed toward this aim.  Hence,
\ODESolve{1+} solves common text-book special cases in preference to
esoteric pathological special cases, and it endeavours to simplify
solutions into convenient forms.


\section{Installation}

The file \texttt{odesolve.in} inputs the full set of source files that
are required to implement \ODESolve{1+} \emph{assuming that the
current directory is the \ODESolve{1+} source directory}.  Hence,
\ODESolve{1+} can be run without compiling it in any implementation of
\REDUCE~3.7 by starting \REDUCE{} in the \ODESolve{1+} source
directory and entering the statement
\begin{verbatim}
1: in "odesolve.in"$
\end{verbatim}

However, the recommended procedure is to compile it by starting
\REDUCE{} in the \ODESolve{1+} source directory and entering the
statements
\begin{verbatim}
1: faslout odesolve;
2: in "odesolve.in"$
3: faslend;
\end{verbatim}
In CSL-\REDUCE{}, this will work only if you have write access to the
\REDUCE{} image file (\texttt{reduce.img}), so you may need to set up
a private copy first.  In PSL-\REDUCE{}, you may need to move the
compiled image file \texttt{odesolve.b} to a directory in your PSL
load path, such as the main fasl directory.  Please refer to the
documentation for your implementation of \REDUCE{} for details.  Once
a compiled version of \ODESolve{1+} has been correctly installed, it
can be loaded by entering the \REDUCE{} statement
\begin{verbatim}
1: load_package odesolve;
\end{verbatim}

A string describing the current version of \ODESolve{1+} is assigned
to the algebraic-mode variable \verb|odesolve_version|, which can be
evaluated to check what version is actually in use.

In versions of \REDUCE{} derived from the development source after 22
September 2000, use of the normal algebraic-mode \odesolve{} operator
causes the package to autoload.  However, the \ODESolve{1+} global
switches are not declared, and the symbolic mode interface provided
for backward compatibility with the previous version is not defined,
until after the package has loaded.  The former is not a huge problem
because all \ODESolve{} switches can be accessed as optional
arguments, and the backward compatibility interface should probably
not be used in new code anyway.


\section{User interface}

The principal interface is via the operator \odesolve{}.  (It also has
a synonym called \texttt{dsolve} to make porting of examples from
Maple easier, but it does not accept general Maple syntax!)  For
purposes of description I will refer to the dependent variable as
``$y$'' and the independent variable as ``$x$'', but of course the
names are arbitrary.  The general input syntax is
\begin{verbatim}
odesolve(ode, y, x, conditions, options);
\end{verbatim}
All arguments except the first are optional.  This is possible
because, if necessary, \ODESolve{1+} attempts to deduce the dependent
and independent variables used and to make any necessary
\texttt{DEPEND} declarations.  Messages are output to indicate any
assumptions or dependence declarations that are made.  Here is an
example of what is probably the shortest possible valid input:
\begin{verbatim}
odesolve(df(y,x));

*** Dependent var(s) assumed to be y

*** Independent var assumed to be x

*** depend y , x

{y=arbconst(1)}
\end{verbatim}
Output of \ODESolve{1+} messages is controlled by the standard
\REDUCE{} switch \texttt{msg}.


\subsection{Specifying the ODE and its variables}

The first argument (\texttt{ode}) is \emph{required}, and must be
either an ODE or a variable (or expression) that evaluates to an
ODE\@.  Automatic dependence declaration works \emph{only} when the
ODE is input \emph{directly} as an argument to the \odesolve{}
operator.  Here, ``ODE'' means an equation or expression containing
one or more derivatives of $y$ with respect to $x$.  Derivatives of
$y$ with respect to other variables are not allowed because
\ODESolve{1+} does not solve \emph{partial} differential equations,
and symbolic derivatives of variables other than $y$ are treated as
symbolic constants.  An expression is implicitly equated to zero, as
is usual in equation solvers.

The independent variable may be either an operator that explicitly
depends on the independent variable, e.g.\ $y(x)$ (as required in
Maple), or a simple variable that is declared (by the user or
automatically by \ODESolve{1+}) to depend on the independent variable.
If the independent variable is an operator then it may depend on
parameters as well as the independent variable.  Variables may be
simple identifiers or, more generally, \REDUCE{} ``kernels'', e.g.
\begin{verbatim}
operator x, y;
odesolve(df(y(x(a),b),x(a)) = 0);

*** Dependent var(s) assumed to be y(x(a),b)

*** Independent var assumed to be x(a)

{y(x(a),b)=arbconst(1)}
\end{verbatim}

The order in which arguments are given must be preserved, but
arguments may be omitted, except that if $x$ is specified then $y$
must also be specified, although an empty list \verb|{}| can be used
as a ``place-holder'' to represent ``no specified argument''.
Variables are distinguished from options by requiring that if a
variable is specified then it must appear in the ODE, otherwise it is
assumed to be an option.

Generally in \REDUCE{} it is not recommended to use the identifier
\verb|t| as a variable, since it is reserved in Lisp.  However, it is
very common practice in applied mathematics to use it as a variable to
represent time, and for that reason \ODESolve{1+} provides special
support to allow it as either the independent or a dependent variable.
But, of course, its use may still cause trouble in other parts of
\REDUCE!


\subsection{Specifying conditions}

If specified, the ``conditions'' argument must take the form of an
(unordered) list of (unordered lists of) equations with either $y$,
$x$, or a derivative of $y$ on the left.  A single list of conditions
need not be contained within an outer list.  Combinations of
conditions are allowed.  Conditions within one (inner) list all relate
to the same $x$ value.  For example:
\begin{description}
\item[Boundary conditions:] ~ \\
\{\{y=y0, x=x0\}, \{y=y1, x=x1\}, ...\}

\item[Initial conditions:] ~ \\
\{x=x0, y=y0, df(y,x)=dy0, ...\}

\item[Combined conditions:] ~ \\
\{\{y=y0, x=x0\}, \{df(y,x)=dy1, x=x1\}, \{df(y,x)=dy2, y=y2, x=x2\}, ...\}
\end{description}
Here is an example of boundary conditions:
\begin{verbatim}
odesolve(df(y,x,2) = y, y, x, {{x = 0, y = A}, {x = 1, y = B}});

        2*x      2*x          2
     - e   *a + e   *b*e + a*e  - b*e
{y=-----------------------------------}
                x  2    x
               e *e  - e
\end{verbatim}
Here is an example of initial conditions:
\begin{verbatim}
odesolve(df(y,x,2) = y, y, x, {x = 0, y = A, df(y,x) = B});

     2*x      2*x
    e   *a + e   *b + a - b
{y=-------------------------}
                x
             2*e
\end{verbatim}
Here is an example of combined conditions:
\begin{verbatim}
odesolve(df(y,x,2) = y, y, x, {{x=0, y=A}, {x=1, df(y,x)=B}});

     2*x      2*x          2
    e   *a + e   *b*e + a*e  - b*e
{y=--------------------------------}
               x  2    x
              e *e  + e
\end{verbatim}

Boundary conditions on the values of $y$ at various values of $x$ may
also be specified by replacing the variables by equations with single
values or matching lists of values on the right, of the form
\begin{center}
y = y0, x = x0
\end{center}
or
\begin{center}
y = \{y0, y1, ...\}, x = \{x0, x2, ...\}
\end{center}
For example
\begin{verbatim}
odesolve(df(y,x) = y, y = A, x = 0);

    x
{y=e *a}

odesolve(df(y,x,2) = y, y = {A, B}, x = {0, 1});

        2*x      2*x          2
     - e   *a + e   *b*e + a*e  - b*e
{y=-----------------------------------}
                x  2    x
               e *e  - e
\end{verbatim}


\subsection{Specifying options and defaults}

The final arguments may be one or more of the option identifiers
listed in the table below, which take precedence over the default
settings.  All options can also be specified on the right of equations
with the identifier ``output'' on the left, e.g.\ ``output = basis''.
This facility if provided mainly for compatibility with other systems
such as Maple, although it also allows options to be distinguished
from variables in case of ambiguity.  Some options can be specified on
the left of equations that assign special values to the option.
Currently, only ``trode'' and its synonyms can be assigned the value 1
to give an increased level of tracing.

The following switches set default options -- they are all off by
default.  Options set locally using option arguments override the
defaults set by switches.
\begin{center}
\begin{tabular}{lll}
\bf Switch         & \bf Option  & \bf Effect on solution \\
\hline
odesolve\_explicit & explicit    & fully explicit \\
odesolve\_expand   & expand      & expand roots of unity \\
odesolve\_full     & full        & fully explicit and expanded \\
odesolve\_implicit & implicit    & implicit instead of parametric \\
                   & algint      & turn on algint \\
odesolve\_noint    & noint       & turn off selected integrations \\
odesolve\_verbose  & verbose     & display ODE and conditions \\
odesolve\_basis    & basis       & output basis solution for linear ODE \\
                   & trode \\
trode              & trace       & turn on algorithm tracing \\
                   & tracing \\
odesolve\_fast     & fast        & turn off heuristics \\
odesolve\_check    & check       & turn on solution checking
\end{tabular}
\end{center}

An ``explicit'' solution is an equation with $y$ isolated on the left
whereas an ``implicit'' solution is an equation that determines $y$ as
one or more of its solutions.  A ``parametric'' solution expresses
both $x$ and $y$ in terms of some additional parameter.  Some solution
techniques naturally produce an explicit solution, but some produce
either an implicit or a parametric solution.  The ``explicit'' option
causes \ODESolve{1+} to attempt to convert solutions to explicit form,
whereas the ``implicit'' option causes \ODESolve{1+} to attempt to
convert parametric solutions (only) to implicit form (by eliminating
the parameter).  These solution conversions may be slow or may fail in
complicated cases.

\ODESolve{1+} introduces two operators used in solutions:
\texttt{root\_of\_unity} and \texttt{plus\_or\_minus}, the latter
being a special case of the former, i.e.\ a second root of unity.
These operators carry a tag that associates the same root of unity
when it appears in more than one place in a solution (cf.\ the
standard \texttt{root\_of} operator).  The ``expand'' option expands a
single solution expressed in terms of these operators into a set of
solutions that do not involve them.  \ODESolve{1+} also introduces two
operators \texttt{expand\_roots\_of\_unity} [which should perhaps be
named \texttt{expand\_root\_of\_unity}] and
\texttt{expand\_plus\_or\_minus}, that are used internally to perform
the expansion described above, and can be used explicitly.

The ``algint'' option turns on ``algebraic integration'' locally only
within \ODESolve{1+}.  It also loads the \texttt{algint} package if
necessary.  Algint allows \ODESolve{1+} to solve some ODEs for which
the standard \REDUCE{} integrator hangs (i.e.\ takes an extremely long
time to return).  If the resulting solution contains unevaluated
integrals then the algint switch should be turned on outside
\ODESolve{1+} before the solution is re-evaluated, otherwise the
standard integrator may well hang again!  For some ODEs, the algint
option leads to better solutions than the standard \REDUCE{}
integrator.

Alternatively, the ``noint'' option prevents \REDUCE{} from attempting
to evaluate the integrals that arise in some solution techniques.  If
\ODESolve{1+} takes too long to return a result then you might try
adding this option to see if it helps solve this particular ODE, as
illustrated in the test files.  This option is provided to speed up
the computation of solutions that contain integrals that cannot be
evaluated, because in some cases \REDUCE{} can spend a long time
trying to evaluate such integrals before returning them unevaluated.
This only affects integrals evaluated \emph{within} the \ODESolve{1+}
operator.  If a solution containing an unevaluated integral that was
returned using the ``noint'' option is re-evaluated, it may again take
\REDUCE{} a very long time to fail to evaluate the integral, so
considerable caution is recommended!  (A global switch called
``noint'' is also installed when \ODESolve{1+} is loaded, and can be
turned on to prevent \REDUCE{} from attempting to evaluate \emph{any}
integrals.  But this effect may be very confusing, so this switch
should be used only with extreme care.  If you turn it on and then
forget, you may wonder why \REDUCE{} seems unable to evaluate even
trivial integrals!)

The ``verbose'' option causes \ODESolve{1+} to display the ODE,
variables and conditions as it sees them internally, after
pre-processing.  This is intended for use in demonstrations and
possibly for debugging, and not really for general users.

The ``basis'' option causes \ODESolve{1+} to output the general
solutions of linear ODEs in basis format (explained below).  Special
solutions (of ODEs with conditions) and solutions of nonlinear ODEs
are not affected.

The ``trode'' (or ``trace'' or ``tracing'') option turns on tracing of
the algorithms used by \ODESolve{1+}.  It reports its classification
of the ODE and any intermediate results that it computes, such as a
chain of progressively simpler (in some sense) ODEs that finally leads
to a solution.  Tracing can produce a lot of output, e.g.\ see the
test log file ``\texttt{zimmer.rlg}''.  The option ``\texttt{trode =
1}'' or the global assignment ``\texttt{!*trode := 1}'' causes
\ODESolve{1+} to report every test that it tries in its classification
process, producing even more tracing output.  This is probably most
useful for debugging, but it may give the curious user further insight
into the operation of \ODESolve{1+}.

The ``fast'' option disables all non-deterministic solution techniques
(including most of those for nonlinear ODEs of order $> 1$).  It may
be most useful if \ODESolve{1+} is used as a subroutine, including
calling it recursively in a hook.  It makes \ODESolve{1+} behave like
the \odesolve{} distributed with \REDUCE{} versions up to and
including 3.7, and so does not affect the \texttt{odesolve.tst} file.
The ``fast'' option causes \ODESolve{1+} to return no solution fast in
cases where, by default, if would return either a solution or no
solution (perhaps much) more slowly.  Solution of sufficiently simple
``deterministically-solvable'' ODEs is unaffected.

The ``check'' option turns on checking of the solution.  This checking
is performed by code that is largely independent of the solver, so as
to perform a genuinely independent check.  It is not turned on by
default so as to avoid the computational overhead, which is currently
of the order of 30\%.  A check is made that each component solution
satisfies the ODE and that a general solution contains at least enough
arbitrary constants, or equivalently that a basis solution contains
enough basis functions.  Otherwise, warning messages are output.  It
is possible that \ODESolve{1+} may fail to verify a solution because
the automatic simplification fails, which indicates a failure in the
checker rather than in the solver.  This option is not yet well
tested; please report any checking failures to me (FJW).

In some cases, in particular when an implicit solution contains an
unevaluated integral, the checker may need to differentiate an
integral with respect to a variable other than the integration
variable.  In order to do this, it turns on the differentiator switch
``allowdfint'' globally.  [I hope that this setting will eventually
become the default.]  In some cases, in particular in symbolic
solutions of Clairaut ODEs, the checker may need to differentiate a
composition of operators using the chain rule.  In order to do this,
it turns on the differentiator switch ``expanddf'' locally only.
Although the code to support both these differentiator facilities has
been in \REDUCE{} for a while, they both require patches that are
currently only applied when \ODESolve{1+} is loaded.  [I hope that
these patches will eventually become part of \REDUCE{} itself.]


\section{Output syntax}

If \ODESolve{1+} is successful it outputs a list of sub-solutions that
together represent the solution of the input ODE\@.  Each sub-solution
is either an equation that defines a branch of the solution,
explicitly or implicitly, or it is a list of equations that define a
branch of the solution parametrically in the form $\{y = G(p), x =
F(p), p\}$.  Here $p$ is the parameter, which is actually represented
in terms of an operator called \texttt{arbparam} which has an integer
argument to distinguish it from other unrelated parameters, as usual
for arbitrary values in \REDUCE{}.

A general solution will contain a number of arbitrary constants
represented by an operator called \texttt{arbconst} with an integer
argument to distinguish it from other unrelated arbitrary constants.
A special solution resulting from applying conditions will contain
fewer (usually no) arbitrary constants.

The general solution of a linear ODE in basis format is a list
consisting of a list of basis functions for the solution space of the
reduced ODE followed by a particular solution if the input ODE had a
$y$-independent ``driver'' term, i.e.\ was not reduced (which is
sometimes ambiguously called ``homogeneous'').  The particular
solution is normally omitted if it is zero.  The dependent variable
$y$ does not appear in a basis solution.  The linear solver uses basis
solutions internally.

Currently, there are cases where \ODESolve{1+} cannot solve a linear ODE
using its linear solution techniques, in which case it will try
nonlinear techniques.  These may generate a solution that is not
(obviously) a linear combination of basis solutions.  In this case, if
a basis solution has been requested, \ODESolve{1+} will report that it
cannot separate the nonlinear combination, which it will return as the
default linear combination solution.

If \ODESolve{1+} fails to solve the ODE then it will return a list
containing the input ODE (always in the form of a differential
expression equated to 0).  At present, \ODESolve{1+} does not return
partial solutions.  If it fails to solve any part of the problem then
it regards this as complete failure.  (You can probably see if this
has happened by turning on algorithm tracing.)


\section{Solution techniques}

The \ODESolve{1+} interface module pre-processes the problem and applies
any conditions to the solution.  The other modules deal with the
actual solution.

\ODESolve{1+} first classifies the input ODE according to whether it
is linear or nonlinear and calls the appropriate solver.  An ODE that
consists of a product of linear factors is regarded as nonlinear.  The
second main classification is based on whether the input ODE is of
first or higher degree.

Solution proceeds essentially by trying to reduce nonlinear ODEs to
linear ones, and to reduce higher order ODEs to first order ODEs.
Only simple linear ODEs and simple first-order nonlinear ODEs can be
solved directly.  This approach involves considerable recursion within
\ODESolve{1+}.

If all solution techniques fail then \ODESolve{1+} attempts to
factorize the derivative of the whole ODE, which sometimes leads to a
solution.


\subsection{Linear solution techniques}

\ODESolve{1+} splits every linear ODE into a ``reduced ODE'' and a
``driver'' term.  The driver is the component of the ODE that is
independent of $y$, the reduced ODE is the component of the ODE that
depends on $y$, and the sign convention is such that the ODE can be
written in the form ``reduced ODE = driver''.  The reduced ODE is then
split into a list of ``ODE coefficients''.

The linear solver now determines the order of the ODE\@.  If it is 1
then the ODE is immediately solved using an integrating factor (if
necessary).  For a higher order linear ODE, \ODESolve{1+} considers a
sequence of progressively more complicated solution techniques.  For
most purposes, the ODE is made ``monic'' by dividing through by the
coefficient of the highest order derivative.  This puts the ODE into a
standard form and effectively deals with arbitrary overall algebraic
factors that would otherwise confuse the solution process.  (Hence,
there is no need to perform explicit algebraic factorization on linear
ODEs.)  The only situation in which the original non-monic form of the
ODE is considered is when checking for exactness, which may depend
critically on otherwise irrelevant overall factors.

If the ODE has constant coefficients then it can (in principle) be
solved using elementary ``D-operator'' techniques in terms of
exponentials via an auxiliary equation.  However, this works only if
the polynomial auxiliary equation can be solved.  Assuming that it can
and there is a driver term, \ODESolve{1+} tries to use a method based
on inverse ``D-operator'' techniques that involves repeated
integration of products of the solutions of the reduced ODE with the
driver.  Experience (by Malcolm MacCallum) suggests that this normally
gives the most satisfactory form of solution if the integrals can be
evaluated.  If any integral fails to evaluate, the more general method
of ``variation of parameters'', based on the Wronskian of the solution
set of the reduced ODE, is used instead.  This involves only a single
integral and so can never lead to nested unevaluated integrals.

If the ODE has non-constant coefficients then it may be of Euler
(sometimes ambiguously called ``homogeneous'') type, which can be
trivially reduced to an ODE with constant coefficients.  A shift in
$x$ is accommodated in this process.  Next it is tested for exactness,
which leads to a first integral that is an ODE of order one lower.
After that it is tested for the explicit absence of $y$ and low order
derivatives, which allows trivial order reduction.  Then the monic ODE
is tested for exactness, and if that fails and the original ODE was
non-monic then the original form is tested for exactness.

Finally, pattern matching is used to seek a solution involving special
functions, such as Bessel functions.  Currently, this is implemented
only for second-order ODEs satisfied by Bessel and Airy-integral
functions.  It could easily be extended to other orders and other
special functions.  Shifts in $x$ could also be accommodated in the
pattern matching.  [Work to enhance this component of \ODESolve{1+} is
currently in progress.]

If all linear techniques fail then \ODESolve{1+} currently calls the
variable interchange routine (described below), which takes it into
the nonlinear solver.  Occasionally, this is successful in producing
some, although not necessarily the best, solution of a linear ODE.


\subsection{Nonlinear solution techniques}

In order to handle trivial nonlinearity, \ODESolve{1+} first
factorizes the ODE algebraically, solves each factor that depends on
$y$ and then merges the resulting solutions.  Other factors are
ignored, but a warning is output unless they are purely numerical.

If all attempts at solution fail then \ODESolve{1+} checks whether the
original (unfactored) ODE was exact, because factorization could
destroy exactness.  Currently, \ODESolve{1+} handles only first and
second order nonlinear exact ODEs.

A version of the main solver applied to each algebraic factor branches
depending on whether the ODE factor is linear or nonlinear, and the
nonlinear solver branches depending on whether the order is 1 or
higher and calls one of the solvers described in the next two
sections.  If that solver fails, \ODESolve{1+} checks for exactness
(of the factor).  If that fails, it checks whether only a single order
derivative is involved and tries to solve algebraically for that.  If
successful, this decomposes the ODE into components that are, in some
sense, simpler and may be solvable.  (However, in some cases these
components are algebraically very complicated examples of simple types
of ODE that the integrator cannot in practice handle, and it can take
a very long time before returning an unevaluated integral.)

If all else fails, \ODESolve{1+} interchanges the dependent and
independent variables and calls the top-level solver recursively.  It
keeps a list of all ODEs that have entered the top-level solver in
order to break infinite loops that could arise if the solution of the
variable-interchanged ODE fails.


\subsubsection{First-order nonlinear solution techniques}

If the ODE is a first-degree polynomial in the derivative then
\ODESolve{1+} represents it in terms of the ``gradient'', which is a
function of $x$ and $y$ such that the ODE can be written as ``$dy/dx =
\textit{gradient}$''.  It then checks \emph{in sequence} for the
following special types of ODE, each of which it can (in principle)
solve:
\begin{description}
\item[Separable] The gradient has the form $f(x)g(y)$, leading
immediately to a solution by quadrature, i.e.\ the solution can be
immediately written in terms of indefinite integrals.  (This is
considered to be a solution of the ODE, regardless of whether the
integrals can be evaluated.)  The solver recognises both explicit and
implicit dependence when detecting separable form.

\item[Quasi-separable] The gradient has the form $f(y+kx)$, which is
(trivially) separable after a linear transformation.  It arises as a
special case of the ``quasi-homogeneous'' case below, but is better
treated earlier as a case in its own right.

\item[Homogeneous] The gradient has the form $f(y/x)$, which is
algebraically homogeneous.  A substitution of the form ``$y = vx$''
leads to a first-order linear ODE that is (in principle) immediately
solvable.

\item[Quasi-homogeneous] The gradient has the form $f(\frac{a_1x +
b_1y + c_1}{a_2x + b_2y + c_2})$, which is homogeneous after a linear
transformation.

\item[Bernoulli] The gradient has the form $P(x) y + Q(x) y^n$, in
which case the ODE is a first-order linear ODE for $y^{1-n}$.

\item[Riccati] The gradient has the form $a(x)y^2 + b(x)y + c(x)$, in
which case the ODE can be transformed into a \emph{linear}
second-order ODE that may be solvable.
\end{description}

If the ODE is not first-degree then it may be linear in either $x$ or
$y$.  Solving by taking advantage of this leads to a parametric
solution of the original ODE, in which the parameter corresponds to
$y'$.  It may then be possible to eliminate the parameter to give
either an implicit or explicit solution.

An ODE is ``solvable for $y$'' if it can be put into the form $y =
f(x,y')$.  Differentiating with respect to $x$ leads to a first-order
ODE for $y'(x)$, which may be easier to solve than the original ODE.
The special case that $y = xF(y') + G(y')$ is called a Lagrange (or
d'Alembert) ODE\@.  Differentiating with respect to $x$ leads to a
first-order linear ODE for $x(y')$.  The even more special case that
$y = x y' + G(y')$, which may arise in the equivalent implicit form
$F(xy'-y) = G(y')$, is called a Clairaut ODE\@.  The general solution is
given by replacing $y'$ by an arbitrary constant, and it may be
possible to obtain a singular solution by differentiating and solving
the resulting factors simultaneously with the original ODE.

An ODE is ``solvable for $x$'' if it can be put into the form $x =
f(y,y')$.  Differentiating with respect to $y$ leads to a first-order
ODE for $y'(y)$, which may be easier to solve than the original ODE.

Currently, \ODESolve{1+} recognises the above forms only if the ODE
manifestly has the specified form and does not try very hard to
actually solve for $x$ or $y$, which perhaps it should!


\subsubsection{Higher-order nonlinear solution techniques}

The techniques used here are all special cases of Lie symmetry
analysis, which is not yet applied in any general way.

Higher-order nonlinear ODEs are passed through a number of
``simplifier'' filters that are applied in succession, regardless of
whether the previous filter simplifies the ODE or not.  Currently, the
first filter tests for the explicit absence of $y$ and low order
derivatives, which allows trivial order reduction.  The second filter
tests whether the ODE manifestly depends on $x+k$ for some constant
$k$, in which case it shifts $x$ to remove $k$.

After that, \ODESolve{1+} tests for each of the following special
forms in sequence.  The sequence used here is important, because the
classification is not unique, so it is important to try the most
useful classification first.
\begin{description}
\item[Autonomous] An ODE is autonomous if it does not depend
explicitly on $x$, in which case it can be reduced to an ODE in $y'$
of order one lower.

\item[Scale invariant or equidimensional in $x$] An ODE is scale
invariant if it is invariant under the transformation $x \to ax, y \to
a^py$, where $a$ is an arbitrary indeterminate and $p$ is a constant
to be determined.  It can be reduced to an autonomous ODE, and thence
to an ODE of order one lower.  The special case $p = 0$ is called
equidimensional in $x$.  It is the nonlinear generalization of the
(reduced) linear Euler ODE.

\item[Equidimensional in $y$] An ODE is equidimensional in $y$ if it
is invariant under the transformation $y \to ay$.  An exponential
transformation of $y$ leads to an ODE of the same order that
\emph{may} be ``more linear'' and so easier to solve, but there is no
guarantee of this.  All (reduced) linear ODEs are trivially
equidimensional in $y$.
\end{description}

The recursive nature of \ODESolve{1+}, especially the thread described
in this section, can lead to complicated ``arbitrary constant
expressions''.  Arbitrary constants must be included at the point
where an ODE is solved by quadrature.  Further processing of such a
solution, as may happen when a recursive solution stack is unwound,
can lead to arbitrary constant expressions that should be re-written
as simple arbitrary constants.  There is some simple code included to
perform this arbitrary constant simplification, but it is rudimentary
and not entirely successful.


\section{Extension interface}\label{OEI}

The idea is that the ODESolve extension interface allows any user to
add solution techniques without needing to edit and recompile the main
source code, and (in principle) without needing to be intimately
familiar with the internal operation of \ODESolve{1+}.

The extension interface consists of a number of ``hooks'' at various
critical places within \ODESolve{1+}.  These hooks are modelled in
part on the hook mechanism used to extend and customize the Emacs
editor, which is a large Lisp-based system with a structure similar to
that of \REDUCE\@.  Each \ODESolve{1+} hook is an identifier which can
be defined to be a function (i.e.\ a procedure), or have assigned to
it (in symbolic mode) a function name or a (symbolic mode) list of
function names.  The function should be written to accept the
arguments specified for the particular hook, and it should return
either a solution to the specified class of ODE in the specified form
or nil.

If a hook returns a non-nil value then that value is used by
\ODESolve{1+} as the solution of the ODE at that stage of the solution
process.  [If the ODE being solved was generated internally by
\ODESolve{1+} or conditions are imposed then the solution will be
re-processed before being finally returned by \ODESolve{1+}.]  If a
hook returns nil then it is ignored and \ODESolve{1+} proceeds as if
the hook function had not been called at all.  This is the same
mechanism that it used internally by \ODESolve{1+} to run sub-solvers.
If a hook evaluates to a list of function names then they are applied
in turn to the hook arguments until a non-nil value is returned and
this is the value of the hook; otherwise the hook returns nil.  The
same code is used to run all hooks and it checks that an identifier is
the name of a function before it tries to apply it; otherwise the
identifier is ignored.  However, the hook code does not perform any
other checks, so errors within functions run by hooks will probably
terminate \ODESolve{1+} and errors in the return value will probably
cause fatal errors later in \ODESolve{1+}.  Such errors are user
errors rather than \ODESolve{1+} errors!

Hooks are defined in pairs which are inserted before and after
critical stages of the solver, which currently means the general ODE
solver, the nonlinear ODE solver, and the solver for linear ODEs of
order greater than one (on the grounds that solving first order linear
ODEs is trivial and the standard \ODESolve{1+} code should always
suffice).  The precise interface definition is as follows.

A reference to an ``algebraic expression'' implies that the \REDUCE{}
representation is a prefix or pseudo-prefix form.  A reference to a
``variable'' means an identifier (and never a more general kernel).
The ``order'' of an ODE is always an explicit positive integer.  The
return value of a hook function must always be either nil or an
algebraic-mode list (which must be represented as a prefix form).
Since the input and output of hook functions uses prefix forms (and
never standard quotient forms), hook functions can equally well be
written in either algebraic or symbolic mode, and in fact \ODESolve{1+}
uses a mixture internally.  (An algebraic-mode procedure can return
nil by returning nothing.  The integer zero is \emph{not} equivalent
to nil in the context of \ODESolve{1+} hooks.)

\noindent\hrulefill

\begin{description}
\item[Hook names:] \verb|ODESolve_Before_Hook|,
\verb|ODESolve_After_Hook|.
\item[Run before and after:] The general ODE solver.
\item[Arguments:] 3
\begin{enumerate}
\item The ODE in the form of an algebraic expression with no
denominator that must be made identically zero by the solution.
\item The dependent variable.
\item The independent variable.
\end{enumerate}
\item[Return value:] A list of equations exactly as returned by
\ODESolve{1+} itself.
\end{description}

\noindent\hrulefill

\begin{description}
\item[Hook names:] \verb|ODESolve_Before_Non_Hook|,
\verb|ODESolve_After_Non_Hook|.
\item[Run before and after:] The nonlinear ODE solver.
\item[Arguments:] 4
\begin{enumerate}
\item The ODE in the form of an algebraic expression with no
denominator that must be made identically zero by the solution.
\item The dependent variable.
\item The independent variable.
\item The order of the ODE.
\end{enumerate}
\item[Return value:] A list of equations exactly as returned by
\ODESolve{1+} itself.
\end{description}

\noindent\hrulefill
\pagebreak

\begin{description}
\item[Hook names:] \verb|ODESolve_Before_Lin_Hook|,
\verb|ODESolve_After_Lin_Hook|.
\item[Run before and after:] The general linear ODE solver.
\item[Arguments:] 6
\begin{enumerate}
\item A list of the coefficient functions of the ``reduced ODE'',
i.e.\ the coefficients of the different orders (including zero) of
derivatives of the dependent variable, each in the form of an
algebraic expression, in low to high derivative order.  [In general
the ODE will not be ``monic'' so the leading (i.e.\ last) coefficient
function will not be 1.  Hence, the ODE may contain an essentially
irrelevant overall algebraic factor.]
\item The ``driver'' term, i.e.\ the term involving only the
independent variable, in the form of an algebraic expression.  The
sign convention is such that ``reduced ODE = driver''.
\item The dependent variable.
\item The independent variable.
\item The (maximum) order ($> 1$) of the ODE.
\item The minimum order derivative present.
\end{enumerate}
\item[Return value:] A list consisting of a basis for the solution
space of the reduced ODE and optionally a particular integral of the
full ODE\@.  This list does not contain any equations, and the dependent
variable never appears in it.  The particular integral may be omitted
if it is zero.  The basis is itself a list of algebraic expressions in
the independent variable.  (Hence the return value is always a list
and its first element is also always a list.)
\end{description}

\noindent\hrulefill

\begin{description}
\item[Hook names:] \verb|ODESolve_Before_Non1Grad_Hook|, \\
\verb|ODESolve_After_Non1Grad_Hook|.
\item[Run before and after:] The solver for first-order first-degree
nonlinear (``gradient'') ODEs, which can be expressed in the form
$dy/dx = \mathrm{gradient}(y,x)$.
\item[Arguments:] 3
\begin{enumerate}
\item The ``gradient'', which is an algebraic expression involving (in
general) the dependent and independent variables, to which the ODE
equates the derivative.
\item The dependent variable.
\item The independent variable.
\end{enumerate}
\item[Return value:] A list of equations exactly as returned by
\ODESolve{1+} itself.  (In this case the list should normally contain
precisely one equation.)
\end{description}

\noindent\hrulefill
\bigskip

The file \texttt{extend.tst} contains a very simple test and
demonstration of the operation of the first three classes of hook.

This extension interface is experimental and subject to change.
Please check the version of this document (or the source code) for the
version of \ODESolve{1+} you are actually running.


\section{Change log}

\begin{description}
\item[27 February 1999] Version 1.06 frozen.
\item[13 July 2000] Version 1.061 added an extension interface.
\item[8 August 2000] Version 1.062 added the ``fast'' option.
\item[21 September 2000] Version 1.063 added the ``trace'', ``check''
  and ``algint'' options, the ``Non1Grad'' hooks, handling of implicit
  dependence in separable ODEs, and handling of the general class of
  quasi-homogeneous ODEs.
\item[28 September 2000] Version 1.064 added support for using `t' as
  a variable and replaced the version identification output by the
  \verb|odesolve_version| variable.
\item[14 August 2001] Version 1.065 fixed obscure bugs in the
  first-order nonlinear ODE handler and the arbitrary constant
  simplifier, and revised some tracing messages slightly.

\end{description}


\section{Planned developments}

\begin{itemize}

\item
Extend special-function solutions and allow shifts in $x$.

\item
Improve solution of linear ODEs, by (a) using linearity more generally
to solve as ``CF + PI'', (b) finding at least polynomial solutions of
ODEs with polynomial coefficients, (c) implementing non-trivial
reduction of order.

\item
Improve recognition of exact ODEs, and add some support for more
general use of integrating factors.

\item
Add a ``classify'' option, that turns on trode but avoids any actual
solution, to report all possible (\@?) top-level classifications.

\item
Improve arbconst and arbparam simplification.

\item
Add more standard elementary techniques and more general techniques
such as Lie symmetry, Prelle-Singer, etc.

\item
Improve integration support, preferably to remove the need for the
``noint'' option.

\item
Solve systems of ODEs, including under- and over-determined ODEs and
systems.  Link to CRACK (Wolf) and/or DiffGrob2 (Mansfield)?

\item
Move more of the implementation to symbolic-mode code.

\end{itemize}


\begin{thebibliography}{99}

\bibitem{CATHODE} CATHODE (Computer Algebra Tools for Handling
Ordinary Differential Equations)
\href{http://www-lmc.imag.fr/CATHODE/}%
{\texttt{http://www-lmc.imag.fr/CATHODE/}},
\href{http://www-lmc.imag.fr/CATHODE2/}%
{\texttt{http://www-lmc.imag.fr/CATHODE2/}}.

\bibitem{Hearn-manual} A. C. Hearn and J. P. Fitch (ed.),
\textit{REDUCE User's Manual 3.6}, RAND Publication CP78 (Rev. 7/95),
RAND, Santa Monica, CA 90407-2138, USA (1995).

\bibitem{MacCallum-ISSAC} M. A. H. MacCallum, An Ordinary Differential
Equation Solver for REDUCE, \textit{Proc.\ ISSAC~'88, ed.\ P. Gianni,
Lecture Notes in Computer Science} \textbf{358}, Springer-Verlag
(1989), 196--205.

\bibitem{MacCallum-doc} M. A. H. MacCallum, ODESOLVE, \LaTeX{} file
\texttt{reduce/doc/odesolve.tex} distributed with \REDUCE~3.6.  The
first part of this document is included in the printed REDUCE User's
Manual 3.6 \cite{Hearn-manual}, 345--346.

\bibitem{Man} Y.-K. Man, \textit{Algorithmic Solution of ODEs and
Symbolic Summation using Computer Algebra}, PhD Thesis, School of
Mathematical Sciences, Queen Mary and Westfield College, University of
London (July 1994).

\bibitem{Man-MacCallum} Y.-K. Man and M. A. H. MacCallum, A Rational
Approach to the Prelle-Singer Algorithm, \textit{J. Symbolic
Computation}, \textbf{24} (1997), 31--43.

\bibitem{Zimmermann} F. Postel and P. Zimmermann, A Review of the ODE
Solvers of \textsc{Axiom}, \textsc{Derive}, \textsc{Maple},
\textsc{Mathematica}, \textsc{Macsyma}, \textsc{MuPAD} and
\textsc{Reduce}, \textit{Proceedings of the 5th Rhine Workshop on
Computer Algebra, April 1-3, 1996, Saint-Louis, France.}
Specific references are to the version dated April 11, 1996.
The latest version of this review, together with log files for each of
the systems, is available from
\href{http://www.loria.fr/~zimmerma/ComputerAlgebra/}%
{\texttt{http://www.loria.fr/\textasciitilde zimmerma/ComputerAlgebra/}}.

\bibitem{Prelle-Singer} M. J. Prelle and M. F. Singer, Elementary
First Integrals of Differential Equations, \textit{Trans.\ AMS}
\textbf{279} (1983), 215--229.

\bibitem{CRACK-doc} T. Wolf and A. Brand, The Computer Algebra Package
\texttt{CRACK} for Investigating PDEs, \LaTeX{} file
\texttt{reduce/doc/crack.tex} distributed with \REDUCE~3.6.  A shorter
document is included in the printed REDUCE User's Manual 3.6
\cite{Hearn-manual}, 241--244.

\bibitem{FJW1} F. J. Wright, An Enhanced ODE Solver for REDUCE.
\textit{Programmirovanie} No 3 (1997), 5--22, in Russian, and
\textit{Programming and Computer Software} No 3 (1997), in English.

\bibitem{FJW2} F. J. Wright, Design and Implementation of
\ODESolve{1+} : An Enhanced REDUCE ODE Solver.  CATHODE Workshop
Report, Marseilles, May 1999, CATHODE (1999). \\
\href{http://centaur.maths.qmw.ac.uk/Papers/Marseilles/}%
{\texttt{http://centaur.maths.qmw.ac.uk/Papers/Marseilles/}}.

\bibitem{Zwillinger} D. Zwillinger, \textit{Handbook of Differential
Equations}, Academic Press.  (Second edition 1992.)

\end{thebibliography}

\end{document}
