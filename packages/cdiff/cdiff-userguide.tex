\documentclass[12pt]{amsart}
\usepackage{xspace}
\usepackage{amssymb}
\usepackage{amsthm}
\usepackage{color}
\usepackage{mathrsfs}
\usepackage{microtype}
\usepackage{upref}
\usepackage{url}
\usepackage{graphicx}
\usepackage[dvips,%
pdftitle={},%
pdfauthor={R. Vitolo},%
pdfkeywords={}%
pdfsubject={},%
colorlinks,linkcolor={blue},citecolor={blue},urlcolor={red}%
]{hyperref}

% LENGTHS

\setlength{\hfuzz}{3pt}
%\addtolength{\arraycolsep}{5pt}
%\smartqed
%-----------------------------------------------------------------------------%
% PAGE SIZES
%-----------------------------------------------------------------------------%
\setlength{\headheight}{32pt}
\setlength{\headsep}{29pt}
\setlength{\footskip}{28pt}
\setlength{\textwidth}{444pt}
\setlength{\textheight}{636pt}
\setlength{\marginparsep}{7pt}
\setlength{\marginparpush}{7pt}
\setlength{\oddsidemargin}{4.5pt}
\setlength{\marginparwidth}{55pt}
\setlength{\evensidemargin}{4.5pt}
\setlength{\topmargin}{-15pt}
\setlength{\footnotesep}{8.4pt}
\allowdisplaybreaks[4]

% CLAIMS

\swapnumbers
\newtheorem{theorem}{Theorem}
\newtheorem{corollary}[theorem]{Corollary}
\newtheorem{lemma}[theorem]{Lemma}
\newtheorem{proposition}[theorem]{Proposition}
\theoremstyle{definition}
\newtheorem{remark}[theorem]{Remark}
\newtheorem{definition}[theorem]{Definition}
\newtheorem{example}[theorem]{Example}

% MACROS

\newcommand{\cprime}{\/{\mathsurround=0pt$'$}}

\newcommand*{\pd}[2]{\mathchoice{\frac{\partial#1}{\partial#2}}
  {\partial#1/\partial#2}{\partial#1/\partial#2}
  {\partial#1/\partial#2}}
\newcommand*{\od}[2]{\mathchoice{\frac{d#1}{d#2}}
  {d#1/d#2}{d#1/d#2}{d#1/d#2}}
\newcommand*{\fd}[2]{\mathchoice{\frac{\delta#1}{\delta#2}}
  {\delta #1/\delta#2}{\delta#1/\delta#2}{\delta#1/\delta#2}}

%    Notation for an expression evaluated at a particular condition.
%    The optional argument can be used to override automatic sizing
%    of the right vert bar, e.g. \eval[\biggr]{...}_{...}
\newcommand{\eval}[2][\right]{\relax
  \ifx#1\right\relax \left.\fi#2#1\rvert}

%    Enclose the argument in vert-bar delimiters.
%    The optional argument can be used to override automatic sizing,
%    e.g. \abs[\bigg]{...}
\newcommand{\envert}[2][\right]{\relax
  \ifx#1\right\relax \left\lvert\else#1\lvert\fi#2#1\rvert}
\let\abs=\envert

%    Enclose the argument in double-vert-bar delimiters:
%    The optional argument can be used to override automatic sizing,
%    e.g. \norm[\bigg]{...}
\newcommand{\enVert}[2][\right]{\relax
  \ifx#1\right\relax \left\lVert\else#1\lVert\fi#2#1\rVert}
\let\matr=\enVert

\newcommand*{\sdim}[2]{#1\vert#2}
\newcommand{\doubell}{\mathcal{L}}

\let\kappa\varkappa
\let\phi\varphi

\newcommand{\hj}{\bar{\jmath}}
\newcommand{\hd}{\bar{d}}
\newcommand{\J}{\bar{\mathcal{J}}}
\newcommand{\Ji}{{\bar{\mathcal{J}}}^{\infty}}
\newcommand{\hL}{\bar{\Lambda}}
\newcommand{\cC}{\mathcal{C}}
\newcommand{\cE}{\mathcal{E}}
\newcommand{\cL}{\mathcal{L}}
\newcommand{\cJ}{\mathcal{J}}
\newcommand{\g}{\mathfrak{g}}
\newcommand{\hH}{\bar{H}^n}
\newcommand{\Eu}{\mathscr{E}}
\newcommand{\N}{\mathbb{N}}
\newcommand{\R}{\mathbb{R}}
\newcommand{\Z}{\mathbb{Z}}

\newcommand{\alg}{\mathcal{F}}

\newcommand{\id}{\mathrm{id}}

\DeclareMathOperator{\CDiff}{\mathcal{C}Dif{}f}
\DeclareMathOperator{\Der}{D}
\DeclareMathOperator{\CL}{CL}
\DeclareMathOperator{\Dv}{D^v}
\DeclareMathOperator{\cl}{cl}
\DeclareMathOperator{\Diff}{Dif{}f}
\DeclareMathOperator{\diff}{dif{}f}
\DeclareMathOperator{\Smbl}{Smbl}
\DeclareMathOperator{\End}{End}
\DeclareMathOperator{\Hom}{Hom}
\DeclareMathOperator{\Sym}{Sym}
\DeclareMathOperator{\CoSym}{CoSym}
\DeclareMathOperator{\im}{im}
\DeclareMathOperator{\pr}{pr}
\DeclareMathOperator{\supp}{supp}
\newcommand*{\CDiffsym}[1]{\CDiff_{(#1)}^{\,\mathrm{sym}}}
\newcommand*{\CDiffself}[1]{\CDiff_{(#1)}^{\,\mathrm{self}}}
\newcommand*{\CDiffskew}[1]{\CDiff_{(#1)}^{\,\mathrm{skew}}}
\newenvironment{system}{\left\{\begin{array}{l}}{\end{array}\right.}
\newcommand{\cdiff}{CDIFF\xspace}
\newcommand{\reduce}{REDUCE\xspace}

%    Cyrillic letter \`E for use in math. mode
%    (from the Univ. of Washington Cyrillic font)
%\DeclareFontFamily{OT1}{wncyi}{}
%\DeclareFontShape{OT1}{wncyi}{m}{it}{
%   <5> <6> <7> <8> <9> gen * wncyi
%   <10> <10.95> <12> <14.4> <17.28> <20.74> <24.88> wncyi10
%  }{}
%\DeclareSymbolFont{cyrletters}{OT1}{wncyi}{m}{it}
%\DeclareSymbolFontAlphabet{\cyrmath}{cyrletters}
%\DeclareMathSymbol{\re}{\cyrmath}{cyrletters}{"03}
%\newcommand{\Ev}{\re}
\newcommand{\Ev}{E}
\newcommand{\EulerOperator}{\mathscr{E}}

%    HyperTeX commands
\providecommand{\href}[2]{#2}
\providecommand{\urlprefix}{URL }
%\newcommand*{\email}[1]{\href{mailto:#1}{\begingroup \urlstyle{rm}\Url{#1}}}
%\newcommand*{\eprint}[2][]{\href{http://arXiv.org/abs/#2}%
%{\begingroup \Url{arXiv:#2}}}

%\makeatletter
%\renewcommand{\@fnsymbol}[1]{}
%\makeatother

%\markboth{Variational brackets in the geometry of PDEs}%
%{P.H.M. Kersten, I.S. Krasil\cprime shchik, A.M.
%  Verbovetsky, R. Vitolo}

\newcommand{\nota}[2]{\color{red}[{Mark: #1}\par{#2}]\color{black}}
%%%%%%%%%%%%%%%%%%%%%%%%%%%%%%%%%%%%%%%%%%%%%%%%%%%%%%%%%%%%%%%%%%%%%

\begin{document}

\title[\cdiff user guide]%
{\cdiff: \reduce packages
\\
for computations in geometry of PDEs}

\author[R. Vitolo]{R. Vitolo}
\address{R. Vitolo\\ Dept.\ of Mathematics
  ``E. De Giorgi'', Universit\`a del Salento \\ via per Arnesano, 73100 Lecce,
  Italy} \email{raffaele.vitolo@unisalento.it}

\date{2010 July 22}

\thanks{}

\keywords{\reduce, Hamiltonian operators, generalized symmetries, higher
  symmetries, conservation laws, nonlocal variables.}

\subjclass[2000]{37K05}

\begin{abstract}
  We describe \cdiff, a symbolic computation package for
  the geometry of Differential Equations (DEs, for short) and
  developed by P. Gragert, P.H.M. Kersten, G. Post and G. Roelofs at the
  University of Twente, The Netherlands.

  The package is part of the official \reduce distribution at Sourceforge
  \cite{red}, but it is also distributed on the Geometry of Differential
  Equations web site \url{http://gdeq.org} (GDEQ for short).

  We start from an installation guide for Linux and Windows. Then we focus on
  concrete usage recipes for the computation of higher symmetries, conservation
  laws, Hamiltonian and recursion operators for polynomial differential
  equations. All programs discussed here are shipped together with this manual
  and can be found at the GDEQ website. The mathematical theory on which
  computations are based can be found in refs.~\cite{Many,KKV}.
\end{abstract}

\maketitle

\section{Introduction}

This brief guide refers to using \cdiff, a set of symbolic computation packages
devoted to computations in the geometry of DEs and developed by P. Gragert,
P.H.M. Kersten, G. Post and G. Roelofs at the University of Twente, The
Netherlands.

Initially, the development of the \cdiff packages was started by Gragert and
Kersten for symmetry computations in DEs, then they have been partly rewritten
and extended by Roelofs and Post. The \cdiff packages consist of 3 program
files plus a utility file; only the main three files are documented
\cite{svec,integ,tools}. The \cdiff packages, as well as a copy of the
documentation (including this manual) and several example programs, can be
found both at Sourceforge in the sources of \reduce \cite{red} and in the
Geometry of Differential Equations (GDEQ for short) web site \cite{gdeq}. The
name of the packages, \cdiff, comes from the fact that the package is aimed at
defining differential operators in total derivatives and do computations
involving them.  Such operators are called \emph{$\mathcal{C}$-differential
  operators} (see \cite{Many}).

The main motivation for writing this manual was that \reduce 3.8 recently became
free software, and can be downloaded here \cite{red}. For this reason, we are
able to make our computations accessible to a wider public, also thanks to the
inclusion of \cdiff in the official \reduce distribution. The readers are
warmly invited to send questions, comments, etc., both on the computations and
on the technical aspects of installation and configuration of \reduce, to the
author of the present manual.

\textbf{Acknowledgements.} My warmest thanks are for Paul H.M. Kersten, who
explained to me how to use the \cdiff packages for several computations of
interest in the Geometry of Differential Equations. I also would like to thank
I.S. Krasil'shchik and A.M. Verbovetsky for constant support and stimulating
discussions which led me to write this text.

\section{Installation}

In order to use \cdiff packages you should be able to write \reduce programs
using \cdiff and run them in the \reduce interactive shell. So, you need two
programs: \reduce and a text editor which is preferably oriented to program
development.

\subsection{Installation of \reduce}
\label{sec:installation-reduce}

In order to install \reduce it is enough to download from here \cite{red} a
precompiled binary for your operating system (\emph{e.g.}, 32-bit or 64-bit
Debian-based Linux like Debian itself or Ubuntu, 32-bit Windows) and uncompress
it in your computer in a location of your choice.

For the moment \cdiff packages have been tested under Linux (both 32bit and
64bit) and Windows XP; please contact the author of this guide if you tested
the packages with positive results under Mac or other versions of Windows like
Vista or Windows 7.

A \reduce program using \cdiff packages can be written with any text editor; it
is customary to use the extension \texttt{.red} for \reduce programs, like
\texttt{program.red}. If you wish to run your program, just run the \reduce
executable. After starting \reduce, you would see something like
\begin{verbatim}
Reduce (Free CSL version), 14-Apr-09 ...

1:
\end{verbatim}

Assume that you wrote the program \texttt{program.red}, using \cdiff macros.
The program must contain the line
\begin{verbatim}
load_package cdiff;
\end{verbatim}
just before the first macro from the package \cdiff.
Then you may run the program with the following instruction at the \reduce
prompt:
\begin{verbatim}
1: in "program.red";
 \end{verbatim}
Note that in what follows we will omit the prompt in \reduce commands.

Of course, if the program file \texttt{program.red} \emph{is not} in the place
where the \reduce executable is, you should indicate the full path of the
program, and this depends on your system.
Remember that each time you run \reduce the path at the \reduce prompt is always
the path of the \reduce executable, unless you use an absolute path as
above. However, if you start \reduce with the graphical interface (see below)
you can always use the leftmost menu item \texttt{File>Open\dots} in order to
avoid to write down the whole absolute path.

\subsection{Installation of an editor for writing \reduce programs}
\label{sec:inst-reduce-ide}

Now, let us deal with the problem of editing \reduce programs.

Generally speaking, any text editor can be used to write a \reduce program. A
more suitable choice is an editor for programming languages. Such editors exist
in Linux and Windows, a list can be found here \cite{ed}.

A suggested text editor in Windows is \texttt{notepad++}. This editor is easy
to install, it has support for many programming languages (but \emph{not} for
\reduce!), and has a GPL free license, see \cite{noteppp}. Similar tools in
Linux are \texttt{kwrite} and \texttt{gedit}.

However, the only IDE (Integrated Development Environment) for developing
programs and running them inside the editor itself exists for the great text
editor \texttt{emacs}, which runs in all operating systems, and in particular
Linux and Windows. We stress that an IDE makes the developing-running-debugging
cycle much faster because every step is performed in the same environment.

Installation of \texttt{emacs} in Linux is quite smooth, although it depends on
the Linux distribution; usually it is enough to select the package
\texttt{emacs} in your favourite package management tool, like
\texttt{aptitude, synaptic,} or \texttt{kpackage}.  In order to install
\texttt{emacs} on Windows one has to work a little bit more. See here
\cite{emacswin} for more information.  Assuming that \texttt{emacs} it is
installed and working, the \reduce IDE for \texttt{emacs} can be found here
\cite{redide}.  We refer to their guide for the installation (the procedure is
the same for both Linux and Windows). I tested the IDE with emacs 23.1 under
Debian-based Linux systems (Debian Etch and Squeeze 32-bit and 64-bit, Ubuntu
10.04 64-bit) and Windows XP and it works fine for me.

Suppose you have \texttt{emacs} and its \reduce IDE installed, then there is a
last configuration step that will make \texttt{emacs} and \reduce work together.
Namely, when opening for the first time a \reduce program file with
\texttt{emacs}, go to the \texttt{REDUCE>Customize\dots} menu item and locate
the `\reduce run Program' item. This item contains the command which is issued
by \texttt{emacs} from the \reduce IDE when the menu item \texttt{Run REDUCE>Run
  REDUCE} is selected. Change the command to:
  \begin{itemize}
  \item under Linux (user and location as above):
   \begin{verbatim}
    reduce -w
    \end{verbatim}
  \item under Windows (user and locations as above):
   \begin{verbatim}
    reduce.exe
    \end{verbatim}
  \end{itemize}
This setting will run \reduce inside \texttt{emacs}. If you prefer the (slower)
graphical interface to \reduce, remove `\texttt{-w}'. Note that the graphical
interface will produce \LaTeX\ output, making it much more readable. This
behaviour can be turned off in the graphical interface by issuing the command
\texttt{off fancy;}.

\section{Working with \cdiff}

All programs that we will discuss in this manual can be found inside the
subfolder \texttt{examples} in the folder which contains this manual.  There
are some conventions that I adopted on writing programs which use \cdiff.
\begin{itemize}
\item Program files have the extension \texttt{.red}. This will load
  automatically the reduce-ide mode in emacs (provided you made the
  installation steps described in the reduce-ide guides).
\item Program files have the following names:
  \begin{center}
    \texttt{equationname\_typeofcomputation\_version.red}
  \end{center}
  where \texttt{equationname} stands for the shortened name of the equation
  (\emph{e.g.} Korteweg--de Vries is always indicated by KdV),
  \texttt{typeofcomputation} stands for the type of geometric object which is
  computed with the given file, for example symmetries, Hamiltonian operators,
  etc., \texttt{version} is a version number.
\item More specific information, like the date and more details on the
  computation done in each version, are included as comment lines at the very
  beginning of each file.
\end{itemize}

If you use a generic editor, as soon as you are finished writing a program, you
may run it from within \reduce by following the instructions in the previous
section.

In \texttt{emacs} with \reduce IDE it is easier: issuing the command
\texttt{M-x run-reduce} (or choosing the menu item \texttt{Run REDUCE>Run
  REDUCE}) will split the window in two halves and start \reduce in the bottom
half. If you are running PSL \reduce you must first issue the command
\texttt{lisp set\_bndstk\_size 1000000;} from within \reduce, in order to avoid
memory problems. If you are running CSL \reduce there is no need of that
instruction. Then you may load the program file that you were editing (suppose
that its name is \texttt{program.red}) by issuing \texttt{in "program.red";} at
the \reduce prompt. In fact, \texttt{emacs} lets \reduce assume as its working
directory the directory of the file that you were editing.

Results of a computation consist of the values of one or more unknown. Suppose
that the unknown's name is \texttt{sym}, and assume that, after a computation,
you wish to save the values of \texttt{sym}, possibly for future use from within
\reduce. Issue the following \reduce commands (of course, after you finish your
computations!):
\begin{verbatim}
off nat;
out file_res.red;
sym:=sym;
shut file_res.red;
\end{verbatim}
The above commands will write the content of \texttt{sym} into a file whose
name is \texttt{file\_res.red}, where \texttt{file} stands for a filename which
follows the above convention. The command \texttt{off nat;} is needed in order
to save the variable in a format which could be imported in future \reduce
sessions. If you wish to translate your results in \LaTeX, see the package
\texttt{rlfi} and its own documentation.

\section{Computing with \cdiff}

Here we describe some examples of computations with \cdiff. The parts of
examples which are shared between all examples are described only once. We
stress that all computations presented in this document are included in the
official \reduce distribution and can be also downloaded at the GDEQ website
\cite{gdeq}. The examples can be run with \reduce by typing \texttt{in
  "program.red";} at the \reduce prompt, as explained above.

\textbf{Remark.} The mathematical theories on which the computations are based
can be found in \cite{Many,KKV}.

\subsection{Higher symmetries}\label{sec:higher-symmetries}

In this section we show the computation of (some) higher symmetries of
Burgers'equation $B=u_t-u_{xx}+2uu_x=0$. The corresponding file is
\texttt{Burg\_hsym\_1.red} and the results of the computation are in
\texttt{Burg\_hsym\_1\_res.red}.

The idea underlying this computation is that one can use the scale symmetries
of Burgers'equation to assign ``gradings'' to each variable appearing in the
equation. As a consequence, one could try different ansatz for symmetries with
polynomial generating function. For example, it is possible to require that
they are sum of monomials of given degrees. This ansatz yields a simplification
of the equations for symmetries, because it is possible to solve them in a
``graded'' way, \emph{i.e.}, it is possible to split them into several
equations made by the homogeneous components of the equation for symmetries
with respect to gradings.

In particular, Burgers'equation translates into the following dimensional
equation:
\begin{displaymath}
  [u_t]=[u_{xx}],\quad [u_{xx}=2uu_x].
\end{displaymath}
By the rules $[u_z]=[u]-[z]$ and $[uv]=[u]+[v]$, and choosing $[x]=-1$, we have
$[u]=1$ and $[t]=-2$. This will be used to generate the list of homogeneous
monomials of given grading to be used in the ansatz about the structure of the
generating function of the symmetries.

The following instructions initialize the total derivatives. The first string
is the name of the vector field, the second item is the list of even variables
(note that \texttt{u1}, \texttt{u2}, ... are $u_x$, $u_{xx}$, \dots), the third
item is the list of odd (non-commuting) variables (`ext' stands for `external'
like in external (wedge) product). Note that in this example odd variables are
not strictly needed, but it is better to insert some of them for syntax reasons.
\begin{verbatim}
super_vectorfield(ddx,{x,t,u,u1,u2,u3,u4,u5,u6,u7,
u8,u9,u10,u11,u12,u13,u14,u15,u16,u17},
{ext 1,ext 2,ext 3,ext 4,ext 5,ext 6,ext 7,ext 8,ext 9,ext 10,ext
11,ext 12,ext 13,ext 14,ext 15,ext 16,ext 17,ext 18,ext 19,ext 20,ext
21,ext 22,ext 23,ext 24,ext 25,ext 26,ext 27,ext 28,ext 29,ext 30,
ext 31,ext 32,ext 33,ext 34,ext 35,ext 36,ext 37,ext 38,ext 39,ext 40,
ext 41,ext 42,ext 43,ext 44,ext 45,ext 46,ext 47,ext 48,ext 49,ext 50,
ext 51,ext 52,ext 53,ext 54,ext 55,ext 56,ext 57,ext 58,ext 59,ext 60,
ext 61,ext 62,ext 63,ext 64,ext 65,ext 66,ext 67,ext 68,ext 69,ext 70,
ext 71,ext 72,ext 73,ext 74,ext 75,ext 76,ext 77,ext 78,ext 79,ext 80
});
\end{verbatim}

\begin{verbatim}
super_vectorfield(ddt,{x,t,u,u1,u2,u3,u4,u5,u6,u7,
u8,u9,u10,u11,u12,u13,u14,u15,u16,u17},
{ext 1,ext 2,ext 3,ext 4,ext 5,ext 6,ext 7,ext 8,ext 9,ext 10,ext
11,ext 12,ext 13,ext 14,ext 15,ext 16,ext 17,ext 18,ext 19,ext 20,ext
21,ext 22,ext 23,ext 24,ext 25,ext 26,ext 27,ext 28,ext 29,ext 30,
ext 31,ext 32,ext 33,ext 34,ext 35,ext 36,ext 37,ext 38,ext 39,ext 40,
ext 41,ext 42,ext 43,ext 44,ext 45,ext 46,ext 47,ext 48,ext 49,ext 50,
ext 51,ext 52,ext 53,ext 54,ext 55,ext 56,ext 57,ext 58,ext 59,ext 60,
ext 61,ext 62,ext 63,ext 64,ext 65,ext 66,ext 67,ext 68,ext 69,ext 70,
ext 71,ext 72,ext 73,ext 74,ext 75,ext 76,ext 77,ext 78,ext 79,ext 80
});
\end{verbatim}

Specification of the vectorfield \texttt{ddx}.  The meaning of the first index
is the parity of variables.  In particular here we have just even variables.
The second index parametrizes the second item (list) in the
\texttt{super\_vectorfield} declaration. More precisely, \texttt{ddx(0,1)}
stands for $\pd{}{x}$, \texttt{ddx(0,2)} stands for $\pd{}{t}$,
\texttt{ddx(0,3)} stands for $\pd{}{u}$, \texttt{ddx(0,4)} stands for
$\pd{}{u_x}$, \dots, and all coordinates $x$, $t$, $u_x$, \dots, are treated as
even coordinates.  Note that \texttt{`\$'} suppresses the output.
\begin{verbatim}
ddx(0,1):=1$
ddx(0,2):=0$
ddx(0,3):=u1$
ddx(0,4):=u2$
ddx(0,5):=u3$
ddx(0,6):=u4$
ddx(0,7):=u5$
ddx(0,8):=u6$
ddx(0,9):=u7$
ddx(0,10):=u8$
ddx(0,11):=u9$
ddx(0,12):=u10$
ddx(0,13):=u11$
ddx(0,14):=u12$
ddx(0,15):=u13$
ddx(0,16):=u14$
ddx(0,17):=u15$
ddx(0,18):=u16$
ddx(0,19):=u17$
ddx(0,20):=letop$
\end{verbatim}
The string \texttt{letop} is treated as a variable; if it appears during
computations it is likely that we went too close to the highest order variables
that we defined in the file. This could mean that we need to extend the
operators and variable list. In case of large output, one can search in it the
string \texttt{letop} to check whether errors occurred.

Specification of the vectorfield \texttt{ddt}. In the evolutionary case we
never have more than one time derivative, other derivatives are
$u_{txxx\cdots}$.
\begin{verbatim}
ddt(0,1):=0$
ddt(0,2):=1$
ddt(0,3):=ut$
ddt(0,4):=ut1$
ddt(0,5):=ut2$
ddt(0,6):=ut3$
ddt(0,7):=ut4$
ddt(0,8):=ut5$
ddt(0,9):=ut6$
ddt(0,10):=ut7$
ddt(0,11):=ut8$
ddt(0,12):=ut9$
ddt(0,13):=ut10$
ddt(0,14):=ut11$
ddt(0,15):=ut12$
ddt(0,16):=ut13$
ddt(0,17):=ut14$
ddt(0,18):=letop$
ddt(0,19):=letop$
sddt(0,20):=letop$
\end{verbatim}

We now give the equation in the form one of the derivatives equated to a
right-hand side expression. The left-hand side derivative is called
\emph{principal}, and the remaining derivatives are called
\emph{parametric}\footnote{This terminology dates back to Riquier, see
  \cite{Mar}}.  For scalar
evolutionary equations with two independent variables internal variables are of
the type $(t,x,u,u_x,u_{xx},\ldots)$.
\begin{verbatim}
ut:=u2+2*u*u1;
\end{verbatim}

\begin{verbatim}
ut1:=ddx ut;
ut2:=ddx ut1;
ut3:=ddx ut2;
ut4:=ddx ut3;
ut5:=ddx ut4;
ut6:=ddx ut5;
ut7:=ddx ut6;
ut8:=ddx ut7;
ut9:=ddx ut8;
ut10:=ddx ut9;
ut11:=ddx ut10;
ut12:=ddx ut11;
ut13:=ddx ut12;
ut14:=ddx ut13;
\end{verbatim}

Test for verifying the commutation of total derivatives.
Highest order defined terms may yield some \texttt{letop}.
\begin{verbatim}
operator ev;

for i:=1:17 do write ev(0,i):=ddt(ddx(0,i))-ddx(ddt(0,i));
\end{verbatim}

This is the list of variables with respect to their grading,
starting from degree \emph{one}.
\begin{verbatim}
graadlijst:={{u},{u1},{u2},{u3},{u4},{u5},
{u6},{u7},{u8},{u9},{u10},{u11},{u12},{u13},{u14},{u15},{u16},{u17}};
\end{verbatim}

This is the list of all monomials of degree $0$, $1$, $2$, \dots which can be
constructed from the above list of elementary variables with their grading.
\begin{verbatim}
grd0:={1};
grd1:= mkvarlist1(1,1)$
grd2:= mkvarlist1(2,2)$
grd3:= mkvarlist1(3,3)$
grd4:= mkvarlist1(4,4)$
grd5:= mkvarlist1(5,5)$
grd6:= mkvarlist1(6,6)$
grd7:= mkvarlist1(7,7)$
grd8:= mkvarlist1(8,8)$
grd9:= mkvarlist1(9,9)$
grd10:= mkvarlist1(10,10)$
grd11:= mkvarlist1(11,11)$
grd12:= mkvarlist1(12,12)$
grd13:= mkvarlist1(13,13)$
grd14:= mkvarlist1(14,14)$
grd15:= mkvarlist1(15,15)$
grd16:= mkvarlist1(16,16)$
\end{verbatim}

Initialize a counter \texttt{ctel} for arbitrary constants \texttt{c};
initialize equations:
\begin{verbatim}
operator c,equ;

ctel:=0;
\end{verbatim}

We assume a generating function \texttt{sym}, \emph{independent of $x$ and
  $t$}, of degree $\leq 5$.
\begin{verbatim}
sym:=
(for each el in grd0 sum (c(ctel:=ctel+1)*el))+
(for each el in grd1 sum (c(ctel:=ctel+1)*el))+
(for each el in grd2 sum (c(ctel:=ctel+1)*el))+
(for each el in grd3 sum (c(ctel:=ctel+1)*el))+
(for each el in grd4 sum (c(ctel:=ctel+1)*el))+
(for each el in grd5 sum (c(ctel:=ctel+1)*el))$
\end{verbatim}

This is the equation $\bar\ell_B(\mathtt{sym})=0$, where $B=0$ is
Burgers'equation and \texttt{sym} is the generating function. From now on all
equations are arranged in a single vector whose name is \texttt{equ}.
\begin{verbatim}
equ 1:=ddt(sym)-ddx(ddx(sym))-2*u*ddx(sym)-2*u1*sym ;
\end{verbatim}

This is the list of variables, to be passed to the equation solver.
\begin{verbatim}
vars:={x,t,u,u1,u2,u3,u4,u5,u6,u7,u8,u9,u10,u11,
u12,u13,u14,u15,u16,u17};
\end{verbatim}

This is the number of initial equation(s)
\begin{verbatim}
tel:=1;
\end{verbatim}

The following procedure uses \texttt{multi\_coeff} (from the package
\texttt{tools}).  It gets all coefficients of monomials appearing in the
initial equation(s).  The coefficients are put into the vector \texttt{equ}
after the initial equations.
\begin{verbatim}
procedure splitvars i;
begin;
ll:=multi_coeff(equ i,vars);
equ(tel:=tel+1):=first ll;
ll:=rest ll;
for each el in ll do equ(tel:=tel+1):=second el;
end;
\end{verbatim}

This command initializes the equation solver.  It passes
\begin{itemize}
  \item the equation vector \texttt{equ} togeher with its length \texttt{tel}
    (\emph{i.e.}, the total number of equations);
  \item the list of variables with respect to which the system \emph{must not}
    split the equations, \emph{i.e.}, variables with respect to which the
    unknowns are not polynomial. In this case this list is just $\{\}$;
  \item the constants'vector \texttt{c}, its length \texttt{ctel}, and the
    number of negative indexes if any; just $\texttt{0}$ in our example;
  \item the vector of free functions \texttt{f} that may appear in
    computations. Note that in \texttt{$\{$f,0,0 $\}$} the second $\texttt{0}$
    stands for the length of the vector of free functions. In this example
    there are no free functions, but the command needs the presence of at least
    a dummy argument, \texttt{f} in this case. There is also a last zero which
    is the negative length of the vector $f$, just as for constants.
  \end{itemize}
\begin{verbatim}
initialize_equations(equ,tel,{},{c,ctel,0},{f,0,0});
\end{verbatim}

Run the procedure splitvars in order to obtain equations on coefficiens
of each monomial.
\begin{verbatim}
splitvars 1;
\end{verbatim}

Next command tells the solver the total number of equations obtained
after running splitvars.
\begin{verbatim}
put_equations_used tel;
\end{verbatim}

It is worth to write down the equations for the coefficients.
\begin{verbatim}
for i:=2:tel do write equ i;
\end{verbatim}

This command solves the equations for the coefficients.
Note that we have to skip the initial equations!
\begin{verbatim}
for i:=2:tel do integrate_equation i;
;end;
\end{verbatim}

In the folder \texttt{computations/NewTests/Higher\_symmetries} it is possible
to find the following files:
\begin{description}
\item[Burg\_hsym\_1.red] The above file, together with its results file.
  \item[KdV\_hsym\_1.red] Higher symmetries of KdV, with the ansatz:
    deg(sym) $\leq$ 5.
\item[KdV\_hsym\_2.red] Higher symmetries of KdV, with the ansatz:
  \begin{center}
    sym = x*(something of degree 3) + t*(something of degree  5)\\
  + (something of degree 2).
  \end{center}
    This yields scale symmetries.
\item[KdV\_hsym\_3.red] Higher symmetries of KdV, with the ansatz:
  \begin{center}
    sym = x*(something of degree 1) + t*(something of degree 3)\\
    + (something of degree 0).
  \end{center}
This yields Galilean boosts.
\end{description}

\subsection{Local conservation laws}
\label{sec:local-cons-laws}

In this section we will find (some) local conservation laws for the KdV
equation $F=u_t-u_{xxx}+uu_x=0$. Concretely, we have to find non-trivial $1$-forms
$f=f_xdx+f_tdt$ on $F=0$ such that $\bar d f=0$ on $F=0$. ``Triviality'' of
conservation laws is a delicate matter, for which we invite the reader to have
a look in \cite{Many}.

The files containing this example is \texttt{KdV\_loc-cl\_1.red,
  KdV\_loc-cl\_2.red} and the corresponding results files.

We make use of \texttt{ddx} and \texttt{ddt}, which in the even part are the
same as in the previous example (subsection~\ref{sec:higher-symmetries}).
After defining the total derivatives we prepare the list of graded variables
(recall that in KdV $u$ is of degree $2$):
\begin{verbatim}
graadlijst:={{},{u},{u1},{u2},{u3},{u4},{u5},
{u6},{u7},{u8},{u9},{u10},{u11},{u12},{u13},{u14},{u15},{u16},{u17}};
\end{verbatim}
We make the ansatz
\begin{verbatim}
fx:=
(for each el in grd0 sum (c(ctel:=ctel+1)*el))+
(for each el in grd1 sum (c(ctel:=ctel+1)*el))+
(for each el in grd2 sum (c(ctel:=ctel+1)*el))+
(for each el in grd3 sum (c(ctel:=ctel+1)*el))$
ft:=
(for each el in grd2 sum (c(ctel:=ctel+1)*el))+
(for each el in grd3 sum (c(ctel:=ctel+1)*el))+
(for each el in grd4 sum (c(ctel:=ctel+1)*el))+
(for each el in grd5 sum (c(ctel:=ctel+1)*el))$
\end{verbatim}
for the components of the conservation law. We have to solve the equation
\begin{verbatim}
equ 1:=ddt(fx)-ddx(ft);
\end{verbatim}
the fact that \texttt{ddx} and \texttt{ddt} are expressed in internal
coordinates on the equation means that the objects that we consider are already
restricted to the equation.

We shall split the equation in its graded summands with the procedure
\texttt{splitvars}, then solve it
\begin{verbatim}
initialize_equations(equ,tel,{},{c,ctel,0},{f,0,0});
splitvars 1;
pte tel;
for i:=2:tel do es i;
end;
\end{verbatim}
As a result we get
\begin{verbatim}
fx := c(3)*u1 + c(2)*u + c(1)$
ft := (2*c(3)*u*u1 + 2*c(3)*u3 + c(2)*u**2 + 2*c(2)*u2)/2$
\end{verbatim}
Unfortunately it is clear that the conservation law corresponding to
\texttt{c(3)} is trivial, because it is the total $x$-derivative of $F$; its
restriction on the infinite prolongation of the KdV is zero. Here this fact is
evident; how to get rid of less evident trivialities by an `automatic'
mechanism? We considered this problem in the file \texttt{KdV\_loc-cl\_2.red},
where we solved the equation
\begin{verbatim}
equ 1:=fx-ddx(f0);
equ 2:=ft-ddt(f0);
\end{verbatim}
after having loaded the values \texttt{fx} and \texttt{ft} found by the
previous program. We make the following ansatz on \texttt{f0}:
\begin{verbatim}
f0:=
(for each el in grd0 sum (cc(cctel:=cctel+1)*el))+
(for each el in grd1 sum (cc(cctel:=cctel+1)*el))+
(for each el in grd2 sum (cc(cctel:=cctel+1)*el))+
(for each el in grd3 sum (cc(cctel:=cctel+1)*el))$
\end{verbatim}
Note that this gives a grading which is compatible with the gradings of
\texttt{fx} and \texttt{ft}. After solving the system
\begin{verbatim}
initialize_equations(equ,tel,{},{cc,cctel,0},{f,0,0});
for i:=1:2 do begin splitvars i;end;
put_equations_used tel;
for i:=3:tel do integrate_equation i;
end;
\end{verbatim}
issuing the commands
\begin{verbatim}
fxnontriv := fx-ddx(f0);
ftnontriv := ft-ddt(f0);
\end{verbatim}
we obtain
\begin{verbatim}
fxnontriv := c(2)*u + c(1)$
ftnontriv := (c(2)*(u**2 + 2*u2))/2$
\end{verbatim}
This mechanism can be easily generalized to situations in which the
conservation laws which are found by the program are difficult to treat by pen
and paper.


\subsection{Local Hamiltonian operators}
\label{sec:local-hamilt-oper}

In this section we will find local Hamiltonian operators for the KdV equation
$u_t=u_{xxx}+uu_x$. Concretely, we have to solve $\bar
\ell_{KdV}(\mathtt{phi})=0$ over the equation
\begin{displaymath}
  \left\{\begin{array}{l}
    u_t=u_{xxx}+uu_x\\
    p_t=p_{xxx}+up_x
  \end{array}\right.
\end{displaymath}
or, in geometric terminology, find the shadows of symmetries on the
$\ell^*$-covering of the KdV equation. The reference paper for this type of
computations is \cite{KKV}.

The file containing this example is \texttt{KdV\_Ham\_1.red}.

We make use of \texttt{ddx} and \texttt{ddt}, which in the even part are the
same as in the previous example (subsection~\ref{sec:higher-symmetries}).  We
stress that the linearization $\bar \ell_{KdV}(\mathtt{phi})=0$ is the equation
\begin{verbatim}
ddt(phi)-u*ddx(phi)-u1*phi-ddx(ddx(ddx(phi)))=0
\end{verbatim}
but the total derivatives are lifted to the $\ell^*$ covering, hence they must
contain also derivatives with respect to $p$'s. This will be achieved by
treating $p$ variables as odd and introducing the odd parts of \texttt{ddx} and
\texttt{ddt},
\begin{verbatim}
ddx(1,1):=0$
ddx(1,2):=0$
ddx(1,3):=ext 4$
ddx(1,4):=ext 5$
ddx(1,5):=ext 6$
ddx(1,6):=ext 7$
ddx(1,7):=ext 8$
ddx(1,8):=ext 9$
ddx(1,9):=ext 10$
ddx(1,10):=ext 11$
ddx(1,11):=ext 12$
ddx(1,12):=ext 13$
ddx(1,13):=ext 14$
ddx(1,14):=ext 15$
ddx(1,15):=ext 16$
ddx(1,16):=ext 17$
ddx(1,17):=ext 18$
ddx(1,18):=ext 19$
ddx(1,19):=ext 20$
ddx(1,20):=letop$
\end{verbatim}
In the above definition the first index `1' says that we are dealing with odd
variables, \texttt{ext} indicates anticommuting variables. Here, \texttt{ext 3}
is $p_0$, \texttt{ext 4} is $p_x$, \texttt{ext 5} is $p_{xx}$, \dots so
\texttt{ddx(1,3):=ext 4} indicates $p_x\pd{}{p}$, etc..

Now, remembering that the additional equation is again evolutionary, we can get
rid of $p_t$ by letting it be equal to \texttt{ext 6 + u*ext 4}, as follows:
\begin{verbatim}
ddt(1,1):=0$
ddt(1,2):=0$
ddt(1,3):=ext 6 + u*ext 4$
ddt(1,4):=ddx(ddt(1,3))$
ddt(1,5):=ddx(ddt(1,4))$
ddt(1,6):=ddx(ddt(1,5))$
ddt(1,7):=ddx(ddt(1,6))$
ddt(1,8):=ddx(ddt(1,7))$
ddt(1,9):=ddx(ddt(1,8))$
ddt(1,10):=ddx(ddt(1,9))$
ddt(1,11):=ddx(ddt(1,10))$
ddt(1,12):=ddx(ddt(1,11))$
ddt(1,13):=ddx(ddt(1,12))$
ddt(1,14):=ddx(ddt(1,13))$
ddt(1,15):=ddx(ddt(1,14))$
ddt(1,16):=ddx(ddt(1,15))$
ddt(1,17):=ddx(ddt(1,16))$
ddt(1,18):=letop$
ddt(1,19):=letop$
ddt(1,20):=letop$
\end{verbatim}

Let us make the following ansatz about the Hamiltonian operators:
\begin{verbatim}
phi:=
(for each el in grd0 sum (c(ctel:=ctel+1)*el))*ext 3+
(for each el in grd1 sum (c(ctel:=ctel+1)*el))*ext 3+
(for each el in grd2 sum (c(ctel:=ctel+1)*el))*ext 3+
(for each el in grd3 sum (c(ctel:=ctel+1)*el))*ext 3+

(for each el in grd0 sum (c(ctel:=ctel+1)*el))*ext 4+
(for each el in grd1 sum (c(ctel:=ctel+1)*el))*ext 4+
(for each el in grd2 sum (c(ctel:=ctel+1)*el))*ext 4+

(for each el in grd0 sum (c(ctel:=ctel+1)*el))*ext 5+
(for each el in grd1 sum (c(ctel:=ctel+1)*el))*ext 5+

(for each el in grd0 sum (c(ctel:=ctel+1)*el))*ext 6
$
\end{verbatim}
Note that we are looking for generating functions of shadows which are
\emph{linear} with respect to $p$'s. Moreover, having set $[p]=-2$ we will look
for solutions of maximal possible degree $+1$.

After having set
\begin{verbatim}
equ 1:=ddt(phi)-u*ddx(phi)-u1*phi-ddx(ddx(ddx(phi)));
vars:={x,t,u,u1,u2,u3,u4,u5,u6,u7,u8,u9,u10,u11,u12,u13,u14,u15,u16,u17};
tel:=1;
\end{verbatim}
we define the procedures \texttt{splitvars} as in
subsection~\ref{sec:higher-symmetries} and \texttt{splitext} as follows:
\begin{verbatim}
procedure splitext i;
begin;
ll:=operator_coeff(equ i,ext);
equ(tel:=tel+1):=first ll;
ll:=rest ll;
for each el in ll do equ(tel:=tel+1):=second el;
end;
\end{verbatim}
Then we initialize the equations:
\begin{verbatim}
initialize_equations(equ,tel,{},{c,ctel,0},{f,0,0});
\end{verbatim}
do \texttt{splitext}
\begin{verbatim}
splitext 1;
\end{verbatim}
then \texttt{splitvars}
\begin{verbatim}
tel1:=tel;
for i:=2:tel1 do begin splitvars i;equ i:=0;end;
\end{verbatim}
Now we are ready to solve all equations:
\begin{verbatim}
put_equations_used tel;
for i:=2:tel do write equ i:=equ i;
pause;
for i:=2:tel do integrate_equation i;
end;
\end{verbatim}
Note that we want \emph{all} equations to be solved!

The results are the two well-known Hamiltonian operators for the KdV:
\begin{verbatim}
phi := c(4)*ext(4) + 3*c(3)*ext(6) + 2*c(3)*ext(4)*u + c(3)*ext(3)*u1$
\end{verbatim}
Of course, the results correspond to the operators
\begin{gather*}
  \text{\texttt{ext(4)}} \to D_x,\\
  \text{\texttt{3*c(3)*ext(6) +
  2*c(3)*ext(4)*u + c(3)*ext(3)*u1}} \to 3D_{xxx} + 2uD_{x} + u_x
\end{gather*}
Note that each operator is multiplied by one arbitrary real
constant, \texttt{c(4)} and \texttt{c(3)}.

\subsection{Non-local Hamiltonian operators}
\label{sec:non-local-hamilt}

In this section we will show an experimental way to find nonlocal Hamiltonian
operators for the KdV equation. The word `experimental' comes from the lack of
a consistent mathematical theory. The result of the computation (without the
details below) has been published in \cite{KKV}.

We have to solve equations of the type \texttt{ddx(ft)-ddt(fx)} as
in~\ref{sec:local-cons-laws}. The main difference is that we will attempt a
solution on the $\ell^*$-covering (see Subsection~\ref{sec:local-hamilt-oper}).
For this reason, first of all we have to determine covering variables with the
usual mechanism of introducing them through conservation laws, this time on the
$\ell^*$-covering.

As a first step, let us compute conservation laws on the $\ell^*$-covering
whose components are linear in the $p$'s.  This computation can be found in the
file \texttt{KdV\_nloc-cl\_1.red} and related results file. When specifying odd
variables in \texttt{ddx} and \texttt{ddt}, we have something like
\begin{verbatim}
ddx(1,1):=0$
ddx(1,2):=0$
ddx(1,3):=ext 4$
ddx(1,4):=ext 5$
ddx(1,5):=ext 6$
ddx(1,6):=ext 7$
ddx(1,7):=ext 8$
ddx(1,8):=ext 9$
ddx(1,9):=ext 10$
ddx(1,10):=ext 11$
ddx(1,11):=ext 12$
ddx(1,12):=ext 13$
ddx(1,13):=ext 14$
ddx(1,14):=ext 15$
ddx(1,15):=ext 16$
ddx(1,16):=ext 17$
ddx(1,17):=ext 18$
ddx(1,18):=ext 19$
ddx(1,19):=ext 20$
ddx(1,20):=letop$
ddx(1,50):=(t*u1+1)*ext 3$ % degree -2
ddx(1,51):=u1*ext 3$ % degree +1
ddx(1,52):=(u*u1+u3)*ext 3$ % degree +3
\end{verbatim}
and
\begin{verbatim}
ddt(1,1):=0$
ddt(1,2):=0$
ddt(1,3):=ext 6 + u*ext 4$
ddt(1,4):=ddx(ddt(1,3))$
ddt(1,5):=ddx(ddt(1,4))$
ddt(1,6):=ddx(ddt(1,5))$
ddt(1,7):=ddx(ddt(1,6))$
ddt(1,8):=ddx(ddt(1,7))$
ddt(1,9):=ddx(ddt(1,8))$
ddt(1,10):=ddx(ddt(1,9))$
ddt(1,11):=ddx(ddt(1,10))$
ddt(1,12):=ddx(ddt(1,11))$
ddt(1,13):=ddx(ddt(1,12))$
ddt(1,14):=ddx(ddt(1,13))$
ddt(1,15):=ddx(ddt(1,14))$
ddt(1,16):=ddx(ddt(1,15))$
ddt(1,17):=ddx(ddt(1,16))$
ddt(1,18):=letop$
ddt(1,19):=letop$
ddt(1,20):=letop$
ddt(1,50):=f1*ext 3+f2*ext 4+f3*ext 5$
ddt(1,51):=f4*ext 3+f5*ext 4+f6*ext 5$
ddt(1,52):=f7*ext 3+f8*ext 4+f9*ext 5$
\end{verbatim}
The variables corresponding to the numbers \texttt{50,51,52} here play a dummy
role, the coefficients of the corresponding vector are the unknown generating
functions of conservation laws on the $\ell^*$-covering.
More precisely, we look for conservation laws
of the form
\begin{verbatim}
fx= phi*ext 3
ft= f1*ext3+f2*ext4+f3*ext5
\end{verbatim}
The ansatz is chosen because, first of all, \texttt{ext 4} and \texttt{ext 5}
can be removed from fx by adding a suitable total divergence (trivial
conservation law); moreover it can be proved that \texttt{phi} is a symmetry of
KdV. We can write down the equations
\begin{verbatim}
equ 1:=ddx(ddt(1,50))-ddt(ddx(1,50));
equ 2:=ddx(ddt(1,51))-ddt(ddx(1,51));
equ 3:=ddx(ddt(1,52))-ddt(ddx(1,52));
\end{verbatim}
However, the above choices make use of a symmetry which contains \texttt{`t'}
in the generator. This would make automatic computations more tricky, but still
possible. In this case the solution of \texttt{equ 1} has been found by hand
and passed to the program:
\begin{verbatim}
f3:=t*u1+1$
f1:=u*f3+ddx(ddx(f3))$
f2:=-ddx(f3)$
\end{verbatim}
together with the ansatz on the coefficients for the other equations
\begin{verbatim}
f4:=(for each el in grd5 sum (c(ctel:=ctel+1)*el))$
f5:=(for each el in grd4 sum (c(ctel:=ctel+1)*el))$
f6:=(for each el in grd3 sum (c(ctel:=ctel+1)*el))$

f7:=(for each el in grd7 sum (c(ctel:=ctel+1)*el))$
f8:=(for each el in grd6 sum (c(ctel:=ctel+1)*el))$
f9:=(for each el in grd5 sum (c(ctel:=ctel+1)*el))$
\end{verbatim}
The previous ansatz keep into account the grading of the starting
symmetry in \texttt{phi*ext 3}. The resulting equations are solved in the usual
way (see the example file).

Now, we solve the equation for shadows of nonlocal symmetries in a covering of
the $\ell^*$-covering. We can choose between three new nonlocal variables
\texttt{ra,rb,rc}. We are going to look for non-local Hamiltonian operators
depending linearly on one of these variables. Higher non-local Hamiltonian 
operators could be found by introducing total derivatives of the r's.
As usual, the new variables are specified through the components of the 
previously found conservation laws according with the rule
\begin{verbatim}
ra_x=fx, ra_t=ft,
\end{verbatim}
and analogously for the others. We define
\begin{verbatim}
ddx(1,50):=(t*u1+1)*ext 3$ % degree -2
ddx(1,51):=u1*ext 3$ % degree +1
ddx(1,52):=(u*u1+u3)*ext 3$ % degree +3
\end{verbatim}
and
\begin{verbatim}
ddt(1,50) := ext(5)*t*u1 + ext(5) - ext(4)*t*u2 + ext(3)*t*u*u1 +
ext(3)*t*u3 + ext(3)*u$
ddt(1,51) := ext(5)*u1 - ext(4)*u2 + ext(3)*u*u1 + ext(3)*u3$
ddt(1,52) := ext(5)*u*u1 + ext(5)*u3 - ext(4)*u*u2 - ext(4)*u1**2 -
ext(4)*u4 + ext(3)*u**2*u1 + 2*ext(3)*u*u3 + 3*ext(3)*u1*u2 + ext(3)*u5$
\end{verbatim}
as it results from the computation of the conservation laws.
The following ansatz for the nonlocal Hamiltonian operator
comes from the fact that local Hamiltonian operators have 
gradings $-1$ and $+1$ when written in terms of $p$'s. So we are looking 
for a nonlocal Hamiltonian operator of degree $3$.
\begin{verbatim}
phi:=
(for each el in grd6 sum (c(ctel:=ctel+1)*el))*ext 50+
(for each el in grd3 sum (c(ctel:=ctel+1)*el))*ext 51+
(for each el in grd1 sum (c(ctel:=ctel+1)*el))*ext 52+

(for each el in grd5 sum (c(ctel:=ctel+1)*el))*ext 3+
(for each el in grd4 sum (c(ctel:=ctel+1)*el))*ext 4+
(for each el in grd3 sum (c(ctel:=ctel+1)*el))*ext 5+
(for each el in grd2 sum (c(ctel:=ctel+1)*el))*ext 6+
(for each el in grd1 sum (c(ctel:=ctel+1)*el))*ext 7+
(for each el in grd0 sum (c(ctel:=ctel+1)*el))*ext 8
$
\end{verbatim}
As a solution, we obtain
\begin{verbatim}
phi := c(1)*(ext(51)*u1 - 9*ext(8) - 12*ext(6)*u - 18*ext(5)*u1 -
4*ext(4)*u**2 - 12*ext(4)*u2 - 4*ext(3)*u*u1 - 3*ext(3)*u3)$
\end{verbatim}
where \texttt{ext51} stands for the nonlocal variable \texttt{rb} fulfilling
\begin{verbatim}
rb_x:=u1*ext 3$
rb_t:=ext(5)*u1 - ext(4)*u2 + ext(3)*u*u1 + ext(3)*u3$
\end{verbatim}

\textbf{Remark.} In the file \texttt{KdV\_nloc-Ham\_2.red} it is possible to
find another ansatz for a non-local Hamiltonian operator of degree $+5$.

\subsection{Computations for systems of PDEs}
\label{sec:comp-syst-pdes}

There is no conceptual difference when computing for systems of PDEs. We will
look for Hamiltonian structures for the following Boussinesq equation:
\begin{equation}
  \label{eq:1}
  \left\{
  \begin{array}{l}
  u_t-u_xv-uv_x-\sigma v_{xxx}=0\\
  v_t-u_x-vv_x=0
\end{array}
\right.
\end{equation}
where $\sigma$ is a constant. This example also shows how to deal with jet
spaces with more than one dependent variable. Here gradings can be taken as
\begin{displaymath}
  [t]=-2,\quad [x]=-1,\quad [v]=1,\quad [u]=2,\quad [p]=[\pd{}{u}]=-2,\quad
  [q]=[\pd{}{v}]=-1
\end{displaymath}
where $p$, $q$ are the two coordinates in the space of generating functions of
conservation laws.

The linearization of the above system and its adjoint are, respectively
\begin{displaymath}
  \ell_{\text{Bou}}=
  \begin{pmatrix}
    D_t-vD_x-v_x & -u_x-uD_x-\sigma D_{xxx}\\
    -D_x & D_t-v_x-vD_x
  \end{pmatrix},\ 
  \ell^*_{\text{Bou}}=
  \begin{pmatrix}
    -D_t+vD_x & D_x\\
    uD_x+\sigma D_{xxx} & -D_t+vD_x
  \end{pmatrix}
\end{displaymath}
and lead to the $\ell^*_{\text{Bou}}$ covering equation
\begin{displaymath}
  \label{eq:2}
  \left\{
  \begin{array}{l}
    -p_t+vp_x+q_x=0\\
    up_x+\sigma p_{xxx}-q_t+vq_x=0\\
  u_t-u_xv-uv_x-\sigma v_{xxx}=0\\
  v_t-u_x-vv_x=0
\end{array}
\right.
\end{displaymath}
We have to find shadows of symmetries on the above covering.
Total derivatives must be defined as follows:
\begin{verbatim}
super_vectorfield(ddx,{x,t,u,v,u1,v1,u2,v2,u3,v3,u4,v4,u5,v5,u6,v6,u7,
v7,u8,v8,u9,v9,u10,v10,u11,v11,u12,v12,u13,v13,u14,v14,u15,v15,
u16,v16,u17,v17},
{ext 1,ext 2,ext 3,ext 4,ext 5,ext 6,ext 7,ext 8,ext 9,ext 10,ext
11,ext 12,ext 13,ext 14,ext 15,ext 16,ext 17,ext 18,ext 19,ext 20,ext
21,ext 22,ext 23,ext 24,ext 25,ext 26,ext 27,ext 28,ext 29,ext 30,
ext 31,ext 32,ext 33,ext 34,ext 35,ext 36,ext 37,ext 38,ext 39,ext 40,
ext 41,ext 42,ext 43,ext 44,ext 45,ext 46,ext 47,ext 48,ext 49,ext 50,
ext 51,ext 52,ext 53,ext 54,ext 55,ext 56,ext 57,ext 58,ext 59,ext 60,
ext 61,ext 62,ext 63,ext 64,ext 65,ext 66,ext 67,ext 68,ext 69,ext 70,
ext 71,ext 72,ext 73,ext 74,ext 75,ext 76,ext 77,ext 78,ext 79,ext 80
});

super_vectorfield(ddt,{x,t,u,v,u1,v1,u2,v2,u3,v3,u4,v4,u5,v5,u6,v6,u7,
v7,u8,v8,u9,v9,u10,v10,u11,v11,u12,v12,u13,v13,u14,v14,u15,v15,
u16,v16,u17,v17},
{ext 1,ext 2,ext 3,ext 4,ext 5,ext 6,ext 7,ext 8,ext 9,ext 10,ext
11,ext 12,ext 13,ext 14,ext 15,ext 16,ext 17,ext 18,ext 19,ext 20,ext
21,ext 22,ext 23,ext 24,ext 25,ext 26,ext 27,ext 28,ext 29,ext 30,
ext 31,ext 32,ext 33,ext 34,ext 35,ext 36,ext 37,ext 38,ext 39,ext 40,
ext 41,ext 42,ext 43,ext 44,ext 45,ext 46,ext 47,ext 48,ext 49,ext 50,
ext 51,ext 52,ext 53,ext 54,ext 55,ext 56,ext 57,ext 58,ext 59,ext 60,
ext 61,ext 62,ext 63,ext 64,ext 65,ext 66,ext 67,ext 68,ext 69,ext 70,
ext 71,ext 72,ext 73,ext 74,ext 75,ext 76,ext 77,ext 78,ext 79,ext 80
});
\end{verbatim}
In the list of coordinates we alternate derivatives of $u$ and derivatives of
$v$. The same must be done for coefficients; for example,
\begin{verbatim}
ddx(0,1):=1$
ddx(0,2):=0$
ddx(0,3):=u1$
ddx(0,4):=v1$
ddx(0,5):=u2$
ddx(0,6):=v2$
...
\end{verbatim}
After specifying the equation
\begin{verbatim}
ut:=u1*v+u*v1+sig*v3;
vt:=u1+v*v1;
\end{verbatim}
we define the (already introduced) time derivatives:
\begin{verbatim}
ut1:=ddx ut;
ut2:=ddx ut1;
ut3:=ddx ut2;
...
vt1:=ddx vt;
vt2:=ddx vt1;
vt3:=ddx vt2;
...
\end{verbatim}
up to the required order (here the order can be stopped at $15$).  Odd
variables $p$ and $q$ must be specified with an appropriate length (here it is
OK to stop at \texttt{ddx(1,36)}). Recall to replace $p_t$, $q_t$ with the
internal coordinates of the covering:
\begin{verbatim}
ddt(1,1):=0$
ddt(1,2):=0$
ddt(1,3):=+v*ext 5+ext 6$
ddt(1,4):=u*ext 5+sig*ext 9+v*ext 6$
ddt(1,5):=ddx(ddt(1,3))$
...
\end{verbatim}
The list of graded variables:
\begin{verbatim}
graadlijst:={{v},{u,v1},{u1,v2},{u2,v3},{u3,v4},{u4,v5},
{u5,v6},{u6,v7},{u7,v8},{u8,v9},{u9,v10},{u10,v11},{u11,v12},{u12,v13},
{u13,v14},{u14,v15},{u15,v16},{u16,v17},{u17}};
\end{verbatim}
The ansatz for the components of the Hamiltonian operator is
\begin{verbatim}
phi1:=
(for each el in grd2 sum (c(ctel:=ctel+1)*el))*ext 3+
(for each el in grd1 sum (c(ctel:=ctel+1)*el))*ext 5+
(for each el in grd1 sum (c(ctel:=ctel+1)*el))*ext 4+
(for each el in grd0 sum (c(ctel:=ctel+1)*el))*ext 6
$

phi2:=
(for each el in grd1 sum (c(ctel:=ctel+1)*el))*ext 3+
(for each el in grd0 sum (c(ctel:=ctel+1)*el))*ext 5+
(for each el in grd0 sum (c(ctel:=ctel+1)*el))*ext 4
$
\end{verbatim}
and the equation for shadows of symmetries is
\begin{verbatim}
equ 1:=ddt(phi1)-v*ddx(phi1)-v1*phi1-u1*phi2-u*ddx(phi2)
-sig*ddx(ddx(ddx(phi2)));
equ 2:=-ddx(phi1)-v*ddx(phi2)-v1*phi2+ddt(phi2);
\end{verbatim}
After the usual procedures for decomposing polynomials we obtain the following
result:
\begin{verbatim}
phi1 := c(6)*ext(6)$
phi2 := c(6)*ext(5)$
\end{verbatim}
which corresponds to the vector $(D_x,D_x)$.  Extending the ansatz to
\begin{verbatim}
phi1:=
(for each el in grd3 sum (c(ctel:=ctel+1)*el))*ext 3+
(for each el in grd2 sum (c(ctel:=ctel+1)*el))*ext 5+
(for each el in grd1 sum (c(ctel:=ctel+1)*el))*ext 7+
(for each el in grd0 sum (c(ctel:=ctel+1)*el))*ext 9+
(for each el in grd2 sum (c(ctel:=ctel+1)*el))*ext 4+
(for each el in grd1 sum (c(ctel:=ctel+1)*el))*ext 6+
(for each el in grd0 sum (c(ctel:=ctel+1)*el))*ext 8
$

phi2:=
(for each el in grd2 sum (c(ctel:=ctel+1)*el))*ext 3+
(for each el in grd1 sum (c(ctel:=ctel+1)*el))*ext 5+
(for each el in grd0 sum (c(ctel:=ctel+1)*el))*ext 7+
(for each el in grd1 sum (c(ctel:=ctel+1)*el))*ext 4+
(for each el in grd0 sum (c(ctel:=ctel+1)*el))*ext 6
$
\end{verbatim}
allows us to find a second (local) Hamiltonian operator
\begin{verbatim}
phi1 := (c(3)*(2*ext(9)*sig + ext(6)*v + 2*ext(5)*u + ext(3)*u1))/2$
phi2 := (c(3)*(2*ext(6) + ext(5)*v + ext(3)*v1))/2$
\end{verbatim}
There is one more higher local Hamiltonian operator, and a whole hierarchy of
nonlocal Hamiltonian operators~\cite{KKV}.


\subsection{Explosion of denominators and how to avoid it}

Here we propose the computation of the repeated total derivative of a
denominator. This computation fills up the whole memory after some time, and
can be used as a kind of speed test for the system. The file is
\texttt{KdV\_denom\_1.red}.

After having defined total derivatives on the KdV equation, run the following
iteration:
\begin{verbatim}
phi:=1/(u3+u*u1)$
for i:=1:100 do begin
                phi:=ddx(phi)$
                write i;
end;
\end{verbatim}
The program shows the iteration number. At the 18th iteration the program uses
about 600MB of RAM, as shown by \texttt{top} run from another shell, and 100\%\
of one processor.

There is a simple way to avoid denominator explosion. The file is
\texttt{KdV\_denom\_2.red}.

After having defined total derivatives with respect to $x$ (on the KdV
equation, for example) consider in the same \texttt{ddx} a component with a
sufficently high index \textbf{immediately after `letop'} (otherwise
\texttt{super\_vectorfield} does not work!), say \texttt{ddx(0,21)}, and think
of it as being the coefficient to a vector of the type
\begin{verbatim}
aa21:=1/(u3+u*u1);
\end{verbatim}
In this case, its coefficient must be
\begin{verbatim}
ddx(0,21):=-aa21**2*(u4+u1**2+u*u2)$
\end{verbatim}
More particularly, here follows the detailed definition of \texttt{ddx}
\begin{verbatim}
ddx(0,1):=1$
ddx(0,2):=0$
ddx(0,3):=u1$
ddx(0,4):=u2$
ddx(0,5):=u3$
ddx(0,6):=u4$
ddx(0,7):=u5$
ddx(0,8):=u6$
ddx(0,9):=u7$
ddx(0,10):=u8$
ddx(0,11):=u9$
ddx(0,12):=u10$
ddx(0,13):=u11$
ddx(0,14):=u12$
ddx(0,15):=u13$
ddx(0,16):=u14$
ddx(0,17):=u15$
ddx(0,18):=u16$
ddx(0,19):=u17$
ddx(0,20):=letop$
ddx(0,21):=-aa21**2*(u4+u1**2+u*u2)$
\end{verbatim}

Now, suppose that we want to compute the 5th total derivative of
\texttt{phi}. Write the following code:
\begin{verbatim}
phi:=aa30;
for i:=1:5 do begin
                phi:=ddx(phi)$
                write i;
end;
\end{verbatim}
The result is then a polynomial in the additional `denominator' variable
\begin{verbatim}
phi := aa21**2*( - 120*aa21**4*u**5*u2**5 - 600*aa21**4*u**4*u1**2*u2**4 - 600*
aa21**4*u**4*u2**4*u4 - 1200*aa21**4*u**3*u1**4*u2**3 - 2400*aa21**4*u**3*u1**2*
u2**3*u4 - 1200*aa21**4*u**3*u2**3*u4**2 - 1200*aa21**4*u**2*u1**6*u2**2 - 3600*
aa21**4*u**2*u1**4*u2**2*u4 - 3600*aa21**4*u**2*u1**2*u2**2*u4**2 - 1200*aa21**4
*u**2*u2**2*u4**3 - 600*aa21**4*u*u1**8*u2 - 2400*aa21**4*u*u1**6*u2*u4 - 3600*
aa21**4*u*u1**4*u2*u4**2 - 2400*aa21**4*u*u1**2*u2*u4**3 - 600*aa21**4*u*u2*u4**
4 - 120*aa21**4*u1**10 - 600*aa21**4*u1**8*u4 - 1200*aa21**4*u1**6*u4**2 - 1200*
aa21**4*u1**4*u4**3 - 600*aa21**4*u1**2*u4**4 - 120*aa21**4*u4**5 + 240*aa21**3*
u**4*u2**3*u3 + 720*aa21**3*u**3*u1**2*u2**2*u3 + 720*aa21**3*u**3*u1*u2**4 + 
240*aa21**3*u**3*u2**3*u5 + 720*aa21**3*u**3*u2**2*u3*u4 + 720*aa21**3*u**2*u1**
4*u2*u3 + 2160*aa21**3*u**2*u1**3*u2**3 + 720*aa21**3*u**2*u1**2*u2**2*u5 + 1440
*aa21**3*u**2*u1**2*u2*u3*u4 + 2160*aa21**3*u**2*u1*u2**3*u4 + 720*aa21**3*u**2*
u2**2*u4*u5 + 720*aa21**3*u**2*u2*u3*u4**2 + 240*aa21**3*u*u1**6*u3 + 2160*aa21
**3*u*u1**5*u2**2 + 720*aa21**3*u*u1**4*u2*u5 + 720*aa21**3*u*u1**4*u3*u4 + 4320
*aa21**3*u*u1**3*u2**2*u4 + 1440*aa21**3*u*u1**2*u2*u4*u5 + 720*aa21**3*u*u1**2*
u3*u4**2 + 2160*aa21**3*u*u1*u2**2*u4**2 + 720*aa21**3*u*u2*u4**2*u5 + 240*aa21
**3*u*u3*u4**3 + 720*aa21**3*u1**7*u2 + 240*aa21**3*u1**6*u5 + 2160*aa21**3*u1**
5*u2*u4 + 720*aa21**3*u1**4*u4*u5 + 2160*aa21**3*u1**3*u2*u4**2 + 720*aa21**3*u1
**2*u4**2*u5 + 720*aa21**3*u1*u2*u4**3 + 240*aa21**3*u4**3*u5 - 60*aa21**2*u**3*
u2**2*u4 - 90*aa21**2*u**3*u2*u3**2 - 120*aa21**2*u**2*u1**2*u2*u4 - 90*aa21**2*
u**2*u1**2*u3**2 - 780*aa21**2*u**2*u1*u2**2*u3 - 180*aa21**2*u**2*u2**4 - 60*
aa21**2*u**2*u2**2*u6 - 180*aa21**2*u**2*u2*u3*u5 - 120*aa21**2*u**2*u2*u4**2 - 
90*aa21**2*u**2*u3**2*u4 - 60*aa21**2*u*u1**4*u4 - 1020*aa21**2*u*u1**3*u2*u3 - 
1170*aa21**2*u*u1**2*u2**3 - 120*aa21**2*u*u1**2*u2*u6 - 180*aa21**2*u*u1**2*u3*
u5 - 120*aa21**2*u*u1**2*u4**2 - 540*aa21**2*u*u1*u2**2*u5 - 1020*aa21**2*u*u1*
u2*u3*u4 - 360*aa21**2*u*u2**3*u4 - 120*aa21**2*u*u2*u4*u6 - 90*aa21**2*u*u2*u5
**2 - 180*aa21**2*u*u3*u4*u5 - 60*aa21**2*u*u4**3 - 240*aa21**2*u1**5*u3 - 990*
aa21**2*u1**4*u2**2 - 60*aa21**2*u1**4*u6 - 540*aa21**2*u1**3*u2*u5 - 480*aa21**
2*u1**3*u3*u4 - 1170*aa21**2*u1**2*u2**2*u4 - 120*aa21**2*u1**2*u4*u6 - 90*aa21
**2*u1**2*u5**2 - 540*aa21**2*u1*u2*u4*u5 - 240*aa21**2*u1*u3*u4**2 - 180*aa21**
2*u2**2*u4**2 - 60*aa21**2*u4**2*u6 - 90*aa21**2*u4*u5**2 + 10*aa21*u**2*u2*u5 +
 20*aa21*u**2*u3*u4 + 10*aa21*u*u1**2*u5 + 110*aa21*u*u1*u2*u4 + 80*aa21*u*u1*u3
**2 + 160*aa21*u*u2**2*u3 + 10*aa21*u*u2*u7 + 20*aa21*u*u3*u6 + 30*aa21*u*u4*u5 
+ 50*aa21*u1**3*u4 + 340*aa21*u1**2*u2*u3 + 10*aa21*u1**2*u7 + 180*aa21*u1*u2**3
 + 60*aa21*u1*u2*u6 + 80*aa21*u1*u3*u5 + 50*aa21*u1*u4**2 + 60*aa21*u2**2*u5 + 
100*aa21*u2*u3*u4 + 10*aa21*u4*u7 + 20*aa21*u5*u6 - u*u6 - 6*u1*u5 - 15*u2*u4 - 
10*u3**2 - u8)$
\end{verbatim}
where the value of \texttt{aa21} can be replaced back in the expression.


\begin{thebibliography}{99}

\bibitem{red} Obtaining \reduce: \url{http://reduce-algebra.sourceforge.net/}.
  \bibitem{gdeq} Geometry of Differential Equations web site:
    \url{http://gdeq.org}.
  \bibitem{noteppp} \texttt{notepad++}:
    \url{http://notepad-plus.sourceforge.net/}
  \bibitem{ed} List of text editors:
    \url{http://en.wikipedia.org/wiki/List_of_text_editors}
  \bibitem{emacswin} How to install \texttt{emacs} in Windows:
    \url{http://www.cmc.edu/math/alee/emacs/emacs.html}. See also
    \url{http://www.gnu.org/software/emacs/windows/ntemacs.html}
  \bibitem{reducewin} How to install \reduce in Windows:
    \url{http://reduce-algebra.sourceforge.net/windows.html}
  \bibitem{svec} \textsc{G.H.M. Roelofs}, The SUPER VECTORFIELD package for
    REDUCE. Version 1.0, Memorandum 1099, Dept. Appl. Math., University of
    Twente, 1992. Available at \url{http://gdeq.org}.
  \bibitem{integ} \textsc{G.H.M. Roelofs}, The INTEGRATOR package for
    REDUCE. Version 1.0, Memorandum 1100, Dept. Appl. Math., University of
    Twente, 1992. Available at \url{http://gdeq.org}.
  \bibitem{tools} \textsc{G.F. Post}, A manual for the package TOOLS 2.1,
    Memorandum 1331, Dept. Appl. Math., University of Twente, 1996. Available
    at \url{http://gdeq.org}.
  \bibitem{redide} \reduce IDE for \texttt{emacs}:
    \url{http://centaur.maths.qmul.ac.uk/Emacs/REDUCE_IDE/}
  \bibitem{Many} \textsc{A. V. Bocharov, V. N. Chetverikov, S. V.  Duzhin, N.
      G.  Khor{\cprime}kova, I. S.  Krasil{\cprime}shchik, A.  V.  Samokhin,
      Yu.\ N.  Torkhov, A. M. Verbovetsky and A. M.  Vinogradov}: Symmetries
    and Conservation Laws for Differential Equations of Mathematical Physics,
    I.  S.  Krasil{\cprime}shchik and A. M.  Vinogradov eds., Translations of
    Math.  Monographs \textbf{182}, Amer.\ Math.\ Soc. (1999).
  \bibitem{KKV} \textsc{P.H.M. Kersten, I.S. Krasil'shchik, A.M. Verbovetsky,}
    \emph{Hamiltonian operators and $\ell^*$-covering}, Journal of Geometry and
    Physics \textbf{50} (2004), 273--302.
  \bibitem{Mar} \textsc{M. Marvan}, \emph{Sufficient set of integrability
    conditions of an orthonomic system}.  Foundations of Computational
    Mathematics \textbf{9} (2009) 651--674.
\end{thebibliography}

\end{document}
