\documentstyle[12pt]{article}
\hoffset=-0.5in \addtolength{\topmargin}{-30pt}
\addtolength{\textheight}{120pt} \addtolength{\textwidth}{70pt}
\begin{document}

\bigskip

\bigskip

\begin{center}

{\huge {THE SYSTEM ORTOCARTAN}} \\

\bigskip

{\huge{USER'S MANUAL}} \\

\vspace{4cm}

{\Large {Andrzej Krasi\'nski}} \\

{\large{N. Copernicus Astronomical Center, Polish Academy of Sciences}} \\

{\large{Bartycka 18, 00-716 Warszawa, Poland}} \\

email: akr@camk.edu.pl \\

\bigskip

{\large {and}} \\

\bigskip

{\Large{Marek Perkowski}} \\

{\large{Department of Electrical Engineering, Portland State University}} \\

{\large{P. O. Box 751, Portland, Oregon 97 207, U. S. A.}}

email: mperkows@ee.pdx.edu \\

\vspace {10cm}

{\large{FIFTH EDITION}} \\

{\large{Warszawa, April 2000}}

\end{center}

\newpage

\ \ \ \

\newpage

\tableofcontents

\newpage

\setcounter{page}{5}

\section{Application of the program.}

The program Ortocartan is devised for automatic  calculation of the curvature
tensors and some related quantities in general relativity from a  given
orthonormal-tetrad  representation of the metric tensor. It was originally
written
     in the University of Texas Lisp 4.1 programming language and
     implemented on a CDC Cyber 73 computer. That version, and  a
few later ones, went out of operation together with the computers on which they
were implemented.  As  for  today,  the only existing versions are 1. In
Cambridge  Lisp on the Atari Mega STE  computers and 2. In Codemist Standard
Lisp for the Windows 98 and Linux operating systems. The latter is now the main
version, and this description will deal only with this one. (An older edition
of this manual is available for the Cambridge Lisp version.) Additional
programs, based on Ortocartan, that perform other kinds of calculation, are
described in the Appendices B and C.

Please send all correspondence concerning  the  program
     to A. Krasi\'nski.

\section{The algorithm modelled by the program.}

The input data for the program is the tetrad of differential forms:

\begin{equation}
e^i \ \ {\stackrel {{\rm def}} =} \ {e^i}_{\alpha}dx^{\alpha}
\end{equation}

\noindent $i = 0, 1, 2, 3, \alpha = 0, 1, 2, 3$, summation over all the values
of a repeated index is implied. The forms $e^i$ represent the  metric tensor
according to the formula:

\begin{equation}
g_{\alpha \beta} dx^{\alpha} dx^{\beta} = \eta_{ij}e^i e^j
\end{equation}

\noindent where $\eta_{ij}$ is assumed to be the matrix

\begin{equation}
\eta_{ij} = \left[ \begin{array}{rrrr}
  1 & 0 & 0 & 0 \\
  0 & -1 & 0 & 0 \\
  0 & 0 & -1 & 0 \\
  0 & 0 & 0 & -1
\end{array} \right]
\end{equation}

\noindent (i.e. the tetrad $e^i$  is orthonormal).

The program calculates the determinant  of  the  matrix ${e^i}_{\alpha}$, the
inverse  matrix ${e^{\alpha}}_i$,  the  antisymmetrized parts of the Ricci
rotation coefficients  ${\Gamma^i}_{[jk]}$ defined by

\begin{equation}
de^i = {\Gamma^i}_{[jk]} e^j \wedge e^k ,
\end{equation}

\noindent and the full Ricci rotation coefficients ${\Gamma^i}_{jk}$ defined by

\begin{equation}
\Gamma_{ijk} = \Gamma_{i[jk]} - \Gamma_{j[ik]} - \Gamma_{k[ij]} ,
\end{equation}

\noindent where

\begin{equation}
\Gamma_{i[jk]} = \eta_{is}{\Gamma^s}_{[jk]} ,
\end{equation}

\noindent and

\begin{equation}
{\Gamma^i}_{jk} = \eta^{is} \Gamma_{sjk}
\end{equation}

\noindent The matrix $\eta^{ij}$ is the inverse  matrix  to $\eta_{ij}$,
numerically identical to $\eta_{ij}$.

Further, the program calculates the tetrad  components of the Riemann tensor,
$R_{ijkl}$ , defined by:

\begin{equation}
d{\Gamma^i}_j + {\Gamma^i}_s \wedge {\Gamma^s}_j = (1/2){R^i}_{jkl} e^k \wedge
e^l ,
\end{equation}

\noindent where

\begin{equation}
{\Gamma^i}_j = {\Gamma^i}_{jk} e^k ,
\end{equation}

\begin{equation}
R_{ijkl} = \eta_{is} {R^s}_{jkl} ,
\end{equation}

\noindent the tetrad components of the Ricci tensor, $R_{ij}$, defined by:

\begin{equation}
R_{ij} = {R^s}_{isj} ,
\end{equation}

\noindent the scalar curvature $R$ defined by:


\begin{equation}
R = \eta^{ij} R_{ij} ,
\end{equation}

\noindent and finally the tetrad components of the Weyl tensor, $C_{ijkl}$,
defined by:

\begin{equation}
{C^{ij}}_{kl} = {R^{ij}}_{kl} + (1/2) \delta^{ijr}_{kls}{R^s}_r - (1/3)
\delta^{ij}_{kl} R ,
\end{equation}

\noindent where

\begin{equation}
C_{ijkl} = \eta_{ir} \eta_{js} {C^{rs}}_{kl} ,
\end{equation}

\noindent and $\delta^{ijr}_{kls}$ and $\delta^{ij}_{kl}$ are multiple
Kronecker deltas defined by the following properties:

1. $\delta^{...}_{...}$ is equal to +1 when none of the values of the upper
indices  is  repeated,
     while the set of lower indices is an even permutation of the
     upper ones.

2. $\delta^{...}_{...}$ is equal to -1 when none of the upper indices is
repeated, while the set of
     lower indices is an odd permutation of the upper ones.

2. $\delta^{...}_{...} = 0$ in all other cases, i.e. if either  any  of  the
values  of
     upper  or  lower indices is repeated or if the lower indices
     are not just a permutation of the upper ones.


The quantities defined above are  calculated  in  every
     run of the program. However, on special request of the user,
the program may also calculate the tetrad components of  the
     Einstein tensor defined by:

\begin{equation}
G_{ij} = R_{ij} - (1/2) \eta_{ij} R ,
\end{equation}

\noindent as well as the coordinate components of all the  quantities,
     including  the  metric  tensor and Christoffel symbols. With
the exception of the Christoffel symbols for which  the  valence  is fixed, the
indices of the tensor components may be
     in any desired positions, e.g. for the  Riemann  tensor  one
may obtain $R_{\alpha \beta \gamma \delta}$ and ${R^{\alpha}}_{\beta \gamma
\delta}$ and ${R^{\alpha \beta}}_{\gamma \delta}$,  and  so on. The tensor
components are calculated  as  secondary  objects  by contractions  of the sets
of tetrad components of the appropriate quantity with the tetrad vectors given
on  input,  or
     with the inverse tetrad vectors found in the first step. For
     example:

\begin{equation}
R_{\alpha \beta \gamma \delta} = {e^i}_{\alpha} {e^j}_{\beta} {e^k}_{\gamma}
{e^l}_{\delta} R_{ijkl}
\end{equation}

\begin{equation}
{{{R^{\alpha}}_{\beta}}^{\gamma}}_{\delta} = e^{\alpha i} {e^j}_{\beta}
e^{\gamma k} {e^l}_{\delta} R_{ijkl}
\end{equation}

\noindent Just how these, and some other additional  requests  of
     the  user may be communicated to the program, will be explained further.

To use the program one does not need any special knowledge. We  only  expect
our  users  to  be familiar with simple text-editing and copying files. All the
remaining tiny amounts of information  will  be  supplied by the present text.
In the first
     place, the user must know two elementary  notions  of  Lisp:
     the atom and the list.

\section {Atoms.}


The definition of an atom given below is, with  respect to  the  general Lisp
definiton, a restricted one so that it fits the needs of Ortocartan. An atom is
either  an  integer number  or  a continuous string of up to 72
characters\footnote{The Codemist Standard Lisp has 72 characters per line on
the screen. Longer atoms may be tolerable within the program code, but they
would look untidy when showed on the screen.} which does not begin with a
number, and does not contain any of the following special characters:

\bigskip

\begin{tabular}{|l|c|}
  % after \\: \hline or \cline{col1-col2} \cline{col3-col4} ...
\hline
 NAME    &  PRINT NAME \\ \hline
 left parenthesis    & ( \\ \hline
 right parenthesis   & ) \\ \hline
 period & . \\ \hline
 comma & , \\ \hline
 blank & \  \\ \hline
 dollar sign & \$ \\ \hline
 colon & : \\ \hline
 backslash & $\backslash$ \\ \hline
 slash & / \\ \hline
 exponentiation sign & \verb+^+ \\ \hline
 multiplication sign & * \\ \hline
 plus or minus & + or - \\ \hline
\end{tabular}

\bigskip

The user is warned that different keyboards are not necessarily compatible, and
various  special  signs  on  them
     often correspond to different symbols. For safety, the  user
     is  therefore advised to avoid special signs like \#, [, ] or
     \& unless one is aware what one really does  with  them.  Any
other  characters  available on the input equipment are permissible. Each of
the prohibited characters  has  a  special
     meaning  for  the Lisp system and, when used inside a string
     of characters, would cause splitting the  string  into  more
     atoms or other structures.

Examples of atoms:

\bigskip

\verb+A 12 DELTA X1 ATOM+

\bigskip

Examples of strings which are not atoms:

\bigskip

\verb+(B) A,BC AT.L 1.5 DOLLAR\$+ - each of these contains forbidden
characters,

\bigskip

     2DELTA - this one starts with a number

\bigskip

Floating-point  numbers,  which  are  atoms   for   the
     Lisp-system,  are  not used in precise calculations as their
use introduces automatically decimal approximations of fractional numbers, very
much unwanted here. Therefore floating-point numbers are illegal in Ortocartan.

\section{Lists.}

A list is a string of characters starting with the left
     parenthesis, ending with the right parenthesis, and containing
     atoms or other lists separated one from the other by  single
     or  multiple  blanks  or  by single commas. Many consecutive
     blanks have the same meaning as a single blank. Two or  more
     consecutive  commas or commas separated only by blanks form,
in some Lisp systems, an illegal character set, and they result in an error.
Blanks  are  not necessary (but allowed) in front of
     and behind a parenthesis. Examples of lists:

\begin{verbatim}
(      )+ - this is an empty list, having no elements
(ATOM)
(SOME MORE ATOMS)
((QUITE) (A (COMPLICATED)(((LIST)))))
\end{verbatim}

\bigskip

Examples of strings which are not lists:

\bigskip

     (SOMETHING (((IS) MISSING)) - one right parenthesis missing.

\bigskip

\noindent This is a very common type of error and one should be careful to
avoid it. It leads to unpredictable error-messages that can be very misleading.

\bigskip

(AGAIN ((SOMETHING  (MISSING)))))  -  one  left  parenthesis missing or one
extra right parenthesis.

\bigskip

     (WHAT) (IS THIS) - this is a series of two lists

\bigskip

Atoms and lists together are called  "symbolic  expressions" or
"S-expressions".

\section{The representation of mathematical formulae in Ortocartan.}

The notation required for Ortocartan is similar to  the
     conventional  mathematical  notation.  The  restrictions  or
     changes result from the fact that the mathematical  notation
     is  in  some  points nonunique, and is intelligible uniquely
     only when accompanied by some explanatory  text,  which,  of
     course,  cannot  be  supplied to the computer in the form of
     English phrases.

Sums are represented exactly as in mathematics. One can
     omit  the  "+" sign before the first term of a sum or insert
it - both cases are legal. One must only be careful to  separate the "+" and
"-" signs from neighbouring terms by blanks
     or parentheses, so that they form separate atoms. Also  when
     a sum is too long to fit into one line of input, and must be
     continued in the next line, then the "+" or "-" sign may  be
     placed either at the end of the preceding line,  or  at  the
     beginning of the following line,  but  not  in  both  these
places simultaneously as this would cause an error (the second sign would be
understood as a symbol of some quantity).

Products are represented as in  mathematics,  with  the
     restriction that the multiplication sign,  *,  must  not  be
     omitted, and must be separated by blanks or parentheses from
     neighbouring factors.

Exponentiations are  represented  differently,  because writing them in coupled
parallel lines would cause hard problems for the Lisp input  procedure.  So
exponentiation  is
     written in one-line format in the following form:

\bigskip

\verb+<the base> ^ <the exponent>+

\bigskip

\noindent just like in FORTRAN.  For instance, $A^B$   will be represented
     by  (A \verb+^+ B). In some versions of Lisp, the symbol \verb+^+
automatically splits any sequence of characters into separate atoms. In those
versions the symbol A\verb+^+B would be understood as the sequence of 3 atoms,
the A, the \verb+^+ and the B. However, in other versions of Lisp, A\verb+^+B
would be understood as a single atom consisting of the 3 characters. Therefore,
it is always safer to separate all signs of mathematical operations from their
arguments by blanks, like in all examples in this manual.

Division is denoted by the slash /, (A / B)  representing A / B.

Ortocartan uses the same hierarchy of algebraic  operations  as  is  used in
ordinary algebra, i. e. the algebraic
     operations are performed in the  following  sequence:  first
     exponentiations,  then multiplications, then divisions, then
     additions with subtractions. This order is  changed  by  the
     use of parentheses, just like in algebra.

Negative integers may be represented either  as  single
     atoms whose first character is the sign "-", e.g. -2 or -10,
     or as a list of two atoms where the first atom is  the  sign
     "-",  and  the  second  atom  is  the  absolute value of the
number, e.g. (- 2) or (- 10). Both representations are  correct.  For
non-numerical quantities however,  the second representation is the only
correct one: (- A) means "the negative of A", while -A, depending on the Lisp
system, would be understood either as a sequence of two atoms or as a single
atom
     which would be treated by the program Ortocartan as a single
     symbol of some quantity.


Non-integer  rational  numbers   are   represented   as
     two-element  lists,  the  first element being the numerator,
     and the second element the denominator of a  fraction,  e.g.
     1/2  will be represented as (1 2). Negative fractions may be
     also represented in two ways, e.g. -2/3 may  be  written  as
     (-2 3) or as (-(2 3)). Both representations are correct, but
     the first one is recommended.

Functional expressions must be  represented  as  lists, the  first  element  of
the list being the name of the function, and the following elements being
consecutive arguments
     of that function. Note: each argument (or the argument if it
     is single) of a function must be a single S-expression, i.e.
either a single atom or a single list. For example, the correct  representation
of  sin(xy)  is  (sin(X * Y)),   while
     (sin X * Y)  would  be understood as a functional expression
     in which the function sin is given three  arguments:  X,  *,
and Y. Of course, such a three-argument sin  would  be  processed by the
program quite incorrectly.

The program "knows" and can process  automatically  the
     following  functions: exp, log (which stands, of course, for
     natural logarithms to base e), cos, sin,  tan,  ctan,  cosh,
sinh,  tanh,  ctanh,  arctan, arcsin, arsh (the function inverse to sinh), arch
(inverse to cosh), and arth (inverse to
     tanh). All these functions are correctly differentiated, and
     the most important simplifications, like (exp (log X)) =  X,
     (log (X * Y)) = (log X) + (log Y), (tan X) * (ctan X) = 1 or
     (sin X) \verb+^+ 2 + (cos X) \verb+^+ 2 = 1 are made on them.

If it is necessary to introduce a symbol for  a  derivative of some function of
unspecified form, like dF/dx, then
     the derivative should be written as \verb+(der <the variable> <the
function>)+.  In this example, one  should  write  (der X F).
     Multiple  derivatives require only inserting the full series
     of variables between der and the name of the function,  e.g.
    $F,_{xxy}$ should be written as (der X X Y F) or (der X Y X F) or
(der Y X X F) (The program is not sensitive to such  subtleties as
commutativity of derivatives or even differentiability of the functions. It
just assumes that all the functions to  be differentiated are differentiable,
and that in multiple derivatives all the differentiations  commute).  Partial
     derivatives  are denoted by the same symbol as total derivatives.

Sometimes it is hard to calculate by hand  an  explicit
     derivative  of  a  large  expression, which must be inserted
into the data. In this case the task of calculating the derivative  may  be
left to the program. The format for writing
     such a derivative is the same as above, only instead of  the
atom der one should write the atom deriv. For instance, suppose one is too lazy
to calculate explicitly the expression:

$$ (d/dx)(\sin^2 x + a x \sin^4 x + a x \log x \sin x) $$

\noindent but such a derivative must be inserted into the  data.  Then
     one should write:

\bigskip

\begin{verbatim}
(deriv X ((sin X) + A ^ 2 * X * (sin X) + A ^ 4 * X * (log X) * (sin X)))
\end{verbatim}

\bigskip

\noindent and the program will do the work. Actually, der may  be  always
substituted by deriv, and it is even advisable to do so
     in case of multiple partial derivatives. In this  case,  the
series  of  variables should be properly ordered for uniqueness. The
deriv-functional expression will be ordered, while
     der  will be left intact literally as given by the user. For
instance, if the user writes (der X Y X F), then such an expression  will not
be changed, while if one writes (deriv X Y X F), then such an expression will
be changed either  to (der X X Y F) or to (der Y X X F) depending if X or
     Y has the higher priority.

An indefinite integral should be  represented  as \verb+(int <the variable>
<the integrand>)+. For example $\int G dx$ should be represented as (int x G),
and $\int {\rm e}^{\sin^2 x} dx$ should be represented as (int x (exp((sin
x)\verb+^+ 2))).

The program is not able to process definite  integrals.
     If a definite integral must be introduced, then  it  can  be
declared as a function of  unspecified  shape  whose  derivatives should be
substituted  by  the  required  expressions
     supplied by the user on input (how  such  substitutions  may
     be requested - see further).

Let us now give a few examples of more complicated  expressions represented on
input for the program Ortocartan:

\bigskip

\noindent $ (1/3)Lr^2(r^2 + a^2) + r^2 - 2mr + a^2 $ should be represented as:

\bigskip

\begin{verbatim}
((1 3) * L * r ^ 2 * (r ^ 2 + a ^ 2) + r ^ 2 - 2 * m * r + a ^ 2)
\end{verbatim}

\bigskip

\noindent $-1 /(1-2m/r)$ should be represented as (-1 / (1 - 2 * m / r)), or as
(-(1 / (1 - 2 * m / r))), or as \verb+(-(1 - 2 * m / r) ^ -1)+.

\bigskip

\noindent $a(df/dx) + \int (x/V)dx$ should be represented as:

\bigskip

(a * (der x f) + (int x (x / V))).

\bigskip

The program prints  the  results  in  the  mathematical
     format, i.e. with superscripts, subscripts and exponents all
     in their proper places. Derivatives are printed according to
     the common convention in general relativity: the name of the
     function  is  followed  by the comma and by subscripts being
     the names of the variables of differentiation. For  example,
dF/dx,  written  on  input  as  (der x F) will be printed as $F,_x$, while
$\partial^3G/\partial x \partial y^2$ will be printed as $G,_{xyy}$.

Unfortunately, most Lisp systems do not support sophisticated text-editing  and
cannot print more elaborate signs like the integral. Therefore  indefinite
integrals are printed similarly as  they  stand  in the input, e. g. $\int Fdx$
will be printed as  int (x F). (See, however, section 23: in Ortocartan and in
the associated programs one can write the output on a disk in the form of Latex
code, and then the printout may be passed through Latex, with more satisfactory
results in some cases.)

\section{Typing the input.}

The whole input described in  the  sections  that  will follow can be typed in
directly  from  the  keyboard. However, it will often contain misprints or will
require adding new substitutions and resubmitting  for calculation. Therefore
the user will usually want to  prepare  an
     input file and let Ortocartan read the data from there.  How
to write the data is described in the next section. You can save the data file
using any text-editor. When you want Lisp to read the data from the file, you
have to write, while working within Lisp:

\bigskip

\begin{verbatim}
(rdf '"<the name of the file>")
\end{verbatim}

\bigskip

\noindent where \verb+<the name of the file>+ should include the whole path of
access, unless the data file is in the same directory as the Lisp core-image.
Remember to write the apostrophe -- it means that Lisp should not look for a
value of \verb+"<the name of the file>"+, but just take the name literally. The
quotation marks are necessary in order that Lisp tolerates untypical characters
like the backslashes \verb+\+ or slashes / or dots. Without the quotation
marks, each such untypical sign would have to be preceded by the exclamation
mark. For example, this is how I myself ask Lisp to read data for the
Robertson-Walker metric (to be included in this manual further on) from a disk
file:

\bigskip

\begin{verbatim}
(rdf '"\akr\ortocar\robwal.dat")
\end{verbatim}

\bigskip

\noindent After reading such a command, Lisp will read the file
 \verb+\akr\ortocar\robwal.dat+ and carry out whatever task it was given in
those data. Then it will automatically go back to the keyboard for more input.

When  typing  the input directly from the keyboard, be careful not to continue
any atom across the right margin -  the  Lisp  system might split one of the
next atoms into two. If an atom seems  too  long  to fit into the current line,
then simply press "Return" (or "Enter"). The program will not start running
until you close all parentheses and press "Return" after the last one.

\section{Starting Lisp and Ortocartan.}

Codemist Standard Lisp must be bought from its owners, the Codemist Limited
(see Appendix A). Installing it on your computer will most probably require the
assistance of your computer staff. We assume that you have the Lisp already at
your disposal and describe how to load the Ortocartan programs into it.

Since the system Ortocartan will be written into the computer's core on top of
Lisp as a core-image, make sure first that you have a second copy of the pure
Lisp core-image, to which you can revert if you wish to use Lisp without
Ortocartan.

The diskette with the Ortocartan programs contains, among other things, the
following files:

1. Ortcsl.lis, containing the program for calculating the curvature tensors as
described in Section 2. This is the main program, and all the other programs
make use of parts of this one. The main program can be used alone, the other
programs must be loaded into core only together with the main one. How to use
the other programs is described in the Appendices B and C.

2. Calcsl.lis - this is the "algebraic abacus" program.

3. Elliscsl.lis. This is the program that calculates the Ellis evolution
equations.

4. Ncurvcsl.lis. This is the program for calculating the curvature tensor for
given connection coefficients in an arbitrary number of dimensions.

5. Lanlagcs.lis - the program for calculating the lagrangian by the
Landau-Lifshitz method for a given metric.

6. Eulagcsl.lis - the program for calculating the Euler-Lagrange equations from
a given lagrangian.

7. Squintcs.lis - the program for checking the first integrals quadratic in
first derivatives.

The files *.tes contain sets of test examples for the corresponding programs.

The sequence of actions is now the following. Copy all the *.lis files onto
your disk. Let the directory where the files are stored have the name

$\backslash$akr$\backslash$ortocar

\noindent (this is my directory - you can choose any name you like). Call Lisp
and then, from within Lisp, write:

\bigskip

\begin{verbatim}
 (rdf '"\akr\ortocar\ortcsl.lis")
 (rdf '"\akr\ortocar\calcsl.lis")
 (rdf '"\akr\ortocar\elliscsl.lis")
 (rdf '"\akr\ortocar\ncurvcsl.lis")
 (rdf '"\akr\ortocar\lanlagcs.lis")
 (rdf '"\akr\ortocar\eulagcsl.lis")
 (rdf '"\akr\ortocar\squintcs.lis")
\end{verbatim}

\bigskip

\noindent Lisp will respond to each (rdf ...) by confirming that it has defined
several functions whose names and meanings need not bother you. Then write:

\bigskip

(reclaim)

\bigskip

\noindent This command will cause that the Lisp system will clear the core
of all useless data. Next write:

\bigskip

(preserve)

\bigskip

\noindent After this command Lisp will write the core-image to the disk file
and quit. Next time when you call Lisp from the same image, all the Ortocartan
definitions will already be parts of the Lisp system.

At this point, you do not know yet how to use the program Ortocartan. This will
be described in the sections that follow. The thing to remember now is this:

Codemist Standard Lisp (the same is true for Cambridge Lisp and for at least
some other Lisp systems) can either consider upper case letters
     identical to their lower  case  counterparts  (this  is  the
     default mode), or can treat them as  different  symbols.  If
     you wish the upper- and lower case letters to be  considered
     different, then write:

\bigskip

     (setq !*lower nil)

\bigskip

     From now on, however, be careful to write  all  commands  in
     lower case letters or else Lisp will not recognize
them. In particular, you should write \verb+(rdf <filename>)+;  if you write
(RDF ....), then Lisp will protest that the function RDF is undefined. When you
     wish to give up this additional convenience, write:

\bigskip

     (setq !*lower t)

\bigskip

When you wish to finish the session with Lisp, click on the cross in the
upper-right corner of the screen.

Should you wish to preserve the Lisp output on a disk, click the "File" in the
upper-left corner of the screen, and then click "To file" in the dropdown menu.
Then you will be guided by the icons so that you can choose the directory where
the output should be written. When you wish to stop writing the output to the
file, click "File" again and then click "Terminate log". Note: if you choose to
write anything to the same file again after you have "terminated", the old
contents of the file will be overwritten with the new data.

\section{The dictionary for communicating with the program.}

For the purposes of communicating the  user's  requests
     to  the  program, all of the quantities that the program can
     calculate were assigned their unique names.  The  dictionary
     of those names is given in the table below.

\bigskip

\begin{tabular}{|l|c|c|}
  % after \\: \hline or \cline{col1-col2} \cline{col3-col4} ...
  \hline
  STANDARD NAME & \multicolumn{2}{c|} {PROGRAM NAMES} \\ \cline{2-3}
  OF THE QUANTITY & TETRAD & TENSOR \\
                  & COMPONENTS & COMPONENTS \\ \hline
  metric tensor &  &  \\
  and its inverse & -- & g \\ \hline
  the array of coefficients & \multicolumn{2}{c|}{\ } \\
  of the tetrad of forms ${e^i}_{\alpha}$ & \multicolumn{2}{c|} {ematrix} \\
  \hline
  determinant of & \multicolumn{2}{c|}{\ } \\
  the matrix ${e^i}_{\alpha}$ & \multicolumn{2}{c|} {determinant} \\ \hline
  the array of coefficients & \multicolumn{2}{c|}{\ } \\
  of the inverse tetrad ${e^{\alpha}}_i$ & \multicolumn{2}{c|} {ie} \\ \hline
  antisymmetrized Ricci & & \\
  rotation coefficients & agamma & -- \\ \hline
  full Ricci & & \\
  rotation coefficients & gamma & -- \\ \hline
  Christoffel symbols & -- & christoffel \\ \hline
  Riemann tensor & riemann & rie \\ \hline
  Ricci tensor & ricci & ric \\ \hline
  scalar curvature & \multicolumn{2}{c|} {curvature} \\ \hline
  Einstein tensor & einstein & ein \\ \hline
  Weyl tensor & weyl & c \\ \hline
\end{tabular}

\bigskip

The program names are easy to  remember  because,  with
     two  exceptions,  they are the same as the names that appear
     on output. The exceptions are: the "determinant" (appears  on
     output  as  "determinant ematrix"), and the "curvature" (appears
     on output as "curvature invariant").

\section{The printout.}

All the results of the program are printed in  the  mathematical many-line
format. They are printed in the form of
     equations:

\bigskip

\begin{verbatim}
<name of the quantity> <indices attached to the name  (sometimes  none)>  =
<value of the appropriate component of the quantity>
\end{verbatim}

\bigskip

The tetrad components of the Riemann, Ricci,  Weyl  and Einstein tensors are
printed with all indices down. The tensor components (if requested by the user
to  be  calculated)
     are  printed  with the appropriate positions of the indices,
corresponding to the valence requested. The tetrad and  tensor  components  of
the same quantity are printed under different names (see the table in section
8) so that one cannot
     get confused about distinguishing them. However, the  tetrad
${e^i}_{\alpha}$ and the inverse tetrad ${e^{\alpha}}_i$   have each one
tetrad- and one tensor-index. To denote which is  which,  a  dot  is  placed
     above or below the tensorial index in print.  For  instance,
the component ${e^0}_0$ of the matrix ${e^i}_{\alpha}$ will be printed as:

\begin{verbatim}
       0
EMATRIX  . = <appropriate expression>
         0
\end{verbatim}

\noindent while the component ${e^0}_0$ of the inverse matrix ${e^{\alpha}}_i$
will be printed as:

\begin{verbatim}
  0
IE.   = <appropriate expression> .
    0
\end{verbatim}

The program follows a general convention used in  relativity theory: for all
the quantities only their independent nonzero components are printed. For
instance,  advantage  is taken  of  the symmetry of the Ricci tensor, and of
the components $R_{ij}$ only those with $i \leq j$ will appear  (if nonzero) in
print. The same is true for the Riemann  tensor: of the components $R_{ijkl}$
only those will be printed for which simultaneously $\{i < j\}, \{i \leq k <
l\}$ and $\{j \leq l$ while $i = k\}$, and whose value is nonzero. If for some
quantity all the components happen to be zero, then, in
     order to avoid the user's confusion, the message is printed:

\bigskip

\begin{verbatim}
     ALL COMPONENTS OF THE <name of the quantity> ARE ZERO.
\end{verbatim}

\bigskip

Now if all the Ricci rotation coefficients turn out  to
     be zero, then everyone can guess that the Riemann tensor and
     all its concommitants will also be zero. In  that  case  the
     program stops after printing the message:

\bigskip

     (ALL COMPONENTS OF THE AGAMMA ARE ZERO).

\bigskip

\noindent If the Riemann tensor turns out to be zero,  then  again  it
     makes  no  sense  to  go on calculating the other quantities
     which are all zero, so the work is  stopped  after  printing
     the appropriate message. If the Ricci tensor turns out to be
     zero, then the Weyl tensor will be just equal to the Riemann
     tensor, so also in this  case  the  work  is  stopped  after
printing the message about the Ricci tensor being zero.

The message END OF WORK marks the  proper  end  of  the
     printout unless an error occured during the calculation (see
     below).

The printout are the results of the calculation,
     interspersed with the time-messages of the form:

\begin{verbatim}
     (ematrix completed)
     (TIME =  <n1>  msec)

     (DETERMINANT EMATRIX calculated)
     (TIME =  <n2>  msec)

     (ie calculated)
     (TIME = <n3>  msec)

     (agamma calculated)
     (TIME = <n4>  msec)

     (gamma completed)
     (TIME = <n5>  msec)

\end{verbatim}

\noindent and so on, where \verb+< n1 > to < n5 >+ are times in  milliseconds
     elapsed  from  the  start of the work on the current problem
     until completing the appropriate step of the calculation.  A
     quantity  is  calculated when all its independent components
were found, and it is completed when also the dependent components were found
with use of the symmetry properties.  The time-messages are not always useful
and they can be suppressed (see sec. 20).

If an error occurred during the calculation,  then  the
     printout  is  just  suddenly  terminated  and followed by an
error-message. If no error occurred, then  the  end  of  the
     printout  is  marked by the message END OF WORK, followed by
     the information \verb+(RUN TIME = <n> msec)+.

Note: All the input data are processed by the procedure of  algebraic
simplification, so that they acquire the standard form required by the program.
They are reprinted in the mathematical format after they are standardized.
Therefore, they may differ from
     the original data by ordering the sums and  products  or  by
the replacement of $\cos^2 x$ by $1 - \sin^2 x$, and so on. You should
     study this part of the printout carefully because  it  shows
     what the program is really processing.

\section{The frame for the data.}

The first and the last  item  of  the  input  data  are always  the same, and
for clarity are best placed in a separate line each. The first item has the
form:

\bigskip

      (ortocartan '(

\bigskip

\noindent (do not overlook the apostrophe!).

The second item  of  the  data  is  the  title  of  the
     problem.  It  is  a completely arbitrary single S-expression
     (i.e. either a single atom or a single  list).  The  precise
     shape  of  the  title  has no meaning for the program, it is
     simply printed as the heading of the printout  in  order  to
     mark  it  by an easily recognizable identifier chosen by the
     user. For example, here are some possible titles:

\bigskip

     CURVATURES

     (SCHWARZSCHILD METRIC)

     (GENERAL SPHERICAL METRIC)

\bigskip

\noindent Note however that the title:

\bigskip

     AXIAL METRIC

\bigskip

\noindent would result in an error: it consists of two  S-expressions.
     In  this  case  the  program  would understand that the atom
     AXIAL is the title while the atom METRIC is a  part  of  the
relevant  data  (see  further sections). Soon it would recognize that the atom
METRIC has none of  the  required  forms
     for the data, and would print the message:

\bigskip

\begin{verbatim}
******  ///  METRIC  ///  *****  IS AN ILLEGAL ARGUMENT FOR OUR SYSTEM. SORRY,
CANT GO ON.
\end{verbatim}

\bigskip

\noindent A convenient title is a reference to the  paper  from  which
     the metric was taken.

The last item of the data  is  the  set  of  two  right
     parentheses:

\bigskip

                        ))

\bigskip

\noindent They close the two parentheses from the first line.

All the other items of the data, described in the  sections  that  follow,
should be placed between the title and
     the last line with parentheses, in an arbitrary order.

\section{Declaration of coordinates.}

The user tells the program the names he has chosen  for
     the  coordinates  by inserting into the data the item of the
     form:

\bigskip

\begin{verbatim}
(coordinates <atomic names for the coordinates> ).
\end{verbatim}

\bigskip

\noindent The expected order is $x^0, x^1, x^2, x^3$ where $x^0$  is  the  time
     coordinate. Examples:

\bigskip

\begin{verbatim}
     (coordinates  X0  X1  X2  X3)
     (coordinates  T   X   Y   Z )
     (coordinates  t   r   theta   phi)
     (coordinates  T   PH1  X1   X2)
\end{verbatim}

\bigskip

\noindent There are no default names for non-declared coordinates,  so
     this  piece of the data cannot be omitted. If it is omitted,
     then Ortocartan will print an error-message  and  refuse  to
     work.

\section{Declaration of arbitrary functions.}

It has the following form:

\bigskip

\begin{verbatim}
(functions <name of function 1> <list of arguments of  function 1>
           <name of function 2> <list of arguments of function 2> .... )
\end{verbatim}

\bigskip

\noindent For instance,  if  the  user  wants  to  use  the  functions
$f(x,y,z)$ and $g(x,y)$, then the appropriate declaration is:

\bigskip

\begin{verbatim}
(functions f (x y z) g (x y) )
\end{verbatim}

\bigskip

If composite functions are to be used,  then  for  each function  only those
arguments must be given on which it depends directly. The arguments which are
themselves functions
     must  be  declared  on  their  own. For example, if the user
     wants to use the functions g(x,y) and f(x,y,h(t,u)), where u
     is a function of z, then the appropriate declaration is:

\bigskip

\begin{verbatim}
(functions  g (x y)  f (x y h)  h (t u)  u (z))
\end{verbatim}

\bigskip

The order in which the declared functions appear is irrelevant  from  the point
of  view of syntax. However, the functions will be ordered by the program in
sums  and  products according to the same order.

If the program Ortocartan is applied to some  explicitly  given  tetrad which
does not contain any arbitrary functions, then this piece of data should be
simply omitted.

Note: The built-in functions listed in section  5  must
     not  be  declared. Of course, their names should not be used
     for user-defined arbitrary functions.  The  reserved  names,
     not  to  be used as the names of functions, include also the
     atoms nil, times, plus, expt, minus, +, - and *.

In the formulae described in sections 14 and following,
     only the names of  the  functions  must  be  written,  their
     arguments do not have to be written out (see examples I -  V
     in  Appendix  E).  However,  if  the  user  wishes  so,  the
arguments can be written out.  Then  Ortocartan  will  automatically pick up
this style and will write  out  the  arguments of the functions throughout the
whole calculation, see Example VI in Appendix E. The derivatives of functional
expressions are then printed differently,  on  the  assumption
     that the arguments are not necessarily  atoms,  but  can  be
     functional expressions themselves. For example, let $f$ be  a
function of $(x^2 + y^2)^{1/2}$ and $z$. One can then  write  in  the
     input data:

\bigskip

\begin{verbatim}
(f ((x ^ 2 + y ^ 2) ^ (1 2)) z)
\end{verbatim}

\bigskip

\noindent and the derivatives of F will be printed as:

\bigskip

\begin{verbatim}
                2    2 -1/2        2    2 1/2
f,  = (1/2) x (x  + y )     f,  ((x  + y )   , z)
  x                           1


                2    2 -1/2        2    2 1/2
f,  = (1/2) y (x  + y )     f,  ((x  + y )   , z)
  y                           1


             2    2 1/2
f,  = f,  ((x  + y )   , z),
  z     2
\end{verbatim}

\bigskip

\noindent the subscripts on the right-hand sides referring to the consecutive
arguments of f. Such an output is less easily readable, however, and we  do not
recommend this  style.  The
     function f from the last example should be declared as being
     dependent on x, y and z. Note the correct format for writing
     the functional expression in the input data, it  must  be  a
     list of the form \verb+(<function name> <argument 1> <argument  2>
     ... <argument n>)+.  The  whole functional expression must be
     enclosed in parentheses, the series of arguments must not be
     enclosed in an extra pair of parentheses, and the  arguments
     must be separated from each other only by spaces, do not use
     any commas.

\section{Declaration of the constants.}

This piece of the data tells the program which  symbols
     should produce zero when differentiated. The declaration has
     the form:

\bigskip

\begin{verbatim}
     (constants <names of the constants> )
\end{verbatim}

\bigskip

\noindent For example:

\bigskip

\begin{verbatim}
(constants M)
(constants KAPPA RHO H)
\end{verbatim}

\bigskip

\noindent The order of the constants has the same meaning as the order
     of  the  functions.  If no arbitrary constants appear in the
     metric, then this piece of data should be omitted.

\section{Declaration of the tetrad.}

Let us recall that the program Ortocartan  can  process
     only  orthonormal  tetrads  in  the signature $(+ - - -)$. The
     tetrad is declared in the form:

\bigskip

\noindent \verb+(ematrix  <components of the matrix+ ${e^i}_{\alpha}$, \verb+in
the order+
\begin{verbatim}
 i=0,A=0 / i=0,A=1 / i=0,A=2 / i=0,A=3 / i=1,A=0 / ... /
 i=3,A=3, and in the notation described in section 5>)
\end{verbatim}

\bigskip

For instance, for the Schwarzschild metric:

\bigskip

$$ ds^2  = (1 - 2GM/c^2 r) dt^2 - 1/(1 - 2GM/c^2r) dr^2 - r^2(d\vartheta^2 +
\sin^2\vartheta d\varphi^2 ) $$

\bigskip

\noindent the declaration should look as follows:

\bigskip

\begin{verbatim}

(ematrix ((1 - 2 * G * M / c ^ 2 * r) ^ (1 2))  0  0  0
    0 (1 / (1 - 2 * G * M / c ^ 2 * r) ^ (1 2)) 0 0
    0  0  r  0
    0  0  0 (r * (sin theta))     )

\end{verbatim}

\bigskip

\noindent Note that each component of ${e^i}_{\alpha}$ in the above list must
be a single S-expression, i.e. must be embraced by parentheses if
     it is not an atom.

This piece of the data is the heart of the problem, so, needless to say, its
omission would push the program into a fatal error. The omission of some
relevant parentheses  usually  causes  that  the list becomes too long, i.e.
has more
     S-expressions than the expected 17. This kind  of  error  is
     directly communicated by the system Ortocartan.

The pieces of data described up to this  place  form  a
     minimal  set of data for Ortocartan. This is all if the user
     has no special requests  concerning  the  results.  All  the
pieces described further serve to adjust the results of the program to the
momentary needs of the user.

\section{Symbols for sums and other expressions.}

In by-hand calculations it is sometimes  convenient  to
     develop  a  product  involving a sum by applying the rule of
     distributivity of multiplication, and sometimes it  is  not.
     The  choice  requires a small piece of intelligent thinking:
one must be conscious of the goal one wants to achieve. But intelligence, even
in such  small amounts, is something (as
     yet) inaccessible to computers: the program must  follow  an
algorithm which specifies unique decisions or unique criteria of choice.
Consequently, in our program the  intelligent  pieces  of work are left to the
user himself. The program applies the rule of distributivity always,  unless
the user  requested  this not to be done for a specific sum. The format for
such a request is described below.

If a sum which is present in the ematrix is not  to  be
     expanded when in a product, then the user should introduce a
     separate symbol for the sum, use the symbol in the  ematrix,
     and insert an additional item into the data, of the form:

\bigskip

\begin{verbatim}
(symbols <the symbol> = <the sum>)
\end{verbatim}

\bigskip

\noindent Actually, one can declare in this way an arbitrary number of special
symbols for sums:

\bigskip

\begin{verbatim}
(symbols <symbol 1> = <sum 1>
    <symbol 2> = <sum 2>
    ....................
    <symbol n> = <sum n> )
\end{verbatim}

\bigskip

\noindent Then, in all the algebraic operations the left-hand sides of the
equations will be used, while differentiation will operate on the right-hand
sides. Example: take the Nariai metric:

$$ ds^2 = \{a(t) \cos[log (r/l)] + b(t) \sin[log (r/l))]\}^2 dt^2 -
l^2(dr^2/r^2 + d\vartheta^2 + \sin^2\vartheta d\varphi^2) ) $$

\noindent where $l$ is a constant and $a$ and $b$ are arbitrary functions of
time. Suppose, you want the coefficient of $dt^2$  not to  be developed in
products, and you call it $\psi$. Then the declaration should be:

\bigskip

\begin{verbatim}
(symbols psi = (a * (cos (log (r / l))) + b * (sin (log (r / l)))))
\end{verbatim}

\bigskip

\noindent Of course, this should be accompanied by:

\bigskip

\begin{verbatim}
 (coordinates t r theta phi)
 (constants l)
 (functions a(t) b(t))
 (ematrix  psi  0  0  0
           0  (l / r)  0  0
           0  0  l  0
           0  0  0  (l * (sin theta)))
\end{verbatim}

\bigskip

\noindent After such a call to Ortocartan, psi will be used in all the
algebraic operations, but for $\partial\psi/\partial t$ and $\partial
\psi/\partial r$ the program will automatically substitute, respectively, the
quantities $(\partial a/\partial t) \cos[\log(r/l)] + (\partial b/\partial t)
\sin[\log(r/l)]$, and $-(a/r) \sin[\log(r/l)] + (b/r) \cos[\log(r/l)]$. This
example is shown in full in Appendix E.

The usefulness of symbols is most evident in  the  case of sums, but the
expressions represented by the symbols need not be sums.

The symbols declared on  the  left-hand  sides  of  the
     equations should not be declared separately as functions (if
     they are, this will have practically no result).

The equations  listed  in  the  "symbols"-argument  are printed  in the
mathematical format at the very beginning of the printout, preceded by the
heading "symbols". Note: Before being  printed and applied they are
standardized by the simplifying procedure.  See  therefore  if  their
standardized
     forms  are  convenient for you. If not, make the appropriate
     adjustments in your data.

\section{Calculating coordinate components of various quantities.}

It is sometimes useful to calculate also (or just)  the tensor  components  of
some quantities; e.g. the metric tensor, to check the correctness of the tetrad
components,  or the Christoffel  symbols,  to  solve the equations of a
geodesic. The program Ortocartan can do that: it  can  calculate
     the tensorial components of the Riemann, Ricci, Einstein and
     Weyl tensors, with any required positions of their  indices,
     the  metric  tensor, the inverse metric, and the Christoffel
symbols. In order to have some tensorial  components  calculated  and  printed
one should insert into the data the item
     of the form:

\bigskip

\begin{verbatim}
(tensors
       <name of quantity 1> <valence I required for quantity 1>
                  <valence II required for quantity 1>
                  ....................................
                  <valence  n required for quantity 1>
       <name of quantity 2> <valence I required for quantity 2>
                  ........ )
\end{verbatim}

\bigskip

\noindent where the \verb+<name of the quantity>+ should be the identifier of
the quantity to be calculated and printed in its tensorial form.
         Essentially (with two exceptions) it is the name of  the
tetrad-quantity which is the source to calculate the relevant tensor. The
dictionary of these names is:

\bigskip

\begin{tabular}{|l|l|}
  % after \\: \hline or \cline{col1-col2} \cline{col3-col4} ...
  \hline
  THE TENSOR WANTED &  NAME TO BE INSERTED IN (tensors ...) \\ \hline \hline
   metric tensor and its inverse & metric \\ \hline
  Christoffel symbols & christoffel  \\ \hline
   Riemann tensor & riemann  \\ \hline
  Ricci tensor &  ricci   \\ \hline
   Weyl tensor  & weyl  \\ \hline
  Einstein tensor  & einstein  \\ \hline
\end{tabular}

\bigskip

\verb+<valence N required for quantity K>+ is a list of  the  signs
         "+"  and  "$-$",  the "+" in the i-th position of the list
         meaning that the i-th index should be an upper one,  and
         the "$-$" in the j-th position meaning that the j-th index
         should be a lower one. For instance:

\bigskip

\begin{verbatim}
(tensors riemann (+ - + -) (+ - - -))
\end{verbatim}

\bigskip

\noindent means that the user wants the components
${{{R^{\alpha}}_{\beta}}^{\gamma}}_{\delta}$ and ${R^{\alpha}}_{\beta \gamma
\delta}$ of the Riemann tensor to be calculated and printed.

If not all indices are on the same level (i.e. are  not
     all  contravariant or not all covariant), then the practical
     use of the symmetry properties may be difficult as  lowering
or raising an index with a nondiagonal metric often requires much algebra. In
this case the symmetries which are not trivial  (i.e.  do  not  amount just to
equality of some components, or to the equality of a component with  the
negative
     of some other component, or to vanishing of a component) are
not taken into account, and the program prints the dependent components, too.
For instance, when printing ${{{R^{\alpha}}_{\beta}}^{\gamma}}_{\delta}$, only
the trivial symmetry ${{{R^{\alpha}}_{\beta}}^{\gamma}}_{\delta} =
{{{R^{\gamma}}_{\delta}}^{\alpha}}_{\beta}$ will be used, while the symmetries
corresponding to $R_{\alpha \beta \gamma \delta} = - R_{\beta \alpha \gamma
\delta}$ and $R_{\alpha \beta \gamma \delta} = - R_{\alpha \beta \delta
\gamma}$ will be ignored (i.e. 136 components, if all  nonzero,  will appear in
print). When printing ${R^{\alpha}}_{\beta \gamma \delta}$, only
${R^{\alpha}}_{\beta \gamma \delta} = - {R^{\alpha}}_{\beta \delta \gamma}$
will be used, while $R_{\alpha \beta \gamma \delta} =  R_{\gamma \delta \alpha
\beta}$ and $R_{\alpha \beta \gamma \delta} = - R_{\beta \alpha \gamma \delta}$
will be ignored (i.e. 96 components, if all nonzero, will  appear
     in print).

One can use the same name repeated a  few  times,  each
     time followed by a different valence, e.g.:

\bigskip

\begin{verbatim}
(tensors weyl(- - - -) weyl(+ - - -) weyl(+ + - -) ),
\end{verbatim}

\bigskip

\noindent but this is equivalent to the simpler expression  where  the
     name is used only once:

\bigskip

\begin{verbatim}
(tensors weyl(- - - -)  (+ - - -)  (+ + - -) ).
\end{verbatim}

\bigskip

\noindent If  a  name  of  some  quantity  is  not   followed   by   a
     valence-specification,  then  it  is understood that all the
     indices of the tensor should be down (covariant). There  are
     two exceptions to this:

1. The "christoffel" has an obvious valence, which thus  need
     not be specified.

2. The "einstein" not followed by a valence-specification  is
     understood  as  a request to calculate the tetrad components
of the Einstein tensor (which are not calculated in the routine run of the
program).

Each specification of valence (or name of a quantity if
     no  valence  is present) may be followed by a series of sets
of indices, each set defining a single component of the tensor.  Such  a form
will be understood as a request to calculate only the components thus defined.
For instance:

\bigskip

\begin{verbatim}
(tensors riemann (+ - - -)
    ricci (- -) (+ -) (0 1) (0 2) (1 1)
    weyl (- - - -) (0 1 0 1) )
\end{verbatim}

\bigskip

\noindent means that the user wants the program to print all the  components
${R^{\alpha}}_{\beta \gamma \delta}$ of the Riemann tensor, and all the
components $R_{\alpha \beta}$ of the Ricci tensor while of the components
${R^{\alpha}}_{\beta}$ of the Ricci tensor and $C_{\alpha \beta \gamma \delta}$
of the Weyl tensor only ${R^0}_1$, ${R^0}_2$, ${R^1}_1$ and $C_{0 1 0 1}$
should be printed. The single components specifically requested in this way are
printed  even  if  they  are equal to zero.

Note that if the symmetries of some quantity (like $R_{\alpha \beta}$) are
trivial in practical use,  then  each  set  of  indices
     referring  to  that quantity should be ordered in the proper
     way, e.g.:

\bigskip

\begin{verbatim}
(tensors ricci (- -) (1 0) )
\end{verbatim}

\bigskip

\noindent will result in printing no components of $R_{\alpha \beta}$, as the
combination (1 0) of the indices of $R_{\alpha \beta}$ is not considered at all
by the program. The correct request is:

\bigskip

\begin{verbatim}
(tensors ricci (- -) (0 1) ).
\end{verbatim}

\section{Expanding natural powers of sums.}

For purposes of uniqueness, natural powers of  sums  are
     automatically expanded with use of rules of the type:

$$(a + b)^2 = a^2 + 2ab + b^2. $$

\noindent This, as well as automatic application of the rule  of distributivity
of multiplication, is necessary to carry out all
     the algebraic simplifications  that  are  possible.  Without
this,  quite  trivial flaws inescapably occur, e.g. the program would not
recognize that

$$(1 + x)^2 - 1 - 2x - x^2 = 0. $$

\noindent However, if the exponent is larger than 3, then  the  expansion  is
suspended, and the expression is left in a non-expanded form to avoid too large
expressions  which  would probably appear in such a case. This might, of
course, be inconvenient in some cases, so  the  user  may  change  this rule,
if necessary. In order to do it, one should insert into the data the item of
the form:

\bigskip

\begin{verbatim}
(expand powers from <n1> to <n2>)
\end{verbatim}

\bigskip

\noindent where \verb+<n1>+ and \verb+<n2>+ are the minimal and the  maximal
value, respectively,  of  the  exponents  with  which the expansion
     should be automatically done everywhere.  If  $n1 = n2$,
     then  only the single exponent $n = n1 = n2$ will be processed
     in this way. Any value $n1 < O$ is equivalent to $n1 = O$:  all
     powers  up to n2-th will be expanded. If $n2 < O$ or $n2 < n1$,
     then no powers will be expanded.

If one wants some definite expressions which appear  in
     some  definite places to be expanded even if their exponents
     exceed the upper limit (whether changed  before  by  (expand
     powers ... )  or  not),  then one should use the facility of
     substitutions described in the next section.

\section {Substitutions.}

As already mentioned before, the program cannot  be  as
     intelligent  as  a  human being. Consequently, a human being
     can use much of his (her?) ingenuity to simplify the process
     of  calculation, or the results, while a program will always
     follow rigid rules, and it will miss some simplifications if
     they  are  not  of an algebraic nature, or if they involve a
calculation with rational functions. For instance,  Ortocartan  cannot
recognize  that $\sin x / \cos x = \tan x$ what is
     sometimes necessary. Therefore, the user may introduce  some
     simplifications  by  his (her) own. The request to make such
     user-defined substitutions is expressed  by  inserting  into
     the data the item of the form:

\bigskip

\begin{verbatim}
(substitutions  <a series of addresses> <equation 1>
           <a series of addresses> <equation 2>
           ....................................
           <a series of addresses> <equation n>  )
\end{verbatim}

\bigskip

Each equation defines a substitution to  be done: the left-hand side is the old
expression which should be replaced by the right-hand  side.  The  addresses
specify  in which  formulae the substitution that follows should be carried
out. The form of the addresses can be  best  explained  on
     examples.  If  the  address is an atomic name of a quantity,
e.g. gamma, riemann or rie, then the substitution that  follows  should be
attempted in each component of the quantity. If the name is followed by sets of
indices, then the substitution should be attempted only in the components
defined by
     these sets of indices. For example, if the addresses are

\bigskip

\begin{verbatim}
gamma (0 0 2) (0 0 3) riemann (0 3 0 3) rie L = R
\end{verbatim}

\bigskip

\noindent then R should be substituted for L in the components ${\Gamma^0}_{0
2}$ and ${\Gamma^0}_{0 3}$ of the "gamma", in the tetrad component $R_{0 3 0
3}$ of the Riemann tensor and in all the coordinate components  of  the
     Riemann tensor.

The atomic names can be also followed by symbols defining  the  valence,  the
ones described in sec. 16. For some
     quantities (e.g. for all the tetrad components  or  for  the
     Christoffel  symbols)  the  valence  is fixed forever and so
     need not be specified. For some others it is  relevant.  For
     instance

\bigskip

\begin{verbatim}
 rie (+ + + +) (+ + - -) (0 1 0 1) (0 3 0 3) (- - - -)  ricci
 ric (+ -) (0 1) L = R
\end{verbatim}

\bigskip

\noindent  means that R should be substituted for L in all the  coordinate
components $R^{\alpha \beta \gamma \delta}$ of the Riemann tensor, in the
coordinate components ${R^{0 1}}_{0 1}$ and ${R^{0 3}}_{0 3}$, and in all the
coordinate components $R_{\alpha \beta \gamma \delta}$, then in all the tetrad
components of the Ricci  tensor  and  in  the  coordinate component ${R^0}_1$.
If an equation is not preceded by any address (i.e. is immediately
     preceded  by another equation or by the atom "substitutions"),
     then the substitution it defines should  be  attempted  just
     everywhere.   This  is  rarely  reasonable  and  results  in
     an unnecessary extension of the calculation time.

Note that those substitutions which should be carried out
     everywhere  are  attempted  first, before the addressed ones
     are considered. Thus, if some  substitutions  are  addressed
and  some others are not, the user does not have a full control of the order in
which they will be executed.

The following example may show the usefulness  of  substitutions.  If  the
Einstein field equations are yet to be solved for a metric, then the "ematrix"
will contain some  unknown functions. After the Einstein tensor is calculated
and the field equations are solved, these functions  become  explicit
expressions. For instance, let:

$$ ds^2 = f(r) dt^2 - g(r) dr^2 - r^2(dh^2 + \sin^2 h dp^2)$$.

\noindent For this metric, the data are:

\bigskip

\begin{verbatim}
 (coordinates t r h p)
 (functions f (r) g (r) )
 (ematrix  (f ^ (1 2))   0   0   0
           0   (g ^ (1 2))   0   0
           0   0   r   0
           0   0   0   (r * (sin h))  )
\end{verbatim}

\bigskip

\noindent From the field equations $G_{ij} = 0$ it follows then that $f = 1/g =
1 - 2m/r$, where $m$ is a constant. In order to check that
     this  is  a solution it is enough to add just two more items
     to the data listed above:

\bigskip

\begin{verbatim}
  (constants m)
  (substitutions f = ((1 - 2 * m / r) ^ (1 2))
                 g = (1 / (1 - 2 * m / r) ^ (1 2))  )
\end{verbatim}

\bigskip

\noindent Actually, it would be more reasonable to write

\bigskip

\begin{verbatim}
  (substitutions ematrix f = ((1 - 2 * m / r) ^ (1 2))
                 ematrix g = (1 / (1 - 2 * m / r) ^ (1 2)))
\end{verbatim}

\bigskip

\noindent because after $f$ and $g$ are replaced by their  values  in  the
     ematrix,  they  will never appear later and the program need
     not look for any other opportunity to replace them.

This rather simple example was chosen just for its simplicity. Ortocartan was
successfully tested on several  much
     more complicated metrics (see Appendix E).

In general, the user is advised to specify  the  places
     in  which the substitutions should be carried out as precisely
     as possible. It makes the calculation  faster:  the  program
     does not look for opportunities to make the substitutions in
     the formulae where these substitutions  are  not  necessary.
     Every  unsuccessful  attempt at performing a substitution is
communicated to the user in the printout.  Suppose  for  instance that the user
has written in the substitutions:

\bigskip

\begin{verbatim}
riemann (0 1 0 1) (0 2 0 2) (0 3 0 3)
     (x ^ 2) = (r ^ 2 - y ^ 2 - z ^ 2)
\end{verbatim}

\bigskip

\noindent but the variable $x$ did not appear in $R_{0 2 0 2}$ at all. Then the
program will print the following message above $R_{0 2 0 2}$:

\bigskip

\begin{verbatim}
> riemann         DID NOT PROVIDE ANY OPPORTUNITY TO PERFORM
          0 2 0 2

> THE FOLLOWING SUBSTITUTIONS

   2    2    2    2
> X  = R  - Y  - Z
\end{verbatim}

\bigskip

\noindent This means that in the next run of the program with the same data,
the address of this substitution should be riemann (0 1 0 1) (0 3 0 3). This is
not an error message, and the program will continue to work.

These messages can be turned off altogether. This  happens  when  one  inserts
the atom "messages" into the argument (dont print ...) (see sec. 20 and example
IV in Appendix E).

Sometimes the whole value of some  quantity  should  be
     substituted  by  some other expression. The user may in this
     case avoid the necessity to rewrite the whole expression  by
writing  the atom "actual" as the left-hand side of the appropriate equation.
So for example the equation:

\bigskip

\begin{verbatim}
 riemann (O 1 O 1)  actual = (x ^ 2)
\end{verbatim}

\bigskip

\noindent means: "from now  on  the  tetrad  component $R_{0 1 0 1}$ of the
Riemann tensor will be equal to $x^2$". The previous value  is forgotten.

The equations defining the substitutions are printed in the  mathematical
format  above  all the results, after the "symbols", preceded by the heading
"substitutions", and with the
     sub-headings  which  either say "everywhere" or are the series
     of addresses as written by the user. One  can  avoid  having
     the  list of substitutions printed (sometimes it is long) by
inserting the atom "substitutions" into  the  argument  (dont print ...) (see
sec. 20).

If some of the left-hand sides is a sum or  a  product,
     then  the substitution required may be missed in some cases.
     One of such cases is when the user wants the sum $x + y$ to be
     substituted  by $U$, and in the expression now processed there
is the sum $x + 2y$. Then the program will not be able to  recognize  that $x +
2y$ should be replaced by $y + U$. The same concerns a product like $ab$: the
program will  not  recognize it as being a part of an expression like $ab^2$.
If this difficulty  is  likely to appear, then it is advised that instead
     of $x + y = U$ the substitution  $x = U - y$ be  declared.  The
     same concerns products: instead of $(a * b) = c$ one may write
     $a = (c / b)$ or $b = (c / a)$. The general  principle  is:  the
     left-hand  sides  of  the  substitutions should be literally
     present in the expressions where they are to be  substituted
     for.

Another example of a trouble: if a sum is multiplied by
     a  factor,  then the substitution for the non-multiplied sum
might be missed. For example, if $X + Y = U$, then  the  program  will not
recognize that $aX + aY = aU$. However, the facility of substitutions is
flexible enough to overcome  such
     difficulties with a little help from the user.

The facility of substitutions may help to  bypass  some conventions of our
system, were they inconvenient in a particular case. For example, the program
changes any even power of $\cos x$ to the appropriate power of $(1 - \sin^2 x)$
just for uniqueness. Were this inconvenient, one may define an atomic
     symbol, e.g. COSX, to stand for $(\cos x)$,  and  then  write
     the equation:

\bigskip

\begin{verbatim}
COSX = (cos x)
\end{verbatim}

\bigskip

\noindent in the "symbols", and the equation:

\bigskip

\begin{verbatim}
(cos x) = COSX
\end{verbatim}

\bigskip

\noindent in the "substitutions". Then,  COSX  will  appear  instead  of (cos
x) in  all  the  prints,  and  will not be replaced by (sin x). See example V
in Appendix E.

In addition  to  the  literal  substitutions  described
     above  one can also carry out another kind of substitutions in
     which the left-hand side of an equation defines a pattern to
     be  looked for, and the right-hand side says what to do with
an expression that matches the pattern. For these  substitutions one should
insert an item of the form:

\bigskip

\begin{verbatim}
(markers <a series of arbitrary atomic symbols>)
\end{verbatim}

\bigskip

\noindent into the data. For example:

\bigskip

\begin{verbatim}
(markers M1 M2 M3 M4)
\end{verbatim}

\bigskip

\noindent If such a command is found in the  data,  then  the program
     will understand that M1, M2, M3 and M4 will be used to stand
     for anything in the substitutions. With the markers,  the  user
can define just a pattern to which an expression should conform in order to be
replaced by some other  expression.  For
     example, if one writes in the substitutions:

\bigskip

\begin{verbatim}
((sin x) ^ M1) = ((1 - (CS x) ^ 2) * (sin x) ^ (M1 - 2))
\end{verbatim}

\bigskip

\noindent then any power of $\sin x$ will be replaced by $(\sin x)^{{\rm the \
old \ power} - 2}(1 - CS^2 x)$, where $CS(x)$ is a user-defined function (note
the danger:  if $\sin x$ happens to appear with an exponent that is negative or
fractional or even is not a number at all, then the  substitution  will  be
performed anyway). If one would like such a
     substitution to be performed not only for $\sin x$, but for sin
     of  any arbitrary argument, then it is enough to use another
     of the markers instead of x, e.g.:

\bigskip

\begin{verbatim}
((sin M2) ^ M1) = ((1 - (CS M2) ^ 2) * (sin M2) ^  (M1 - 2))
\end{verbatim}

\bigskip

\noindent One should not worry about using the same marker in  different
formulae, each time with a different meaning - the program assigns a meaning to
each marker in each  substitution
     anew  and  does  not remember the meanings the markers could
     have had before.

Suppose, the cartesian radius $r = (x^2 + y^2 + z^2)^{1/2}$ appears  in  a
calculation. One can then place r (x y z) in the "functions" and write:

\bigskip

\begin{verbatim}
(substitutions (der x r) = (x / r)
          (der y r) = (y / r)
          (der z r) = (z / r)  )
\end{verbatim}

\bigskip

\noindent With the "markers", the same effect can  be  achieved  by  writing
     just one equation:

\bigskip

\begin{verbatim}
(substitutions (der M1 r) = (M1 / r))
\end{verbatim}

\bigskip

Let us stress  that  markers  can  symbolize  not  only
     atoms,  but any arbitrarily complicated expressions. For instance, with:

\bigskip

\begin{verbatim}
(substitutions (M1 ^ (1 2)) = (M1 / M1 ^ (1 2)) )
\end{verbatim}

\bigskip

\noindent also an expression like $(a^2 + b^3)^{1/2}$ will  be  replaced by
$[a^2 / (a^2 + b^3)^{1/2} + b^3 / (a^2 + b^3)^{1/2}]$. If M1 and M2 are
     markers, then (M1 + M2) means "any sum", and (M1 * M2) means
     "any product". For instance, the equation  (2  *  (log (M1 +
     M2) )) =  (log ((M1 + M2) \verb+^+ 2)) means: whenever a logarithm
     of any sum is multiplied by 2, square the sum and  drop  the
coefficient  2.  The "markers" can also be used to represent arbitrary parts of
sums and products, provided their meaning can
     be guessed uniquely. For instance, (A + B + M1 + C + M2 + D)
     means "a sum in which anything stands between B  and  C  and
     anything  stands  between  C and D", (A + B + M1) means "any
     sum starting  with  A + B",  but  (A + M1 + M2 + C)  is  not
     unique:  if  several terms stand between A and C, then there
     will be no way to tell which of them should go into  M1  and
     which  into  M2.  Such  a  pattern  will result in an error-message.

The equation:

\bigskip

\begin{verbatim}
M = (M * a * b ^ 2 / c)
\end{verbatim}

\bigskip

\noindent with M being a marker, means "multiply the equation, whatever it was,
by $ab^2/c$". In this case, M stands for the whole old expression. This is
sometimes useful, but the address(-es) of this substitution must be specified,
or else {\bf all} equations will be multiplied by the factor $ab^2/c$.

The disadvantage of the substitutions by pattern-matching is that they are
slower than the literal  substitutions,
     and  sometimes  can be very much slower (when the right-hand
     sides are complicated functions of the "markers").

Note that Ortocartan understands the functions "quotient"
     and "remainder" acting on the  "markers".  However,  it  is  the
     user's  exclusive  responsibility  to  make  sure  that  the
     arguments of these functions will in each case turn  out  to
     be integer numbers.

Expanding integer powers of sums is a part of the substitutions. For instance,
in order to expand $(A + B)^4$ one  should
     write in the substitutions:

\bigskip

\begin{verbatim}
((A + B) ^ 4) = expand
\end{verbatim}

\bigskip

\noindent Should just any power of $(A + B)$ be expanded, one writes:

\bigskip

\begin{verbatim}
((A + B) ^ M1) = expand
\end{verbatim}

\bigskip

\noindent where M1 should be one of the markers.

The user writes the substitutions  by  pattern-matching together  with  the
literal ones in the argument (substitutions.........), without marking them in
any way. The  presence  of  the  argument  (markers..........) is a sufficient
     warning for Ortocartan - it will then be able to distinguish
     the two kinds of substitutions by itself.

\section{Substitutions in the data.}

This facility was introduced long ago, before easy-to-use text-editors became
commonly available, and before the additional programs of the Appendices B and
C were written. It is unlikely that any user would find this useful today.
However, the facility still exists, so its description is retained.

As mentioned before, the data written by the user  are,
     before  applying,  processed  by  the procedure of algebraic
     simplification, so that the user may write them  in  a  form
     that  is more compact than the final one (e.g. one may leave
the task of carrying out a multiplication by a sum, or  calculating  a
derivative, to the program itself). For this reason, however, the final form of
the data may appear  inconvenient  (e.g.  clumsy  or too extended). In such a
case the
     result may be smoothed out by the user, either by  rewriting
     the  input data, or by applying the substitutions. Consider,
however, the following example (taken from a paper by J. Kowalczy\'nski). Let:

$$ S {\stackrel {\rm def} =} [2m + 2b \log(ZC - ap) + b \log(q)] a/p - 4bZC/p^2
- (1/3)hp^2, \eqno{(1)}  $$

\noindent where a = 1 or a = -1, b, c, h and m are constants, t, $\phi$, r and
z are coordinates and

$$ ZC = z + c, \qquad p = (t^2 - r^2)^{1/2},\qquad q = p^8/(t - r
\cos(\phi))^8. $$

\noindent The symbol S appears in the metric and so is  differentiated during
the calculation. However, the automatically calculated derivatives of S are
rather messy, for example:

\bigskip

\begin{verbatim}
              -2  2    2 -1/2               -2  2    2 -1/2
dS/dt = - abtp  (t  - r )     log(q) - 2abtp  (t  - r )     log(ZC-ap)

          6 -1                -8   2    2 -1/2
   + 8abtp q  (t - r cos(phi))   (t  - r )

         7 -1                -9        -2  2   2 -1/2
   - 8abp q  (t - r cos(phi))   - 2amtp  (t - r )

         -3     2    2 -1/2              2    2 -1/2
   + 8btp  ZC (t  - r )     - (2/3)htp (t  - r )

       2   -1         -1  2    2 -1/2
   - 2a btp  (ZC - ap)  (t  - r )    .                                (2)
\end{verbatim}

\bigskip

\noindent The missing simplifications are:

\bigskip

\begin{verbatim}
 2     2    2
t   = r  + p ,                                                        (3)

 -1                   8  8
q   = (t - r cos(phi)) /p ,                                           (4)

                                 -1  3     -1          -1       -1
log(q) = - 2 log(ZC -ap) + (1/3)ab  hp  + ab  Sp  + 4ap  ZC - 2b  m,  (5)

 2
a  = 1.                                                               (6)
\end{verbatim}

\bigskip

\noindent After they are done, the more neat result is obtained:

\bigskip

\begin{verbatim}
             -3       -1                -1       -4
dS/dt = 8abtp   - 8abp  (t - r cos(phi))   + 4btp  ZC

           -2         -1           -2
     - 2btp  (ZC - ap)   - ht - tSp  .                                (7)
\end{verbatim}

\bigskip

\noindent The net result of having the substitution (7) carried out may
     be effected by declaring the equation (1) in the "symbols", and
     then writing the equations (3), (4), (5) and (6),  in  turn,
     in  the  "substitutions". Then, however, such a coupled series
     of substitutions would be performed a  lot  of  times,  each
time inside a large expression, and this would make the calculation slow. It is
more economical to force the program to
     replace every occurrence of dS/dt directly by (7).

This can be done. One should declare S as  one  of  the
     "functions", and then write:

\bigskip

\begin{verbatim}
(substitutions
        (der t S) = (deriv t (<the right-hand side of (1)>)))
(data substitutions
        (der t S) (t ^ 2)  = (r ^ 2  +  p ^ 2)
        (der t S) (q ^ -1) = ((t - r * (cos phi)) ^ 8 / p ^ 8)
        (der t S) (log q) = (-2 * (log(ZC  -  a * p)) +
                                          (1  3) * a * h * p ^ 3 / b
                       + a * S * p / b + 4 * a * ZC / p  -  2 * m / b)
        (der t S) (a ^ 2) = 1 )
\end{verbatim}

\bigskip

\noindent (For brevity we  omit  the  corresponding  substitutions  in
     dS/d$\phi$, dS/dr and dS/dz).

The "data substitutions" specify the replacements  to  be
     made  inside  the "substitutions" before the program starts to
run. They have the same syntax as  the  substitutions  themselves,  with the
difference that here the addresses are the appropriate left-hand sides of the
equations from the "substitutions", and the replacements are performed in the
corresponding right-hand sides. If no address is given, then the appropriate
data substitution is performed just everywhere in the
     "substitutions".

In the example given above the value of dS/dt, initially  equal to the
right-hand side of (2), after being in turn
     transformed by (3), (4), (5), and (6)  will  finally  change
into  the right-hand side of (7), and then just this expression will be
directly substituted for dS/dt in the  calculation proper.

The use of the data substitutions facility has the following result:

1. The substitutions are standardized  and  printed  as
     explained  in section 19, just as if there were no "data substitutions".

2. The data substitutions are printed analogously,  immediately below, with the
heading:

\bigskip

(THE SUBSTITUTIONS LISTED ABOVE WILL  BE  THEMSELVES  TRANSFORMED BY THE
FOLLOWING SUBSTITUTIONS)

\bigskip

\noindent (Don't get confused if you had turned off printing the list of
substitutions - then they will not be "listed above" although the heading of
data substitutions will say so).  Each
     equation  printed  has its own sub-heading which says either
     EVERYWHERE, or:

\bigskip

\begin{verbatim}
> IN THE VALUE OF THE EXPRESSION <the appropriate  left-hand
         side from substitutions>
\end{verbatim}

\bigskip

3. Another heading is printed:

\bigskip

(THE SUBSTITUTIONS YOU ASKED ME TO MODIFY WILL HAVE THE FOLLOWING FINAL FORM)

\bigskip

4. Those substitutions which were modified are  printed
     in their final, working form.

One can avoid having the list of modified substitutions printed  by  inserting
the atom "modifications" into the argument (dont print ...) (see sec. 20).

\section{Suppressing some parts of the printout.}

Sometimes the user is not interested in having all  the results  of  the
routine run of the program. In this case he or she may instruct the program not
to print some intermediate results or to stop its work earlier than usually,
before
     calculating the Weyl tensor. This results in saving paper as
     well  as computer's time and core, and, consequently, user's
     time, too.

The request to stop the work earlier  is  expressed  by
     inserting into the data the item of the form:

\bigskip

\begin{verbatim}
(stop after <name>)
\end{verbatim}

\bigskip

\noindent where \verb+<name>+ is one of the names from the dictionary in
section 8 (only scalars and tetrad-names will have the required
     result). For example, if the expression:

\bigskip

\begin{verbatim}
(stop after riemann)
\end{verbatim}

\bigskip

\noindent is found in the data, then the Ricci and Weyl  tensors  will
     not be calculated.

If the user is not interested in some intermediate  results, then the
appropriate request is:

\bigskip

\begin{verbatim}
(dont print <a series of names>)
\end{verbatim}

\bigskip

\noindent where \verb+<a series of names>+ consists of any  number  of  names
     from  the  dictionary  in  section 8 (again only scalars and
     tetrad names except einstein, i.e. only the quantities  that
     are  calculated  in  the routine run, will have the required
     result). All the quantities that bear  these  names,  though
     calculated  for  the  needs of further calculation, will not
     appear in print. For example, if the request:

\bigskip

\begin{verbatim}
(dont print ie determinant agamma gamma ricci)
\end{verbatim}

\bigskip

\noindent is found in the data, then none  of  the  quantities
${e^{\alpha}}_i$, det$({e^i}_{\alpha})$, ${\Gamma^i}_{[jk]}$, ${\Gamma^i}_{jk}$
and $R_{ij}$ will be printed. The series of names may also contain any of the
atoms "substitutions", "modifications", "messages" and "timemessages".
Inserting them into the series will have, respectively, the following  results:
the list of substitutions will not be printed, the list of substitutions
modified by data substitutions will not be printed,  the messages about the
unsuccessful attempts at substitutions will not be printed and the information
about timings shown in sec. 9 will not be printed. The order of the atoms  in
the series is irrelevant.

\section{Formatting the output.}

The linelength of the output on the screen is 72 characters.  However,  the
user may wish to produce a narrower
     output (e. g. in order to be printed and inserted in a typed
     manuscript) or a wider output (e. g. in  order  to  use  the
     paper  more economically). It is then possible to change the
     linelength of the output by  inserting  the  following  item
     into the data for Ortocartan:

\bigskip

\begin{verbatim}
(rmargin <n+1>)
\end{verbatim}

\bigskip

\noindent where n is the required new linelength. This will be useful mainly if
the user wishes to print the results on wide paper with continuous feeding; in
this case the typical parameters are (rmargin 120) or (rmargin 132).

Similarly, you can adjust the left margin by  inserting
     the item

\bigskip

\begin{verbatim}
(lmargin <n+1>)
\end{verbatim}

\bigskip

\noindent into the data. This  will  have  an  effect  only  in  those
     formulae that extend over more than one line: the second and
     following lines will be printed starting at the (n + 1)-st
     column. The default value for lmargin is 8. The  first  line
     of each formula always begins at column 5, and this  is  not
     adjustable.

\section{Storing the results on a disk file and printing them.}

In a normal run, Ortocartan will show  the  results  of
     the calculation only  on  the  screen.  When  the
     calculation is simple (such  as  the  Schwarzschild  or  the
     Robertson-Walker  metric), the  formulae  will  flash
     through the screen much faster than you can read  them.  You
can go back and forth along the results to see the details by pressing the
buttons "PgUp" and "PgDn" on the keyboard. However, it is often convenient to
print the results on paper.  In  order  to  do  so,  you  have  to click on the
"File" in the upper left corner of the screen, and then click the "To file"
option in the dropdown menu. Then you can choose the name and the path of the
directory where the complete output from Lisp will be stored. Later, you can
view it with any text-editor, edit and print it.

\section{Special forms of the output.}

Ortocartan produces formulae that are easily readable. However, sometimes these
formulae have to be used as input for another calculation or inserted in a text
for publication. Retyping is always annoying, and carries the risk of typos.
This can be largely avoided thanks to the following two conveniences.

If the formulae produced by Ortocartan should be used as input for another run
of Ortocartan, then you should insert the following item into the data:

\bigskip

\begin{verbatim}
(output for input)
\end{verbatim}

\bigskip

\noindent Then, the whole output will be written in the same notation as is
used on input, and every piece of it will be directly readable for Ortocartan
as its data. Usually, you will want to use only parts of this output, but then
all you have to do is very simple editing (cutting and pasting), without any
rewriting.

If the (some of the) formulae produced by Ortocartan should be used as a part
of a text for publication, then you should insert the following item into the
data:

\bigskip

\begin{verbatim}
(output for latex)
\end{verbatim}

\bigskip

\noindent Then, all the formulae will be written as expressions readable for
the system Latex. You will only have to add your favourite Latex preamble, and
run Latex on them, they will then show up as a neat Latex output, with all
equations numbered. Ortocartan will automatically replace the names of the
Greek letters with their Latex codes, e.g. if you use "alpha" as the name of
the variable, then it will be replaced by \verb+\alpha+ in the Latex code, and
appear as $\alpha$ in the output from Latex.

The printing in Latex format is not 100\% safe against Latex errors. It may
happen that the end of line in the output from Ortocartan will occur in a place
that is not acceptable for Latex. Then Latex will signal an error. You may also
prefer to have the Latex linebreaks in other places than Ortocartan will place
them. Such failures will have to be corrected by hand.

\section{Errors.}
The program Ortocartan has some built-in tests for correctness of the data. If
one of our tests  finds  an  error, then an error-message is printed which
should directly identify the kind of error that was made. However, only some of
the possible errors can be identified uniquely.

For beginners in Lisp computing, the most likely  error
     is incorrect placement of parentheses. If Ortocartan refuses
     to start working, then almost surely some  left  parentheses
     in your data are unmatched. If you  have  written  the  data
     directly from the keyboard, you may try your luck by writing
     a  few  right  parentheses  and  pressing  "RETURN"   again.
     However, if the parentheses  were  distributed  incorrectly,
     then a fatal error of one kind or another is sure to  occur,
     and the  error-message  can  be  quite  beside the  point,
     especially when the data are read from a disk. There  is  no
     way for the Lisp system to recognize incorrect  distribution
of parentheses. It knows what you want  it  to  do  only  by
     reading your data as lists, and the parentheses  divide  the
     data into sublists. Placing the parentheses incorrectly  may
     turn some sublists into a mathematical  nonsense,  but  they
     will still be meaningful S-expressions, and Lisp  will  take
     them literally. Hence, whenever you get an obvious  nonsense
as a result, or an overtly absurd  or  unintelligible  error-message, please
check carefully the positions of all  parentheses in your data.

There is one more possible error that is not signalized
     at all, and  may  cause  trouble.  Each  item  of  the  data
     described  in  one of the sections 11 to 21 may be used only
     once. This is not really a limitation of the power  of  the
     program,  it  is just a small piece of rigour forced upon the
     user, as each item can hold many requests of its appropriate
     kind. If any of the items is used twice or more times,  then
     only  its last appearance has the expected results, and the
     previous ones are simply ignored. For example, in the call:

\bigskip

\begin{verbatim}
(ortocartan '(
 (SPHERICAL METRIC IN STANDARD FORM)
 (functions  MU(T R)  NU(T R))
 (coordinates T R THETA PHI)
 (ematrix  (exp NU)  0  0  0  0  (exp MU) 0  0  0  0  R
             0  0  0  0  (r *(sin theta)))
 (tensors riemann  (+ - - -) ricci)
 (stop after ricci)
 (tensors riemann  (- - - -))
 (stop after weyl)
                   ))
\end{verbatim}

\bigskip

\noindent only the components $R_{\alpha \beta \gamma \delta}$ of  the Riemann
tensor will be calculated and printed (second appearance of  (tensors....))
while the requests to calculate ${R^{\alpha}}_{\beta \gamma \delta}$  and
$R_{\alpha \beta}$ will  be ignored. Also, the program will finish its work
after calculating the Weyl tensor, i.e. after completing the whole routine run,
while the request to stop after the  Ricci  tensor
     will be ignored.

We recall what has already been said before: if you wish to use capitals and
lower case letters as different symbols, then you should insert the command:

\bigskip

\begin{verbatim}
(setq !*lower nil)
\end{verbatim}

\bigskip

\noindent before the "(ortocartan'(" line. Without it, all capitals will be
mapped into lower case letters and printed as such. Even worse things can occur
if you do insert the (setq !*lower nil) command, and then carelessly use
capitals and lower case letters. With this command inserted, all function names
should be written in lower case, for example sin, but not SIN (the SIN would
then be an unknown function for Ortocartan). You can then use capitals only as
names of the symbols introduced by yourself in the data for Ortocartan, and in
the title.

If the "(setq !*lower nil)" has been used, then it is advisable to place the
command

\bigskip

\begin{verbatim}
(setq !*lower t)
\end{verbatim}

\bigskip

\noindent at the end of the data for Ortocartan (i.e. behind the two right
parentheses). Then Lisp will revert to its default mode after it has finished
the calculation, otherwise it will continue to see capitals and lower case
letters as different.

\bigskip

\bigskip

\bigskip

\centerline {\Large {\bf {ACKNOWLEDGEMENTS}}}

\nopagebreak

\bigskip

This manual was rewritten into a computer file for  its
     third  edition  thanks to the courtesy of Professor F. Hehl.
     We express our gratitude to him and to his  collaborators  at
     the  University of Cologne who actually did the hard work of
     retyping. Our thanks are due to Drs J. Richer and A.  Norman
who reworked Ortocartan into Cambridge Lisp  and  implemented it on an IBM
computer. This enabled one of the authors (A. K .)  to  rewrite Ortocartan into
Slisp/360  (now  a defunct version). The kind assistance of Professor J. Fitch
in this last task is hereby gratefully acknowledged. The program Calculate (see
Appendix B) and two other, now defunct, programs were written and implemented
as a part of a project supported  by  the A. von Humboldt Foundation at the Max
Planck Institute in Garching (Germany). The  Slisp/360  version  of Ortocartan
was implemented as a part of a project supported by the Deutsche
Forschungsgemeinschaft at the Konstanz University (Germany). The
pattern-matching  substitutions  were added  together with several other
improvements as a part of a project supported by the  Deutsche
Forschungsgemeinschaft at  the University of Cologne. A. K. is grateful to
Professors J. Ehlers, H. Dehnen and F.  Hehl  who  were  the  respective  hosts
of those projects for their kind hospitality.

The Atari computer, on which a newer Cambridge Lisp version of Ortocartan was
implemented, was bought for funds kindly provided by Professor M. Demia\'nski
from his grant. The implementation of Ortocartan on it was facilitated by the
assistance of Professor M. A. H. MacCallum. The programs described in Appendix
C were first written and implemented with that version.

The most recent implementation in the Codemist Standard Lisp to be run under
the  Windows 98 and Linux operating systems was made possible thanks to the
grant no 2 P03B 060 17 awarded by the Polish Research Committee to a research
group at the Institute of Theoretical Physics led by Professor Andrzej
Trautman. The funds from the grant were used to upgrade A.K.'s computer at the
N. Copernicus Center and to buy a license for the Codemist Standard Lisp from
Codemist Limited. I (A.K.) am very grateful to Professor Trautman for including
me in his research group.

\appendix

\section{How to acquire Ortocartan.}

In order to use Ortocartan one must first buy the Codemist Standard Lisp. It
can be bought from:

\bigskip

     Professor John Fitch, Director

     CODEMIST Limited

     "Alta", Horsecombe Vale

     Combe Down

     BATH, Avon, BA2 5QR

     England

     phone and fax (44-1225) 837 430

     email jpffitch@maths.bath.ac.uk

\bigskip

Ortocartan is free of charge. If you wish to have it, just contact A.
Krasi\'nski, I will send you a diskette. How to install Ortocartan when Lisp is
already working is described in sec. 7.

\section {How to use the program Calculate.}

The definition of the function "calculate" is contained in the file
"calcsl.lis" in the distribution diskette of Ortocartan  (see  sec.  7). In
order to use the program you
     simply have to follow the instructions from sec. 7. To  make
     sure that the program is  indeed  at  your  disposal,  after
     starting Ortocartan, write:

\bigskip

infocalculate

\bigskip

\noindent and press "RETURN". The response should be reassuring.

With the exceptions of integration and factoring  polynomials,  this  program
can carry out any kind of elementary algebraic operations, but lacks the
facility of writing programs  in  it  without resorting to Lisp, i.e. is not a
programming language in itself.

The format for the user's data is nearly  identical  to
     the  one  described in the manual (see example V in Appendix
     E). The following differences with respect to the description in the
     manual must be observed here:

\medskip

Section 1:

The program can simplify symbolic  expressions  of  (in
     principle) arbitrary degree of complexity. In fact, it gives
the user direct access to the operations  which  are carried out as
intermediate tasks in the program Ortocartan.

\medskip

Section 2:

Does not apply here at all.

\medskip

Section 7:

Whenever you start Ortocartan, the function "calculate" is  ready  for
     your use, too.

\medskip

Section 8:

Does not apply here at all.

\medskip

Section 9:

Several expressions can be simplified in one run of the
     program. First, the heading is printed:

\medskip

I UNDERSTAND YOU REQUEST THE FOLLOWING EXPRESSION TO BE SIMPLIFIED

\medskip

\noindent Then, the program prints the expression as  written  by  the
     user  on input translated from the input format of sec. 5 to
     the normal mathematical format (without yet simplifying it).
     Next, the program prints:

\medskip

THE RESULT IS

\medskip

\noindent and the simplified expression follows.  The  simplified  expression
is printed as the equation

\bigskip

\begin{verbatim}
result    = <the expression>
       <i>
\end{verbatim}

\bigskip

\noindent where \verb+<i>+ is a subscript running consecutively  from  1  for
     the  first expression. Such labeling of the printed results
     makes it possible to direct the substitutions  to  different
     results.

No time-messages are  printed  for  single  operations,
     only  the  total  run-time for the call to calculate will be
     displayed.

\medskip

Section 10:

The first line here should be:

\medskip

(calculate '(

\medskip

The second item of the data is again a title, the last  line
     must contain the two right parentheses.

\medskip

Section 11:

The number of coordinates (i.e.  independent  variables
     in  differentiation)  is  arbitrary here. Coordinates can be
     omitted. If no differentiations are going to be  done,  then
     all the symbols used can be called constants.

\medskip

Section 14:

The heart of the data are here the  expressions  to  be
     simplified. They are written as:

\bigskip

\begin{verbatim}
 (operation <an arbitrary expression written according to the rules of
               section 5>)
 (operation <another arbitrary expression>)
 ..........................................
 (operation <another arbitrary expression>)
\end{verbatim}

\bigskip

The number of operations is arbitrary. It is understood that the program will
be used to calculate complicated derivatives or to multiply large polynomials
or to carry out series of substitutions in given expressions.

\medskip

Section 16:

Does not apply here at all.

\medskip

Section 18:

The substitutions should be  directed  here  to  \verb+result (<index>)+ where
the \verb+<index>+ is the subscript that appears in print. If no address is
specified, then they  will  be carried out everywhere, i. e. in each result in
the present call
     to calculate. All the rules explained in sec. 18 apply  here
     without any modification.

\medskip

Section 20:

Does not apply here at all.

\section{The programs "ellisevol", "curvature", "landlagr", "eulagr" and
"squint".}

\subsection{The program Ellisevol.}

This program calculates all the quantities appearing in the evolution equations
of the kinematical tensors of fluid flow, as defined by Ellis\footnote{G. F. R.
Ellis, in {\it General relativity and cosmology. Proceedings of the
International School of Physics "Enrico Fermi", Course 47: General Relativity
and Cosmology.} Edited by R. K. Sachs, Academic Press, New York 1971.}. Since
all these equations are consequences of the Ricci identity $u_{\alpha;\beta
\gamma} - u_{\alpha; \gamma \beta} = u_{\mu}{R^{\mu}}_{\beta \gamma \delta}$,
they will be fulfilled identically in many cases. However, they may fail to be
identically fulfilled when we make assumptions about separate parts of the
flow, e.g. if we assume that the shear is zero. As is well known, such
assumptions have consequences for the other characteristics of the flow, and
the Ellis equations will show what the consequences are. Along the way, the
program calculates the expansion, acceleration, the shear tensor and scalar,
the rotation tensor and scalar, and the electric and magnetic parts of the Weyl
tensor with respect to the velocity vector.

In order to make sure that the program Ellisevol is indeed contained in the
core-image of Lisp that you use, write:

\bigskip

infoellis

\bigskip

\noindent and press "RETURN".

Since the signature assumed in the calculation is $(+ - - -)$, the formulae may
differ from those given in textbooks, and so some of them are quoted below for
reference.

The program is called by writing:

\bigskip

\begin{verbatim}
(ellisevol'(
\end{verbatim}

\bigskip

\noindent in the first line of the data file. All the remaining parts of the
data are the same as for the program Ortocartan described in the manual, except
that there is one additional item here:

\bigskip

\begin{verbatim}
 (velocity <contravariant tetrad components of the velocity field
 u0, u1, u2, u3>)
\end{verbatim}

\bigskip

\noindent (see Example VII in Appendix E). This is the vector field for which
all the kinematical quantities will be calculated. This argument cannot be
omitted, the omission would result in an error, communicated by the program
Ellisevol.

Note: It is implicitly assumed that this will be a velocity field of some
continuous medium because this is the most frequent application of these
formulae. However, the calculation makes sense for any timelike vector field
whose length is normalized to 1. In particular, this may the the unit vector
collinear with a timelike Killing vector.

The second item of the data must be the title (this rule applies to all the
additional programs described in this appendix).

The quantities printed are the "ematrix", the velocity field (tetrad components
as given by the user), the determinant of the ematrix, the inverse matrix to
the "ematrix", the contravariant coordinate components of the velocity field
(named "uvelo"), the covariant coordinate components of the velocity field
(named "lvelo"), the components of the metric tensor (named "metric"), the
inverse metric (named "invmetric"), the "agamma" and the "gamma", and the
Christoffel symbols (named "christoffel"). Then come:

The matrix of covariant derivatives of the covariant velocity field (named
"vtida" -- for tidal matrix of velocity).

The contravariant ("uacce") and covariant ("lacce") components of the
acceleration field.

The rotation tensor $\omega_{\alpha \beta}$, named "rotdd".

The mixed rotation tensor ${\omega_{\alpha}}^{\beta}$, named "rotdu".

The square of the rotation scalar.

The covariant ("projdd") and mixed ("projdu") components of the projection
tensors $g_{\alpha \beta} - u_{\alpha} u_{\beta}$ and
${\delta^{\alpha}}_{\beta} - u^{\alpha} u_{\beta}$.

The expansion scalar.

The covariant ("sheardd") and mixed ("sheardu") components of the shear tensor,
and the square of the shear scalar.

The matrix of covariant derivatives of the covariant acceleration,
$\dot{u}_{\alpha ; \beta}$ (called "atida" for the tidal matrix of the
acceleration).

The rotation constraint equations:

$$ \omega_{[\alpha \beta; \gamma]} + \dot{u}_{[\alpha; \gamma}u_{\beta]} +
\dot{u}_{[\alpha}\omega_{\beta \gamma]} = 0, $$

\noindent (square brackets on indices denote antisymmetrization, round brackets
on indices denote symmetrization). The components of these equations are
printed with the name "rotcons".

The shear constraint equations:

$${h^{\alpha}}_{\beta}({\omega^{\beta \gamma}}_{;\gamma} - {\sigma^{\beta
\gamma}}_{;\gamma} + {2 \over 3} \theta^{;\beta}) - ({\omega^{\alpha}}_{\beta}
+ {\sigma^{\alpha}}_{\beta})\dot{u}^{\beta} = 0 $$

\noindent under the name "shearcons".

The rotation evolution equations:

$${h_{\alpha}}^{\gamma} {h_{\beta}}^{\delta}\dot{\omega}_{\gamma \delta} -
{h_{\alpha}}^{\gamma} {h_{\beta}}^{\delta} \dot{u}_{[\gamma; \delta]} + 2
\sigma_{\delta [\alpha} {\omega^{\delta}}_{\beta]} + {2 \over 3} \theta
\omega_{\alpha \beta} = 0, $$

\noindent under the name "rotevol".

The tetrad components of the Riemann and Ricci tensors, and the curvature
scalar.

The Raychaudhuri equation:

$$\dot{\theta} + {1 \over 3}\theta^2 - {\dot{u}^{\alpha}}_{; \alpha} +
\sigma^{\alpha \beta} \sigma_{\alpha \beta} - \omega^{\alpha \beta}
\omega_{\alpha \beta} + R_{\alpha \beta}u^{\alpha} u^{\beta} = 0 $$

\noindent under the name "raychaudhuri equation".

The tetrad components of the Weyl tensor.

The (coordinate) electric components of the Weyl tensor:

$$E_{\alpha \beta} = C_{\alpha \rho \beta \sigma} u^{\rho} u^{\sigma} \equiv
E_{\beta \alpha} $$

\noindent under the name "elweyl".

The shear evolution equations:

$${h_{\alpha}}^{\gamma} {h_{\beta}}^{\delta}\dot{\sigma}_{\gamma \delta} -
{h_{\alpha}}^{\gamma} {h_{\beta}}^{\delta} \dot{u}_{(\gamma; \delta)} +
\dot{u}_{\alpha} \dot{u}_{\beta} + \omega_{\alpha \gamma}
{\omega^{\gamma}}_{\beta} + \sigma_{\alpha \gamma} {\sigma^{\gamma}}_{\beta} +
{2 \over 3} \theta \sigma_{\alpha \beta} $$

$$+ {1 \over 3} h_{\alpha \beta}[2(\omega^2 - \sigma^2) +
{{\dot{u}^{\gamma}}}_{; \gamma}] + E_{\alpha \beta} = 0, $$

\noindent under the name "shearevol".

The magnetic components of the Weyl tensor:

$$ H_{\alpha \beta} = {1 \over 2} {\sqrt {- g}}\varepsilon_{\alpha \gamma \mu
\nu} {C^{\mu \nu}}_{\beta \delta} u^{\gamma} u^{\delta} \equiv H_{\beta \alpha}
$$

\noindent under the name "magweyl", the $\varepsilon_{\alpha \gamma \mu \nu}$
is the Levi-Civita symbol.

The "magnetic constraint" equations:

$$ 2\dot{u}_{(\alpha}w_{\beta)} - {\sqrt {- g}}{h_{\alpha}}^{\gamma}
{h_{\beta}}^{\delta} ({\omega_{(\gamma}}^{\mu ; \nu} + {\sigma_{(\gamma}}^{\mu
; \nu})\varepsilon_{\delta) \rho \mu \nu} u^{\rho} = H_{\alpha \beta}, $$

\noindent under the name "magcons", where $w_{\beta}$ is the rotation vector
field defined by:

$$w^{\alpha} = {1 \over {2 \sqrt{- g}}} \varepsilon^{\alpha \beta \gamma
\delta} u_{\beta} \omega_{\gamma \delta}. $$

Each of the names of the quantities printed can be used as an address in the
substitutions and as an entry in the (dont print ...) or (stop after ...), just
like in the main program Ortocartan. For the print-names that consist of two
words (like the "raychaudhuri equation"), only the first word should be used as
an address or as an entry in (dont print ...) and (stop after ...).

All the facilities described for Ortocartan exist also here, including the
"tensors" -- except that the Christoffel symbols and the metric are calculated
in every run because they are needed at later stages of the calculation (but
their printing can be suppressed with (dont print ...)).

Please do not forget about the last line with the two right parentheses. This
applies to all the other programs of this appendix.

\subsection{The program "curvature".}

This program calculates the curvature tensor from given connection coefficients
in any number of dimensions (see example VIII in Appendix E). The connection
coefficients are assumed symmetric (i.e. torsion-free), but need not be
metrical. The program was written for one special application, and hence the
assumption of zero torsion; it can be removed without any difficulty.

To make sure that this program is in your core-image, write:

\bigskip

infoncurva

\bigskip

\noindent and press "RETURN".

The first item of the data for this program is the line:

\bigskip

\begin{verbatim}
(curvature <n> '(
\end{verbatim}

\bigskip

\noindent where \verb+<n>+ is the number of dimensions of the manifold on which
the connection coefficients and the curvature are defined. The next item should
be the title of the problem, just like in every other program of the Ortocartan
collection. The main part of the data is the specification of the connection
coefficients that has the form:

\bigskip

\begin{verbatim}
 (connection <list of components of the connection coefficients
 in the order D000 D001 D002 D003 ... D00n D011 D012 ... D01n ...
 Dn00 Dn01 ... Dnnn>)
\end{verbatim}

\bigskip

\noindent i.e. only the algebraically independent components of ${D^i}_{jk}$
with $j \leq k$ are given.

All the other parts of input, like constants, functions, symbols,
substitutions, etc, can be used just like in Ortocartan, if only they make
sense here (for example (stop after ...) or (dont print ...) would not work
here because there is only one quantity that is being calculated).

The last item of the data are the two right parentheses.

\subsection{The program "landlagr".}

This program calculates the reduced lagrangian for the Einstein equations by
the Landau--Lifshitz method\footnote{L. D. Landau and E. M. Lifshitz, Teoria
polya, 6th Russian edition. Izdatelstvo "Nauka", Moskva 1973, sec, 93.}. This
is the Hilbert lagrangian with the divergences containing second derivatives of
the metric already removed. In short, this (noncovariant) lagrangian is
obtained by deleting from the scalar curvature those terms in which the
Christoffel symbols are differentiated, and taking the remaining part with the
reverse sign.

To make sure that this program is in your core-image, write:

\bigskip

infolandlagr

\bigskip

\noindent and press "RETURN".

The program is called by writing at the beginning of the list of data:

\bigskip

\begin{verbatim}
(landlagr'(
\end{verbatim}

\bigskip

\noindent The remaining data are exactly like for "ortocartan".

The main part of the data is the "ematrix", and the quantities printed are the
ematrix and its inverse, the metric and its inverse, the "agamma", the "gamma",
the Crhistoffel symbols and the lagrangian. You can direct substitutions to it
with the address "lagrangian". See example IX in Appendix E.

The last item of the data are the two right parentheses.

Users are warned that deriving the Einstein equations from a variational
principle with assumptions limiting the generality of the metric is tricky and
requires detailed knowledge about the problem. It may happen that the
Euler-Lagrange equations will have nothing to do with the Einstein equations;
this is known, for example, to occur for certain Bianchi-type
models\footnote{See M. A. H. MacCallum, in {\it General relativity. An Einstein
centenary survey.} Edited by S. W. Hawking and W. Israel. Cambridge University
Press 1979, p. 552.}. Therefore, the user must make sure, before using the
program "landlagr", that the situation is appropriate for using the lagrangian
methods.

\subsection{The program "eulagr".}

This program calculates the Euler-Lagrange equations starting from a lagrangian
specified by the user. It is assumed that the resulting E-L equations will be
ordinary differential equations (i.e. that there is only one independent
variable in the lagrangian action integral).

To make sure that this program is in your core-image, write:

\bigskip

infoeulagr

\bigskip

\noindent and press "RETURN".

The program is called by writing as the first item of the data:

\bigskip

\begin{verbatim}
(eulagr'(
\end{verbatim}

\bigskip

The second item of the data is the title.

The main part of the data are the following three items (see example X in
Appendix E):

\bigskip

\begin{verbatim}
 (parameter <an atomic name of the independent variable>)
 (variables <the list of names of the lagrangian variables,
             an arbitrary number of them>)
 (lagrangian <the formula for the lagrangian written in accordance
              with the rules of Section 5>)
\end{verbatim}

\bigskip

\noindent The "variables" must be declared as functions in the (functions ...)
list because the program allows them to depend also on other arguments in
addition to the "parameter" specified in the data. (For example, you may wish
to differentiate some of the E-L equations with respect to another variable.)

The consecutive Euler-Lagrange equations are printed with the headings:

\bigskip

\begin{verbatim}
THIS IS THE VARIATIONAL DERIVATIVE BY <the name of the lagrangian variable>
\end{verbatim}

\bigskip

\noindent as the equations
\bigskip

\begin{verbatim}
eulagr        = <the appropriate equation>
      <index>
\end{verbatim}

\bigskip

\noindent The names and indices are needed to address the substitutions.
Whatever devices of the main program Ortocartan make sense here, can be used,
and all of Ortocartan's conventions apply. If you wish to stop the calculation
at an earlier stage, you can write:

\bigskip

\begin{verbatim}
(stop after lagrangian)
\end{verbatim}

\bigskip

\noindent (this would make sense if you want to just check whether the
lagrangian you wrote in the data has no errors in it), or:

\bigskip

\begin{verbatim}
(stop after (eulagr <n>))
\end{verbatim}

\bigskip

\noindent where \verb+<n>+ is the index of the last Euler-Lagrange equation
that you want to have. Then, the whole expression \verb+eulagr <n>+ has to be
in parentheses.

The last item of the data are the two right parentheses.

\subsection{The program "squint".}

This program verifies whether a given expression is a first integral of a given
set of ordinary differential equations. The program was written for a specific
application, therefore it is rather limited in its abilities. It is assumed
that the (hypothetical or actual) first integral is a polynomial of first or
second degree in the first derivatives of the functions that should obey the
set of equations. It is also assumed that the equations in the set are all of
second or first order.

To make sure that this program is in your core-image, write:

\bigskip

infosquint

\bigskip

\noindent and press "RETURN".

The program is called by writing as the first item of the data:

\bigskip

\begin{verbatim}
(squint'(
\end{verbatim}

\bigskip

\noindent (This is an abbreviation for "square integral".)

The second item of the data is the title.

The main part of the data are the following three items (see example XI in
Appendix E):

\bigskip

\begin{verbatim}
 (parameter <an atomic name of the independent variable
            in the set of equations>)
 (variables <the list of names of the functions that should obey
            the set of equations>)
 (integral <the formula for the first integral
            to be tested by the program,
            written in accordance with the rules of Section 5>)
\end{verbatim}

\bigskip

\noindent The "variables" must be declared as functions in the (functions ...)
list because the program allows them to depend also on other arguments in
addition to the "parameter" specified in the data. The "integral" may either be
an explicit expression which is being tested by the program whether it is a
first integral indeed, or a polynomial of second or first degree in the first
derivatives of the "variables" with unknown coefficients. The unknown
coefficients must then be declared as (functions ...) of the appropriate
variables.

Example\footnote{For an actual example of a simple application of this program,
see Example XI in Appendix E.}: suppose you have a set of 3 second-order
ordinary differential equations to be fulfilled by the functions $f^1(t)$,
$f^2(t)$ and $f^3(t)$, of the form:

$$ {{d^2f^i} \over {dt^2}} = {W^i}_{jk} {{d f^i} \over {dt}} {{d f^j} \over
{dt}} + {V^i}_j {{d f^j} \over {dt}} + U^i \eqno{(C.1)}$$

\noindent (the coefficients ${W^i}_{jk}$, ${V^i}_j$ and $U^i$ are functions,
but their exact forms are irrelevant here). Suppose you expect the following
expression to be a first integral of this set:

$$ I = Q_{ij} {{d f^i} \over {dt}} {{d f^j} \over {dt}} + L_i {{d f^i} \over
{dt}} + E, $$

\noindent where $Q_{ij}$, $L_i$ and $E$ are functions (as yet unknown) of the
variables $f^i$. Then the "integral" in the data for "squint" should be:

\bigskip

\begin{verbatim}
(integral (Q11 * (der t f1) ^ 2 + 2 * Q12 * (der t f1) * (der t f2)
    + 2 * Q13 * (der t f1) * (der t f3) + Q22 * (der t f2) ^ 2
    + 2 * Q23 * (der t f2) * (der t f3) + Q33 * (der t f3) ^ 2
    + L1 * (der t f1) + L2 * (der t f2) + L3 * (der t f3) + E) )
\end{verbatim}

\bigskip

\noindent In this case, the other parts of the main data should be:

\bigskip

\begin{verbatim}
 (parameter t)
 (variables f1 f2 f3)
 (functions f1(t) f2(t) f3(t) Q11(f1 f2 f3) Q12(f1 f2 f3) Q13(f1 f2 f3)
            Q22(f1 f2 f3) Q23(f1 f2 f3) Q33(f1 f2 f3) L1(f1 f2 f3)
            L2(f1 f2 f3) L3(f1 f2 f3) E(f1 f2 f3) )
\end{verbatim}

\bigskip

\noindent (you may allow all the "functions" to depend on more variables if you
wish, for example when you suspect that the coefficients of the first integral
will explicitly depend also on t.) The number of the "variables", and
consequently of the equations in the set, is arbitrary.

The program will then print the resulting partial differential equations to be
fulfilled by $Q_{ij}$, $L_i$ and $E$ (but it is up to you then to solve them).

First, the program calculates and prints the total derivative $dI/dt$ in the
form of the equation:

\bigskip

\begin{verbatim}
maineq = <the calculated derivative dI/dt>
\end{verbatim}

\bigskip

\noindent You can prevent this expression from being printed (it will likely be
quite long) by writing:

\bigskip

\begin{verbatim}
(dont print maineq)
\end{verbatim}

\bigskip

\noindent However, it is not reasonable to allow the program to calculate
$dI/dt$ without taking into account the set of equations (C.1). Hence, you
should insert the following item into the data:

\bigskip

\begin{verbatim}
(substitutions
   maineq (der t t f1) = <the right-hand side from (C.1)>
   maineq (der t t f2) = <the right-hand side from (C.1)>
   maineq (der t t f3) = <the right-hand side from (C.1)>
          )
\end{verbatim}

\bigskip

\noindent In this way, the program will eliminate second derivatives of the
$f^i$, thus making use of the set (C.1). The "maineq" means that these three
substitutions should be carried out within the $dI/dt$ (the "main equation"),
before the program goes on with the calculation (see below).

Next, the program finds and prints the coefficients of all the expressions ${{d
f^i} \over {dt}} {{d f^j} \over {dt}} {{d f^k} \over {dt}}$, ${{d f^i} \over
{dt}} {{d f^j} \over {dt}}$, ${{d f^i} \over {dt}}$ and the terms that do not
contain any derivatives of the $f^i$. These coefficients are printed with the
headings that have the following form:
\bigskip

\begin{verbatim}
                                     2
>   THIS IS THE COEFFICIENT OF  f2,    f3,
                                   t      t
\end{verbatim}

\bigskip

\noindent and the coefficients themselves are printed in the following form:

\bigskip

\begin{verbatim}
>   equation  = <the appropriate expression>
            6
\end{verbatim}

\bigskip

\noindent For a set of $n$ equations to determine $n$ functions, there will be
${1 \over 6} (n + 1)(n + 2) (n + 3)$ such coefficients altogether. If you do
not need some of them, you can insert the following item into the data:

\bigskip

\begin{verbatim}
(dont print equation (<n1>) (<n2>) (<n3>) ...)
\end{verbatim}

\bigskip

\noindent where the \verb+<n1>, <n2> <n3>+ are any numbers. (Note: just like in
the program Ortocartan, each of these numbers should be put in parentheses.) If
you wish the program to stop working before it has calculated all the
coefficients, write the following item in the data:

\bigskip

\begin{verbatim}
(stop after (equation (<n>)) )
\end{verbatim}

\bigskip

\noindent where \verb+<n>+ is the number of the last equation to be printed.
Note again: here the \verb+equation <n>+ must be put in parentheses, and the
number \verb+<n>+ must be in its own parentheses. This may be confusing, but
has a justification in the algorithm of the program. Since the justification
would be highly technical, we shall skip it. You can also write:

\bigskip

\begin{verbatim}
(stop after maineq)
\end{verbatim}

\bigskip

\noindent and then the program will not print any of the coefficients.

In practice, you will solve the equations one by one, and substitute the
solutions in the remaining equations until all of them are satisfied. When this
finally happens, the program does not print either the "maineq" or the
"equations", but just prints the following message:

\bigskip

\begin{verbatim}
 THE FIRST INTEGRAL IS ALREADY MAXIMALLY SIMPLIFIED
 AND IS EXPLICITLY CONSTANT
\end{verbatim}

\bigskip

The substitutions are carried out when you write more equations in the
(substitutions ...) list shown above. You can direct the additional
substitutions either to "maineq" or to any "equation". Directing the additional
substitutions to "maineq" is in most cases not reasonable. This is often a long
and complicated expression, and the substitution may take quite a while. It is
more reasonable to direct those additional substitutions to the "equations".
This is done by writing:

\bigskip

\begin{verbatim}
equation (<n1>) (<n2>) (<n3>) .....
\end{verbatim}

\bigskip

\noindent in front of the appropriate substitution, where the numbers
\verb+<n1>, <n2>, <n3> ...+ (each necessarily in parentheses) refer to the
equation numbers.

Whatever devices of the main program Ortocartan make sense here, can be used.

The last item of the data are the two right parentheses.

The "integral" need not necessarily be a polynomial of second degree in the
${f^i},_t$. It can be a polynomial of first degree in ${f^i},_t$, or it may be
independent of the derivatives. However, the program "squint" will go wild and
produce a nonsense result when you try an "integral" that is a polynomial in
${f^i},_t$ of any degree higher than 2 or if the second derivatives of the
${f^i}$ are not eliminated from $dI/dt$.

\section{Versions of Ortocartan for different computers.}

As  mentioned  in  the  introduction,  Ortocartan   was
     originally written and  implemented  in  the  University  of
     Texas Lisp 4.1 on a CDC Cyber 73 computer.  Those  computers
     were scrapped in all sites  where  Ortocartan  was  used  on
     them. The file with the U.T.  Lisp  code  of  Ortocartan  is
     still preserved and can be obtained  from  the
     author (A. K.), but since I do not have access to the U.  T.
     Lisp myself, this version will not  be  maintained  further,
     and has probably already become defunct (which is  a
     pity  because  the  U.T.  Lisp  was  a  dialect  of   superb
     elegance).

The versions of Ortocartan  written  in  an  older  implementation of Cambridge
Lisp for IBM 360/370 computers and in the Slisp/360 version  for  Siemens  4004
computers have been defunct for some years already. Information is missing on
the Lisp 1108 version for UNIVAC computers, but it was not under  our
     care anyway. You may try to obtain information  on  it  from
     Dr. Gokturk Ucoluk, Fizik Bolumu, ODTU, Ankara, Turkey.

The Cambridge Lisp version for the Atari Mega ST computers still exists and
works, although the algebraic computing community seems to have taken divorce
from these computers quite a while ago. It will probably become defunct
together with A.K.'s Atari computer that may be among the last ones still
working. The previous edition of this manual describes how to use that version,
it can be obtained from A. Krasi\'nski.

The only version now under maintenance is the Codemist Standard Lisp version
that will run wherever CSL can be used. See Appendix A for more details.

\section{Sample prints.}

In this appendix, we present copies of original  outputs
     from the computer. The examples exhibit the various features
     of the program.

The printouts are broken into lines without obeying the standard  rules  of
neat printing, such as: do not divide expressions of the form $f(x)$, do not
jump to the  next  line
     just  before  the  right  parenthesis or just after the left
     parenthesis, do not separate the base from its exponent, and
so on. If the printout is to be inserted in a text for  publication, then the
user is advised to use the (output for latex) command described in Section 23.

This untidiness of Ortocartan printouts is the price we had to pay  for  making
     the printing procedure more powerful. Consider the following
example (these are copies of parts of the  original  input and output for  the
program "calculate", transferred here by a text-editor, with some irrelevant or
empty lines removed):

\bigskip

\begin{verbatim}
(calculate '(
    (print example)
    (constants a b c d)
    (coordinates x)
    (functions f (x))
    (operation (deriv x (a ^ (b ^ ((der x f) ^ (c ^ d))))))
        ))
\end{verbatim}

\bigskip

\noindent The output for this example will be:

\bigskip

{\samepage
\begin{verbatim}
(print example)

(I UNDERSTAND YOU REQUEST THE FOLLOWING EXPRESSION TO BE SIMPLIFIED)

                        d
                       c
                   f,
                     x
                  b
>       deriv (x,a       )
\end{verbatim} }
\begin{verbatim}

THE RESULT IS

                     d
                    c        d
                f,          c                                       d
                  x
               b        f,      d                            - 1 + c
                          x
>   result  = a        b       c  log (a) log (b) f,     f,
          1                                         x x    x

(I REALLY LIKED THIS ! CAN I HAVE MORE ? PLEASE ?!?)

END OF WORK (RUN TIME = 50 msec)
\end{verbatim}

\bigskip

\noindent You have  never  seen  a  problem  involving  this  kind  of
     expression? Well, honestly, neither have we. But you can  be
     safely assured that, no matter  how  wildly  complicated  an
     expression is, ortocartan and calculate  will  know  how  to
     handle and print it. This generality made  it  difficult  to
     instruct the program to end each line only at a $+$ or a $-$  in
     the base level, but someday we may come back to the  problem
and solve it. By the way, try to print the same expression with Latex -- we
dare say Ortocartan does it in a more readable way.

Each base-level line of the printout begins with the sign \verb+>+, to
facilitate  reading  those  lines
     that were broken in unusual places. For  each  example,  the
     input data was read from a disk file.

For each of the examples, the input data as prepared by
     the user is shown separately. The output is copied from disk
     files produced according to the instructions  from  sec. 7.

\subsection{Example I: The Robertson-Walker Metric}

$$ ds^2 = dt^2 - R^2 [dr^2 /(1 - kr^2) + r^2 (d\vartheta^2 + \sin^2\vartheta d
\varphi^2)], $$

\noindent where $R = R (t)$ is an arbitrary function and $k$ is an arbitrary
constant.

\bigskip

\centerline{Reference}

W. Rindler, {\it Essential relativity}. Van Nostrand Reinhold Company,  New
York-Cincinnati-Toronto-London-Melbourne 1969, p.
     234.

The input data is here:

\bigskip

\begin{verbatim}
(setq !*lower nil)
 (ortocartan '(
    (Robertson-Walker metric)
    (coordinates t r theta phi)
    (functions R (t) )
    (constants k)
    (ematrix 1 0 0 0 0 (R / (1 - k * r ^ 2) ^ (1 2)) 0 0 0 0
             (r * R) 0 0 0 0 (r * R * (sin theta))  )
     ))
(setq !*lower t)
\end{verbatim}

\bigskip

\centerline{Notes}

The metric is conformally flat  -  please  verify  that
     Ortocartan has recognized it. Here we asked the Lisp  system
     to treat upper- and lower case letters as different  symbols
     (the first line of input). The command in the last
     line of the input reverses the command from the first  line.

The output is this (some irrelevant or empty lines were deleted by the
text-editor. The irrelevant lines are not parts of Ortocartan's output, but are
responses of the Lisp system that are not interesting for the user.):

\bigskip

\begin{verbatim}
(Robertson - Walker metric)


           0
>   ematrix  . = 1
             0

           1               2 - (1/2)
>   ematrix  . = R (1 - k r )
             1

           2
>   ematrix  . = r R
             2

           3
>   ematrix  . = r R sin (theta)
             3

(ematrix completed)
(TIME = 190 msec)

                           2  3         2 - (1/2)
>   DETERMINANT EMATRIX = r  R  (1 - k r )        sin (theta)


(DETERMINANT EMATRIX calculated)
(TIME = 260 msec)


      0
>   ie.   = 1
        0

      1      -1         2 (1/2)
>   ie.   = R   (1 - k r )
        1

      2      -1  -1
>   ie.   = r   R
        2

      3      -1  -1    -1
>   ie.   = r   R   sin  (theta)
        3

(ie calculated)
(TIME = 400 msec)

          1              -1
>   agamma      = (1/2) R   R,
            0 1               t

          2              -1
>   agamma      = (1/2) R   R,
            0 2               t

          2              -1  -1         2 (1/2)
>   agamma      = (1/2) r   R   (1 - k r )
            1 2

          3              -1
>   agamma      = (1/2) R   R,
            0 3               t

          3              -1  -1         2 (1/2)
>   agamma      = (1/2) r   R   (1 - k r )
            1 3

          3              -1  -1                -1
>   agamma      = (1/2) r   R   cos (theta) sin  (theta)
            2 3

(agamma calculated)
(TIME = 590 msec)

(agamma completed)
(TIME = 590 msec)

         0        -1
>   gamma      = R   R,
           1 1         t


         0        -1
>   gamma      = R   R,
           2 2         t


         0        -1
>   gamma      = R   R,
           3 3         t


         1          -1  -1         2 (1/2)
>   gamma      = - r   R   (1 - k r )
           2 2


         1          -1  -1         2 (1/2)
>   gamma      = - r   R   (1 - k r )
           3 3


         2          -1  -1                -1
>   gamma      = - r   R   cos (theta) sin  (theta)
           3 3


(gamma calculated)
(TIME = 680 msec)

(gamma completed)
(TIME = 690 msec)

                      -1
>   riemann        = R   R,
           0 1 0 1         t t

                      -1
>   riemann        = R   R,
           0 2 0 2         t t

                      -1
>   riemann        = R   R,
           0 3 0 3         t t

                          -2    -2     2
>   riemann        = - k R   - R   R,
           1 2 1 2                   t

                          -2    -2     2
>   riemann        = - k R   - R   R,
           1 3 1 3                   t

                          -2    -2     2
>   riemann        = - k R   - R   R,
           2 3 2 3                   t

(riemann calculated)
(TIME = 930 msec)

(riemann completed)
(TIME = 980 msec)

                    -1
>   ricci    = - 3 R   R,
         0 0             t t

                    -2      -2     2    -1
>   ricci    = 2 k R   + 2 R   R,    + R   R,
         1 1                     t           t t

                    -2      -2     2    -1
>   ricci    = 2 k R   + 2 R   R,    + R   R,
         2 2                     t           t t

                    -2      -2     2    -1
>   ricci    = 2 k R   + 2 R   R,    + R   R,
         3 3                     t           t t

(ricci calculated)
(TIME = 1000 msec)

(CURVATURE INVARIANT calculated)
(TIME = 1000 msec)

                                 -2      -2     2      -1
>   CURVATURE INVARIANT = - 6 k R   - 6 R   R,    - 6 R   R,
                                              t             t t
(weyl calculated)
(TIME = 1090 msec)


(ALL COMPONENTS OF THE WEYL TENSOR ARE ZERO)


(I REALLY LIKED THIS! CAN I HAVE MORE ?   PLEASE ?!?)

END OF WORK
(RUN TIME = 1090 msec)
\end{verbatim}

\subsection{Example II: The most general spherically symmetric metric in the
Schwarzschild coordinate system.}

$$ ds^2 = {\rm e}^{2\nu} dt^2 - {\rm e}^{2\mu} dr^2 - r^2(d\vartheta^2 + \sin^2
\vartheta d\varphi^2), $$

\noindent where $\nu = \nu(t,r)$ and $\mu = \mu(t,r)$ are arbitrary functions.

\bigskip

\centerline{Reference}

Most textbooks on general relativity, e.g. J.L. Synge, {\it Relativity,  the
general theory}. North Holland Publishing Company, Amsterdam 1960, p. 265.

The input data is here:

\bigskip

\begin{verbatim}
(ortocartan '(
    (spherically symmetric standard)
    (coordinates t r theta phi)
    (functions nu (t r) mu (t r))
    (ematrix (exp nu) 0 0 0 0 (exp mu) 0 0 0 0
             r 0 0 0 0 (r * (sin theta)))
    (rmargin 80)
            ))
\end{verbatim}

\bigskip

\centerline{Notes}

nu stands for $\nu$, mu stands for $\mu$. (rmargin  80)  will suit the output
to the pagewidth of this text. Note the untidy linebreaks in the Riemann and
Ricci tensors. Some irrelevant and empty lines have been deleted again. The output is

\bigskip

\begin{verbatim}
(spherically symmetric standard)

           0
>   ematrix  . = exp (nu)
             0

           1
>   ematrix  . = exp (mu)
             1

           2
>   ematrix  . = r
             2

           3
>   ematrix  . = r sin (theta)
             3

(ematrix completed)
(TIME = 20 msec)

                           2
>   DETERMINANT EMATRIX = r  exp (nu + mu) sin (theta)


(DETERMINANT EMATRIX calculated)
(TIME = 40 msec)


      0
>   ie.   = exp (- nu)
        0

      1
>   ie.   = exp (- mu)
        1

      2      -1
>   ie.   = r
        2

      3      -1    -1
>   ie.   = r   sin  (theta)
        3

(ie calculated)
(TIME = 200 msec)


          0
>   agamma      = - (1/2) exp (- mu) nu,
            0 1                         r

          1
>   agamma      = (1/2) exp (- nu) mu,
            0 1                       t

          2              -1
>   agamma      = (1/2) r   exp (- mu)
            1 2

          3              -1
>   agamma      = (1/2) r   exp (- mu)
            1 3

          3              -1                -1
>   agamma      = (1/2) r   cos (theta) sin  (theta)
            2 3

(agamma calculated)
(TIME = 360 msec)

(agamma completed)
(TIME = 360 msec)

         0
>   gamma      = exp (- mu) nu,
           1 0                 r


         0
>   gamma      = exp (- nu) mu,
           1 1                 t


         1          -1
>   gamma      = - r   exp (- mu)
           2 2


         1          -1
>   gamma      = - r   exp (- mu)
           3 3


         2          -1                -1
>   gamma      = - r   cos (theta) sin  (theta)
           3 3


(gamma calculated)
(TIME = 380 msec)


(gamma completed)
(TIME = 380 msec)

                                       2
>   riemann        = exp (- 2 nu) mu,    + exp (- 2 nu) mu,     - exp (- 2 nu)
           0 1 0 1                   t                     t t

                                        2
>       nu,   mu,   - exp (- 2 mu) nu,    - exp (- 2 mu) nu,     + exp (- 2 mu)
           t     t                    r                     r r


>       nu,   mu,
           r     r

                        -1
>   riemann        = - r   exp (- 2 mu) nu,
           0 2 0 2                         r

                        -1
>   riemann        = - r   exp (- nu - mu) mu,
           0 2 1 2                            t

                        -1
>   riemann        = - r   exp (- 2 mu) nu,
           0 3 0 3                         r

                        -1
>   riemann        = - r   exp (- nu - mu) mu,
           0 3 1 3                            t

                        -1
>   riemann        = - r   exp (- 2 mu) mu,
           1 2 1 2                         r

                        -1
>   riemann        = - r   exp (- 2 mu) mu,
           1 3 1 3                         r

                      -2                 -2
>   riemann        = r   exp (- 2 mu) - r
           2 3 2 3

(riemann calculated)
(TIME = 690 msec)


(riemann completed)
(TIME = 740 msec)

                  -1                                        2
>   ricci    = 2 r   exp (- 2 mu) nu,   - exp (- 2 nu) mu,    - exp (- 2 nu) mu
         0 0                         r                    t

                                                             2
>       ,     + exp (- 2 nu) nu,   mu,   + exp (- 2 mu) nu,    + exp (- 2 mu)
         t t                    t     t                    r


>       nu,     - exp (- 2 mu) nu,   mu,
           r r                    r     r

                  -1
>   ricci    = 2 r   exp (- nu - mu) mu,
         0 1                            t

                  -1                                        2
>   ricci    = 2 r   exp (- 2 mu) mu,   + exp (- 2 nu) mu,    + exp (- 2 nu) mu
         1 1                         r                    t

                                                             2
>       ,     - exp (- 2 nu) nu,   mu,   - exp (- 2 mu) nu,    - exp (- 2 mu)
         t t                    t     t                    r

\end{verbatim}
{\samepage
\begin{verbatim}
>       nu,     + exp (- 2 mu) nu,   mu,
           r r                    r     r
\end{verbatim}
       }
\begin{verbatim}

                  -2                 -1                       -1
>   ricci    = - r   exp (- 2 mu) - r   exp (- 2 mu) nu,   + r   exp (- 2 mu)
         2 2                                            r

                 -2
>       mu,   + r
           r

                  -2                 -1                       -1
>   ricci    = - r   exp (- 2 mu) - r   exp (- 2 mu) nu,   + r   exp (- 2 mu)
         3 3                                            r

                 -2
>       mu,   + r
           r

(ricci calculated)
(TIME = 820 msec)

(CURVATURE INVARIANT calculated)
(TIME = 850 msec)

                             -2                   -1                         -1
>   CURVATURE INVARIANT = 2 r   exp (- 2 mu) + 4 r   exp (- 2 mu) nu,   - 4 r
                                                                     r

                                                 2
>       exp (- 2 mu) mu,   - 2 exp (- 2 nu) mu,    - 2 exp (- 2 nu) mu,     +
                        r                      t                       t t

                                                         2
>       2 exp (- 2 nu) nu,   mu,   + 2 exp (- 2 mu) nu,    + 2 exp (- 2 mu) nu,
                          t     t                      r

                                               -2
>            - 2 exp (- 2 mu) nu,   mu,   - 2 r
        r r                      r     r


\end{verbatim}
{\samepage
\begin{verbatim}
                           -2                       -1
>   weyl        = - (1/3) r   exp (- 2 mu) + (1/3) r   exp (- 2 mu) nu,   - (1/
        0 1 0 1                                                        r
\end{verbatim}
          }
\begin{verbatim}

            -1                                              2
>       3) r   exp (- 2 mu) mu,   + (1/3) exp (- 2 nu) mu,    + (1/3) exp (- 2
                               r                          t


>       nu) mu,     - (1/3) exp (- 2 nu) nu,   mu,   - (1/3) exp (- 2 mu) nu,
               t t                          t     t                          r

        2
>         - (1/3) exp (- 2 mu) nu,     + (1/3) exp (- 2 mu) nu,   mu,   + (1/3)
                                  r r                          r     r

         -2
>       r


                         -2                       -1
>   weyl        = (1/6) r   exp (- 2 mu) - (1/6) r   exp (- 2 mu) nu,   + (1/6)
        0 2 0 2                                                      r

         -1                                              2
>       r   exp (- 2 mu) mu,   - (1/6) exp (- 2 nu) mu,    - (1/6) exp (- 2 nu)
                            r                          t

                                                                           2
>       mu,     + (1/6) exp (- 2 nu) nu,   mu,   + (1/6) exp (- 2 mu) nu,    +
           t t                          t     t                          r

                                                                             -2
>       (1/6) exp (- 2 mu) nu,     - (1/6) exp (- 2 mu) nu,   mu,   - (1/6) r
                              r r                          r     r

                         -2                       -1
>   weyl        = (1/6) r   exp (- 2 mu) - (1/6) r   exp (- 2 mu) nu,   + (1/6)
        0 3 0 3                                                      r

         -1                                              2
>       r   exp (- 2 mu) mu,   - (1/6) exp (- 2 nu) mu,    - (1/6) exp (- 2 nu)
                            r                          t

                                                                           2
>       mu,     + (1/6) exp (- 2 nu) nu,   mu,   + (1/6) exp (- 2 mu) nu,    +
           t t                          t     t                          r

                                                                             -2
>       (1/6) exp (- 2 mu) nu,     - (1/6) exp (- 2 mu) nu,   mu,   - (1/6) r
                              r r                          r     r

                           -2                       -1
>   weyl        = - (1/6) r   exp (- 2 mu) + (1/6) r   exp (- 2 mu) nu,   - (1/
        1 2 1 2                                                        r

            -1                                              2
>       6) r   exp (- 2 mu) mu,   + (1/6) exp (- 2 nu) mu,    + (1/6) exp (- 2
                               r                          t


>       nu) mu,     - (1/6) exp (- 2 nu) nu,   mu,   - (1/6) exp (- 2 mu) nu,
               t t                          t     t                          r

        2
>         - (1/6) exp (- 2 mu) nu,     + (1/6) exp (- 2 mu) nu,   mu,   + (1/6)
                                  r r                          r     r

         -2
>       r


                           -2                       -1
>   weyl        = - (1/6) r   exp (- 2 mu) + (1/6) r   exp (- 2 mu) nu,   - (1/
        1 3 1 3                                                        r

            -1                                              2
>       6) r   exp (- 2 mu) mu,   + (1/6) exp (- 2 nu) mu,    + (1/6) exp (- 2
                               r                          t


>       nu) mu,     - (1/6) exp (- 2 nu) nu,   mu,   - (1/6) exp (- 2 mu) nu,
               t t                          t     t                          r

        2
>         - (1/6) exp (- 2 mu) nu,     + (1/6) exp (- 2 mu) nu,   mu,   + (1/6)
                                  r r                          r     r

\end{verbatim}
{\samepage
\begin{verbatim}
         -2
>       r
\end{verbatim}
          }
\begin{verbatim}


                         -2                       -1
>   weyl        = (1/3) r   exp (- 2 mu) - (1/3) r   exp (- 2 mu) nu,   + (1/3)
        2 3 2 3                                                      r

         -1                                              2
>       r   exp (- 2 mu) mu,   - (1/3) exp (- 2 nu) mu,    - (1/3) exp (- 2 nu)
                            r                          t

                                                                           2
>       mu,     + (1/3) exp (- 2 nu) nu,   mu,   + (1/3) exp (- 2 mu) nu,    +
           t t                          t     t                          r

                                                                             -2
>       (1/3) exp (- 2 mu) nu,     - (1/3) exp (- 2 mu) nu,   mu,   - (1/3) r
                              r r                          r     r

(weyl calculated)
(TIME = 1430 msec)


(I REALLY LIKED THIS! CAN I HAVE MORE ?   PLEASE ?!?)

END OF WORK
(RUN TIME = 1430 msec)
\end{verbatim}

\bigskip

This example provides a good opportunity to demonstrate the output in the Latex
format. Let us run the same example with the data modified as follows:

\bigskip

\begin{verbatim}
(ortocartan '(
    (spherically symmetric standard)
    (coordinates t r theta phi)
    (functions nu (t r) mu (t r))
    (ematrix (exp nu) 0 0 0 0 (exp mu) 0 0 0 0
             r 0 0 0 0 (r * (sin theta)))
    (output for latex)
    (dont print ematrix determinant ie agamma gamma riemann)
    (stop after ricci)
            ))
\end{verbatim}

\bigskip

\noindent The command (rmargin ...) is not reasonable here (in fact, with the
"(output for latex)" command, Ortocartan would ignore any (rmargin ...)
command) because the final layout will be made by Latex anyway. In order to
save space we have asked Ortocartan not to print anything before the Ricci
tensor, and to stop after calculating the Ricci tensor, i.e. to give up on the
scalar curvature and the Weyl tensor. The output was produced in the Latex
code, that is first shown verbatim:

\bigskip

\begin{verbatim}
(spherically symmetric standard)

(ematrix completed)
(TIME = 240 msec)


(DETERMINANT EMATRIX calculated)
(TIME = 270 msec)

(ie calculated)
(TIME = 460 msec)

(agamma calculated)
(TIME = 780 msec)


(agamma completed)
(TIME = 780 msec)

(gamma calculated)
(TIME = 830 msec)

(gamma completed)
(TIME = 830 msec)

(riemann calculated)
(TIME = 1070 msec)


(riemann completed)
(TIME = 1110 msec)

\begin{equation}
ricci_{00} = 2r^{-1}\exp (- 2{\mu}){\nu},_{r} - \exp (- 2{\nu}){{\mu},_{t}}^{2}
$$

$$
- \exp (- 2{\nu}){\mu},_{t t} + \exp (- 2{\nu}){\nu},_{t}{\mu},_{t} + \exp (- 2{\mu}){{\nu},_{r}}^{2} + \exp (- 2{\mu})
$$


$$
{\nu},_{r r} - \exp (- 2{\mu}){\nu},_{r}{\mu},_{r}
\end{equation}

\begin{equation}
ricci_{01} = 2r^{-1}\exp (- {\nu} - {\mu}){\mu},_{t}
\end{equation}

\begin{equation}
ricci_{11} = 2r^{-1}\exp (- 2{\mu}){\mu},_{r} + \exp (- 2{\nu}){{\mu},_{t}}^{2}
$$

$$
+ \exp (- 2{\nu}){\mu},_{t t} - \exp (- 2{\nu}){\nu},_{t}{\mu},_{t} - \exp (- 2{\mu}){{\nu},_{r}}^{2} - \exp (- 2{\mu})
$$

$$
{\nu},_{r r} + \exp (- 2{\mu}){\nu},_{r}{\mu},_{r}
\end{equation}

\begin{equation}
ricci_{22} = - r^{-2}\exp (- 2{\mu}) - r^{-1}\exp (- 2{\mu}){\nu},_{r}
$$

$$
+ r^{-1}\exp (- 2{\mu}){\mu},_{r} + r^{-2}
\end{equation}

\begin{equation}
ricci_{33} = - r^{-2}\exp (- 2{\mu}) - r^{-1}\exp (- 2{\mu}){\nu},_{r}
$$

$$
+ r^{-1}\exp (- 2{\mu}){\mu},_{r} + r^{-2}
\end{equation}

(ricci calculated)
(TIME = 1250 msec)


(I REALLY LIKED THIS! CAN I HAVE MORE ?   PLEASE ?!?)

END OF WORK
(RUN TIME = 1250 msec)
\end{verbatim}

\bigskip

\noindent Yes, some lines are too long to fit into this page. This output was
not meant to be shown to humans, but to be read by Latex. But verbatim is
verbatim -- we do not cheat. Now the same output will be inserted into this
text as part of the Latex code (with the time-messages deleted). Note how
Ortocartan has recognized the Greek letters and printed them in Latex's
favourite way, and what Latex will now do with them.

\begin{equation}
ricci_{00} = 2r^{-1}\exp (- 2{\mu}){\nu},_{r} - \exp (- 2{\nu}){{\mu},_{t}}^{2}
$$

$$
- \exp (- 2{\nu}){\mu},_{t t} + \exp (- 2{\nu}){\nu},_{t}{\mu},_{t} + \exp (- 2{\mu}){{\nu},_{r}}^{2} + \exp (- 2{\mu})
$$

$$
{\nu},_{r r} - \exp (- 2{\mu}){\nu},_{r}{\mu},_{r}
\end{equation}

\begin{equation}
ricci_{01} = 2r^{-1}\exp (- {\nu} - {\mu}){\mu},_{t}
\end{equation}

\begin{equation}
ricci_{11} = 2r^{-1}\exp (- 2{\mu}){\mu},_{r} + \exp (- 2{\nu}){{\mu},_{t}}^{2}
$$

$$
+ \exp (- 2{\nu}){\mu},_{t t} - \exp (- 2{\nu}){\nu},_{t}{\mu},_{t} - \exp (- 2{\mu}){{\nu},_{r}}^{2} - \exp (- 2{\mu})
$$

$$
{\nu},_{r r} + \exp (- 2{\mu}){\nu},_{r}{\mu},_{r}
\end{equation}

\begin{equation}
ricci_{22} = - r^{-2}\exp (- 2{\mu}) - r^{-1}\exp (- 2{\mu}){\nu},_{r}
$$

$$
+ r^{-1}\exp (- 2{\mu}){\mu},_{r} + r^{-2}
\end{equation}

\begin{equation}
ricci_{33} = - r^{-2}\exp (- 2{\mu}) - r^{-1}\exp (- 2{\mu}){\nu},_{r}
$$

$$
+ r^{-1}\exp (- 2{\mu}){\mu},_{r} + r^{-2}
\end{equation}

\noindent (The equation numbers above are continued from Section 1 because we
have not readjusted the equation counter. You can do it in your own Latex
preamble.)

\subsection{Example III: The Stephani solution.}

$$ ds^2 = D^2 dt^2 - (R / V)^2 (dx^2 + dy^2 + dz^2), $$

\noindent where

$$ V = 1 + {1 \over 4} k [(x - x_0)^2  + (y - y_0 )^2  + (z - z_0)^2], $$

$$ D = F (V,_t / V - R,_t / R), $$

\noindent $R_0, k, x_0 , y_0 , z_0$ and $F$ are arbitrary functions of $t$.

\bigskip

\centerline{Reference:}

H. Stephani, {\it Commun. Math. Phys.} {\bf 4}, 137 (1967).

The input data is here:

\bigskip

\begin{verbatim}
(setq !*lower nil)
(ortocartan '(
    (the Stephani solution)
    (coordinates t x y z)
    (functions C (t) k (t) R (t) X0(t) Y0(t) Z0(t) F (t)
               V (t x y z)  D (t x y z)  )
    (ematrix D  0  0  0  0  (R / V)  0  0  0  0  (R / V)
                0  0  0  0  (R / V)  )
    (tensors einstein)
    (markers m)
    (substitutions
    riemann (0 1 0 1) (0 2 0 2) (0 3 0 3)
 ((der t D)*(der t V)) = ((der t D) * V * (D / F + (der t R) / R))
    agamma riemann
 (der m D) = (deriv m (F * ((der t V) / V - (der t R)/ R)))
    riemann
 (der t R) = (R * ((der t V) / V - D / F))
    riemann
 ((der z V) ^ 2) = ((V - 1) * k - (der x V) ^ 2 - (der y V) ^ 2)
    riemann
 ((der t V) / V) = (D / F + (der t R) / R)
    riemann
 V = (1 + (1 4) * k * ((x - X0) ^ 2 + (y - Y0) ^ 2 + (z - Z0) ^ 2))
    riemann
 k = ((C ^ 2 - 1 / F ^ 2) * R ^ 2)
                    )
    (rmargin 61)
    (dont print messages agamma)
                   ))
   (setq !*lower t)
\end{verbatim}

\bigskip

\centerline{Notes:}

This is the most  general  conformally  flat  nonstatic
     perfect fluid solution of Einstein's equations  (see  Kramer
     et al., {\it Exact  solutions  of  Einstein's  field  equations},
     Cambridge University Press 1980, p. 371, theorem 32.15). The
matter density in it, denoted $3C^2(t)$ here, is a function  of
     $t$ only and is related to $k$, $F$ and $R $by:

$$ k = (C^2 - 1/ F^2) R^2. $$

\noindent The Einstein tensor is calculated here in  addition  to  the
     other  quantities,  and  the  messages  about   unsuccessful
     attempts at substitutions are suppressed. Also, printing the
antisymmetrized Ricci rotation coefficients agamma  is  suppressed. In this
example, instead of  substituting  the  explicit forms of the components of the
metric tensor,  it  is
     more reasonable to feed the information about them into  the
     formulae step  by  step.  Each  substitution  results  in  a
partial simplification of the formulae. The  fourth  substitution stems from
the identity:

$$ {V,_x}^2 + {V,_y}^2 + {V,_z}^2   = k (V - 1). $$

\noindent The second substitution  makes  use  of  the  marker  m,  it results
in replacing $D,_x$, $D,_y$ and $D,_z$ by  the  appropriate derivatives of $[F
(V,_t / V - R,_t / R)]$.

The  first,  third  and  fifth  substitutions  use  the definition of $D$, but
apply it only in certain contexts.  For example, in the  first  substitution,
$V,_t$ is  replaced  by $V (D / F + R,_t / R)$, but only in those instances
where $V,_t$ is multiplied by $D,_t$.

Each  consecutive  substitution   was   guessed   after
     inspecting  the  output  obtained  without   it.   You   are
     encouraged to repeat this procedure. Run this example  first
     without any substitutions, then add the first  one  and  see
     what has changed, then add the second one, and so on. If you
     are clever with using your editor, then you may  use  it  to
     cut the data for this example out of the file containing
      this manual.

The output is here (again, with irrelevant lines deleted):

\bigskip

\begin{verbatim}
(the Stephani solution)

substitutions

riemann (0 1 0 1) (0 2 0 2) (0 3 0 3)

                 -1                -1
>   V,   D,   = R   V R,   D,   + F   V D D,
      t    t            t    t              t

agamma riemann

               -1         -1
>   D,   = (F V   V,   - R   F R,  ),
      m             t            t   m

riemann

                -1        -1
>   R,   = - R F   D + R V   V,
      t                        t

riemann

        2                   2       2
>   V,    = - k + k V - V,    - V,
      z                   x       y

riemann

     -1         -1         -1
>   V   V,   = R   R,   + F   D
          t          t

riemann


>   V = 1 - (1/2) x k X0 - (1/2) y k Y0 - (1/2) z k Z0 + (1/


               2             2             2          2
>       4) k X0  + (1/4) k Y0  + (1/4) k Z0  + (1/4) x  k +


               2            2
>       (1/4) y  k + (1/4) z  k


riemann

         2  2    2  -2
>   k = C  R  - R  F


           0
>   ematrix  . = D
             0

           1        -1
>   ematrix  . = R V
             1

           2        -1
>   ematrix  . = R V
             2

           3        -1
>   ematrix  . = R V
             3

(ematrix completed)
(TIME = 500 msec)

                           3  -3
>   DETERMINANT EMATRIX = R  V   D


(DETERMINANT EMATRIX calculated)
(TIME = 520 msec)


      0      -1
>   ie.   = D
        0

      1      -1
>   ie.   = R   V
        1

      2      -1
>   ie.   = R   V
        2

      3      -1
>   ie.   = R   V
        3

(ie calculated)
(TIME = 760 msec)

(agamma calculated)
(TIME = 1060 msec)

(agamma completed)
(TIME = 1060 msec)

         0          -1    -1  -1              -1    -1
>   gamma      = - R   F V   D   V,   V,   + R   F D   V,
           1 0                     t    x                t

>
        x
\end{verbatim}

\bigskip

\noindent Oops! This is very untidy printing -- the $t$ and $x$ in the second
derivative of $V$ have been separated. But you can easily learn to live with
this, and use the (output for latex) when you need a really neat printout.

\bigskip

\begin{verbatim}

         0        -1  -1         -1  -1
>   gamma      = R   D   R,   - V   D   V,
           1 1             t              t


         0          -1    -1  -1              -1    -1
>   gamma      = - R   F V   D   V,   V,   + R   F D   V,
           2 0                     t    y                t


>
        y


         0        -1  -1         -1  -1
>   gamma      = R   D   R,   - V   D   V,
           2 2             t              t


         0          -1    -1  -1              -1    -1
>   gamma      = - R   F V   D   V,   V,   + R   F D   V,
           3 0                     t    z                t


>
        z


         0        -1  -1         -1  -1
>   gamma      = R   D   R,   - V   D   V,
           3 3             t              t


         1          -1
>   gamma      = - R   V,
           2 1           y


         1        -1
>   gamma      = R   V,
           2 2         x


         1          -1
>   gamma      = - R   V,
           3 1           z


         1        -1
>   gamma      = R   V,
           3 3         x


         2          -1
>   gamma      = - R   V,
           3 2           z


         2        -1
>   gamma      = R   V,
           3 3         y


(gamma calculated)
(TIME = 1160 msec)

(gamma completed)
(TIME = 1170 msec)

                            -1         2
>   riemann        = - C F D   C,   + C
           0 1 0 1               t

                            -1         2
>   riemann        = - C F D   C,   + C
           0 2 0 2               t

                            -1         2
>   riemann        = - C F D   C,   + C
           0 3 0 3               t

                        2
>   riemann        = - C
           1 2 1 2

                        2
>   riemann        = - C
           1 3 1 3

                        2
>   riemann        = - C
           2 3 2 3

(riemann calculated)
(TIME = 5931 msec)

(riemann completed)
(TIME = 6001 msec)

                      -1           2
>   ricci    = 3 C F D   C,   - 3 C
         0 0               t

                      -1           2
>   ricci    = - C F D   C,   + 3 C
         1 1               t

                      -1           2
>   ricci    = - C F D   C,   + 3 C
         2 2               t

                      -1           2
>   ricci    = - C F D   C,   + 3 C
         3 3               t

(ricci calculated)
(TIME = 6021 msec)

(CURVATURE INVARIANT calculated)
(TIME = 6031 msec)

\end{verbatim}
{\samepage
\begin{verbatim}
                                 -1            2
>   CURVATURE INVARIANT = 6 C F D   C,   - 12 C
                                      t
\end{verbatim}
   }
\begin{verbatim}

                     2
>   einstein    = 3 C
            0 0


                         -1           2
>   einstein    = 2 C F D   C,   - 3 C
            1 1               t


                         -1           2
>   einstein    = 2 C F D   C,   - 3 C
            2 2               t


                         -1           2
>   einstein    = 2 C F D   C,   - 3 C
            3 3               t


(einstein calculated)
(TIME = 6061 msec)

(weyl calculated)
(TIME = 6141 msec)

(ALL COMPONENTS OF THE WEYL TENSOR ARE ZERO)

(I REALLY LIKED THIS! CAN I HAVE MORE ?   PLEASE ?!?)

END OF WORK
(RUN TIME = 6141 msec)
\end{verbatim}

\subsection{Example IV: The Nariai solution.}

$$ ds^2 = P^2 dt^2 - (L^2/r^2) (dx^2 + dy^2 + dz^2), $$

\noindent where $P = a (t) \cos (\log (r / L)) + b (t) \sin (\log (r / L))$,
$L$ is a constant, $r^2 = x^2 + y^2 + z^2$, $a(t)$ and $b(t)$ are arbitrary
functions of time.

\bigskip

\centerline{Reference:}

A. Krasi\'nski, J. Pleba\'nski, {\it Rep. Math. Phys.} {\bf 17}, 217 (1980).

The input data is:

\bigskip

\begin{verbatim}
(setq !*lower nil)
 (ortocartan '(
    (Nariai solution)
    (coordinates t x y z)
    (constants L)
    (markers m)
    (ematrix P   0   0   0
        0   (L / r)   0   0
        0   0   (L / r)   0
        0   0   0   (L / r))
    (symbols
      P = (a * (cos (log (r / L))) + b * (sin (log (r / L)))) )
    (substitutions
    agamma riemann
     (der m r) = (m / r)
    agamma riemann
     (cos (log (r / L))) = (P / a - b * (sin (log (r / L)))/ a)
    agamma riemann
     (z ^ 2) = (r ^ 2 - x ^ 2 - y ^ 2)
            )
    (functions a (t) b (t) r(x y z) )
    (dont print messages)
    (rmargin 61)
          ))
(setq !*lower t)
\end{verbatim}

\bigskip

\centerline{Notes}

This is a solution of the Einstein's equations in empty
     space  with  the cosmological term (note  the  form  of  the
     Ricci tensor in the output!). Note the use of markers  which
     implies substitutions by pattern-matching, they are  simpler
     than in the previous example, so may be  more  readable.  It
     was not specified here in which  components  of  agamma  and
     riemann the substitutions should be performed.  The  command
(dont print messages) suppresses the information about  unsuccessful attempts
at substitutions.

The output is:

\bigskip

\begin{verbatim}
(Nariai solution)

symbols

>   P = a cos (- log (L) + log (r)) + b sin (- log (L) + log

>       (r))

substitutions

agamma riemann

              -1
>   r,   = m r
      m


agamma riemann

                                   -1
>   cos (- log (L) + log (r)) = - a   b sin (- log (L) + log


                -1
>       (r)) + a   P


agamma riemann

     2      2    2    2
>   z  = - x  - y  + r


           0
>   ematrix  . = P
             0

           1        -1
>   ematrix  . = L r
             1

           2        -1
>   ematrix  . = L r
             2

           3        -1
>   ematrix  . = L r
             3

(ematrix completed)
(TIME = 300 msec)


                           3  -3
>   DETERMINANT EMATRIX = L  r   P


(DETERMINANT EMATRIX calculated)
(TIME = 340 msec)


      0      -1
>   ie.   = P
        0

      1      -1
>   ie.   = L   r
        1

      2      -1
>   ie.   = L   r
        2

      3      -1
>   ie.   = L   r
        3

(ie calculated)
(TIME = 490 msec)


          0              -1    -1    -1
>   agamma      = (1/2) L   x r   a P   sin (- log (L) + log
            0 1

                      -1    -1  -1            -1    -1  -1
>       (r)) - (1/2) L   x r   a   b + (1/2) L   x r   a   b


        2  -1
>         P   sin (- log (L) + log (r))


\end{verbatim}
{\samepage
\begin{verbatim}
          0              -1    -1    -1
>   agamma      = (1/2) L   y r   a P   sin (- log (L) + log
            0 2
\end{verbatim}
     }
\begin{verbatim}

                      -1    -1  -1            -1    -1  -1
>       (r)) - (1/2) L   y r   a   b + (1/2) L   y r   a   b


        2  -1
>         P   sin (- log (L) + log (r))


          0              -1    -1    -1
>   agamma      = (1/2) L   z r   a P   sin (- log (L) + log
            0 3

                      -1    -1  -1            -1    -1  -1
>       (r)) - (1/2) L   z r   a   b + (1/2) L   z r   a   b


        2  -1
>         P   sin (- log (L) + log (r))


          1              -1    -1
>   agamma      = (1/2) L   y r
            1 2

          1              -1    -1
>   agamma      = (1/2) L   z r
            1 3

          2                -1    -1
>   agamma      = - (1/2) L   x r
            1 2

          2              -1    -1
>   agamma      = (1/2) L   z r
            2 3

          3                -1    -1
>   agamma      = - (1/2) L   x r
            1 3

          3                -1    -1
>   agamma      = - (1/2) L   y r
            2 3

(agamma calculated)
(TIME = 850 msec)

(agamma completed)
(TIME = 860 msec)

         0          -1    -1    -1
>   gamma      = - L   x r   a P   sin (- log (L) + log (r))
           1 0

           -1    -1  -1      -1    -1  -1  2  -1
>       + L   x r   a   b - L   x r   a   b  P   sin (- log



>       (L) + log (r))



         0          -1    -1    -1
>   gamma      = - L   y r   a P   sin (- log (L) + log (r))
           2 0

           -1    -1  -1      -1    -1  -1  2  -1
>       + L   y r   a   b - L   y r   a   b  P   sin (- log



>       (L) + log (r))



         0          -1    -1    -1
>   gamma      = - L   z r   a P   sin (- log (L) + log (r))
           3 0

           -1    -1  -1      -1    -1  -1  2  -1
>       + L   z r   a   b - L   z r   a   b  P   sin (- log



>       (L) + log (r))



         1          -1    -1
>   gamma      = - L   y r
           2 1


         1        -1    -1
>   gamma      = L   x r
           2 2


         1          -1    -1
>   gamma      = - L   z r
           3 1


         1        -1    -1
>   gamma      = L   x r
           3 3


         2          -1    -1
>   gamma      = - L   z r
           3 2


         2        -1    -1
>   gamma      = L   y r
           3 3


(gamma calculated)
(TIME = 970 msec)

(gamma completed)
(TIME = 980 msec)

                      -2  2  -2
>   riemann        = L   x  r
           0 1 0 1

\end{verbatim}
{\samepage
\begin{verbatim}
                      -2      -2
>   riemann        = L   x y r
           0 1 0 2
\end{verbatim}
       }
\begin{verbatim}

                      -2      -2
>   riemann        = L   x z r
           0 1 0 3

                      -2  2  -2
>   riemann        = L   y  r
           0 2 0 2

                      -2      -2
>   riemann        = L   y z r
           0 2 0 3

                        -2  2  -2    -2  2  -2    -2
>   riemann        = - L   x  r   - L   y  r   + L
           0 3 0 3

                      -2  2  -2    -2  2  -2    -2
>   riemann        = L   x  r   + L   y  r   - L
           1 2 1 2

                      -2      -2
>   riemann        = L   y z r
           1 2 1 3

                        -2      -2
>   riemann        = - L   x z r
           1 2 2 3

                        -2  2  -2
>   riemann        = - L   y  r
           1 3 1 3

                      -2      -2
>   riemann        = L   x y r
           1 3 2 3

                        -2  2  -2
>   riemann        = - L   x  r
           2 3 2 3

(riemann calculated)
(TIME = 3260 msec)

(riemann completed)
(TIME = 3300 msec)

                  -2
>   ricci    = - L
         0 0

                -2
>   ricci    = L
         1 1

                -2
>   ricci    = L
         2 2

                -2
>   ricci    = L
         3 3

(ricci calculated)
(TIME = 3330 msec)

(CURVATURE INVARIANT calculated)
(TIME = 3330 msec)

                               -2
>   CURVATURE INVARIANT = - 4 L



                   -2  2  -2          -2
>   weyl        = L   x  r   - (1/3) L
        0 1 0 1

                   -2      -2
>   weyl        = L   x y r
        0 1 0 2

                   -2      -2
>   weyl        = L   x z r
        0 1 0 3

                   -2  2  -2          -2
>   weyl        = L   y  r   - (1/3) L
        0 2 0 2

                   -2      -2
>   weyl        = L   y z r
        0 2 0 3

                     -2  2  -2    -2  2  -2          -2
>   weyl        = - L   x  r   - L   y  r   + (2/3) L
        0 3 0 3

                   -2  2  -2    -2  2  -2          -2
>   weyl        = L   x  r   + L   y  r   - (2/3) L
        1 2 1 2

                   -2      -2
>   weyl        = L   y z r
        1 2 1 3

                     -2      -2
>   weyl        = - L   x z r
        1 2 2 3

                     -2  2  -2          -2
>   weyl        = - L   y  r   + (1/3) L
        1 3 1 3

                   -2      -2
>   weyl        = L   x y r
        1 3 2 3

                     -2  2  -2          -2
>   weyl        = - L   x  r   + (1/3) L
        2 3 2 3

(weyl calculated)
(TIME = 3420 msec)


(I REALLY LIKED THIS! CAN I HAVE MORE ?   PLEASE ?!?)

END OF WORK
(RUN TIME = 3420 msec)
\end{verbatim}

\subsection{Example V: The Laplace equation in the cylindrical coordinates.}

The input data is here:

\bigskip

\begin{verbatim}
 (calculate '(
    (Laplace equation in cylindrical coordinates)
    (coordinates x y z)
    (functions F (r phi z))
    (symbols
 r = ((x ^ 2 + y ^ 2) ^ (1 2))
 phi = (arctan (y / x))
 cosphi = (cos phi)
       )
    (substitutions
 (x ^ 2) = (r ^ 2 - y ^ 2)
 (1 + y ^ 2 / x ^ 2) = (cosphi ^ -2)
 x = (r * cosphi)
 y = (r * (sin phi))
 ((sin phi) ^ 3) = ((sin phi) * (1 - cosphi ^ 2))
 cosphi = (cos phi)
    )
 (operation ((deriv x x F) + (deriv y y F) + (deriv z z F)))
    ))
\end{verbatim}

\bigskip

\centerline{Notes}

This example shows how to use  the  program  "calculate".
     Since coordinate transformations as such are  not  available
in Ortocartan, the transformation to the cylindrical  coordinates is achieved
through a trick: the unknown  function  $F$ is defined to depend not directly
on  the  cartesian  coordinates $x$ and $y$, but on $r$ and $\phi$ which are
defined as  explicit expressions in $x$ and $y$. In the result thus  obtained,
$x$ and $y$ are replaced by the appropriate functions of  $r$  and $\phi$. Note
how  the  program  was  prevented  from  replacing $\cos^2 \phi$  by $(1 -
\sin^2 \phi)$ at a too early stage of the calculation (this last trick is
described in sec. 18).

The output is:

\bigskip

\begin{verbatim}
(laplace equation in cylindrical coordinates)

symbols

          2    2 (1/2)
>   r = (x  + y )

                   -1
>   phi = arctan (x   y)

>   cosphi = cos (phi)

substitutions

everywhere

     2      2    2
>   x  = - y  + r

everywhere

         -2  2         -2
>   1 + x   y  = cosphi


everywhere

>   x = r cosphi


everywhere

>   y = r sin (phi)


everywhere

       3                2
>   sin (phi) = - cosphi  sin (phi) + sin (phi)


everywhere

>   cosphi = cos (phi)


(I UNDERSTAND YOU REQUEST THE FOLLOWING EXPRESSION TO BE SIMPLIFIED)

>       deriv (x,x,f) + deriv (y,y,f) + deriv (z,z,f)


THE RESULT IS

               -2               -1
>   result  = r   f,         + r   f,   + f,     + f,
          1         phi phi          r      z z      r r


(I REALLY LIKED THIS ! CAN I HAVE MORE ? PLEASE ?!?)

END OF WORK (RUN TIME = 550 msec)

\end{verbatim}

\subsection{Example VI: The spherically symmetric metric in the standard
coordinates with the arguments of functions written out explicitly.}

The input data here is:

\bigskip

\begin{verbatim}
(setq !*lower nil)
 (ortocartan '(
     (SPHERICAL WITH ARGUMENTS)
     (coordinates T R THETA PHI)
     (functions MU (T R) NU (T R))
     (ematrix (exp (NU T R))  0  0  0  0  (exp (MU T R))  0
              0  0  0  R  0  0  0  0  (R * (sin THETA))  )
     (dont print ie agamma riemann)
     (stop after ricci)
     (rmargin 61)
           ))
(setq !*lower t)
\end{verbatim}

\bigskip

\centerline{Notes}

This example is in fact a duplicate copy of the example
     II, it is only meant to show that, if the  user  wishes  so,
     then the arguments of functional expressions can be  written
     out explicitly. Most of the output is suppressed.

The output is:

\bigskip

\begin{verbatim}
(SPHERICAL WITH ARGUMENTS)


           0
>   ematrix  . = exp (NU (T,R))
             0

           1
>   ematrix  . = exp (MU (T,R))
             1

           2
>   ematrix  . = R
             2

           3
>   ematrix  . = R sin (THETA)
             3

(ematrix completed)
(TIME = 50 msec)

                           2
>   DETERMINANT EMATRIX = R  exp (MU (T,R) + NU (T,R)) sin (


>       THETA)


(DETERMINANT EMATRIX calculated)
(TIME = 90 msec)

(ie calculated)
(TIME = 180 msec)

(agamma calculated)
(TIME = 290 msec)

(agamma completed)
(TIME = 300 msec)

         0
>   gamma      = exp (- MU (T,R)) (NU,   (T,R))
           1 0                        2


         0
>   gamma      = exp (- NU (T,R)) (MU,   (T,R))
           1 1                        1


\end{verbatim}
{\samepage
\begin{verbatim}
         1          -1
>   gamma      = - R   exp (- MU (T,R))
           2 2
\end{verbatim}
    }
\begin{verbatim}


         1          -1
>   gamma      = - R   exp (- MU (T,R))
           3 3


         2          -1                -1
>   gamma      = - R   cos (THETA) sin  (THETA)
           3 3


(gamma calculated)
(TIME = 340 msec)

(gamma completed)
(TIME = 340 msec)

(riemann calculated)
(TIME = 530 msec)

(riemann completed)
(TIME = 560 msec)

                  -1
>   ricci    = 2 R   exp (- 2 MU (T,R)) (NU,   (T,R)) + exp
         0 0                                2

                                    2
>       (- 2 MU (T,R)) (NU,   (T,R))  + exp (- 2 MU (T,R)) (
                           2


>       NU,     (T,R)) - exp (- 2 MU (T,R)) (MU,   (T,R)) (
           2 2                                  2

                                                       2
>       NU,   (T,R)) - exp (- 2 NU (T,R)) (MU,   (T,R))  -
           2                                  1


>       exp (- 2 NU (T,R)) (MU,     (T,R)) + exp (- 2 NU (T,
                               1 1


>       R)) (MU,   (T,R)) (NU,   (T,R))
                1             1

                  -1
>   ricci    = 2 R   exp (- MU (T,R) - NU (T,R)) (MU,   (T,R
         0 1                                         1


>       ))


                  -1
>   ricci    = 2 R   exp (- 2 MU (T,R)) (MU,   (T,R)) - exp
         1 1                                2

                                    2
>       (- 2 MU (T,R)) (NU,   (T,R))  - exp (- 2 MU (T,R)) (
                           2


>       NU,     (T,R)) + exp (- 2 MU (T,R)) (MU,   (T,R)) (
           2 2                                  2

                                                       2
>       NU,   (T,R)) + exp (- 2 NU (T,R)) (MU,   (T,R))  +
           2                                  1


>       exp (- 2 NU (T,R)) (MU,     (T,R)) - exp (- 2 NU (T,
                               1 1


>       R)) (MU,   (T,R)) (NU,   (T,R))
                1             1

                  -2                       -1
>   ricci    = - R   exp (- 2 MU (T,R)) + R   exp (- 2 MU (T
         2 2

\end{verbatim}
{\samepage
\begin{verbatim}
                              -1
>       ,R)) (MU,   (T,R)) - R   exp (- 2 MU (T,R)) (NU,   (
                 2                                      2
\end{verbatim}
    }
\begin{verbatim}

                 -2
>       T,R)) + R


                  -2                       -1
>   ricci    = - R   exp (- 2 MU (T,R)) + R   exp (- 2 MU (T
         3 3

                              -1
>       ,R)) (MU,   (T,R)) - R   exp (- 2 MU (T,R)) (NU,   (
                 2                                      2

                 -2
>       T,R)) + R


(ricci calculated)
(TIME = 689 msec)


(I REALLY LIKED THIS! CAN I HAVE MORE ?   PLEASE ?!?)

END OF WORK
(RUN TIME = 689 msec)
\end{verbatim}

\subsection{Example VII: Application of the program Ellisevol to check the
Ellis evolution equations for the Lanczos metric.}

The Lanczos metric is:

$$ ds^2 = (dt + Crd\varphi)^2 - \psi d\varphi^2 - {1 \over 4} {\rm e}^{- r}
dr^2/\psi - {\rm e}^{- r}dz^2, $$

\noindent where

$$ \psi = (C^2 r + \Lambda - \Lambda {\rm e}^{- r}).$$

\bigskip

\centerline{References}

The original paper: K. Lanczos, {\it Zeitschrift f\"{u}r Physik} {\bf 21}, 73
(1924).

English translation: {\it Gen. Rel. Grav.} {\bf 29}, 363 (1997).

The input data is here:

\bigskip

\begin{verbatim}
 (setq !*lower nil)
 (ellisevol'(
    (LANCZOS METRIC)
    (coordinates t phi r z)
    (velocity 1 0 0 0)
    (constants C Lambda)
    (symbols psi = (C ^ 2 * r + Lambda - Lambda * (exp (-  r))) )
    (ematrix  1  (C * r)  0  0
             0  ((C ^ 2 * r + Lambda - Lambda * (exp (-  r)))^ (1 2 ))
         0 0 0 0 ((1 2) * (exp ((-1 2) * r)) * (C ^ 2 * r + Lambda
                   - Lambda * (exp (- r))) ^ (-1 2))  0
             0  0  0  (exp (- (1 2) * r)))
 (substitutions (C ^ 2 * r + Lambda - Lambda * (exp (-  r))) = psi )
   (dont print messages)
   (tensors einstein)
          ))
(setq !*lower t)
\end{verbatim}

\bigskip

\centerline{Notes}

This is a stationary cylindrically symmetric solution of Einstein's equations
with a rotating dust source and with a nonvanishing cosmological constant
$\Lambda$. The coordinates used in the metric shown above are comoving and the
velocity vector field of the dust is one of the orthonormal tetrad vectors,
hence the tetrad components of velocity field are \verb+(1 0 0 0)+. Since this
is a solution of Einstein's equations, this vector field is uniquely determined
by the metric, and so, as expected, all the constraint and evolution equations
will be identities. However, the acceleration (= 0), rotation, expansion (= 0),
and shear (= 0) are all calculated, along with the electric and magnetic parts
of the Weyl tensor. This example is well suited to try out the (output for
latex) option -- try it yourself. The substitution in line 5 from the bottom
was requested to be done everywhere -- this is usually not a reasonable option
because it makes the calculation slower. It was done this way here in order to
use it together with the (dont print messages) option that suppresses all the
messages about unsuccessful attempts at substitutions. The Einstein tensor
calculated along the way makes it possible to easily see that it is a dust
solution indeed.

\bigskip

\begin{verbatim}
(LANCZOS METRIC)

symbols
                                       2
>   psi = Lambda - Lambda exp (- r) + C  r

substitutions

everywhere
\end{verbatim}
{\samepage
\begin{verbatim}
                                 2
>   Lambda - Lambda exp (- r) + C  r = psi
\end{verbatim}
   }
\begin{verbatim}



           0
>   ematrix  . = 1
             0

           0
>   ematrix  . = C r
             1

           1        (1/2)
>   ematrix  . = psi
             1

           2                              - (1/2)
>   ematrix  . = (1/2) exp (- (1/2) r) psi
             2

           3
>   ematrix  . = exp (- (1/2) r)
             3

(ematrix completed)
(TIME = 210 msec)


            0
>   velocity  = 1


>   DETERMINANT EMATRIX = (1/2) exp (- r)


(DETERMINANT EMATRIX calculated)
(TIME = 270 msec)


      0
>   ie.   = 1
        0

      0              - (1/2)
>   ie.   = - C r psi
        1

      1        - (1/2)
>   ie.   = psi
        1

      2                        (1/2)
>   ie.   = 2 exp ((1/2) r) psi
        2

      3
>   ie.   = exp ((1/2) r)
        3

(ie calculated)
(TIME = 450 msec)


         0
>   uvelo  = 1



>   lvelo  = 1
         0


>   lvelo  = C r
         1



>   metric    = 1
          0 0


>   metric    = C r
          0 1


                         2  2
>   metric    = - psi + C  r
          1 1

                                     -1
>   metric    = - (1/4) exp (- r) psi
          2 2


>   metric    = - exp (- r)
          3 3


(metric calculated)
(TIME = 520 msec)


             0 0        2  2    -1
>   invmetric    = 1 - C  r  psi



             0 1          -1
>   invmetric    = C r psi



             1 1        -1
>   invmetric    = - psi



             2 2
>   invmetric    = - 4 exp (r) psi



             3 3
>   invmetric    = - exp (r)


(invmetric calculated)
(TIME = 560 msec)

          0
>   agamma      = - C exp ((1/2) r)
            1 2

          1                                         - (1/2)          2
>   agamma      = - (1/2) Lambda exp (- (1/2) r) psi        - (1/2) C
            1 2

                         - (1/2)
>       exp ((1/2) r) psi


          3                                (1/2)
>   agamma      = - (1/2) exp ((1/2) r) psi
            2 3

(agamma calculated)
(TIME = 750 msec)

(agamma completed)
(TIME = 760 msec)

         0
>   gamma      = - C exp ((1/2) r)
           1 2


         0
>   gamma      = C exp ((1/2) r)
           2 1


         1
>   gamma      = - C exp ((1/2) r)
           2 0


         1                                 - (1/2)    2
>   gamma      = Lambda exp (- (1/2) r) psi        + C  exp ((1/2) r)
           2 1

           - (1/2)
>       psi


         2                        (1/2)
>   gamma      = exp ((1/2) r) psi
           3 3


(gamma calculated)
(TIME = 810 msec)

(gamma completed)
(TIME = 810 msec)

               0              2      -1
>   christoffel      = (1/2) C  r psi
                 0 2


               0                                               -1
>   christoffel      = (1/2) C - (1/2) C Lambda r exp (- r) psi   - (1/
                 1 2

            3      -1          3  2    -1
>       2) C  r psi   + (1/2) C  r  psi



               1                    -1
>   christoffel      = - (1/2) C psi
                 0 2


               1                                 -1          2      -1
>   christoffel      = (1/2) Lambda exp (- r) psi   - (1/2) C  r psi
                 1 2

                 2    -1
>       + (1/2) C  psi


               2
>   christoffel      = 2 C exp (r) psi
                 0 1


               2                           2                    2
>   christoffel      = - 2 Lambda psi + 4 C  r exp (r) psi - 2 C  exp (
                 1 1


>       r) psi


               2                                           -1
>   christoffel      = - (1/2) - (1/2) Lambda exp (- r) psi   - (1/2) C
                 2 2

        2    -1
>         psi


               2
>   christoffel      = 2 psi
                 3 3


               3
>   christoffel      = - (1/2)
                 2 3


(CHRISTOFFEL SYMBOLS calculated)
(TIME = 1000 msec)

(CHRISTOFFEL SYMBOLS completed)
(TIME = 1000 msec)


>   vtida      = (1/2) C
         1 ; 2


>   vtida      = - (1/2) C
         2 ; 1


((TIDAL MATRIX OF lvelo) completed)
(TIME = 1050 msec)


ACCELERATION = 0


>   rotdd    = (1/2) C
         1 2

(rotdd calculated)
(TIME = 1060 msec)

(rotdd completed)
(TIME = 1070 msec)

           2
>   rotdu    = - 2 C exp (r) psi
         1

           0            2      -1
>   rotdu    = - (1/2) C  r psi
         2

           1              -1
>   rotdu    = (1/2) C psi
         2

(rotdu completed)
(TIME = 1100 msec)


                        2
>   ROTATION SQUARED = C  exp (r)


(ROTATION SCALAR calculated)
(TIME = 1100 msec)


>   projdd    = - psi
          1 1


                                     -1
>   projdd    = - (1/4) exp (- r) psi
          2 2


>   projdd    = - exp (- r)
          3 3


(projdd calculated)
(TIME = 1110 msec)

(projdd completed)
(TIME = 1110 msec)

            0
>   projdu    = - C r
          1

            1
>   projdu    = 1
          1

            2
>   projdu    = 1
          2

            3
>   projdu    = 1
          3

(projdu calculated)
(TIME = 1140 msec)


>   EXPANSION SCALAR = 0


(EXPANSION SCALAR calculated)
(TIME = 1150 msec)

(sheardd calculated)
(TIME = 1160 msec)

SHEAR = 0

(lacce IS COVARIANTLY CONSTANT)
((TIDAL MATRIX OF lacce) completed)
(TIME = 1160 msec)

(ALL THE ROTATION CONSTRAINTS ARE FULFILLED IDENTICALLY)
(ROTATION CONSTRAINTS calculated)
(TIME = 1160 msec)

(ALL THE SHEAR CONSTRAINTS ARE FULFILLED IDENTICALLY)
(SHEAR CONSTRAINTS calculated)
(TIME = 2500 msec)

(ALL THE ROTATION EVOLUTION EQUATIONS ARE FULFILLED IDENTICALLY)
(ROTATION EVOLUTION EQUATIONS calculated)
(TIME = 2800 msec)

                        2
>   riemann        = - C  exp (r)
           0 1 0 1

                        2
>   riemann        = - C  exp (r)
           0 2 0 2

                                    (1/2)
>   riemann        = - C exp (r) psi
           0 2 1 2

                                  (1/2)
>   riemann        = C exp (r) psi
           0 3 1 3

                                   2
>   riemann        = - Lambda - 2 C  exp (r)
           1 2 1 2

                                 2
>   riemann        = - Lambda - C  exp (r)
           1 3 1 3

                                 2
>   riemann        = - Lambda - C  exp (r)
           2 3 2 3

(riemann calculated)
(TIME = 3050 msec)

(riemann completed)
(TIME = 3090 msec)

                  2
>   ricci    = 2 C  exp (r)
         0 0

\end{verbatim}
{\samepage
\begin{verbatim}
                             2
>   ricci    = 2 Lambda + 2 C  exp (r)
         1 1
\end{verbatim}
      }
\begin{verbatim}

                             2
>   ricci    = 2 Lambda + 2 C  exp (r)
         2 2

                             2
>   ricci    = 2 Lambda + 2 C  exp (r)
         3 3

(ricci calculated)
(TIME = 3120 msec)

(CURVATURE INVARIANT calculated)
(TIME = 3120 msec)

                                          2
>   CURVATURE INVARIANT = - 6 Lambda - 4 C  exp (r)


                                2
>   einstein    = 3 Lambda + 4 C  exp (r)
            0 0


>   einstein    = - Lambda
            1 1


>   einstein    = - Lambda
            2 2


>   einstein    = - Lambda
            3 3

(einstein calculated)
(TIME = 3140 msec)

>   RAYCHAUDHURI EQUATION = 0

(RAYCHAUDHURI EQUATION calculated)
(TIME = 3150 msec)

                           2
>   weyl        = - (1/3) C  exp (r)
        0 1 0 1

                           2
>   weyl        = - (1/3) C  exp (r)
        0 2 0 2

                                 (1/2)
>   weyl        = - C exp (r) psi
        0 2 1 2

                         2
>   weyl        = (2/3) C  exp (r)
        0 3 0 3

                               (1/2)
>   weyl        = C exp (r) psi
        0 3 1 3

                           2
>   weyl        = - (2/3) C  exp (r)
        1 2 1 2

                         2
>   weyl        = (1/3) C  exp (r)
        1 3 1 3

                         2
>   weyl        = (1/3) C  exp (r)
        2 3 2 3

(weyl calculated)
(TIME = 3250 msec)

                         2
>   elweyl    = - (1/3) C  exp (r) psi
          1 1

                          2    -1
>   elweyl    = - (1/12) C  psi
          2 2

                       2
>   elweyl    = (2/3) C
          3 3


(elweyl calculated)
(TIME = 3310 msec)

(ALL THE SHEAR EVOLUTION EQUATIONS ARE FULFILLED IDENTICALLY)
(SHEAR EVOLUTION EQUATIONS calculated)
(TIME = 3820 msec)

>   magweyl    = - (1/2) C
           2 3

(magweyl calculated)
(TIME = 3890 msec)

(ALL THE MAGNETIC CONSTRAINTS ARE FULFILLED IDENTICALLY)
(magcons calculated)
(TIME = 5170 msec)

(I REALLY LIKED THIS! CAN I HAVE MORE ?   PLEASE ?!?)

END OF WORK
(RUN TIME = 5170 msec)
\end{verbatim}

\bigskip

\subsection{Example VIII: Application of the program "curvature".}

In this example, the program will calculate the curvature tensor for a
3-dimensional manifold of constant positive curvature (in fact, a 3-sphere; the
connection coefficients used here as input data were previously calculated by
another program of the Ortocartan set as the Christoffel symbols for the metric
of a 3-sphere). This example is so simple in order that the readers can look up
the answer in textbooks and verify that it is correct. The number of dimensions
of the manifold, $n$, can be arbitrary and it is the first argument of the
function "curvature". The connection coefficients are assumed symmetric, and so
there should be ${1 \over 2}n^2(n + 1)$ of them. The program checks whether the
number of the connection coefficients actually given is equal to this. The
order in which the conection coefficients should be given is described in
Appendix C.2, it is the obvious one.

The input data is here:

\bigskip

\begin{verbatim}
(setq !*lower nil)
(curvature 3 '(
   (the curvature of christoffels of spherical space)
   (coordinates r th ph)
   (connection 0 0 0 (- (sin r) * (cos r)) 0 (- (sin r)*(cos r)*(sin th) ^ 2)
               0 ((cos r) / (sin r)) 0 0 0 (- (cos th) * (sin th))
               0 0 ((cos r) / (sin r)) 0 ((cos th) / (sin th)) 0
                       )
            ))
 (setq !*lower t)
\end{verbatim}

\bigskip

\noindent and the results are:

\bigskip

\begin{verbatim}
(the curvature of christoffels of spherical space)

              0
>   connection      = - cos (r) sin (r)
                1 1

              0                            2
>   connection      = - cos (r) sin (r) sin (th)
                2 2

              1                  -1
>   connection      = cos (r) sin  (r)
                0 1

              1
>   connection      = - cos (th) sin (th)
                2 2

              2                  -1
>   connection      = cos (r) sin  (r)
                0 2

              2                   -1
>   connection      = cos (th) sin  (th)
                1 2

(CONNECTION COEFFICIENTS completed)
(TIME = 60 msec)


              0            2
>   ncurvature        = sin (r)
                1 0 1

              0            2       2
>   ncurvature        = sin (r) sin (th)
                2 0 2

              1
>   ncurvature        = -1
                0 0 1

              1            2       2
>   ncurvature        = sin (r) sin (th)
                2 1 2

              2
>   ncurvature        = -1
                0 0 2

              2              2
>   ncurvature        = - sin (r)
                1 1 2

(ncurvature calculated)
(TIME = 190 msec)

(I REALLY LIKED THIS! CAN I HAVE MORE ?   PLEASE ?!?)

END OF WORK
(RUN TIME = 190 msec)
\end{verbatim}

\bigskip

\subsection{Example IX: Application of the program "landlagr".}

The Landau-Lifshitz lagrangian equals $\sqrt{- g} \cal{R}$, where $g$ is the
determinant of the metric tensor and $- \cal{R}$ is the Ricci scalar with the
derivatives of the Christoffel symbols dropped. The example will be a diagonal
Bianchi type I metric, for which the reduced lagrangian is known to provide the
correct Einstein equations.

The input data is:

\bigskip

\begin{verbatim}
(landlagr '(
    (lagrangian for a diagonal Bianchi I metric)
    (coordinates t x y z)
    (functions f1(t) f2(t) f3(t))
    (ematrix 1 0 0 0 0 f1 0 0 0 0
             f2 0 0 0 0 f3)
    (rmargin 61)
            ))
\end{verbatim}

\bigskip

\noindent and the result is:

\bigskip

\begin{verbatim}
(lagrangian for a diagonal bianchi i metric)

           0
>   ematrix  . = 1
             0

           1
>   ematrix  . = f1
             1

           2
>   ematrix  . = f2
             2

           3
>   ematrix  . = f3
             3

(ematrix completed)
(TIME = 20 msec)


>   DETERMINANT EMATRIX = f1 f2 f3


(DETERMINANT EMATRIX calculated)
(TIME = 30 msec)


      0
>   ie.   = 1
        0

      1       -1
>   ie.   = f1
        1

\end{verbatim}
{\samepage
\begin{verbatim}
      2       -1
>   ie.   = f2
        2
\end{verbatim}
      }
\begin{verbatim}

      3       -1
>   ie.   = f3
        3

(ie calculated)
(TIME = 220 msec)



>   metric    = 1
          0 0


                    2
>   metric    = - f1
          1 1


                    2
>   metric    = - f2
          2 2


                    2
>   metric    = - f3
          3 3


(metric calculated)
(TIME = 250 msec)



             0 0
>   invmetric    = 1


             1 1       -2
>   invmetric    = - f1



             2 2       -2
>   invmetric    = - f2



             3 3       -2
>   invmetric    = - f3


(invmetric calculated)
(TIME = 310 msec)


          1               -1
>   agamma      = (1/2) f1   f1,
            0 1                 t

          2               -1
>   agamma      = (1/2) f2   f2,
            0 2                 t

          3               -1
>   agamma      = (1/2) f3   f3,
            0 3                 t

(agamma calculated)
(TIME = 520 msec)

(agamma completed)
(TIME = 540 msec)

         0         -1
>   gamma      = f1   f1,
           1 1           t


         0         -1
>   gamma      = f2   f2,
           2 2           t


         0         -1
>   gamma      = f3   f3,
           3 3           t


(gamma calculated)
(TIME = 590 msec)

(gamma completed)
(TIME = 590 msec)

               0
>   christoffel      = f1 f1,
                 1 1         t


               0
>   christoffel      = f2 f2,
                 2 2         t


               0
>   christoffel      = f3 f3,
                 3 3         t


               1         -1
>   christoffel      = f1   f1,
                 0 1           t


               2         -1
>   christoffel      = f2   f2,
                 0 2           t


               3         -1
>   christoffel      = f3   f3,
                 0 3           t


(CHRISTOFFEL SYMBOLS calculated)
(TIME = 710 msec)

(CHRISTOFFEL SYMBOLS completed)
(TIME = 720 msec)


>   landlagr = 2 f1 f2,   f3,   + 2 f2 f1,   f3,   + 2 f3 f1
                       t     t            t     t

>       ,   f2,
         t     t


(I REALLY LIKED THIS! CAN I HAVE MORE ?   PLEASE ?!?)

END OF WORK
(RUN TIME = 870 msec)
\end{verbatim}

This lagrangian can then be used as data for the program "eulagr", and the
resulting Euler-Lagrange equations can be compared with the Einstein equations
derived for the same metric in the ordinary way.

\bigskip

\subsection{Example X: Application of the program "eulagr".}

The program will derive the Newtonian equations of motion for a point particle
of mass $m$ in the cartesian coordinates $\{x, y, z\}$ from the lagrangian

$$ L = {1 \over 2} m ({\dot {x}}^2 + {\dot {y}}^2 + {\dot {z}}^2) - V(x, y, z),
$$

\noindent where $V$ is a potential and $x(t), y(t), z(t)$ are the equations of
a trajectory of the particle. The input data are:

\bigskip

\begin{verbatim}
(setq !*lower nil)
(eulagr '(
   (The lagrangian for the Newtonian equations of motion
    in 3 dimensions)
   (constants m)
   (parameter t)
   (functions x(t) y(t) z(t) V(x y z) )
   (variables x y z)
   (lagrangian ((1 2) * m * ((der t x) ^ 2 + (der t y) ^ 2
      + (der t z) ^ 2) - V))
        ))
(setq !*lower t)
\end{verbatim}

\bigskip

\noindent and the results are:

\bigskip

\begin{verbatim}
(The lagrangian for the Newtonian equations of motion in 3 dimensions)

                                   2               2               2
>   lagrangian = - V + (1/2) m x,    + (1/2) m y,    + (1/2) m z,
                                 t               t               t

(THIS IS THE VARIATIONAL DERIVATIVE BY x)

>   eulagr  = m x,     + V,
          0       t t      x


(THIS IS THE VARIATIONAL DERIVATIVE BY y)

>   eulagr  = m y,     + V,
          1       t t      y


(THIS IS THE VARIATIONAL DERIVATIVE BY z)

>   eulagr  = m z,     + V,
          2       t t      z

(I REALLY LIKED THIS! CAN I HAVE MORE ?   PLEASE ?!?)

END OF WORK
(RUN TIME = 100 msec)
\end{verbatim}

\bigskip

\subsection{Example XI: Application of the program "squint".}

In order to make the result easy to verify, we shall use the program "squint"
to find a first integral of the equations found in the previous example. We
shall at first pretend that we do not know what the integral should be and will
assume that it is a general polynomial of second degree in the first
derivatives by $t$ of the functions $x(t)$, $y(t)$ and $z(t)$. Note how the
(markers ...) were used to simplify the substitutions: the single equation (der
t t M) = (der M V) represents the 3 equations $d^2x^i/dt^2 = \partial V
/\partial x^i$ for $i = 1, 2, 3$ simultaneously.

The input data are:

\bigskip

\begin{verbatim}
(setq !*lower nil)
(squint'(
   (a first integral of the Newtonian equations of motion)
   (constants m)
   (parameter t)
   (functions x(t) y(t) z(t) V(x y z) Q11(x y z) Q12(x y z) Q13(x y z) Q22(x y z)
   Q23(x y z) Q33(x y z) L1(x y z) L2(x y z) L3(x y z) E(x y z) )
   (variables x y z)
(integral (Q11 * (der t x) ^ 2 + 2 * Q12 * (der t x) * (der t y)
    + 2 * Q13 * (der t x) * (der t z) + Q22 * (der t y) ^ 2
    + 2 * Q23 * (der t y) * (der t z) + Q33 * (der t z) ^ 2
    + L1 * (der t x) + L2 * (der t y) + L3 * (der t z) + E) )
(markers M)
   (substitutions
    maineq
(der t t M) = (- (der M V) / m)
     )
(dont print maineq)
          ))
(setq !*lower t)
\end{verbatim}

\bigskip

\noindent and the results are:

\bigskip

\begin{verbatim}
(a first integral of the Newtonian equations of motion)


                           2
>   integral = E + Q11 x,    + 2 Q12 x,   y,   + 2 Q13 x,   z,   + Q22
                         t             t    t            t    t

            2                             2
>       y,    + 2 Q23 y,   z,   + Q33 z,    + L1 x,   + L2 y,   + L3 z,
          t             t    t          t          t         t


>
        t


substitutions

maineq

                -1
>   M,     = - m   V,
      t t            M


                                     3
>   THIS IS THE COEFFICIENT OF   x,
                                   t

>   equation  = Q11,
            1       x

                                     2
>   THIS IS THE COEFFICIENT OF   x,    y,
                                   t     t

>   equation  = 2 Q12,   + Q11,
            2         x        y

                                     2
>   THIS IS THE COEFFICIENT OF   x,    z,
                                   t     t

>   equation  = 2 Q13,   + Q11,
            3         x        z

                                          2
>   THIS IS THE COEFFICIENT OF   x,   y,
                                   t    t

>   equation  = Q22,   + 2 Q12,
            4       x          y

>   THIS IS THE COEFFICIENT OF   x,   y,   z,
                                   t    t    t

>   equation  = 2 Q23,   + 2 Q13,   + 2 Q12,
            5         x          y          z

                                          2
>   THIS IS THE COEFFICIENT OF   x,   z,
                                   t    t

>   equation  = Q33,   + 2 Q13,
            6       x          z

                                     3
>   THIS IS THE COEFFICIENT OF   y,
                                   t

>   equation  = Q22,
            7       y

                                     2
>   THIS IS THE COEFFICIENT OF   y,    z,
                                   t     t

>   equation  = 2 Q23,   + Q22,
            8         y        z

                                          2
>   THIS IS THE COEFFICIENT OF   y,   z,
                                   t    t

>   equation  = Q33,   + 2 Q23,
            9       y          z

                                     3
>   THIS IS THE COEFFICIENT OF   z,
                                   t

>   equation   = Q33,
            10       z

                                     2
>   THIS IS THE COEFFICIENT OF   x,
                                   t

>   equation   = L1,
            11      x

>   THIS IS THE COEFFICIENT OF   x,   y,
                                   t    t

>   equation   = L2,   + L1,
            12      x       y

>   THIS IS THE COEFFICIENT OF   x,   z,
                                   t    t

>   equation   = L3,   + L1,
            13      x       z

                                     2
>   THIS IS THE COEFFICIENT OF   y,
                                   t

>   equation   = L2,
            14      y

\end{verbatim}
{\samepage
\begin{verbatim}
>   THIS IS THE COEFFICIENT OF   y,   z,
                                   t    t
\end{verbatim}
}
\begin{verbatim}

>   equation   = L3,   + L2,
            15      y       z

                                     2
>   THIS IS THE COEFFICIENT OF   z,
                                   t

>   equation   = L3,
            16      z

>   THIS IS THE COEFFICIENT OF   x,
                                   t

                      -1               -1               -1
>   equation   = - 2 m   Q11 V,   - 2 m   Q12 V,   - 2 m   Q13 V,   + E
            17                 x                y                z

>       ,
         x

>   THIS IS THE COEFFICIENT OF   y,
                                   t

                      -1               -1               -1
>   equation   = - 2 m   Q12 V,   - 2 m   Q22 V,   - 2 m   Q23 V,   + E
            18                 x                y                z

>       ,
         y

>   THIS IS THE COEFFICIENT OF   z,
                                   t

                      -1               -1               -1
>   equation   = - 2 m   Q13 V,   - 2 m   Q23 V,   - 2 m   Q33 V,   + E
            19                 x                y                z

>       ,
         z

>   THESE ARE THE TERMS THAT ARE   FREE OF THE DERIVATIVES

                    -1            -1            -1
>   equation   = - m   L1 V,   - m   L2 V,   - m   L3 V,
            20              x             y             z


(I REALLY LIKED THIS! CAN I HAVE MORE ?   PLEASE ?!?)

END OF WORK
(RUN TIME = 820 msec)
\end{verbatim}

\bigskip

Now we shall substitute the well-known solution of these equations into the
data and see what happens.

\bigskip

\begin{verbatim}
(setq !*lower nil)
(squint'(
   (a first integral of the Newtonian equations of motion - the final result)
   (constants m)
   (parameter t)
   (functions x(t) y(t) z(t) V(x y z))
   (variables x y z)
(integral ((1 2) * m * ((der t x) ^ 2 + (der t y) ^ 2 + (der t z) ^ 2) + V) )
(markers M)
   (substitutions
    maineq
(der t t M) = (- (der M V) / m)
     )
     (dont print maineq)
          ))
(setq !*lower t)
\end{verbatim}

\bigskip

\noindent The result is:

\bigskip

\begin{verbatim}
(a first integral of the Newtonian equations of motion - the final result)

                               2               2               2
>   integral = V + (1/2) m x,    + (1/2) m y,    + (1/2) m z,
                             t               t               t

substitutions

maineq

                -1
>   M,     = - m   V,
      t t            M


(THE FIRST INTEGRAL IS ALREADY MAXIMALLY SIMPLIFIED
AND IS EXPLICITLY CONSTANT)
>   maineq = 0



(I REALLY LIKED THIS! CAN I HAVE MORE ?   PLEASE ?!?)

END OF WORK
(RUN TIME = 130 msec)
\end{verbatim}

\bigskip

\centerline{***************************************************}

\bigskip

You can produce many more examples by yourself  if  you use the sets of input
data recorded in the *.tes-files on the Ortocartan distribution diskette. Do
not rewrite them, but use the editor to cut out single calls to Ortocartan or
to the other functions. Good luck and enjoy it!

\bigskip

\bigskip

\centerline{***** THIS IS THE END OF THE MANUAL *****}

\end{document}
