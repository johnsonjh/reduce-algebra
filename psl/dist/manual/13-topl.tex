\chapter{Top Level Loop}


\section{Introduction}

  Normally   one   interacts   with   PSL   through  a  toplevel
read-eval-print loop, it is the highest  level  of  control  and
consists  of an endless loop that reads an expression, evaluates
it, and prints the result.  One has an effect on  the  state  of
PSL  by  invoking actions that have side effects.  This toplevel
loop will catch all throws (this includes errors).

\section{The General Purpose Top Loop Function}

  The user interacts with PSL through toploop.  The  purpose  of
this loop is to read expressions typed at the terminal, evaluate
them,   and  type  the  result  back  on  the  terminal.    Such
expressions may be entered to see what value they return  or  to
produce a side-effect on the global environment.

An expression which has a side effect on the global environment
\begin{verbatim}
  (setq base 10)
\end{verbatim}
Another example of a side effect, the definition of a function
\begin{verbatim}
  (de square (n) (* n n))
\end{verbatim}
Evaluation, just to see what value the expression returns
\begin{verbatim}
  (square 2)
\end{verbatim}
A  close  approximation to the behavior of this loop is given by
the following functional description:

\begin{verbatim}
(de toploop (toploopread*
             toploopeval*
             toploopprint*)
  (while t
    (funcall toploopprint*
             (funcall toploppeval*
                      (funcall toploopread*)))))
\end{verbatim}
The actual toploop of PSL is  somewhat  more  complex  than  the
simple one defined here because it provides a few extra features
(these  are  described  below).  The important point is that the
primitives required for writing such a loop are all available to
the user.

  Syntactically correct expressions have a well defined  mapping
into  PSL  data structures, read accomplishes this mapping.  For
example, linear list notation is  converted  into  the  internal
tree  structure  representation,  strings  are  stored in a more
efficient non-list form, and numbers are internalized to a  form
compatible  with  the  arithmetic unit of the machine.  The most
primitive piece of read is the scanner.  This routine recognizes
characters special to PSL.  For example, space, (, and ).    The
scanner   is   also   responsible   for  building  the  internal
representation of identifiers.   The  scanner  must  make  every
reference  to  a  particular  id  point  to  the  same  internal
structure.  This is accomplished at the time an id  is  created.
The  character sequence is compared against sequences which have
already been converted into ids.   This  comparison  employs  an
efficient  search  technique  so  that  not every id is compared
(currently a hash algorithm is used).

  The eval function is responsible for the  evaluation  of  data
structures  interpreted  as  programs.   A through discussion of
this can be found in Chapter 11.

  Print displays data structures in a way which could  later  be
typed  back  in  to  read.    Some of the more interesting print
routines do prettyprinting.  That is,  they  format  the  output
using  conventions  based  on  the  structural  nesting  of  the
expressions.

  Giving the user the power to invoke  all  of  these  operation
from  his  code is a very powerful feature of LISP which sets it
apart from most other languages.

\de{toploop}{(toploop {\small TOPLOOPREAD*:function TOPLOOPPRINT*:function\\
TOPLOOPEVAL*:function TOPLOOPNAME*:string WELCOMEBANNER:string)}: nil}{expr}
{
    This function is called to establish a new  Top  Loop.    It
    prints  the WELCOMEBANNER and then invokes a Read-Eval-Print
    loop,  using  the   functions   defined   by   TOPLOOPREAD*,
    TOPLOOPEVAL*,   and   TOPLOOPPRINT*.    Since  each  of  the
    parameters to toploop is  fluid  they  may  be  examined  or
    changed.    Timing  and  history mechanisms are provided.  A
    prompt is constructed by  prefixing  TOPLOOPNAME*  with  the
    history  count.   As a convention, the name is followed by a
    number of right angle brackets ($>$), to indicate the depth of
    toploop invocations.
}

\variable{toploopread*}{[Initially: nil]}{global}
{
    When toploop is called this id is bound to the function used
    for reading.
}

\variable{toploopeval*}{[Initially: nil]}{global}
{
    When toploop is called this id  is  bound  to  the  function
    which evaluates input.
}

\variable{toploopname*}{[Initially: string]}{global}
{
    Bound to a string (currently "lisp"), which will be part  of
    the prompt for input.
}

\variable{toplooplevel*}{[Initially: 0]}{global}
{
    Depth of top loop invocations.
}

\variable{initforms*}{[Initially: nil]}{global}
{
    A  list  of  forms  to  be  evaluated  at startup, (prior to
    calling main).  The forms are evaluated in a left  to  right
    order,  once the last form is evaluated initforms* is set to
    nil.
}

\variable{*output}{[Initially: T]}{switch}
{
    If non-nil, the result of  evaluating  top  level  forms  is
    printed.
}

\variable{*time}{[Initially: nil]}{switch}
{
    If non-nil, a step evaluation time is printed after each top
    level form is processed within toploop.
}

\de{hist}{(hist [N:integer]): nil}{nexpr}
{    Hist  is  called  with 0, 1 or 2 integers, which control how
    much history is to be printed out:

\begin{center}
\begin{tabular}{ll}
    (hist) & Display full history.\\

    (hist n m) & Display history from n to m.\\

    (hist n) & Display history from n to present.\\

    (hist -n) & Display last n entries.\\
\end{tabular}
\end{center}
}

  The following functions permit the user to access and resubmit
previous expressions, and to re-examine previous results.


\de{inp}{(inp N:integer): any}{expr}
{    Return N'th input at this level.
}

\de{redo}{(redo N:integer): any}{expr}
{    Reevaluate N'th input.
}

\de{ans}{(ans N:integer): any}{expr}
{    Return N'th result.
}

\variable{historycount*}{[Initially: 0]}{global}
{
    The number of entries which have been read so far.
}

\variable{historylist*}{[Initially: nil]}{global}
{
    A list of pairs, the first element of each  pair  represents
    an  input  form,  the second is the result of evaluating the
    frst.  Note that the top level  evaluation  of  historylist*
    will result in a circular list.
}

\section{Changing the Default Top Level Function}

  As  PSL starts up, it first sets the stack pointer and various
other variables, and then calls the function main inside a catch
form.

\begin{verbatim}
(catch 'reset (main))
\end{verbatim}
By default, main calls standardlisp.

\begin{verbatim}
(de main ()
  (funcall 'standardlisp))

(de standardlisp ()
  (let ((currentreadmacroindicator* 'lispreadmacro)
        (currentscantable* lispscantable*))
    (toploop 'read 'print 'eval "lisp" "Portable Standard LISP")))
\end{verbatim}
  In order to have a saved PSL come up in a different top  loop,
the function Main should be appropriately redefined by the user.


\de{main}{(main): Undefined}{expr}
{    An  initialization  function which is called after the stack
    is set.  Its redefinition allows  the  user  to  change  the
    default top loop.
}
