\chapter*{Addendum}

Since its earlier versions starting in the 80's, Portable Standard LISP 
has undergone several changes. 

PSL was developed for mainframe systems
e.g. of /370 and Vax type running proprietary operating systems of the  
manufacturers. Today most PSL versions run on workstations with RISC processors
running a Unix type operating system. 

Today PSL is mostly used a deliverly vehicle for LISP based systems, e.g.
the computer algebra system REDUCE and not as a LISP dialect of its own 
right.

The support and maintainance for PSL is mostly done at ZIB in Berlin.

Recently, the internal representation of identifiers was changed in order
to support a 256 character set and case sensitivity. This causes a
binary incomptibility of .b files with older versions such as PSL 3.4.

This document describes the changes and newer developments for PSL
done at ZIB in order to improve performance and to work with new
application types such as parallel processing.

\section{Installation}

This section gives instructions for the Installation of PSL 4.2 on
your UNIX system. It does not apply for other operating systems such
as VMS, MVS, DOS etc, and it does not apply for distribution on
diskettes (e.g. all PC systems, IBM RS/6000, NeXT). Please read
special instructions (below) for the system mentioned above.

The Installation procedure described below uses {\tt csh} syntax.
If you don't have csh installed on your system, please change the
commands accordingly.

PSL is stored as a single file tree with two directories on the
top level, called bin and dist, where the later contains all sources
and some more binary files.

The installation is relatively simple.

\subsection{Reading the tape}

The tape contains a compressed file tree. Please cd to the directory
where you want to store the files (called PSL root directory later on)
and enter 

\begin{verbatim}
dd if=/dev/tape ibs=51200 | uncompress | tar -xf -
\end{verbatim}

Please replace /dev/tape in the above command with the correct
name of your tape drive.

\subsection{Reading the tape for IBM RS/6000}

The PSL file tree for the IBM RS/6000 is configured in the same way as for
other Unix type workstations. The installation needs an additional
step.

The installation procedure described below assumes that you use the
C-Shell. If you use different shells please change commands accordingly.

To read the tape, make a directory with a name like .../psl, cd to that
directory, and do, after putting the tape into the drive, 

\begin{verbatim}
     tar -xf /dev/rfd0
\end{verbatim}

You first need to install a syscall into the AIX system. 
{\bf In order to do this, YOU NEED SUPER USER PRIVILEGE.}
To install the syscall, do
 
\begin{verbatim}
   su
   cd ./dist/kernel/ibmrs/syscall
   ./install
   ./installsysc
   exit
\end{verbatim}

The script install compiles and installs the required system call.
The script installsysc in the same directory installs the system call only.
With any reboot of the system it is mandatory to run installsysc in the
directory ./dist/kernel/ibmrs/syscall.
It is suggested that this is included into the AIX boot files. The sources
of the syscall can be found in ./dist/kernel/ibmrs/syscall  too.

\subsection{Reading Diskettes for LINUX 386}

The PSL file tree for LINUX is compressed on to 3 diskettes for
distribution. The 3 diskettes require about 4 megabytes of disk space to
dump.  However, you must then rebuild the binary files for your machine
before you can do anything. Instructions on doing this are given below.
If the files for your machine are built, the file tree then requires
about 30 megabytes during the build (but less when complete).

To read the diskettes, make a directory with a name like .../psl, cd to that
directory, and do, after putting DISK \# 1 into the floppy disk reader,

\begin{verbatim}
     tar -xf /dev/fd0H1440
     install_psl
\end{verbatim}

The script install\_psl prompts for the other disks and will uncompress the
files.

If there is an incompatibility between the delivered exeutable and
your Linux version, or when you upgrade your Linux version
and {\small PSL} becomes inoperable, you must recompile the
executable $bpsl$:
\begin{verbatim}
    cd $pxk/linux
    cclnk
\end{verbatim}
After this step you should rebuild the image files using the
scripts in \verb+$psl/distrib+.

\subsection{Customizing Makefiles and scripts}

In the next step the system must get knowledge about the PSL root directory
and the machine architecture. The correct name for the architecture in the
PSL file tree can be found in the table below.  
\begin{verbatim}

        Machine        Operating system      PSL MACHINE name 
        
        CDC 4xxx        EP/IX                   mips_cdc     ++
        Convex Cxxxx                            convex       ++
        Convex SPP                              convex_spp
        Cray 1, X-MP    UNICOS                  crayxmp      
        Cray Y-MP       UNICOS                  crayymp
        Cray C90        UNICOS                  crayymp
        Cray T3D/E      UNICOS                  cray_t3d
        DECStation      ULTRIX                  Mips_dec
        DEC Alpha       OSF/1 (DEC Unix)        Alpha
        DEC Alpha       Linux                   Alpha_Linux
        DEC VAX         ULTRIX                  vax
        DG AViiON       DG-UX                   88k-aviion    ++
        HP9000/300 400  HP-UX                   bobcat        ++
        HP9000/700      HP-UX                   Snake
        HP9000/800      HP-UX                   spectrum      ++
        IBM RS/6000     Aix                     Ibmrs	
        IBM SP2         (the RS/6000 version happens to work)
        Intel 386       Linux                   Linux
        Intel 386 ELF   Linux (ELF)             Linux_elf
        Intel x86       Solaris 2.x             Solarisx86
        SGI Iris/Indigo Irix                    Mips_iris
        SGI Iris/Indigo Irix64                  Mips_iris_64
        Sun 3           SunOS                   sun           ++
        Sparc           SunOS (Solaris1.x)      Sun4
        Sparc           Solaris 2.x             Solaris
        UltraSparc      Solaris 2.x             Ultrasparc

++ This version may be not supported in future version.
        
\end{verbatim}

The script dist/distrib/newroot.csh will modify various Makefiles and
scripts, especially dist/psl-names. It must be started with the
PSL root directory and MACHINE name as parameter, e.g. for a Sparc under
SUNOS:

\begin{verbatim}
dist/distrib/newroot.csh `pwd` Sun4
\end{verbatim}

After that the file dist/psl-names contains the settings for the  
PSL specific environment variables such as \$psys. It must be
read in by the PSL user to insure correct operation,e.g. for a
Sparc machine under SunOS:

\begin{verbatim}
setenv MACHINE Sun4
source <PSL root directory>/dist/psl-names
\end{verbatim}

This installation is designed for usage with multiple machine types,
it saves disk space by sharing the PSL code (.sl files).
The variable  {\bf MACHINE} is used to specify the machine type.
If you install a single version of PSL it may be useful to
replace {\bf MACHINE} by the correct value in the file
$<PSL root directory>$/dist/psl-names. In this case the
user does not need to specify the MACHINE environment variable.

To test the installation, you can try the following commands which
are supposed to run without error message:

\begin{verbatim}   
$psys/psl
(load inum)
(quit)
\end{verbatim}   

If you change files, e.g. when you receive a bug fix, please 
put the file into the correct place , cd to the directory \$psl and
simply say {\tt make}. This will compile the file (and maybe some
files which depend on this), and produce new binaries.

\subsection{Printing Documentation}

The documentation for PSL (User's Manual and some more documents)
can be found in the directories 
dist/doc and dist/lpt and dist/manual (in {\LaTeX} or plain ASCII print format)

\section{New unexec procedure, Image model}

PSL's former unexec model (Savesystem) was to produce executables
in the sense of the operating system (mostly a.out files). 
This has led into maintainance 
problems with various new file formats (e.g. ELF, COFF) and dynamic
loading. Influenced by the model developed for the PSL on Personal
Computers, for all new PSL versions for new architectures the
so called {\bf image model} is used. This simply means that
the command {\tt savesystem} does not write an executable file but
it puts all the LISP data into a file such that the information
can be reloaded afterwards. The reload is done by the program
{\tt \$pxk/bpsl} when the -f parameter is used.
The dumpfiles names get the suffix {\tt .img}, e.g. the command

\begin{verbatim}
(savesystem "Banner" "dir/file" initform)
\end{verbatim}

produces a file {\tt dir/file.img} which can be restarted by
issueing the command

\begin{verbatim}
$pxk/bpsl -f dir/file.img
\end{verbatim}

The LISP data written to the file are: 
{\tt SYMVAL, SYMPRP, SYMFNC, SYMNAM and SYMGET, hashtable}  and 
the allocated parts of {\tt bps and heap.} 
This reduces the amount of disk storage used drastically.

To insure proper reloading, the program bpsl can be used for reloading
of images only if these were written from its own core image.
A check for this
is provided using a datetag scheme which must be the same in bpsl
and image file. The error message {\bf Cant start image with this bpsl}
means that the check was not successful and you have to produce a new
image file. This normally happens when the file bpsl is rebuild.

In very rare cases after dramatic changes in the environment (generally
speaking), you may get the message {\bf Cant relocate the image}.
In this case, please try to rebuild the images. If this fails
(it should not), please try

\begin{verbatim}
$pxk/bpsl -f dir/file.img -g 1000000
\end{verbatim}

If this works properly, you may try to reduce the g value.
In any case please send a note the the author of this document.

\def \thing #1#2#3#4 {{\noindent {\tt(#1): #2} \hfill {\it #3}}
                      \begin{itemize}\item[] #4
                      \end{itemize}}

\section{Dynamic configuration of Heap Size and Binding Stack}           

The PSL workspaces {\tt Heap and Bindingstack}
can be enlarged at run time. This is often necessary for
programs that allocate a lot dynamic storage (heap)
or use variable
binding intensively. 

If heap space gets low, the garbage collector tries to
increase heap size automatically. This is a reaction to an
emergency situation though and it is tried after many garbage
collections.  Moreover, depending on the history of the PSL session,
the system may find itself unable to allocate the requested memory causing
an LISP error condition. A manual increase of heap size 
is much better for overall performance.

The Binding stack is not enlarged automatically, because
in case of on infinite loop the PSL memory would blow up
and it would take a very long time until the user would
see a problem (if at all).

Two functions have been added to PSL
for handling heap and binding stack size:

\thing{set-heap-size N:Integer or NIL}{Integer or NIL}{expr}{If called
with NIL it returns the number of LISP items in heap.
If called with a number (which is counted in items and should be
bigger that the current size in items) PSL tries to allocate the
memory. If successful, the argument is returned, if not, NIL is 
returned.}

\thing{set-bndstk-size N:Integer or NIL}{Integer or NIL}{expr}{If called with NIL
it returns the number of LISP items in bndstk.
If called with a number (which is counted in items and should be
bigger that the current size in items) PSL tries to allocate the
memory. If successful, the argument is returned, if not, NIL is
returned.}

\section{Size of Address Space}

The tradition PSL model is {\tt high tagged}, i.e. a tag which describes
the data type is stored in the high bits of a LISP item. In most cases
the number of bits used for tagging is 5, which limits the address space
to bitsperword - 5 bits. This has not been a problem for a long time
for the widely used 32 bit microprocessor architectures, 
since the swap space available was much less that this number. E.g
on a Sparc system you can use about 100 MB LISP workspace, about 7000000 
LISP items.

For the time being the 64 bit microprocessors do not suffer from this
limitation. A {\tt low tagged} version of PSL is under development
already since 1990.

\section{Arbitrary Precision Integer Support}

The efficient support of Arbitrary Precision Integers ("Bignums")
is crucial for a computer algebra system. This version of PSL tries to
exploit the given machine hardware as far as possible, e.g. using double
size multiplication and division in hardware where available. Because of
the different processor architectures the optimal internal representation 
of bignums may vary between workstations even with same number 
of bits per word. 

The PSL manual mentions the modules ``big'' and ``nbig'' which are
written in the true spirit of portable software but these are unfortunately
not very efficient. We recommend to use the load module
{\bf zbig} instead. In all versions this load module contains 
the bignum version which we have found to be optimal.

\section{Monitoring of Performance}

\subsection{SPY (Unix only)}

The Unix profil system call provides a facility to get information 
about the cpu consumption in a LISP program. A simple interface was
build to be able to turn profiling on and off and to map the results
on to the LISP address space. For details, please refer to the source
code in \verb+$pxu/spy.sl+.
\\

\thing{spywhole N:Integer}{void}{expr}{Turns on profiling for the
whole bps space. N must be a power of 2, defining the grain for the
profil call. Typical values are 4 and 8.
Returns {\tt nil}.}

\thing{spyprint}{void}{expr}{Turns off the profiler.
Prints profiler information
after mapping it to the LISP address space, i.e  the names of
LISP functions are printed together with their relative cpu consumption.
The variable \&spyminimum\& defines the cutoff
for spyprint, i.e. functions with less than \&spyminimum\& ticks will
not be printed.
Returns {\tt nil}.}

\thing{spyon fwa lwa tslice bucketsize power}{void}{expr}{This is a complete
interface for the profil call. Allocates
Warray space for the Unix profiler and turns on profiling. fwa and lwa are the
addresses of the first and last word of code memory which should be
inspected, tslice is the timeslice and not suported on most Unix systems.
It is a parameter for compatibility reasons.
Bucketsize must be a power of 2 and power is the base 2 logarithm of
bucketsize.  Typical values are 4 or 8 for bucketsize and 2 or 3 for
power resp.  Returns {\tt nil}.}

\thing{spyoff}{void}{expr}{Turns off profiling using the Unix
profil system call.  Returns {\tt nil}.}

Example:
\begin{verbatim}
(load spy zbig)

NIL

(spywhole 4)

NIL

(null (setq aa(expt 3000 10000)))

NIL

(spyprint)  % This was done on a SUN4 PSL

	214       39.5%     BPLUSA2
	204       37.7%     INITCODE
	67        12.3%     WTIMESDOUBLE
	15        2.7%      COPYFROMONESTATICHEAP
	12        2.2%      XXCOPYFROMSTACK
	12        2.2%      SET_HEAP_SIZE
NIL
\end{verbatim}

It is easy to see that the command \verb+(null (setq aa(expt 3000 10000)))+
leads to a lot of bignum computations. The spyprint output proves that.
Please note that \verb+INITCODE+ sums up all the functions that belong
to PSL's C runtime support. In this case the high percentage is due
to the fact that the (old) SPARC implementation performs integer
multiply and divide in software i.e. in non LISP code.

\subsection{Qualified timing}           

Qualified counting is used to measure the cpu time spend between
call and return from a function, not necessarily spend within the 
body of the function. It gives an overview about the costs that
a call of particular function causes. The sum shown as result may
be bigger that 100 percent, if a function and its callee are both
measured or if gc time is accumulated.
\\

\thing{qualtime S:list}{void}{macro}{Starts qualified timing for the
list of functions contained in list.  Returns {\tt nil}.}

\thing{print!-qualtime}{void}{expr}{Prints the accumulated timings
and number of calls for all functions. Resets counters.  Returns {\tt nil}.}

Example:

\begin{verbatim}
(load qualified-timing zbig)
(off usermode)

(qualtime gtpos times2)

 (null (setq aa(expt 3000 10000))) % this will be measured

(print-qualtime)
 *********** Qualified Timing **************
 **** Overall Cpu time :  5083  *****

 *** TIMES2                    * calls : 18        * time : 5797  * % 114
 *** GTPOS                     * calls : 5560      * time : 935   * % 18

\end{verbatim}

\subsection{Qualified counting}

For our own applications it had turned out to be useful to know the
callers of a function especially when the function is called very,
very often, e.g. basic routines like generic arithmetic. To 
provide this a module qualified counting was generated using a similar
syntax like qualified timing.
\\

\thing{qualcount S:list}{void}{macro}{Starts qualified counting for the
list of functions contained in list.  Returns {\tt nil}.}

\thing{print!-qualcount}{void}{expr}{Prints the accumulated call
counting and the callers for all functions. Resets counters.  
Printing will be suppressed if the number of calls is less than
\*bordervalue\* (default = 20)
Returns {\tt nil}.}

\thing{reset!-qualcount}{void}{expr}{Resets the accumulated call
counting and the callers for all functions. 
Returns {\tt nil}.}

Example:

\begin{verbatim}
 (load qualified-counting)
 (off usermode)

 (qualcount faslin intern)
*** Function `G0003' has been redefined
*** Function `FASLIN' has been redefined
*** Function `G0025' has been redefined
*** Function `INTERN' has been redefined

 (load nbig30) % this will be counted

(print-qualcount)
************* calls for function FASLIN *************

************* calls for function INTERN *************

number of calls : 306 from READ-ID-TABLE
\end{verbatim}


\section{Compiler Modifications}

In order to optimize the code generation for the RISC processors which
(in most cases) have a concept of delayed branches a special optimization
phase for the PSL compiler was build to fill the delay slots. This 
new compiler phase is called {\bf lapopt}. It collects several optimizations
which work on the generated stream of instructions 
(the lap code). This  optimization phase is enabled by default,
it is controled by the switch {\bf lapopt}, and it can be turned off and on by
\begin{verbatim}
(on lapopt)    and  (off lapopt)
\end{verbatim}

A sample of optimized code can be found in the following section
on disassemble.

\section{Disassembler}

For most implementations of PSL, a disassembler for compiled functions
is available which gives insight into the 'real' code. It is a tool
for code optimization only and not meant to give insight for the
'normal' LISP user.  \\
Example:
\begin{verbatim}
(load disassemble)
NIL
(disassemble 'get)
( ** Function  GET  at 16#36110)

00036110    9DE3BFA0    save       r14,-x'60,r14
00036114    3B3E0000    sethi      -x'20000,r29
00036118    21360002    sethi      -x'9FFFE,r16             Alloc 2
0003611C    A2100002    or         r0,r2,r17                (frame 1)
00036120    9930601B    srl        r1,x'1B,r12
00036124    80A3201E    subcc      r12,x'1E,r0
00036128    32800007    bne,a      L0003
0003612C    82100006    or         r0,r6,r1
00036130    9930A01B    srl        r2,x'1B,r12
00036134    80A3201E    subcc      r12,x'1E,r0
00036138    22800005    be,a       L0004
0003613C    88100006    or         r0,r6,r4
00036140    82100006    or         r0,r6,r1
                    L0003:
00036144    81C7E008    jmpl       r31+8,r0
00036148    81E82000    restore    r0,0,r0

              ... etc ...
\end{verbatim}

\section{More unsupported software}

A few parts of the PSL system or applications are no longer useful
or cant be supported.

\subsection{Oload not supported}

We discontinue the support of OLOAD, which allows on some versions 
to link in external routines into PSL at run time using the system loader. 
As alternative we recommend the usage of shared memory (see below).
An example how to use an shared memory based oload substitute is under 
development.

\subsection{Portable Common Lisp Subset (PCLS) not supported}

If you want Common LISP , please use Common LISP.

\section{Shared Memory Interface (Unix only)}

The PSL shared memory interface provides all function for
operating with shared memory regions and semaphores
(See e.g. Unix man pages for shmop,shmget,shmctl,semop,semctl,semget etc.)
The definitions of these man pages are used in the paragraph.
Using the memory address map mechanism described below,it
is easy to write one's own shared memory application.

In the rest of this paragraph we describe a simple model
implementation of a 'pipe' using shared memory and a semaphore.
This code is contained in the file \$pu/shmem.sl.
\\

\thing{shm!-open S:pair M:Mode}{any}{expr}{If S = 0 a new shared
memory area is allocated. Otherwise S is expected to be a dotted pair
of shmid and semid of an existing shared memory. Legal modes are
{\bf input\_create, output\_create, input and output}.
A list consisting of channelnumber shmid and semid is returned.}

\thing{independentdetachshm C:channel}{any}{expr}{Detaches the
shared memory region used by C, closes the channel.}

\thing{readfromshm C:channel}{any}{expr}{Waits until
the shared memory region is readable, reads an expression and 
resets the mode to writable. Returns the expression.}

\thing{writetoshm C:channel E:expression}{list}{expr}{Waits until
the shared memory region is writeable, prints the expression
using prin2. Returns the value of prin2.}


The following C program works together with the PSL part
such that it prints the messages read from shared memory
when they are ready for printing. It must be started with
two paramemters, namely the shmid and the semid to synchronize
with the PSL part.

\begin{verbatim}

#include <stdio.h>
#include <sys/types.h>
#include <sys/ipc.h>
#include <sys/sem.h>

struct sembuf ;

struct sembuf sembu;
struct sembuf *sembuptr;

main (argc,argv)
     int argc;
     char *argv[];

{ int sema , shmemid;
  char * shmaddress;

  sembuptr = &sembu;

  sscanf(argv[1],"%d",&shmemid);
  sscanf(argv[2],"%d",&sema);

  /* open shared memory */

 printf("the data is : %d %d\n",shmemid,sema);
  shmaddress = shmat(shmemid,0,0);

  while (1)
  { waitsema(sema) ; /* wait for a 0 */
    printf("the message is : %s \n",shmaddress +4);
    setsema (sema,2) ; /* ok, eaten */
  }
}

setsema (sema,val)
int sema,val;
{ semctl(sema,0,SETVAL,val); }

waitsema (sema)

int sema;
{ 
  sembu.sem_num =0; sembu.sem_op = 0; sembu.sem_flg =0;

  semop (sema, sembuptr , 1);
}
\end{verbatim}

\section{Socket interface (Unix only)}

to be added....

\section{Pipe Interface (Unix only)}

The Pipe interface establishes a pipe as a PSL channel from which
(or to which) normal PSL read/write operations can be used.
Pipes are specially useful to supply data to other processes, e.g.
REDUCE uses it to send data to a plot program.
\\

\thing{pipe!-open S:string M:Mode}{integer}{expr}{Starts a process and
executes the command in <string>. The mode must be either Input or Output.
Returns a channel.}

Pipes are closed with a normal close call. This will kill the process
that has been started. If this process should survive, you have to use
abandonpipe.
\\

\thing{abandonpipe channel}{void}{expr}{Closes channel without killing
the process started by pipe!-open. Returns {\tt nil}.}
                                               
Example:
\begin{verbatim}
(load pipes)
(setq chn (pipe!-open "uname"))
(channelread chn)
(close chn)
\end{verbatim}

\section{Mapping of LISP Addresses to C addresses}

This paragraph applies for some architectures only, e.g. 
the {\tt IBM/RS, HP PA-Risc and MIPS} where the system architecture
uses 'high' bits in the addresses which conflict with the
PSL tagging mechanism.(See the paragraph on the Size of the
Address space above).  The consequence of this is that using the 
inf operation
for calculating the C equivalent from a LISP address does not
work here. The easiest way (without explaining the reason)
of mapping for the architectures is:

\begin{tabular}{rl}
	IBM/RS&(mkstr item) \\
	HP PA-Risc&(mkvec item)\\
	Mips&(wshift (inf item) 2) \\
	Linux\_elf&(mkfixn item)
\end{tabular}

On Convex systems the 'high' bit is the leftmost one. Therefore the
tag bits start at bit 1 and the inf range is limited by 26 bits.
The inf operation works as expected.

