\chapter*{Miscellaneous Useful Features}

\section{Exiting PSL}

\de{quit}{(quit): Undefined}{expr}
{    Return  from  LISP  and  terminate  the PSL process.  }

\de{exitlisp}{(exitlisp): Undefined}{expr}
{  Return  from  LISP  and  terminate  the PSL process.  }

\de{exitwithstatus}{(exitwithstatus STATUS:integer): Undefined}{expr}
{  Return  from  LISP  and  terminate  the PSL process.
   STATUS is passed to the calling command shell as return value.  }

\section{Saving an Executable PSL}

\DE{savesystem}{(savesystem MSG:string FILE:string FORMS:form-list):\\
Undefined}{expr} {    
    This records the welcome message (after attaching a date) in
    the  global variable lispbanner* used by standardlisp's call
    on toploop, and then calls  dumplisp  to  compact  the  core
    image  and  write  it  out as a machine dependent executable
    file with the name FILE.  FILE should have  the  appropriate
    extension  for  an  executable  file.   Savesystem also sets
    *usermode to t.
}

    The forms in the list FORMS will be evaluated when  the  new
    core image is started.  For example

\begin{verbatim}
    (savesystem "PSL 4.2" "PSL" '((read-init-file "psl")))
\end{verbatim}
\variable{lispbanner*}{[Initially: ]}{global}
{
    Records  the  welcome  message given by a call to savesystem
    from PSL.  Also contains  the date, given  by  the  function
    date.  }

\de{dumplisp}{(dumplisp FILE:string): Undefined}{expr}
{    The  core  image  is written as an executable file, with the
    name FILE. Better use the function savesystem.
}
\section{Init Files}

Init files are available to make it easier for the user to
customize PSL to his/her own needs.  When psl or pslcomp
is  executed, if a file .pslrc or .pslcomprc on UNIX boxes
or psl.rc or pslcomp.rc on DOS or Windows systems
is on the users home directory, it will be read and evaluated.

The remainder of this section describes functions defined in
the module {\bf init-file}. These functions can be used to implement
init files. 

\de{user-homedir-string}{(user-homedir-string): string}{expr}
{    Returns a full pathname for the user's home directory.
}

\de{init-file-string}{(init-file-string progname:string): string}{expr}
{    Returns  the  full  pathname of the user's init file for the
    program progname.
}

\de{read-init-file}{(read-init-file progname:string): nil}{expr}
{   
   \index{init file}
    Reads and evaluates the init  file  with  name  progname.
    Read-init-file  calls  init-file-string   with   argument
    progname. Under UNIX the initfilename is concatenated from
    "." progname and "rc", such that a typical initfilename is
    .pslrc for psl
}
\begin{verbatim}
    (read-init-file "psl")
\end{verbatim}

\section{Miscellaneous Functions}


\de{reset}{(reset): Undefined}{expr}
{    Return to the top level of PSL, unwind-protect forms  get  a
    chance to run.  }

\de{time}{(time): integer}{expr}
{    CPU time in milliseconds since login time.  }

\de{date}{(date): string}{expr}
{    The date in the form "day-month-year" }
\begin{verbatim}
    1 lisp> (date)
    "21-Jan-1997"
\end{verbatim}

\de{date-and-time}{(date-and-time): string}{expr}
{   \index{time}
     A  string  is returned which describes the date, followed by
    the time.  The string is of the form
}
    "day-month-year hours:minutes:seconds"

\begin{verbatim}
    1 lisp> (date-and-time)
    "21-Jan-1997 17:04:14"
\end{verbatim}
\section{Garbage Collection}


\de{reclaim}{(reclaim): nil}{expr}
{    Reclaim is the user level call  to  the  garbage  collector.
    Active  data  in  the heap is made contiguous and all tagged
    pointers into the heap from active local stack  frames,  the
    binding stack and the symbol table are relocated.  If *gc is
    t,  prints  some  statistics.  Increments gcknt* and updates
    gctime*.
}

\de{known-free-space}{(known-free-space): integer}{expr}
{    Returns the number of items available in the heap.  }

\de{free-bps}{(free-bps): integer}{expr}
{    Returns the number of items available  in  the  BPS  (Binary
    Program Space).  }

\variable{*gc}{[Initially: t]}{switch}
{
    \index{garbage collection}
    *Gc  controls  the printing  of garbage  collector messages.
    If nil no indication  of  garbage  collection  occurs.    If
    non-nil various messages will be displayed.
}

\begin{verbatim}
    1 lisp> (reclaim)
    *** Gargabe collection starting
    *** GC 1:  8-Nov-97 15:49:54, 50 ms (50 %), 218 recovered, 260679 free
    nil
\end{verbatim}
    As  the example shows, the amount of time (in milliseconds),
    required to do the garbage collection, the number  of  items
    recovered, and the number of items available in the heap are
    displayed.  The first number which follows GC represents the
    value of gcknt*. The inclusion of the date into the message
    has been found useful
    for calculations which take extremely long.

\variable{gctime*}{[Initially: 0]}{global}
{
    The   total   time   (in   milliseconds)  spent  in  garbage
    collection.  When the garbage collector is invoked, the time
    spent performing the collection is added  to  the  value  of
    gctime* }

\variable{gcknt*}{[Initially: 0]}{global}
{
    Records  the  number of times that the garbage collector has
    been invoked.  Gcknt* may  be  reset  to  another  value  to
 			record counts incrementally, as desired.}
