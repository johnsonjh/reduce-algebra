\chapter{Identifiers}

\section{Introduction}

Ids  or identifiers can be used in a number of different ways.
Every id has a name, called its print name.  Given  an  id,  one
can  obtain its name in the form of a string.  Conversely, given
the name of an id as a string one can obtain the id itself.\\


\noindent
Ids have a component called the  property  list.    This  list
consists  of  pairs  and ids.  A pair contains two elements, the
first is the name of the  property,  the  second  is  the  value
associated  with  that  property.    An  id on the property list
represents a flag.\\


\noindent
Each id also references a package, see Chapter [packages]  for
more information on the package system.\\


\noindent
In PSL an id can be used simultaneously as a variable and as a
name for a function.  Aside from the functions described in this
chapter,  there  are  additional  functions for dealing with the
values associated with an id.\\


\noindent
An id can be referenced simply by writing its name. If the
name  consists  only of uppercase alphabetic characters, digits,
or a subset of the special characters (listed below), and if the
name of the id cannot be mistaken for a number,  then the id can
be notated by the sequence of characters in its name.
\begin{center}
\begin{verbatim}
$+ - $ & * / : ; | < = > ? ^ _ { } ~ @$
\end{verbatim}
\end{center}
\noindent
An id may have uppercase letters, lowercase letters, or both  in
its  print  name. The  PSL reader normally (i.e. version 4.2 and
above) converts uppercase
letters to the corresponding lowercase letters when reading ids.
Therefore, most of the time case makes  no  difference  when
notating ids. 

The conversion of letters is controlled by the functions 
{\bf input-case} and {\bf output-case}.


\DE{input--case}{(input-case modus): modus}{expr}
{    If modus equals lower, characters on input are converted to the
corresponding lowercase characters. If modus equals raise, characters
are raised during input. A modus NIL leaves characters unchanged.
The former modus is returned. The default is {\tt lower}.
}

\DE{output--case}{(output-case modus): modus}{expr}
{    If modus equals lower, characters on output are converted to the
corresponding lowercase characters. If modus equals raise, characters
are raised during output. A modus NIL leaves characters unchanged.
The former modus is returned. The default is NIL.
}


Ids are kept in a table  which  is  called  the
symbol table (or id-hash-table).  Two ids are considered
different if their corresponding print names are different.  For
example, the id whose name is "that" is different  from  the  id
whose  name is "THAT". The ids which name PSL functions have
lowercase names. The reason you can type such names with
uppercase letters  is  that  the reader is converting
uppercase letters to lowercase by default.

\begin{verbatim}
1 lisp> (Add1 2)
3
2 lisp> (input-case nil)
lower
3 lisp> (Add1 2)
***** `!Add1' is an undefined function
\end{verbatim}
\noindent  
If the user tries to use a PSL function name for a function he
is defining a warning message appears.

\begin{verbatim}
Do you really want to redefine the system function `NAME'? 
(Y or N)
\end{verbatim}

\noindent
If the user responds "Y", his definition  replaces  the  current
definition.    (See  Chapter  9  for a description of the switch
*usermode which controls the printing of this message).\\

\noindent
There is an escape convention for notating an  id  whose  name
contains special characters.  Any character which follows a ! is
considered  to  be  an  ordinary  character.    In  addition  to
lowercase  letters,  the  following  characters  are  considered
special:
\begin{verbatim}
! " % ' ( ) . [ ] ` , # |
\end{verbatim}
If  it  is  not  clear  from  the  printed  output,  this set of
characters includes both quote and accent grave.  Note  that  if
any  character  wthin  a  name is preceded by a !, then the name
will not be interpreted as a number.\\

\begin{tabular}{ll}
SUSAN &  \% three ways to notate the same id\\
susan\\
SuSan\\
+\$ &     \% an id without alphabetic characters\\
1+ &    \% an id whose first character is a digit\\
+1 &     \% this is a number\\
\verb/x^2+y^2/ & \% an id which looks like an expression\\
!9 &     \% the id whose name is "9", not the number 9\\
\end{tabular}

\section{Identifiers and the Id Hash Table}

The method used by PSL to retrieve  information  about  an  id
makes  use  of  a  symbol  table.    PSL uses a technique called
hashing to implement this table (id hash table is  another  name
for the symbol table).\\

\noindent
The  process  of putting an id into the symbol table is called
interning.  The  PSL  reader  interns  ids  as  they  are  read.
Consider  what  happens  after  the  name of an id is read.  The
symbol table is examined to see if  it  contains  an  identifier
with the same name.  If there is a match then a reference to the
matching  id is returned.  Otherwise, a new id is created, it is
added to the symbol table, and a reference to it is returned.

\subsection{Identifier Functions}

The following functions deal with identifiers and  the  symbol
table.


\de{gensym}{(gensym): id}{expr}
{    An  id  is  created  which is not interned.  Since it is not
    interned it is not eq to any other id.  The  id  is  derived
    from  a  string  of the form "G0000".  The numeric suffix is
    incremented upon each call to gensym.
}

\de{interngensym):}{(interngensym): id}{expr}
{    Similar to gensym but returns an interned id.
}

\de{stringgensym}{(stringgensym string):}{expr}
{    Similar to gensym but returns a string of the  form  "L0000"
    instead of an id.
}

\de{remob}{(remob U:id): id}{expr}
{    If  U  has  been  interned  in  the  symbol table then it is
    removed.  The values associated with U will not be affected.
    U is returned.  It is not possible to remove from the symbol
    table  an  identifier  whose  name  consists  of  a   single
    character.
}
\begin{verbatim}
1 lisp> (setq what (intern "THIS"))
THIS
2 lisp> (set what "SOMETHING")
"SOMETHING"
3 lisp>  % Remove the id whose name is "THIS".
3 lisp> (remob what)
THIS
4 lisp>  % Although the id whose name is "THIS"
has been removed from
4 lisp>  
% the symbol table, it remains in 
%existance and its value %	cell
4 lisp>  % is still defined as "SOMETHING".
4 lisp> (eval what)
"SOMETHING"
\end{verbatim}

\de{newid}{(newid U:string): id}{expr}
{    Creates  an  uninterned  identifier with the specified name.
    The string is used as the print name without  being  copied.
    See  section  2.3  for  the  full definition.  This function
    makes it possible to create a number of distinct  ids  which
    have the same name.  To illustrate the use of this function,
    the   implementation   of  a  package  system  (see  Chapter
    [packages]), requires a function like newid.
}

\de{internp}{(internp U:{id,string}): boolean}{expr}
{    Returns t if U is interned.
}

\de{mapobl}{(mapobl FNAME:function): Undefined}{expr}
{    Mapobl applies function FNAME to  each  interned  id.    The
    following  expression  will  print  each id which is flagged
    global.  Note that there should be only one formal parameter
    to FNAME.
}
\begin{verbatim}
(mapobl '(lambda (item) (if (flagp item 'global)
(print item))))
\end{verbatim}
\subsubsection{Find}

These functions take a string or id as an argument,  and  scan
the  symbol table to collect a list of ids whose names contain a
prefix or suffix which matches the argument.    These  functions
are defined in the library module {\bf find}.


\de{findprefix}{(findprefix KEY:{id, string}): id-list}{expr}
{    Each  interned id whose name contains a prefix which matches
    the KEY is  added  to  the  result.    The  ids  are  sorted
    alphabetically.  The expression
}
\begin{verbatim}
    (findprefix '*)
\end{verbatim}
    will  return  a  list  of all of the interned ids whose name
    begins with *.


\de{findsuffix}{(findsuffix KEY:{id, string}): id-list}{expr}
{    Each interned id whose name contains a suffix which  matches
    the  KEY  is  added  to  the  result.    The  ids are sorted
    alphabetically.  The expression
}
\begin{verbatim}
    (findsuffix "STRING")
\end{verbatim}
    will return a list of all of the  interned  ids  whose  name
    ends with STRING.

\section{Property List Functions}

A  property  list  is  used  to  associate an id with a set of
entities;  those  entities  are  called  flags  if   their   use
associates a boolean value with the id, and properties if the id
is to be associated with an arbitrary attribute.


\de{put}{(put U:id IND:id PROP:any): any}{expr}
{    The  indicator  IND with the property PROP is placed on  the
    property list of U.  If the action of put occurs, the  value
    of  PROP is returned.  If either U or IND are not ids then a
    type mismatch error occurs.
}
\begin{verbatim}
    ***** An attempt was made to do PUT on `U', which is not  
    an identifier
\end{verbatim}
    The  definition  of  a  property  will  cause  the  previous
    definition to be lost.

\de{get}{(get U:id IND:id): any}{expr}
{    Returns the  property associated with indicator IND from the
    property list of U.  If U does not have indicator  IND,  nil
    is returned.  Get returns nil if U is not an id.
}

\de{deflist}{(deflist U:list IND:id): list}{expr}
{    U  is  a  list  in which each element is a two-element list:
    (ID:id PROP:any).  Each id in U has the indicator  IND  with
    property  PROP  placed  on its property list by the function
    put function.  The value of deflist is a list of  the  first
    elements of each two-element list.
}
\begin{verbatim}
    1 lisp> (deflist '((plus2 'two)
                       (plus 'many))
                     'no-operands)
    (plus2 plus)
    2 lisp> (get 'plus 'no-operands)
    many
\end{verbatim}

\de{remprop}{(remprop U:id IND:id): any}{expr}
{    Removes  the  property  with indicator IND from the property
    list of U.  Returns the removed property or nil if there was
    no such indicator.
}

\de{rempropl}{(rempropl U:id-list IND:id): nil}{expr}
{    Removes the property IND from all of the ids in U.
}

\noindent
The following example is intended to illustrate  the  idea  of
data  driven  programming.  We define a function called simplify
which will  simplify  symbolic  algebraic  expressions.    These
expressions  are  represented as lists.  To begin, there will be
only  one  operator  (plus),  and  operands  may  be   integers,
variables  or  an application of plus.  Prefix notation is used.
The addition of variable x and 3 would be represented as (plus x
3).  The first version of simplify will certainly do the job.

\begin{verbatim}
(de simplify (expression)
  (cond ((atom expression) expression)
        ((eq (first expression 'plus)
             (add-simplify expression))
        (t expression)))
\end{verbatim}
However, as we  add  operands  it  will  become  necessary  to
redefine  simplify.   A better approach is to allow the operator
to specify the information on how to simplify the expression.

\begin{verbatim}
(de simplify (expression)
  (cond ((atom expression) expression)
        (t (apply (get (first expression) 'simplify)
                  (ncons expression)))))
\end{verbatim}
\begin{verbatim}
(put 'plus 'simplify 'add-simplify)
\end{verbatim}
This version will not have to be rewritten when a new operator
is added.  For example, if the operator times is added  then  we
only  need to define a function called times-simplify and attach
its name to the property  list  of  times  under  the  indicator
simplify.    We  can  design  add-simplify in a similar fashion.
Using this approach we will be able to accomodate numbers  other
than integers.

\begin{verbatim}
  (de add-simplify (expression)
    (let ((left (second expression))
          (right (third expression)))
      (cond ((zerop left) right)
            ((zerop right) left)
            ((and (numberp left)(numberp right))
             (let ((new (common-type left right)))
               (apply (get (data-type (first  new))'add-op) new)))
            (t (list (first expression)
                     (simplify left)
                     (simplify right))))))
\end{verbatim}
\begin{verbatim}
(put 'integer 'add-op 'plus2)
\end{verbatim}
\begin{verbatim}
1 lisp> (simplify '(plus (plus 1 8) (plus x 0)))
(plus 9 X)
\end{verbatim}

\subsection{Functions for Flagging Ids}

\de{flag}{(flag U:id-list V:id): nil}{expr}
{    Flag flags each id in U with V; that is, the effect of  flag
    is  that  for  each  id X in U, (flagp X V) has the value t.
    Both V and all of the elements of U must be identifiers or a
    type mismatch error occurs.  The id V  will  appear  on  the
    property  list  of  each  id in U.  However, flags cannot be
    accessed, placed on, or removed from  property  lists  using
    the  normal  property  list functions get, put, and remprop.
    Note that if an error occurs during execution of flag,  then
    some  of  the ids in U may be flagged with V, and others may
    not be.  The statement below causes  the  flag  Lose  to  be
    placed on the property lists of the ids x and y.
}
\begin{verbatim}
    (flag '(x y) 'lose)
\end{verbatim}
\de{flagp}{(flagp U:id V:id): boolean}{expr}
{    Returns  t  if  U has been flagged with V; otherwise returns
    nil.  Returns NIL if either U or V is not an id.
}

\de{remflag}{(remflag U:id-list V:id): nil}{expr}
{    Removes the flag V from the property list of each member  of
    the list U.  Both V and all the elements of U must be ids or
    the type mismatch error occurs.
}

\de{flag1}{(flag1 U:id V:any): Undefined}{expr}
{    The  identifier  U  is flagged V.  The effect is to add V to
    the property list of U.
}

\de{remflag1}{(remflag1 U:id V:any): Undefined}{expr}
{    The identifier U is no longer flagged V.  The effect  is  to
    removed V from the property list of U.
}
\subsection{Direct Access to the Property Cell}

  Use  of  the  following functions can destroy the integrity of
the property list.  Since PSL uses properties at  a  low  level,
care should be taken in the use of these functions.


\de{prop}{(prop U:id): any}{expr}
{    Returns the property list of U.
}

\de{setprop}{(setprop U:id L:any): L:any}{expr}
{    Store item L as the property list of U.
}
\section{Value Cell Functions}

The contents of the value cell of an id is usually accessed by
eval (Chapter 11) or valuecell (below) and changed by setq, setf
or sometimes set.

\de{setq}{(setq [VARIABLE:id VALUE:any]): any}{fexpr}
{    The value of each VARIABLE is set to the corresponding value
    of  VALUE.   Each argument VALUE is evaluated, each argument
    VARIABLE is not evaluated.  It is not true that
}
\begin{verbatim}
    (setq variable value)
\end{verbatim}
    is equivalent to

\begin{verbatim}
    (set 'variable value)
\end{verbatim}

				Where setq may be used to set any type of  variable  (fluid,
    global or local) the function set is restricted to fluid and
    global variables.


\de{set}{(set EXP:id VALUE:any): any}{expr}
{    Set  is  used  to  define the value cell of fluid and global
    identifiers.  An error occurs if EXP does not evaluate to an
    identifier.
}
\begin{verbatim}
    ***** An attempt was made to do SET on `EXP', which  is  not
    an identifier
\end{verbatim}
%\newpage
    If EXP evaluates to t or nil an error occurs.
\begin{verbatim}
    ***** Cannot change T or NIL
\end{verbatim}
\de{desetq}{(desetq U:any V:any): V:any}{macro}
{    This  function  is  part of the USEFUL package.  Desetq is a
    destructuring setq.  That is, the first argument is  a  list
    whose  elements are ids.  The value of each id is set to the
    corresponding element in the second argument.  For  example,
    evaluation of}
\begin{verbatim}
    (desetq (A (B) . C) '((1) (2) (3) 4))
\end{verbatim}
    defines  the  value of A to be (1), B to be 2, and C to ((3)
    4).

\de{psetq}{(psetq [VARIABLE:id VALUE:any]): Undefined}{macro}
{    This function is defined in the USEFUL package.    Psetq  is
    very  similar  to  setq.   The difference is that with psetq
    each VALUE is evaluated before any assignment is made.}
\begin{verbatim}
    1 lisp> (setq a 'same  b a)
    SAME
    2 lisp> (eq a b)
    T
    3 lisp> (psetq a 'other  b a)
    OTHER
    4 lisp> (eq a b)
    NIL
\end{verbatim}

\de{setf}{(setf [LHS:form RHS:any]): RHS:any}{macro}
{    The ability to assign values to ids allows us  to  think  of
    ids  as  variables.    We  can  generalize  this  notion  of
    variable.  For example, a  list  can  be  thought  of  as  a
    collection   of   anonymous  variables.  Usually  there  are
    seperate access  and  update  functions  for  each  kind  of
    generalized variable.  For example the function cdr accesses
    the cdr of a pair, the function rplacd updates it.  However,
    we  can think of a call on an access function as a reference
    to a storage location.  Just as we consider the  mention  of
    an  id  to  be  a  reference to its value, (cdr pair) can be
    thought of as the name for the cdr for  some  pair.   Rather
    than  having  to  remember  two  functions  for each kind of
    generalized variable (rplacd corresponds  to  cdr),  we  can
    adopt  a uniform syntax for updating storage locations using
    the setf macro.
}\\

\noindent
    The application of setf can take on any one of the following
    forms:\\

{\tt (setf id data)}
\qquad expands into\qquad
{\tt (setq id data)}

{\tt(setf (eval form) data)}
\qquad  expands into\qquad
{\tt (set form data)}\\

The same effect is obtained by substituting value in
place of eval.\\

{\tt (setf (car pair) data)}
\qquad expands into\qquad
{\tt (rplaca pair data)}

{\tt (setf (cdr pair) data)}
\qquad expands into\qquad
{\tt (rplacd pair data)}

{\tt (setf (getv vector index) data)}
\qquad expands into\qquad\\ 
		\qquad	\qquad	{\tt (putv vector index data)}

{\tt (setf (indx form index) data)}
\qquad expands into\qquad\\ 
 \qquad  \qquad  {\tt (setindx form index data)}

{\tt (setf (sub form start size) data)}
\qquad expands into\qquad\\ 
	\qquad	\qquad {\tt (setsub form start size data)}

{\tt (setf (nth pair index) data)}
\qquad expands into an expression similar to\qquad\\ 
		\qquad \qquad	{\tt (rplaca (pnth pair index) data)}\\


If the first argument to setf is a macro then it will be expanded
    before setf is.  For example, if first is defined as\\

{\tt (ds first (p) (car p))} then\\

{\tt (setf (first p) data)}
\qquad is equivalent to\qquad
{\tt (setf (car p) data)}\\

\noindent
The USEFUL module contains an expanded version of setf.  The
    basic definition of setf is  not  consistent  with  that  of
    setq.    The  value  returned from an application of setq is
    always the value assigned. For example, the expression\\


{\tt (setf (car '(a b)) 'd)}
\qquad expands into\qquad
{\tt (rplaca '(a b) 'd)}\\

\noindent
    The value returned after evaluating this  second  expression
    is  (d  b).  The extended version of setf will always return
    the value assigned.\\

\noindent
    An application of the extended version of setf  will  accept
    the additional following forms:\\


\noindent
{\tt (setf (cons left right) pair)}
\qquad will expand into an expression similar to \\
 \qquad {\tt (progn (setf left (car pair)) (setf right (cdr pair)))}\\

{\tt (setf (cXYr pair) data)}
\qquad expands into an expression similar to\qquad\\
	\qquad	\qquad			{\tt (rplacX (cYr pair) data)}\\


where X is either a or d and Y is either a, d, aa, ...,
or ddd\\

{\tt (setf (flagp id name) data)}
\qquad expands into an expression similar to\qquad\\ 
	\qquad	\qquad			{\tt (flag (list id) name)}, if data is non-nil otherwise\\
	\qquad	\qquad			{\tt (remflag (list id) name)}

{\tt (setf (get id name) data)}
\qquad expands into\qquad
{\tt (put id name data)}

{\tt (setf (getd name) data)}
\qquad expands into an expression similar to\qquad\\ 
	\qquad	\qquad			{\tt (putd name (car data) (cdr data))}\\


This expansion assumes that data is similar to an expression which
would be returned by a call on getd. If data is a code-pointer
or a lambda expression then 'expr is used in place of (car data).\\


{\tt (setf (lastcar pair) data)}
\qquad expands into an expression similar to\qquad\\ 
	\qquad	\qquad			{\tt (rplaca (lastpair pair) data).}

{\tt (setf (list a b c ...) pair)}\\
\qquad the expansion of this expression is very similar to the expansion
of\qquad\\
	\qquad	\qquad			{\tt (desetq (a b c ...) pair)}

{\tt (setf (pnth pair index) data)}
\qquad expands into an expression similar to\qquad\\
	\qquad	\qquad		 {\tt (rplacd (pnth pair (sub1 index)) data)}

{\tt (setf (vector al bl cl ...) [ar br cr ...])}\\
\qquad expands into an expression similar to\\
\qquad	\qquad	{\tt (progn (setf al ar)(setf bl br)(setf cl cr)...)}

The setf function is extensible to permit additional operators on the left
hand side. If there is an assign-op property on the property list of the
operator then the value of that property (either a lambda expression or
the name of a function) is used to build the expansion of the macro.
The effect is similar to
\begin{verbatim}
    (apply (get op 'assign-op) (append (cdr lhs) (list rhs)))
\end{verbatim}
    The  property  setf-expand  is searched for when their is no
    assign-op property.  If there is such a property  its  value
    is  applied to the two arguments passed to setf.  The effect
    is similar to
\begin{verbatim}
    (apply op (list lhs rhs))
\end{verbatim}
    If the left hand side operator is flagged as setf-safe, then
    it is assumed that the expansion of the macro will yeild  an
    expression  which  will  return  the value of the right hand
    side.  Otherwise the expansion will take one  of  the  forms
    listed  below.    Within the second expansion, references to
    RHS are replaced with references to VAR.  The second form is
    used when RHS is a list, the assumption  being  that  it  is
    effecient to evaluate an application only once.

\begin{verbatim}
    (progn expansion rhs)

    (let ((var rhs))
      expansion
      var)
\end{verbatim}
\de{psetf}{(psetf [LHS:form RHS:any]): Undefined}{macro}
{    This  function  is  defined in the USEFUL package.  Psetf is
    very similar to setf.  The difference  is  that  with  psetf
    each RHS is evaluated before any assignment is made.
}

\de{makeunbound}{(makeunbound U:id): Undefined}{expr}
{    U  is  made an unbound identifier, that is to say it will no
    longer have a value.  This function  should  be  applied  to
    fluid identifiers only.
}

\de{valuecell}{(valuecell U:id): any}{expr}
{    Safe access to the value cell of an id.  If U is not an id a
    type  mismatch  error  occurs.    If  U is an unbound id, an
    unbound id error occurs.  Otherwise the current value  of  U
    is  returned.    This  function  should  be applied to fluid
    identifiers only.
}

\de{unboundp}{(unboundp U:id): boolean}{expr}
{    Returns t is U is unbound.  This function should be  applied
    to fluid identifiers only.
}
\section{System Global Variables, Switches and Other "Hooks"}

\subsection{Introduction}

  A number of global variables provide global control of the PSL
system,  or  reference  values  which  are  constant  throughout
execution.   Certain  options  are   controlled   by   switches,
variables  which  have a value of either t or nil.  For example,
the value of *verboseload controls the display of messages  when
files  are loaded.  The values of other global variables are not
restricted to be boolean.  For example, the value of outputbase*
is the radix  in  which  numbers  are  printed.   PSL  uses  the
convention that the name of a global variables which is a switch
begins  with  "*".  The names of other global variables end with
"*".  

\subsection{Setting Switches}

Strictly   speaking,   NAME   is  a  switch  and  *NAME  is  a
corresponding global variable that assumes a value of t  or  nil.
Both  NAME  and  *NAME  are  loosely  referred  to  as  switches
elsewhere in the manual.\\

\noindent
The functions on and off functions  are  used  to  change  the
values  of  the  identifiers  associated  with  switches.   Some
switches contain an s-expression on their property  lists  under
the  indicator  simpfg \footnote{The name simpfg comes from its
introduction in the Reduce algebra system, where it was used  to
specify  various  simplifications  to  be  performed  as various
switches were turned on or off.}.  The s-expression has the form
of a cond list:

\begin{verbatim}
((t (action-for-on)) (nil (action-for-off)))
\end{verbatim}
If the  simpfg  indicator  is  present,  then  the  on  and  off
functions   also   evaluate   the   appropriate  action  in  the
s-expression.


\de{on}{(on [U:id]): Undefined}{macro}
{    For each switch U, the associated identifier *U  is  set  to
    nil.   If  the  clause (t (action-for-on) is found under the
    indicator simpfg on U then the expression action-for-on will
    be evaluated.
}

\de{off}{(off [U:id]): Undefined}{macro}
{    For each switch U, the associated identifier *U  is  set  to
    nil.  If the clause (nil (action-for-off) is found under the
    indicator  simpfg  on  U  then the expression action-for-off
    will be evaluated.
}
  The  switch  fast-integers  is  used  by  the  compiler   when
arithmetic  expressions  are compiled.  There are definitions of
numeric  operators  which  do  not  check  the  types  of  their
arguments in order to reduce execution time.  Evaluation of

\begin{verbatim}
(get 'fast-integers 'simpfg)
\end{verbatim}
returns

\begin{verbatim}
((t (enable-fast-numeric-operators))
 (nil (disable-fast-numeric-operators))))
\end{verbatim}
Evaluation  of  (on fast-integers) will result in *fast-integers
being   set   to   t   and    evalution    of    the    function
enable-fast-numeric-operators.

\subsection{Special Global Variables}

\variable{nil}{[Initially: nil]}{global}
{
    A  special global variable whose value cannot be modified by
    set or setq.
}

\variable{t}{[Initially: t]}{global}
{
    A special global variable whose value cannot be modified  by
    set or setq.
}

\subsection{Special Put Indicators}

  Some  actions search the property list of relevant ids for the
following indicators.

\begin{Ventry}{\bf code-pointer}
\item [{\bf breakfunction}] Associates a function to be 
              run with an  id typed
              in  a break loop (see Chapter 16).  For example, q
              is used to exit from a  break  loop  and  (get 'q
              'breakfunction) returns breakquit.

\item [{\bf type}] PSL  uses  the property type to indicate whether
														a function is a fexpr, macro, or  nexpr.    If  this
              property  is absent, expr is assumed. For example,
              (get 'and 'type) returns fexpr.

\item [{\bf vartype}] PSL uses the property vartype to indicate 
whether
              an identifier is of type global or fluid.
\begin {verbatim}
         1 lisp > (fluid '(mary))
         nil
         2 lisp > (get 'mary 'vartype)
         fluid 
\end{verbatim}

\item[{\bf *lambdalink}]   The  interpreter  looks  under  *lambdalink  for a
              lambda  expression  when  a  compiled   definition
              cannot be found.
\begin {verbatim}
         1 lisp > (de list-first (p) (car p))
         list-first
         2 lisp > (get 'list-first '*lambdalink)
         (lambda (p) (car p))
\end{verbatim}
\end{Ventry}

\noindent
The  compiler and loader use a number of other indicators, see
Chapter 19.

\subsection{Special Flag Indicators}

%\TABELLE***
\begin{Ventry}{\bf eval, ignore}
\item [{\bf eval, ignore}]  These flags are  used  primarily  to 
control  the
              evaluation  of expressions during compilation (for
              more information see Chapter 19).
              \index{eval}\index{ignore}

\item [{\bf lose}]          The function putd is used  to 
associate  function
              definitions  with ids.  Its application is aborted
              if the id has been flagged lose.
              \index{lose}
\begin{verbatim}
*** `NAME' has not been defined, because it is flagged LOSE
\end{verbatim}

\item [{\bf user}]          If  *usermode is t, when a function is
defined its
              name will be  flagged  user.    This  is  used  to
              distinguish  user  defined  functions  from system
              functions (see Chapter 9 for more information).
              \index{user}\index{*usermode}
%\end{TABELLE***}
\end{Ventry}
