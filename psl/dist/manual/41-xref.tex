\chapter*{Cross Reference Tools}

\section{Introduction}

  This chapter describes tools which build cross references, and
which aid in  analyzing  LISP  code  for  building  other  cross
reference - like tools.

\subsection{RCREF - Cross Reference Generator for PSL Files}

{\bf rcref} is a Standard LISP program for processing a set of PSL
function definitions to produce:

\begin{enumerate}
\item  A "Summary" showing:
\begin{itemize}

 \item       A list of files processed.


\item        A list of "entry points" (functions  which  are    not
          called or are called only by themselves).

\item A list of undefined  functions  (functions  called but
          not defined in this set of functions).

\item         A list of variables that were used non-locally but
not
          declared GLOBAL or FLUID before their use.

\item        A list of variables that were declared GLOBAL but used
          as FLUIDs (i.e. bound in a function).

 \item        A  list  of  FLUID  variables that were not bound in
a
          function so that one  might  consider  declaring  them
          GLOBALs.

 \item        A list of all GLOBAL variables present.

 \item        A list of all FLUID variables present.

\item         A list of all functions present.
\end{itemize}

\item     A "global variable usage" table, showing for each
non-local variable:
\begin{itemize}

\item          Functions in which it is used as a declared
										FLUID or GLOBAL.
  
\item         Functions in which it is used but not declared
          before.

\item          Functions in which it is bound.

\item         Functions in which it is changed by setq.
\end{itemize}
\item   A "function usage" table showing for each function:

\begin{itemize}
\item         Where it is defined.

\item        Functions which call this function.

\item         Functions called by it.

 \item        Non-local variables used.
\end{itemize}
\end{enumerate}
   The  output  is alphabetized on the first seven characters of
each function name.

   RCREF   also  checks  that  functions  are  called  with  the
correct number of arguments.

\subsection{Restrictions}

  Algebraic  procedures  in   Reduce  are  treated  as  if  they
were symbolic,  so that algebraic constructs  actually appear as
calls to symbolic functions, such as aeval.

  RCREF adds the eval flag to some  functions,  so  it  probably
cannot coexist with the compiler in a core image.

  SYSLisp procedures are not correctly analyzed.

\subsection{Usage}

  RCREF  should  be  used in PSL:RLisp.  To make a file FILE.CRF
which is a cross  reference  listing  for  files  FILE1.EX1  and
FILE2.EX2  do the following in RLisp:

LOAD RCREF;       % RCREF is now autoloading, so this may be omitted.

OUT "file.crf";   % later, CREFOUT ... \\
ON CREF; \\
IN "file1.ex1","file2.ex2"; \\
OFF CREF; \\
   SHUT "file.crf".  % later CREFEND \\

To  process  more files, more IN statements may be added, or the
IN statement may be changed to include more files.

\subsection{Options}

\variable{*crefsummary}{[Initially:NIL]}{switch}                          

{    If the switch CREFSUMMARY is ON then only the summary (see 1
    above) is produced.}

  Functions with the flag NOLIST are  not  examined  or  output.
Initially, all PSL functions are so flagged.  (In fact, they are
kept on a list NOLIST!*, so if you wish to see references to ALL
functions,  then  CREF  should  be first loaded with the command
LOAD RCREF, and this variable then set to NIL).  (RCREF  is  now
autoloading.)

\variable{nolist*}{[Initially: the following list]}{global}
{}
(and
cond
list
max
min
or
plus
prog
prog2
progn
times
lambda
abs
add1
append
apply
assoc
atom car cdr caar
cadr cdar cddr
caaar caadr cadar
caddr cdaar
cdadr
cddar
cdddr
caaaar
caaadr
caadar
caaddr
cadaar
cadadr
caddar
cadddr
cdaaar
cdaadr
cdadar
cdaddr
cddaar
cddadr
cdddar
cddddr
close
codep
compress cons 
constantp de deflist delete df difference digit divide 
dm eject
eq eqn equal error errorset eval evlis expand explode 
expt fix fixp
flag flagp float floatp fluid  fluidp function densym 
get getd  
getv global globalp go greaterp idp intern length 
lessp linelength
liter lposn map map mapc mapcan mapcar mapcon  
maplist max2 member 
memq minus minusp min2 mkvect nconc not  
null numberp onep
open pagelength pair pairp plus2 posn 
princ print prin1 prin2
prog2 put putd putv quote quotient rds read readch 
remainder remd 
remflag remob remprop return reverse rplaca  
rplacd sassoc set  
setq stringp sublis subst sub1  terpri times2 
unfluid upbv vectorp 
wrs zerop)

  It should also be remembered that in RLisp any  macros    with
the  flag  EXPAND or, if FORCE is on, without the flag NOEXPAND,
are expanded  before  the  definition    is    seen    by    the
cross-reference  program,    so  this  flag  can also be used to
select those macros you require expanded and those you  do  not.
The use of ON FORCE; is highly recommended for CREF.

\section{Scanalyzer}

\subsection{Introduction}

  The  scanalyzer  module is a tool for analyzing PSL functions,
expressions, and source files.  It makes it easy to analyze  PSL
code   and   to  write  preprocessors  for  PSL.    It  supports
preprocessing of PSL code in much the way that macroexpand does,
but is more flexible and extensible.  The scanalyzer understands
the various PSL function types, including macros, and  has  some
understanding  of  each  of  the  PSL special forms.  Scanalyzer
lives in SCANALYZER.SL, some support functions
(including scanalyze-file) live in XREF-SUPPORT.SL.

  Apologies  to John Brunner, author of "Stand on Zanzibar", for
the name of this facility.

\subsection{Philosophy}

  The meaning of a piece of LISP code can depend a great deal on
the context in which it is compiled  or  evaluated.    Virtually
everything   in  a  LISP  system  can  change  dynamically,  and
analyzing the  meaning  of  code  in  the  right  context  is  a
significant problem.

  This  facility  focuses  on analysis of code that is compiled,
and it does so for two reasons.  One is that most large  systems
written  in  PSL run compiled.  The other is that once LISP code
is compiled, the meaning of the object code is much more  static
than the meaning of the corresponding interpretive code.

  Reliable system operation depends on modules being compiled in
an  environment that is known and fixed as much as possible.  To
do this, we compile each module with a compiler that is  started
afresh. After  one module has been compiled we do not compile
other modules with the same compiler.

  To precisely analyze a module, we can start a compiler.  To it
we add the  analysis  facility,  which  should  not  change  the
compilation  context. We then analyze the module, just as we
would compile it. The analysis facility must be sensitive  to
the same contextual information that the compiler is.

  Many   modules  have  some  side-effects  on  the  context  of
compilation -- defining a macro is just one of the  things  that
affects  compilation  context. To be sure that a module is
analyzed in the right context, the analysis should have the same
side-effects on compilation context as the compilation would
have.

\subsection{Functions}

\de{scanalyze-file}{(scanalyze-file INPUT-FILE:{string,
pathname}\\ 
OUTPUT-FILE:{string, pathname}):
string}                 {expr}
{    Applies  the  function  scanalyze-form  (defined  in section
    2.3.3) to every form in input-file.  Scanalyze-file  assumes
    that  information  accumulates  in the list xref-assertions.
    After all the forms have been analyzed, scanalyze-file dumps
    all the assertions into output-file as a list (reversing the
    list first).  The assertion ($<input-file>$ IS-SOURCE-FILE) is
    added to the front of the assertion list.
    
				File  name  defaulting:  As  its  arguments   scanalyze-file
    accepts  both  filename  strings and pathnames.  In any case
    the output-file argument is defaulted  from  the  input-file
    argument.  Any missing components of the name in output-file
    are  defaulted using the function merge-pathname-defaults to
    be the same as in  input-file,  except  for  the  file  type
(suffix). If that is not specified in output-file it is set to
".XD", which is the tentative standard for Xref-Databases.

    For  example,  in  the  call  (scanalyze-file   "pk:load.sl"
    "mydir:"), the output filename generated is "mydir:load.xd".
    In   the   call   (scanalyze-file  "\#5:/pslroot/foo/test.sl"
    "mydir:test-out.dat",  the  output  filename  generated   is
    "mydir:test-out.dat".
}

\de{scanalyze-form}{(scanalyze-form FORM:form): form}{expr}
{    This expects an s-expression as its argument and analyzes it
    as  a top-level form in a source file.  By recognizing which
    forms would be evaluated at compile-time, load-time,  or  at
    both  times, this keeps the context for analysis the same as
    the context would be for  compilation.    Those  forms  that
    would be compiled are passed to the function Scanalyze.}

    Scanalyze-Form   recognizes   forms   such   as  compiletime
    bothtimes, and loadtime.  It evaluates each  at  compiletime
    if  the  compiler  would  and  analyzes each if the compiler
    would compile it.  Actually, it recognizes  the  IGNORE  and
    EVAL flags used internally by the compiler.  This means that
    if analysis is started in the same context as a compilation,
    during  analysis  macros are defined in the same way in both
    cases, as are wconsts, if systems expand the same  way,  the
    same  "compiletime-loaded" modules are loaded, etc..  In PSL
    this means that the analysis context  very  closely  matches
    the compilation context.

\de{scanalyze}{(scanalyze FORM:form ENV:form): form}{expr}
{    Analyzes   the   given  form  (s-expression)  in  the  given
    environment.  The value  returned  depends  on  whether  the
    caller  has  set  up  values  for  functional  variables  or
    properties that this function  is  sensitive  to.    If  the
    caller  has  set  up  no special actions, the value returned
    should be the  same  that  the  function  MACROEXPAND  would
    return.
}
\subsection{Environment Arguments}

  The  caller  can  set  up  special  actions  that  are  in two
categories.   These  are  "analysis  hooks"  and  "preprocessing
hooks".

  The  ENV  arguments  to  various  things are environments.  An
environment is currently a list of "frames".  Each  frame  is  a
list  of  2  elements.   The first identifies the type of frame,
currently always 'LOCALS.  The second  is  the  "contents",  for
locals a list of the local variable names.

\subsection{Analysis Hooks}

  The scanalyzer defines a number of variables that the user may
give functional values to. Some support the user of this module
in  analyzing  code. Others provide the means to preprocess or
transform code. The hooks are functional variables  of 2  args
(expression and environment) to attach an analysis function to.
Each of these must either be NIL or a list of functions. The
first three of these hooks are called before any preprocessing
or expansion of the form is done; the last is called after all
preprocessing and expansion of a form is complete. Scan-Macro-Hooks
Called before each macro is expanded. Scan-Fn-Hooks Called  before
descending into any expr to analyze it. Scan-Non-Pair-Hooks Called
with each non-pair examined.

After-Expansion-Hooks Called with each fully expanded
    expression. Not applied to forms such as macro
    calls, because they are expanded further before
    being returned.


\subsection{Properties}

  These are checked respectively before and after an expression
has been expanded. After expansion means that the call has been
expanded until it is not a macro call, and all subexpressions
have been expanded also.  If the car of an expression is an id
that has the appropriate property, each member of the list that
is the property value will be funcalled. These properties are
checked at the same points where the scan-xxx hooks are checked.
Thus it is of no use to give a macro a post-expand-scanners
property. The properties are:
   
			- Pre-Expand-Scanners

   - Post-Expand-Scanners

  Pre-expand-scanners take precedence over scan-macro-hooks and
scan-fn-hooks.

\subsection{Information Made Available By Scanalyzer}

Current-Function This has the name of the current function where
that can be determined. Forms such as DE and DEFUN that expand into
calls on PUTD with constant function names cause this to be rebound
inside analysis of the function body argument.

Current-Top-Level-Form Contains the argument that was passed to
the call on scanalyze-form currently being executed.

Top-Level-Code? Is T for expressions not nested in any invocation of
the function FUNCTION. Expressions for which this variable is T are
known to be performed at initialization time.


\subsection{Expansion And Preprocessing Hooks}

Special-Expander Should  be a functional variable of 2 arguments
when set to a non-NIL value. If it is, and if the similar functional
value of expand-specially? is non-NIL, the special-expander function
will be used to expand the form.

Expand-Specially? If non-NIL, we use special-expander to expand the
form. Non-Pair-Expander Either NIL or a functional value of 2
arguments used to expand expressions that are "atomic".


  If any function or macro, etc.has an Expr-Expander property,
its value must be a 2 argument function that scans and expands
expressions whose first element is that id. This module
provides several such functions so that built-in special forms
can be analyzed.

  Special-Expander and Non-Pair-Expander have first priority.
Then comes any Expr-Expander property. Then normal processing
by type of function.

\subsection{Cross Reference Support}

  The  fluid  xref-assertions  holds the assertions generated by
xref-assert and  xref-assert-list,  dumped  out  to  a  file  by
scanalyze-file.


\de{xref-assert}{(xref-assert A:assertion): assertion}{expr}
{}
\de{xref-assert-list}{(xref-assert-list L:list): assertion-list}{expr}
{}
  These functions add the single assertion or list of assertions
to xref-assertions.


\de{record-usage}{(record-usage EXPR:form ENV:form): nil}{expr}
{    Suitable  as  a  function to be applied to each (non-atomic)
    expression  and  subexpression  being  analyzed.     It   is
    especially  suitable as an after-expansion hook.  Except for
    various special cases described below, record-usage makes an
    assertion for each expression that is a function call.   The
    assertion is of the form, ($<$fn1$>$ CALLS $<$fn2$>$) where
				$<$fn2$>$ is the function being called (car of
    the expression). $<$Fn1$>$ is the current-function, or if 
    that is NIL, it is the current-file.
}

  Record-usage ignores non-pair expressions. Expressions that
are pairs (function calls) are checked for an openfn or opencode
property, which would imply that they are compiled in-line. No
assertion is  made for these or for expressions where the
function part (car) is not an id.

  If the function being called has a usage-asserter property,
the value of that property is funcalled to make whatever
assertions it will. Otherwise if the function is flagged
dont-record-usage, no assertion is made.


\de{record-macro-usage}{(record-macro-usage EXPR:form ENV:form): 
nil}{expr}
{    record-macro-usage is  like  record-usage,  but  it  is  for
    macros  and  cmacros.    This  function  is  suitable  as  a
    scan-macro hook.  The assertions generated by this  function
    in  the  usual  case  are  ($<$fn1$>$ USES-MACRO $<$macro$>$)
    or for cmacros, ($<$fn1$>$ USES-CMACRO $<$cmacro$>$).
}

\de{setup-some-xref-actions}{(setup-some-xref-actions): t}{expr}
{    The programmer who likes some  of  the  facilities  in  this
    module,  but  doesn't want everything here can just load the
    module.  If you want to  use  the  facilities  and  defaults
    available in the module, call this function.  In addition to
    setting  up  record-usage and record-macro-usage as analysis
    hooks, it flags some sets of functions as not  having  their
    usage  recorded.    It  also  sets  up  some  special  usage
    asserters for certain functions.
}

  The value of each of these variables is a list of functions (or 
macros  or  both).  The "setup" function, above,  flags each
function on each of these lists as dont- record-usage.

\variable{kernel-fns}{[Initially: A list of all defined\\
functions in a PSL kernel.]}{global}

\variable{useful-fns}{[Initially: A list of frequently-used functions\\
and macros defined in the USEFUL library module.]}{global}

\variable{object-fns}{[Initially: A list of frequently-used functions\\
and macros in the OBJECTS module.]}{global}
{}
The following are special usage asserters for certain
functions and macros. See the source code for details.\\

   - Load-Usage-Asserter

   - Imports-Usage-Asserter

   - Send-Usage-Asserter

   - Defmethod-Usage-Asserter

