\section{Data Types}

\subsection{Data Types and Structures Supported in PSL}

\subsubsection{Data Types}

In  contrast  to many programming languages, type declarations
are not needed in PSL.  Data objects contain  information  about
their  type.   Some functions, like equal, are "generic" in that
the result they return depends on the types of the arguments.

\begin{verbatim}
1 lisp> (equal "sextuped" "sextuped")
T
2 lisp> (equal [bencolin beef] [bencolin beef])
T
\end{verbatim}
For the purposes of input and output, an appropriate notation is
used for each type of data object used in  PSL.    For  example,
double  quotes  are  used to delimit the characters of a string.
For a full discussion on syntax see Chapter 12.

The basic data types supported in PSL and a  brief  indication
of their representations are described below.

\begin{Ventry}{\bf code-pointer}
\item [{\bf integer}]  The  integers  are   also called  "fixed" numbers.
            The magnitude of integers is essentially unrestricted if
            the "big number" module, {\bf zbig}, is loaded. The notation for
            integers is  a  sequence  of digits in an appropriate
												radix (radix 10 is the default, which can be overridden
           	by a radi prefix, such as 2\#, 8\#, 16\# etc). There are
												three internal representations of integers, chosen  to
            suit the implementation:

\item [{\bf inum}] 	A signed number fitting into info.
                       Inums do not require dynamic storage
                       and are represented in the same form
                      as machine integers. 

\item [{\bf fixnum}] 	A full-word signed integer,
                       allocated in the heap.

\item [{\bf bignum}]   A signed integer of arbitrary precision,
			allocated as a vector of
                       system integers. These integers may be not
                       integers for PSL, because they do not fit into info.
			Beware. Bignums are currently not
                       installed  by  default, to use them
                       load the ZBIG module. 

\item [{\bf float}]    A floating point number, allocated in the heap.
			The precision of floats
			is determined solely by the implementation. 
			Usually, the floating numbers are equivalent
			to system 'doubles', (64 bits).
			The notation for a float is a sequence of digits
			with the addition of  a  single floating  point
			( . ) and an optional exponent
			(E \verb+<+integer\verb+>+).
			(No  spaces  may  occur between the point and
			the  digits). Radix 10 is used for representing
			the  mantissa  and  the  exponent  of floating
			point numbers.

\item [{\bf id}]       An identifier (or id) is an item whose info field
			points to a  five-item  structure  containing  the
			print  name,  property  cell, value cell, function
			cell, and  package  cell. This structure is contained
			in the id space. The notation for an id is  its 
			print  name,  an  alphanumeric  character sequence.
			One always refers to a particular id by giving its
			print name.   When  presented  with  an appropriate
			print name, the PSL reader will find a unique  id 
			to  associate with it.  See Chapters 4 and 12 for 
			more  information  on  ids  and  their syntax.
			The ids t and nil are considered special in that 
			it  is  not  possible  for  the  user  to redefine
			their value cells.

\item [{\bf pair}]     A  primitive two-item structure  which has a  left
			and  right part.  A notation  called  dot-notation
			is used, with the form:
			($<$left-part$>$   .  $<$right-part$>$). 
			The $<$left-part$>$ is  known  as the car  portion 
			and  the  $<$right-part$>$  as  the  cdr portion. 
			The parts may be any item.  (Spaces  are used  to
			resolve  ambiguity  with  floats;  see Chapter 12).

\item [{\bf vector }]  A primitive uniform structure of items; an integer
			index is used  to  access  random  values  in  the
			structure.    The  individual elements of a vector
			may be any item.  Access to vectors is by means of
			functions for indexing, sub-vector extraction  and
			concatenation,  defined  in  Section  7.1.  In the
			notation for vectors, the elements of a vector are
			surrounded by square brackets:
			[item-0 item-1 ... item-n]. 

\item [{\bf string}]   A  packed  vector  (or byte vector) of characters;
			the elements are small integers  representing the
			ASCII  codes  for  the characters (usually inums).
			The  elements  may  be   accessed   by   indexing,
			substring  and concatenation functions, defined in
			Chapter 6.  String notation consists of  a  series
			of  characters  enclosed  in  double quotes, as in
			"THIS IS A  STRING".    A  quote  is  included  by
			doubling  it, as in "HE SAID, ""LISP""".  A string
			may be input across  the  end  of  a  line  but  a
			warning   will   be   given   unless   the  switch
			{\bf  eolinstringok}\index{eolinstringok}
			\index{*eolinstringok} is non-nil (see Chapter 12).

\item [{\bf w-vector}] A vector of machine-sized words, used to implement
			such  things  as  fixnums,  bignums,  etc. The
			elements  are  not considered to be items, and are
			not examined by the garbage collector.

\item [{\bf byte-vector}]   A vector of bytes.  Internally  a  byte-vector  is
              the   same   as   a  string,  but  it  is  printed
              differently as a vector  of  integers  instead  of
              characters.

\item [{\bf code-pointer}] This  item  is used to refer to the entry point of
              compiled functions (exprs, fexprs, macros,  etc.),
              permitting   compiled  functions  to  be  renamed,
              passed around anonymously, etc.  New code-pointers
              are  created  by  the  loader  (lap,   fasl)   and
              associated  functions.   They can be printed;  the
              printing function prints the number  of  arguments
              expected  as  well  as the entry point.  The value
              appears in the convention  of  the  implementation
              (e.g.  \#$<$Code  A  N$>$,  where  A  is  the number of
              arguments and N is the entry point).
\end{Ventry}

\subsubsection{Other Notational Conventions}

Certain functional arguments can be any of a number of  types.
For  convenience,  we  give these commonly used sets a name.  We
refer to these sets as "classes" of primitive data  types.    In
addition  to the types described above and the names for classes
of types given below, we use the following  conventions  in  the
manual.   \{XXX, YYY\} indicates that either data type XXX or data
type YYY will do.  \{XXX\}-\{YYY\} indicates that any object of type
XXX can be used except those of type YYY; in this case, YYY is a
subset of XXX.  For example,  \{integer,  float\}  indicates  that
either an integer or a float is acceptable; \{any\}-\{vector\} means
any type except a vector.

\begin{Ventry}{\bf None Returned}
\item [{\bf any}]          Any  of  the  types  given  above. S-expression is
              another term for any.  All PSL entities have  some
              value unless an error occurs during evaluation.

\item [{\bf atom }]         The class {any}-{pair}.

\item [{\bf boolean}]       The class of global  variables {t, nil}, or  their
              respective    values,    {t,   nil}.  (See Section
              4.6).

\item [{\bf character}]  Integers in the range of 0 to 255 without 128
		representing ASCII character codes. 
		These are distinct from single-character ids.

\item [{\bf constant}]      The class  of  {integer,  float,  string,  vector,
              code-pointer}.    A  constant  evaluates to itself
              (see the definition of eval in Chapter 11).

\item [{\bf extra-boolean}] Any value  in the  system.    Anything  that    is
              not nil has the boolean interpretation t.

\item [{\bf ftype}]         The  set  of  ids (expr, fexpr, macro, and nexpr),
              which represent definable  function  types.    The
              ftype  is only an attribute of identifiers, and is
              not  associated  with   either   executable   code
              code-pointers or lambda expressions.

\item [{\bf io-channel}]    A  small  integer  representing an io channel (see
              Chapter  12   for   a   complete   discussion   of
              io-channels).

\item [{\bf number}]       The class of {integer, float}.

\item [{\bf x-vector}]      Any   kind  of  vector;  i.e.  a  string,  vector,
              w-vector, or word.

\item [{\bf Undefined}]     An implementation-dependent value returned by some
              low-level functions;  i.e.  the  user  should  not
              depend on this value.

\item [{\bf None Returned}] A  notational convenience used to indicate control
              functions that  do  not  return  directly  to  the
              calling  point,  and  hence  do not return a value
              (for example, see the function go in Chapter 8).
\end{Ventry}

\subsubsection{Structures}

 Structures are  entities  created  using  pairs.    Lists  are
structures  very  commonly  required as parameters to functions.
If a list of homogeneous entities is  required  by  a  function,
this class is denoted by xxx-list, in which xxx is the name of a
class  of  primitives  or  structures.  Thus a list of ids is an
id-list, a list of integers is an integer-list, and so on.

\begin{Ventry}{\bf function}
\item [{\bf list}]          A list   is  recursively   defined  as  
nil    or
              the  pair   (any . list).    A  special   notation
              called list-notation is used to  represent  lists.
              List-notation eliminates the extra parentheses and
              dots  required  by  dot-notation,  as  illustrated
              below.   List-notation  and  dot-notation  may  be
              mixed, as shown in the second example.
\end{Ventry}
              \begin{verbatim}
              DOT NOTATION                    LIST NOTATION
              (A . (B . (C . NIL)))             (A B C)
              ((A . (B)) . C)                   ((A B) . C)
              \end{verbatim}

              Note:  ()  is an alternate input representation of
              nil.

\begin{Ventry}{\bf function}
\item [{\bf a-list}]        An a-list, or association list, is a list in
which
              each element is a pair, the car part being  a  key
              associated with the value in the cdr part.

\item [{\bf form}]          A  form  is an S-expression (any) which is
legally
              acceptable to eval; that is, it  is  syntactically
              and  semantically  accepted  by the interpreter or
              the compiler. 

\item [{\bf lambda}]       A lambda expression must be of the following form,
              the square brackets are used to indicate  zero  or
              more occurances of an expression.

              \begin{verbatim}
              (LAMBDA <parameters> [<form>])
              \end{verbatim}
              The expression \verb+<parameters>+ is a list of ids which
              represents  the formal parameters or the body (the
              sequence of \verb+<form>+s). The evaluation of the body
              takes place as if the \verb+<form>+s were enclosed within
              a  progn.    

\item [{\bf function}]      A  lambda  expression  or  a   code-pointer,  
the
              function  type  is assumed to be expr.  This means
              that the arguments will be evaluated, and that the
              number of arguments must agree with the number  of
              parameters.
\end{Ventry}

\subsection{Predicates Useful with Data Types}

Most  functions  in  this  Section  return  t if the condition
defined is met and nil if it is  not.    Exceptions  are  noted.
Defined are type-checking functions and elementary comparisons.

\subsubsection{Functions for Testing Equality}

  Functions  for  testing  equality are listed below.  For other
functions comparing arithmetic values see Chapter 3.


\de{eq}{(eq U:any V:any): boolean}{ open-compiled expr}
{    Returns t if U points to  the same object as V, i.e. if they
    are identical items.    Eq  is  not  a  reliable  comparison
    between  numeric  arguments.    This function should only be
    used in special circumstances.  Normally, equality should be
    tested with equal, described below.
}

\de{eqn}{(eqn U:any V:any): boolean}{expr}
{    Returns t if U and V  are eq or if U  and V are numbers  and
    have the same value and type.
}

\de{equal}{(equal U:any V:any): boolean}{expr}
{    Returns  t  if  U  and  V  are  the  same.   A usually valid
    heuristic is that if two objects look the  same  if  printed
    with   the  function  print,  they  are  equal.    Equal  is
    open-compiled as eq if one argument is known to be an atom.
}
\begin{verbatim}
    (de equal (u v)
      (cond ((and (pairp u) (pairp v))
             (and (equal (car u) (car v))
                  (equal (cdr u) (cdr v))))
            ((and (stringp u) (stringp v))
             (string= u v))
            ((and (vectorp u) (vectorp v))
             (vector-equal u v))
            (t (eqn u v))))
\end{verbatim}
\begin{verbatim}
    1 lisp> (setq x '(lisa) y x)
    (LISA)
    2 lisp> (eq x y)
    T
    3 lisp> (eq x '(lisa))
    NIL
    4 lisp> (equal x '(lisa))
    T
    5 lisp> (eq 1.0 1.0)
    NIL
    6 lisp> (eqn 1.0 1.0)
    T
    7 lisp> (equal 0 0.0)
    NIL
\end{verbatim}
\de{neq}{(neq U:any V:any): boolean}{macro}
{    (not (equal U V)).  }

\de{ne}{(ne U:any V:any): boolean}{open-compiled expr}
{    (not (eq U V)).  }

\de{eqstr}{(eqstr U:any V:any): boolean }{expr}
{    Compare two strings, for exact  (case  sensitive)  equality.
    The  function  string-equal (which is defined in the STRINGS
    module), is not sensitive to case.  Eqstr returns t if U and
    V are eq or if U and V are equal strings.  }

\de{eqcar}{(eqcar U:any V:any): boolean}{expr}
{    Tests whether (eq (car U) V)).  If the first argument is not
    a pair, eqcar returns nil.  }

\subsubsection{Predicates for Testing the Type of an Object}

\de{atom}{(atom U:any): boolean}{open-compiled expr}
{    Returns t if U is not a pair.
}

\de{codep}{(codep U:any): boolean}{open-compiled expr}
{    Returns t if U is a code-pointer.
}
\de{constantp}{(constantp U:any): boolean}{expr}
{    Returns t if U is a constant (that is, neither a pair nor an
    id).  Note that vectors are considered constants.
}
\de{fixp}{(fixp U:any): boolean}{open-compiled expr}
{    Returns t if U is an  integer.    If  BIG  is  loaded,  this
    function also returns t for bignums.
}
\de{floatp}{(floatp U:any): boolean}{open-compiled expr}
{    Returns t if U is a float.
}
\de{idp}{(idp U:any): boolean}{open-compiled expr}
{    Returns t if U is an id.
}
\de{null}{(null U:any): boolean}{open-compiled expr}
{    Returns  t    if  U    is  nil.  This  is exactly  the  same
    function as  not, defined  in Section   2.2.3.   Both    are
    available  solely  to  increase  readability.
}
\de{numberp}{(numberp U:any): boolean}{open-compiled expr}
{    Returns t if U is a number (integer or float).
}
\de{pairp}{(pairp U:any): boolean}{open-compiled expr}
{    Returns t if U is a pair.
}
\de{stringp}{(stringp U:any): boolean}{open-compiled expr}
{    Returns t if U is a string.
}
\de{vectorp}{(vectorp U:any): boolean}{open-compiled expr}
{    Returns t if U is a vector.
}
\subsubsection{Boolean Functions}

Boolean  functions  return  nil for false; anything non-nil is
taken to be true, although a conventional  way  of  representing
truth  is  as  t.   Note  that t always evaluates to itself, its
value cannot be redefined.  Nil may also be represented  as  ().
As  a  matter  of  style, () should be used to refer to an empty
list.  The Boolean functions and, or, and not can be applied  to
an  object  of any type.  And and or may also be used as control
structures (see Section 8.2 for more information).

\noindent
Since  PSL  treats  any  value   which   is   non-nil   as   a
representation  for  true, there is no clear distinction between
an arbitrary function and a  boolean  function.    However,  the
three  functions  presented  here  are by far the most useful in
constructing more complex tests from simple predicates.


\de{not}{(not U:any): boolean}{open-compiled expr}
{    Returns t  if U  is  nil.  This   is  exactly    the    same
    function  as  null,  defined  in  Section  2.2.2.   Both are
    available solely to increase readability.
}

\de{and}{(and [U:form]): extra-boolean}{open-compiled fexpr}
{    And evaluates each U until a value of nil is  found  or  the
    end  of  the    list is encountered.  If  a non-nil value is
    the last value, it is returned; otherwise nil  is  returned.
    Note  that and called with zero arguments returns t.  In the
    example which follows  and  is  used  to  select  the  first
    element of a list, if the call on pairp returns nil then car
    will not be applied.
}
\begin{verbatim}
    1 lisp> ((lambda (p) (and (pairp p) (car p)))  '(robin))
    robin
\end{verbatim}
\de{or}{(or [U:form]): extra-boolean}{open-compiled fexpr}
{    U is any number of expressions  which are evaluated in order
    of  their  appearance.  If one is found to be non-nil, it is
    returned as the value of  or.    If  all  are  nil,  nil  is
    returned.  Note that if or is called with zero arguments, it
    returns nil.  The following function defines a predicate for
    numbers.
}
\begin{verbatim}
    (de number-p (n)
      (or (fixp n) (floatp n)))
\end{verbatim}
\subsection{Converting Data Types}

The following functions are used in converting data items from
one  type  to  another.   They are grouped according to the type
returned.  Numeric types may be converted using  functions  such
as fix and float, described in Section 3.2.


\de{intern}{(intern U:{id,string}): id}{expr}
{    Returns an identifier from the symbol table (also called the
    id-hash-table).   When  the  PSL  reader reads a sequence of
    characters which notate an id, it will apply intern  to  the
    string of characters.  Therefore, it generally does not make
    sense  to  apply  intern  to  an id.  Intern will search the
    symbol table for an id whose print name matches U.   If  the
    search  is  successful  then  the  matching  id is returned.
    Otherwise a new id will be entered into the symbol table and
    a reference to it will be returned.  If U has more than  the
    maximum    number    of    characters   permitted   by   the
    implementation, an error will be signalled.
}
\begin{verbatim}
    ***** Too many characters to INTERN
\end{verbatim}
    The id which is returned from an application of intern to  a
    string  will  have  the  string  as  its  print  name.  Most
    identifiers have lowercase print names (even though you  may
    type  in  lower case letters), but interning "ABC" yields an
    id with a lower case print name.

\begin{verbatim}
    1 lisp> (eq (intern "abc") 'abc)
    NIL
\end{verbatim}
    The maximum number of characters in any token is system dependent,
around 5000 can be expected to be allowed.


\de{newid}{(newid S:string): id}{expr}
{    Allocates a new identifier and sets its print  name  to  the
    string  S.   The identifier is not added to the symbol table
    (an identifier which does not appear in the symbol table  is
    said to be uninterned).  The string is not copied.
}
\begin{verbatim}
    1 lisp> (setq new (newid "NEWONE"))
    NEWONE
    2 lisp> (eq new 'newone)
    nil
\end{verbatim}
    If  one  refers  directly  to  an  identifier  (for  example
    'newone), the reader will apply  intern  to  the  string  of
    characters  it  has  read  ("NEWONE").   In the example, the
    identifier created by the call on newid  is  different  from
    the one created by the reader when it read 'newone.

\de{int2id}{(int2id I:integer): id}{expr}
{    Converts  an integer to an id; this refers to the I'th id in
    the id space. Since  0  ...  255  correspond  to  ASCII
    characters,  int2id  with an argument in this range converts
    an ASCII code to the corresponding single character id.
    The id NIL is always found by {\tt (int2id 128)}.  }

\de{id2int}{(id2int D:id): integer}{expr}
{    Returns the id space position of D as a LISP integer.  }

\de{id2string}{(id2string D:id): string}{expr}
{    Get name from id space.  Id2string returns the print name of
    its  argument  as  a  string.    This  is  not  a  copy,  so
    destructive  operations  should  not  be  performed  on  the
    result.    PSL  uses  an  escape  convention  for   notating
    identifiers which contain special characters.  Any character
    which  follows  the  character  !  is  considered  to  be an
    alphabetic character.   In  the  example,  notice  that  the
    character ! does not appear in the result.  }
\begin{verbatim}
    1 lisp> (id2string 'is-!%)
    "is-%"
\end{verbatim}
\de{string2list}{(string2list S:string): inum-list}{expr}
{    Creates  a  list  of length  (add1 (size S)), converting the
    ASCII characters into small integers.
}
\begin{verbatim}
    1 lisp> (string2list "STRING")
    (83 84 82 73 78 71)
\end{verbatim}

\de{list2string}{(list2string L:inum-list): string}{expr}
{    Allocates a string of the same size as L, and converts small
    integers into characters according to their ASCII code.   An
    integer  outside  the  range  of 0 ... 127 will result in an
    error.
}
\begin{verbatim}
***** An attempt was made to do LISP2CHAR on `N' which is not a character

    1 lisp> (list2string '(83 84 82 73 78 71))
    "STRING"
\end{verbatim}
\de{string}{(string [I:inum]): string}{nexpr}
{    Creates  and  returns  a string containing each of the small
    integers
}
\begin{verbatim}
    1 lisp> (string 83 84 82 73 78 71)
    "STRING"
\end{verbatim}
\de{vector}{(vector [U:any]): vector}{nexpr}
{    Creates and returns a vector containing all the Us given.
}
\begin{verbatim}
    1 lisp> (vector 83 84 82 73 78 71)
    [83 84 82 73 78 71]
\end{verbatim}
\de{vector2string}{(vector2string V:vector): string}{expr}
{    Pack the small integers in the vector into a string  of  the
    same  size,  using the integers as ASCII values.  An integer
    outside the range of 0 ... 255 will result in an error.
}
\begin{verbatim}
    1 lisp> (vector2string [83 84 82 73 78 71])
    "STRING"
\end{verbatim}
\de{string2vector}{(string2vector S:string): vector}{expr}
{    Unpack the string into a vector  of  the  same  size.    The
    elements  of the vector are small integers, representing the
    ASCII values of the characters in S.
}
\begin{verbatim}
    1 lisp> (string2vector "VECTOR")
    [V E C T O R]
\end{verbatim}
\de{vector2list}{(vector2list V:vector): list}{expr}
{    Create a list of the same size as V, the elements are copied
    in a left to right order.
}
\begin{verbatim}
    1 lisp> (vector2list [L I S T])
    (L I S T)
\end{verbatim}
\de{list2vector}{(list2vector L:list): vector}{expr}
{    Copy the elements of the list into  a  vector  of  the  same
    size.
}
\begin{verbatim}
    1 lisp> (list2vector '(V E C T O R))
    [V E C T O R]
\end{verbatim}
