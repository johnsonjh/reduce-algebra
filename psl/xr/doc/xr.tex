\documentstyle[11pt,epsfig,reduce,twoside]{article}
\makeatletter

\define@nomathsize{70}
\define@nomathsize{40}

\makeatother

\def\xr{{\tt XR}}
\def\keybox#1{\raisebox{1.5pt}{\footnotesize\fbox{\vphantom{E}\smash{\sf #1}}}}

\renewcommand{\encodingdefault}{T1}
\renewcommand{\familydefault}{ptm}
\renewcommand{\rmdefault}{ptm}
\renewcommand{\sfdefault}{phv}
\renewcommand{\ttdefault}{pcr}
\renewcommand{\bfdefault}{b}

\setlength{\oddsidemargin}{5mm}
\setlength{\evensidemargin}{-5mm}
\setlength{\textwidth}{159.2mm}
\setlength{\textheight}{235mm}
\addtolength{\topmargin}{-18mm}

\title{{\tt XR}: A User Interface for Reduce under X}
\date{}
\author{Chris Cannam, ZIB Berlin; 1992 -- 93}
\index{Cannam, Chris}      % Haha!  Today the index, tomorrow ... the index
\sloppy
\begin{document}
\pagestyle{empty}
\begin{titlepage}
  \vspace*{\fill}
  \nopagebreak
  \begin{center}
    \fontsize{70}{75pt}\selectfont
    REDUCE \\
    \fontsize{40}{65pt}\selectfont
    XR: User Interface\\
    under X\\[40mm]
    \LARGE
    Chris Cannam\\[45mm]
    \mbox{\epsfig{file=ziblogo.eps,width=21mm}}\\[10mm]
    \Large
    Konrad-Zuse-Zentrum Berlin 1996
  \end{center}
\end{titlepage}

\pagestyle{headings}

\maketitle
\section{Introduction}

\xr\ is a program which interacts with \REDUCE\ to provide a
keyboard-and-mouse operated graphical user interface under UNIX and
the X windowing system.  It requires the presence of \REDUCE{}, and
X11 Release 4 or 5 with the Athena widget set.

\xr\ provides:
\begin{itemize}
\item high-resolution graphical display of formulae;
\item command history recall and editing with bracket matching;
\item file logging and management;
\item printer support in PostScript and ASCII, and \LaTeX\ transcription;
\item a vaguely hypertexty help facility;
\item dynamically user-selectable fonts.
\end{itemize}

\xr\ does {\it not\/} provide:
\begin{itemize}
\item graph-drawing capabilities; an interface to graphics is provided
by other packages, e.g. the gnuplot package which can of course be used
when the \REDUCE\ session is controlled by \xr\ .
\item anything based on the interface styles of Motif, OpenLook or any
other proprietary system.  \xr\ will of course run happily in Motif or
OpenLook environments, but it makes no use of any particular features
of theirs.
\end{itemize}

\newpage
\section{Using \xr}

Provided \xr\ and its associated X defaults file have been correctly
installed, you should be able to run it by executing the file
{\tt\$reduce/xr/bin/xr}.

This is a long command to type; if your shell supports command
aliases, it may be provident to create one, with a command like:
\begin{description}

\item[] {\tt alias xr \$reduce/xr/bin/xr}\hskip 1pc (C shell and derivatives);
\item[] {\tt alias xr=\$reduce/xr/bin/xr}\hskip 1pc (derivatives of
the Bourne shell).
\end{description}

\subsection{The Main \xr\ Window Box}

\unitlength=1cm
\begin{picture}(12,12)(0,0)
\put(1,2){\epsfbox{xr-main.ps}}
\put(3.8,3.05){\line(-1,-1){1.45}}
\put(4.35,2.35){\line(-1,-6){0.24}}
\put(6.6,5.7){\line(1,-5){0.9}}
\put(1.75,9.5){\line(-1,6){0.19}}
\put(7.45,9.3){\line(1,6){0.185}}
\put(8.65,8.45){\line(1,2){1.3}}
\put(0.52,1.146){\footnotesize Input History Window}
\put(3.42,0.426){\footnotesize Input Pane}
\put(6.73,0.78){\footnotesize Question Box}
\put(0.83,10.85){\footnotesize Button Box}
\put(6.98,10.65){\footnotesize Title Bar}
\put(9.17,11.3){\footnotesize Output Pane}
\end{picture}

The \xr\ window contains four sections. From the top: the title bar,
with buttons; the output history window; the input history window; and
the input window.  The input history window is read-only, and \xr\
will beep if you try to type into it.

\subsubsection{The Title Bar}

The title bar has five buttons, as follows.
\begin{description}

\item[Quit] The Quit button is used to end the \xr\ session.  When
pressed ({\it ie\/}.~a mouse button is pressed when the mouse pointer
is over the Quit button's box), it pops up a small confirmation
question box to check your intention, and if confirmed, \xr\ quits.
Note that question boxes are described in detail in section
\ref{questionboxes}.

\item[Interrupt] This button sends an Interrupt signal to the \REDUCE\
process.  The effect is the same as that of pressing
\keybox{Ctrl}~\keybox{C} in the Input window.

\item[Help] Pressing this button invokes the \xr\ built-in info-based
Help system.  This is described in detail in section \ref{help}.

\item[Options] This button calls up the Option box, which provides the
ability to tailor certain aspects of \xr{}'s behaviour as it runs.
This is described in section \ref{options}.

\item[File] This button calls up the File box, which provides various
possibilities for reading from and writing to files, and for printing
output.  This is described in section \ref{file}.

\end{description}

\subsubsection{The Output History Window}

This window contains a copy of each input line from you to \REDUCE{},
with the associated (normally pretty-printed) output.  It is
scrollable --- a scrollbar appears on the left when the window fills
--- and maintains a certain number of lines of output off the top of
the window; it should always retain at least the last two of
\REDUCE{}'s responses, however long they may be.

This window is read-only, and if you type whilst the mouse pointer is
over it, your typing will be redirected to the input window.  You
can't use the X paste mechanism to paste into this window, although
you can use it to copy from the textual bits of it into the input
window if you wish.

\subsubsection{The Input History Window}

This displays all the previous lines you have entered to \REDUCE{}, in
a scrollable read-only copy-and-pasteable text window, with one extra
feature: you can position its text cursor by clicking with the first
mouse button within the window, and then by pressing \keybox{Return},
subsequent lines from the History window are pasted into the Input
window below it.

\subsubsection{The Input Window}

The input window is the window you actually type into.  You compose
your commands for \REDUCE\ in it, using the normal editing keys, and
when you press \keybox{Return}, the contents of the window are sent to
be processed by \REDUCE{}.  There are a few useful features here:

\begin{description}

\item[History Editing] When you press the up or down cursor key, if it
is not possible to move the cursor caret any further up or down in the
current window, the window will instead be cleared and its contents
replaced by the previous or next line in the input history window.
The contents may be restored (before any further changes are made) by
a press on the opposing cursor movement key.

\item[Bracket Matching]\label{bracketmatching} \xr\ can check that
closing brackets match an opening bracket somewhere for you; it can
even prevent you from typing unmatched brackets.  These abilities are
controlled by the {\tt matchBrackets} and {\tt forceBracketMatch} X
Resources; the defaults are True and False, respectively.  If matching
is enabled, every bracket you type will cause either its matching
bracket to flash, or the terminal bell to ring if there is no match.
Matching is carried out for round, square and curly brackets, and for
the \REDUCE\ \verb+<<+\ldots\verb+>>+ and {\tt begin}\ldots{\tt end}
pairs.  If force matching is enabled, \xr\ will render the
close-bracket keys ineffectual unless there is a matching open
bracket; this applies to all of the bracket styles except {\tt
begin}\ldots{\tt end}.

\item[Bracket Checking and Completion] Moving the cursor to the right
of a closing bracket in the input window will cause its matching
opening bracket to flash; pressing Ctrl-TAB when the cursor is on a
line containing unmatched opening brackets will cause enough closing
brackets of the correct types to be inserted to match.  This only
matches round, square and curly brackets, not compound delimiters.
\end{description}

\subsubsection{Question Boxes}\label{questionboxes}

When \xr , or \REDUCE\ via \xr , needs to know something for which a
yes or no answer is possible (such as ``Are you sure you want to quit
this \REDUCE\ session?'', or ``Declare {\tt ocdbwmxz} operator?''), a
Question box is popped up to ask you about it.  These are small boxes
containing only the question and two buttons, labelled Yes and No.
When one appears, your mouse pointer will immediately be warped into
it, probably landing on either Yes or No, whichever is the default for
the particular question\footnote{This only works with a window manager
that supports pointer warping.}.  You can take the following actions:

\begin{itemize}
\item Click on one of the Yes or No buttons;
\item Press \keybox{y} or \keybox{n}, whilst the pointer is within the
question box;
\item Press \keybox{Ctrl}~\keybox{C}, which abandons the question and
sends an Interrupt signal to Reduce;
\item Press \keybox{Escape}, which is equivalent to clicking No;
\item Press \keybox{Return}, which will select whichever box (Yes or No) the
pointer is currently over, or be ignored if it is not over either.
\end{itemize}

\subsection{The Help Window}\label{help}

\unitlength=1cm
\begin{picture}(20,14)(0,0)
\put(2.5,0.6){\epsfbox{xr-hmen.ps}}
\end{picture}

Pressing the Help button on the title bar calls up the Help selection
window. This windows contains button for each item in the REDUCE doc
directory. Pressing the buttons starts up the appropriate display
utility which is the program {\tt more} for plain text files and an own
help system which is based on the X info format. The usage of this
xr feature is described in the rest of this chapter.
The contents of the REDUCE doc directory and the type of available
information is described in the file \$reduce/help/helpdir.info.

\unitlength=1cm
\begin{picture}(12,10)(0,0)
\put(2.5,0.6){\epsfbox{xr-help.ps}}
\end{picture}

This window is divided into a button-and-label section, at the top,
and two textual windows.  The upper textual window contains a help
`page' normally covering a single topic, and the lower contains a
`menu' of sub-pages (`children').  If there is no text in the current
page, the upper is empty, and likewise if there are no children the
lower is empty.

The help system is arranged as a tree-hierarchical set of pages; a
page may have a `parent' page, some child pages and some siblings.
The only parentless page is known as `Top', and contains the main
contents menu; pages with children show their children as a list in
the menu window; and pages with siblings may have `Previous' and
`Next' pages.

There are several ways of moving around between help pages:

\begin{description}
\item[Cross References] Occasionally the text in the textual help page
window may contain words (other than titles) displayed in {\bf bold
type}.  These indicate cross-references, {\it ie\/}.~words which refer
to other help pages.  These are highlighted as the mouse pointer
passes over them, and they may be clicked upon in order to transport
you directly to the page to which they refer.

\item[The Prev, Up and Next buttons] These buttons, when pressed, flip
the current page to the previous sibling page, parent page or next
sibling page respectively.  If there is no page in the given direction
to which to flip, a beep will sound instead.  Note that if you are at
the Top node, the `Up' button reads not `Up: (none)', as you might
expect, but `Up: (dir)'.  This is for historical reasons\footnote{The
actual reason is because the file format is closely based on the GNU
project's Info format, which provides for an above-Top-level directory
of Info files.}.

\item[The Menu] If there is a menu in the bottom (menu) window, each
item in it should be listed with the name of its page in {\bf bold
type} and a short description in normal type.  Clicking the mouse
button on one of the {\bf bold} page names will flip to that page.

\item[Searching] Provided that the Help Index file is available, which
depends on the installation, there should be a Search button at the
top of the window.  When clicked, this takes you to a Search page,
designed to help you find pages relevant to subjects of interest.  On
the Search page you may enter a simple string consisting of one or
more words for which to search (for example, `matrix', or `taylor
series'), and click on the `Execute Search' button; the results of the
search will then be listed in the menu window.  This list is of all
pages which were considered as possibilities, in an approximate order
of likelihood.  Clicking on a name will then take you to that page.

When the Search page is active, the Search button is relabelled
`Return to Help Page', and clicking on it returns you to the page you
were last looking at.

\item[The Show Route button] This button causes a list to be displayed
of all the pages you have looked at so far, in order, with a count of
the number of times each has been visited.  The list is displayed as a
menu, and can be clicked to transport you directly to an earlier page.
When it is shown, the Show Route button is relabelled `Show Current
Page', and when clicked to returns you to the last page you were
looking at.
\end{description}

The other buttons and labels in the top section of the Help window are
the Exit Help button, which removes the Help window, and an
information label telling you which page you are looking at.

In addition, there is a provisional contextual help facility.  Hitting
the contextual help button (by default the key~\keybox{F10}, though it
can be bound to a different key in your resource file) while the
pointer is in an input or output window will cause the Help box to pop
up and a search to be initiated in an attempt to find any pages
relevant to anything you have been typing recently.

The best way to get to know the Help system is to play around with it
a bit and find out what goes where -- it's very intuitive once you
have the general idea.

\subsection{The Option Box}\label{options}

\unitlength=1cm
\begin{picture}(12,10)(0,0)
\put(0.7,2.5){\fbox{\epsfbox{xr-optn.ps}}}
\put(6.8,3.1){\fbox{\epsfbox{xr-file.ps}}}
\put(2.3,1.8){\footnotesize Option Box}
\put(8.5,2.4){\footnotesize File Box}
\end{picture}

This is called up by pressing the Options button on the title bar.  It
may be left up throughout the \xr\ session, or dismissed at any time
by pressing its Done button.  The Option box contains the following
options, whose values may be changed by clicking on the little boxes
in which they are displayed:

\begin{description}

\item[History Length] This controls the number of lines of input and
output which are saved in the output history window's scrollable
area.  The default is 100; using too large values may slow \xr\ down.

\item[Page Mode] If enabled, \xr\ will pause every time a whole output
window's worth of output has appeared since the last pause or since
the last prompt.  Its assessment of ``a whole page'' is not
particularly accurate, but the feature can be useful if you want to
read something like a long demo file; it may be of more use than the
conventional \REDUCE\ {\tt DEMO} switch.  When the output is paused,
the cursor will show a pointing hand whenever it is over the input
window; then clicking on the first mouse button or pressing
\keybox{Return} will free the page mode for another page.

\item[Use Greek Font] If your X server is not equipped with Greek
fonts, you have two options: you can just choose a different font for
Greek characters (such as an Italic font, in which case for example
$\pi$ will display as {\it p\/}), or you can turn off Greek printing
using this option, in which case Greek characters' real names will be
printed instead (\`a la terminal \REDUCE{}).

\item[Reverse Video] This causes graphical formulaic output to be
displayed in reverse video\footnote{Certain pathological X servers may
require Reverse Video to be selected in order to display correctly in
forward video.}.

\item[Graphics] This controls whether or not \xr\ displays proper
fancy-output graphics, or merely ASCII \REDUCE{}-style
prettyprinting.

\item[Match Brackets {\rm and} Force Bracket Matching] These options
control the bracket-matching capabilities which were described in
section \ref{bracketmatching}.

\item[Font Selection] This calls up the Font Selector, which enables
dynamic user font selection.  This is described in detail in section
\ref{fonts}.

\end{description}

\subsubsection{The Font Selector}\label{fonts}

\unitlength=1cm
\begin{picture}(12,8.5)(0,0)
\put(2.3,0.2){\fbox{\epsfbox{xr-font.ps}}}
\end{picture}

The Font Selector window is a smallish square window which provides
the ability for you to tailor the appearance of \xr{}, and to save the
results to the default file for subsequent \xr\ sessions.  It is
divided up as follows:

\begin{description}

\item[Purpose List] This box, in the top-left quadrant of the Font
Selector, lists all the possible purposes for which you can choose
fonts.  Clicking with the mouse pointer over a list item allows you
then to choose the font for that purpose; double-clicking on it will
cause the currently chosen font to be displayed in the font display
window (for which see below), and also allow you to select the font
for that purpose.

\item[Family List] This, in the top-right quadrant, contains a
scrollable list of the various families of fonts which your server
has.  Selecting one of the list items (by clicking when the mouse
pointer is over it) will show that family of fonts in the display
window, allowing you to choose from~it.

\item[Font Display Window] This is the larger box taking up most of
the lower half of the selector.  It lists all the fonts in the
selected family by name; each name is displayed in that font.  In
order to choose one of these fonts for your currently selected
purpose, you simply click on it with the mouse pointer, and click on
the Apply button (see below).

\item[Any Button{\rm , } Bold Button {\rm and }Italic Button] These
select which fonts in the current family to display.  If the Any
button is pressed, all fonts will be shown (this is the default);
pressing the Bold and Italic buttons will restrict the display to bold
or italic fonts.  If none of these buttons are pressed, only fonts
which are neither bold nor italic will be displayed.

\item[Scale Button] If the current family contains some scalable
fonts, this button will be active, and if you press it, you will be
asked for a point size, to which the scalable fonts in the family will
then be scaled.

\item[Show Button] This controls how the displayed fonts are named; it
toggles between names of the form (for instance) ``Adobe 14pt Times
Bold'' and the default names, which are generally of a form something
like ``{\tt -adobe-times-bold-r-normal--14-140-75-75-p-77-iso8859-1}''.

\item[Apply Button] When a purpose and a font are both highlighted, in
the Purpose List and Font Display window respectively, a click on
Apply will choose that font for that purpose.

\item[Reset Button] This will reset the font for the current purpose
to that originally chosen for it in the \xr\ default file.

\item[Propagate Button] This is a cunning bit.  If you have chosen one
or two fonts, and you don't want to bother choosing the rest, simply
hit Propagate and \xr\ will try to choose the other fonts to match.  A
window will pop up listing the chosen fonts and you can choose whether
or not you want to use them.

\item[Done Button] This closes the Font Selector.  If you have changed
the choice of fonts, you will be given the opportunity to save the new
selection in your (personal, not system-wide) X defaults file so that
they will be used automatically next time you use \xr{}.

\end{description}

\subsection{The File Box}\label{file}

This is invoked by pressing the File button on the title bar, and
contains various options for controlling the input and output of
\REDUCE.  Its options are activated in the same way as those of the
Option box.  Most of them, when activated, pop up a File Selection
box, as described in section~\ref{fileselect} to allow the choice of a
file on which to operate.  The options are as follows:

\begin{description}

\item[Load System Package] This allows you to choose one of the
supplied \REDUCE\ packages to be loaded into the \REDUCE\ session.

\item[Load User Package] An option to load a pre-compiled \REDUCE\
package of your own devising (produced by {\tt faslout}).  It's
probable that the casual user will have no particular need for this.

\item[Read File Noisily] This reads a batch file of \REDUCE\ commands
in to the current session, with each command in the file being echoed
when read\footnote{Sadly, any display resulting from reading in a file
using either of the File Box options cannot be printed out using the
normal Print option.  This is because reading in a file has
(technically) no result, and hence its result cannot be printed.  If
you must print output obtained from reading a file, you might consider
cutting and pasting the file into the \xr\ input window using the
standard X mechanisms, instead of using a Read option.}.

\item[Read File Quietly] This reads a batch file in the same manner,
except that commands are not echoed as they are read.  Their results
are still displayed.

\item[Log to File] This controls the capability of \xr\ to send a copy
of the session to a log file as well as to the output window.  If it
is enabled, subsequent interactions with \REDUCE\ will be sent to the
log file as well as to the \xr\ window.  Logging stops as soon as it
is disabled again. The log file contains {\small REDUCE} output
in its internal encoding. There is a tool $log\_latex$
in the module $logutil$ for translating such a dribble file into a true
\LaTeX\ file. E.g. for translating a dribble file $reduce.log$ to
a \LaTeX\ file $log.tex$ type
\begin{verbatim}
   load_package logutil;
   log_latex("reduce.log","log.tex");
\end{verbatim}

\item[Send to Printer] This calls up the Printer Box, which allows you
to send a copy of some of the interactions with \REDUCE{}, specified
by begin and end prompt numbers, to a printer or file.  This is
described in section~\ref{printer}.

\end{description}

\subsubsection{File Selection Boxes}\label{fileselect}

\unitlength=1cm
\begin{picture}(12,8.2)(0,0)
\put(2.25,0.2){\fbox{\epsfbox{xr-fsel.ps}}}
\end{picture}

A File Selection Box is a box popped up by several of the File
options, and also sometimes by the Font Selector, to allow you to
choose a file to read from or write to.  It is divided into several
areas: labels indicating why the file is needed, which directory of
files is currently being inspected, and which file is selected; a list
of file names; and several buttons.

The list of files shows each file name in the current directory, with
a terminating character (in the style of the UNIX command~{\tt ls~-F})
to indicate the type of the file: ``{\tt /}'' for directories, which
can be `entered' by clicking on them; ``{\tt *}'' for executable
files; ``{\tt =}'' for named sockets.  Other files have no special
terminating character.  Clicking on any file's name will, unless that
file is a directory, select the file.

The File Selector's buttons are as follows:

\begin{description}

\item[Up Button] This opens and displays the parent directory of the
current one; for instance, if the current directory is ``{\tt
/usr/local/bin}'', this changes the display to list instead the
directory ``{\tt /usr/local}''.  If the current directory is at the
root of the directory tree, this button will be disabled.

\item[Rescan Button] This causes the current directory to be reread,
so as to update it if it has for some reason changed since it was
last opened.

\item[Show Button] The Show button toggles the displaying convention
of the file listing.  By default the listing shows only those files
neither beginning with the character~``{\tt .}'' nor ending
with~``\verb+~+'', ``{\tt \#}'' or ``{\tt \%}'' (thus excluding many
forms of backup or system files).  If the Show button is then pressed,
such files will be included in the list, and the button will be
relabelled `Hide'; pressing it again reverses the action.

\item[Apply Button] This tells \xr\ that you are happy with the
selection of the file; the file is then (where possible) opened and
used for whichever purpose the selector box had been called.

\item[Cancel Button] This cancels the operation for which the file
name was required.

\end{description}

If the file is to be written to, the label showing the currently
selected filename will be editable, in order to allow the creation of
a new unlisted file; if it is solely for reading, only files actually
listed in the directory may be selected.  For some purposes, normally
those for which it would not be advisable to use an existing file
(because the contents of the old file would be lost), a default name,
unique within the directory, will be chosen at random.

\subsubsection{The Printer Box}\label{printer}

\unitlength=1cm
\begin{picture}(13,6.3)(0,0)
\put(1.4,0.2){\fbox{\epsfbox{xr-prnt.ps}}}
\end{picture}

The Printer box is a window which is popped up when the Send To
Printer option on the Option Box is selected.  It contains an
information label, four input text labels and several option buttons,
as follows:

\begin{description}

\item[Printer Name] In this box you can enter the name of a printer to
which you wish your transcription to be sent.  The default is to use
no name, and the actual printing is always carried out by sending to
the standard input of the UNIX {\tt lpr} command.  If a name is
specified, print will be sent to the printer of that name, using {\tt
lpr}'s standard {\tt -P} option.

\item[File Name] This is for the name of the file to send
transcription to, should you choose file output instead of printer.
The default is chosen uniquely each time the printer box is invoked;
if an existing file is requested, it will be overwritten.  Note that
the Printer Name and File Name boxes can only be typed into when the
Printer and File option buttons, respectively, are selected.

\item[Begin Prompt Number {\rm and} End Prompt Number] These are for
choosing which section of the \xr\ session you want to print.  The end
number must be at least as large as the begin number, and neither may
exceed the current prompt number displayed on the Output window.  If
they are equal, then exactly one input and output will be printed.
The default is for both to contain the number of the previous prompt.
Note that the prompts you select do {\it not\/} have to be still
visible or available on the \xr\ output window.

\item[Plain Text button{\rm , } PostScript button {\rm and} LaTeX
button] These three buttons allow you to choose what format should be
used for the output.  Plain Text transcribes the session as it would
be seen on an ASCII terminal \REDUCE{}.  PostScript transcribes in the
Adobe PostScript language, suitable for sending directly to many
printers and which may be processed by some desktop publishing
utilities.  LaTeX produces a complete \LaTeX\ document; you can only
send \LaTeX{} to a file, not directly to the printer.

\item[Printer button {\rm and} File button] These select where to send
the output.

\item[A4{\rm , } Letter{\rm , } Portrait {\rm and} Landscape buttons]
These are only applicable with PostScript output; they select the size
and orientation of the output page.

\end{description}

There are also OK and Cancel buttons, causing the output to be printed
or thrown away respectively.

\newpage
\section{Resources and Options}

\xr\ recognises a swathe of X resources and command-line options,
which are documented below.  The defaults specified in the tables are
those hardcoded into the \xr\ source; several of them are overridden
by specifications in the distribution {\tt XR} resource file.

\subsection{File Resources}

\vskip 6pt
{\footnotesize\tt
\begin{tabular}{l p{0.8in} l p{1.7in}}
{\it Resource} &{\it Option} &{\it Default} &{\it Description}\\
&&&\\

reducePath &-rpath &\$reduce/psl &{\rm
Path of directory containing \REDUCE{} executable}\\

reduceName &-rname &bpsl &{\rm
Name of \REDUCE{} executable file within {\tt reducePath} directory}\\

reduceImageFile &-rimage &\$reduce/reduce.img &{\rm
Name of image file to feed to \REDUCE{} via {\tt -f} option}\\

reduceInitFile &-rinitfile &\$reduce/xr/bin/xr.red &{\rm
Name of initialisation file to feed to \REDUCE{} at \xr\ startup}\\

reduceHelpFile &-rhelpfile &\$reduce/xr/doc/rhelp.info &{\rm
Name of Help Info file used by \xr\ Help module}\\

reduceOptions &{\rm (none)} &-td 12000000 &{\rm
Command option string to pass to \REDUCE\ process}\\

logToFile &-log, -nolog &False &{\rm
Whether to send transaction to log file as well as screen}\\

logFile &-logfile &/dev/null &{\rm
File to which to send logging information}
\end{tabular}
}

\subsection{Dimensional Resources}

\vskip 6pt
{\footnotesize\tt
\begin{tabular}{l p{0.8in} l p{2.5in}}
{\it Resource} &{\it Option} &{\it Default} &{\it Description}\\
&&&\\

widthInColumns &-linelength &80 &{\rm
Number of characters per line of textual output}\\

graphicalLineSpace &-graphspace &10 &{\rm
Pixels of space between graphical formula output lines}\\

reduceHistoryMax &-history &100 &{\rm
Number of lines of past output saved in output history window}\\

reduceHistoryExcess &-hexcess &50 &{\rm
Number of slack lines allowed beyond history limit before \xr\ notices}\\

historyPaneHeight &-oheight &400 &{\rm
Preferred pixel height of output history window}\\

inputPaneHeight &-iheight &50 &{\rm
Preferred pixel height of input pane window}\\

fontSelectorHeight &-fheight &400 &{\rm
Pixel height of font selector box}\\

fontSelectorWidth &-fwidth &400 &{\rm
Pixel width of font selector box}
\end{tabular}
}


\vbox{
\subsection{Font Resources}

\vskip 6pt
{\footnotesize\tt
\begin{tabular}{l p{1in} l p{2.5in}}
{\it Resource} &{\it Option} &{\it Default} &{\it Description}\\
&&&\\

optionButtonFont &-buttonfn &{\rm (*)} &{\rm
Font for use on option buttons}\\

inputPaneFont &-inputpanefn &{\rm (*)} &{\rm
Font for use in the input pane window}\\

inputHistoryFont &-inputhistfn &{\rm (*)} &{\rm
Font for use in the input history window}\\

outputTextFont &-outputtextfn &{\rm (*)} &{\rm
Font for display of textual output}\\

normalAsciiFont &-asciifn &{\rm (*)} &{\rm
Font for normal-sized ASCII chars in graphical formula output}\\

normalGreekFont &-greekfn &{\rm (*)} &{\rm
Font for normal-sized Greek chars in formulae}\\

normalSymbolFont &-symbolfn &{\rm (*)} &{\rm
Font for normal-sized mathematical symbols in formulae}\\

smallAsciiFont &-smallasciifn &{\rm (*)} &{\rm
Font for exponent-sized ASCII chars in formulae}\\

smallGreekFont &-smallgreekfn &{\rm (*)} &{\rm
Font for exponent-sized Greek chars in formulae}\\

smallSymbolFont &-smallsymfn &{\rm (*)} &{\rm
Font for exponent-sized mathematical symbols in formulae}\\

verbatimFont &-verbfn &{\rm (*)} &{\rm
Font to be used for fixed-width display in Help and output windows}\\

helpTextFont &-helptextfn &{\rm (*)} &{\rm
Font for display of normal text in Help window}\\

helpXrefFont &-helpxreffn &{\rm (*)} &{\rm
Font for display of highlighted cross-references in Help window}\\

helpTitleFont &-helptitlefn &{\rm (*)} &{\rm
Font for titles in Help window}\\

psTextFont &-pstextfn &Courier &{\rm
Font for text and commands in PostScript printer output}\\

psSymbolFont &-pssymbolfn &Symbol &{\rm
Font for Greek characters and mathematical symbols in PostScript}\\

psAsciiFont &-psasciifn &Times-Roman &{\rm
Font for ASCII characters in graphical PostScript formulae}\\

ignoreTeXFonts &-texfns, -notexfns &False &{\rm
Whether to edit out fonts with names resembling \TeX-style {\tt tfm} fonts}\\

useGreekFont &-usegreek, -usenogreek &True &{\rm
Whether to use the Greek font (or instead to write names of Greek chars)}\\

useSymbolFont &-usesymbol, -usenosymbol &True &{\rm
Whether to use the Symbol font (or draw mathematical chars by hand)}

\end{tabular}
}

Resources marked in the Defaults column with `(*)' default to
wildcarded XLFD font names, chosen to provide a reasonable look on our
own displays.  The default names will be reported if they appear to be
absent from your system (in which case choosing different values might
be wise), and can be inspected using the built-in Font Selector.
}

\subsection{Timing and Buffering Resources}

\vskip 6pt
{\footnotesize\tt
\begin{tabular}{l p{0.8in} l p{2.5in}}
{\it Resource} &{\it Option} &{\it Default} &{\it Description}\\
&&&\\

bracketFlashTime &-bflashtime &1000 &{\rm
Time (ms) for which to flash brackets when matching}\\

multiClickTime &-mclicktime &500 &{\rm
Time (ms) within which to accept a second mouse click as multi-clicking}\\

inputBufferSize &-ibuf &10000 &{\rm
Currently unused}\\

outputBufferSize &-obuf &8000 &{\rm
Char size of \REDUCE{} to \xr\ buffer}\\

outputOnDelay &-ondelay &30 &{\rm
Length (ms) of \xr\ buffer polls}\\

outputOffDelay &-offdelay &100 &{\rm
Delay (ms) between \xr\ buffer polls}
\end{tabular}
}

\subsection{Miscellaneous Resources}

\vskip 6pt
{\footnotesize\tt
\begin{tabular}{l p{1in} l p{2.5in}}
{\it Resource} &{\it Option} &{\it Default} &{\it Description}\\
&&&\\

matchBrackets &-bmatch, -bnomatch &True &{\rm
Whether to blink matching brackets and beep for unmatched brackets}\\

forceBracketMatch &-bforcematch, -bnoforcematch &False &{\rm
Whether to prevent the user from typing unmatched brackets}\\

reverseVideo &-reverse, -forward &False &{\rm
Display formulae in reverse video; essential on some demented servers}\\

pageMode &-pagemode, -nopagemode &False &{\rm
Whether to use Page Mode by default}\\

pointerJump &-pointerjump, -nopointerjump &True &{\rm
Whether to try to warp the pointer to default selections in Question boxes}
\end{tabular}
}

\subsection{Options Without Corresponding Resources}

\vskip 6pt
{\footnotesize
\begin{tabular}{p{1.2in} p{4in}}
{\it Option}&{\it Description}\\
&\\

{\tt-quiet} &Causes \xr\ to ignore any unrecognised command-line
options, passing them on to the \REDUCE\ process instead, suppress
inessential notification messages and keep all error messages brief.\\

{\tt-continue} &Causes \xr\ to continue processing despite complaining
about any unrecognised command-line options, and to pass them on to
\REDUCE.\\

{\tt-despair} &Causes \xr\ to give up in hopeless disgust when
confronted with an unknown command-line option.  This is the default
behaviour.\\

{\tt-writehelpix} &Causes \xr\ to start, process the Help Info file,
produce a corresponding Index file (replacing the Info file's suffix
with the suffix {\tt .hnx}), deposit it in the same place as the Info
file and quit; still requires that the {\tt\$DISPLAY} be set and
accessible.\\

{\tt-query, -q, -?} &Causes \xr\ to print a table of known options and
exit.
\end{tabular}
}

\newpage
\section{Installation}

\xr\ has been tested to some extent on the following machines: Sun-3
and Sun-4 under SunOS, Sun Solaris, DEC Alpha under OSF/1, DG Aviion
under DG/UX, HP-700 under HP-UX, and IBM RS6000 under AIX, and IBM PC
and compatibles under linux.

\subsection{Installing \xr\ from a distribution
executable}\label{executable}

Most distributions of \xr\ are supplied with a ready-compiled
executable.  Such versions are normally supplied in a directory tree
called {\tt xr}.  If your \xr\ is supplied as part of a \REDUCE\
distribution, its top-level directory should already be
{\tt\$reduce/xr}, where {\tt\$reduce} is the top-level directory of
the \REDUCE\ tree; otherwise the top-level \xr\ directory must be
moved into {\tt\$reduce} before use.  The \xr\ executable is then
found in {\tt\$reduce/xr/bin/xr}.

Before using \xr, its X resource file, supplied as
{\tt\$reduce/xr/XR}, must be copied (with the same name) to a
directory where the X resource manager will be able to find it when
\xr\ starts up.  \xr\ itself has no control over where to look for the
file.  The X resource manager is supposed to look in the following
directories for this file (listed in reverse order of precedence ---
later take precedence over earlier):

\begin{itemize}
\item each directory in the path specified in the environment variable
{\tt\$XFILESEARCHPATH};
\item the directory specified by the environment variable {\tt\$HOME};
\item the directory specified by the environment variable
{\tt\$XAPPLRESDIR};
\item each directory in the path specified in the environment variable
{\tt\$XUSERFILESEARCHPATH}.
\end{itemize}

Taking precedence over all these are any resources which were read
from the contents of the following three files.  These, however, are
read not at \xr\ startup but at the server's startup, or whenever you
reread them with {\tt xrdb}.  If you're really desperate, you could
append the contents of the \xr\ resource file to one of these, and
then run {\tt xrdb {\it filename\/}}:
\begin{itemize}
\item {\tt\$HOME/.Xdefaults}
\item {\tt\$HOME/.Xdefaults-{\it hostname\/}} (noting that the
hostname is that of the \xr\ host, not the server);
\item {\tt\$XENVIRONMENT} (which should be a file, not a directory).
\end{itemize}

Bear in mind that if you include the resources in one of the files in
{\tt\$HOME}, each user of \xr\ will require their own copy of or link
to the resource file.

As an aside, there are two further informational files used by \xr,
although neither is strictly necessary and neither should need
moving before \xr\ can be run.  These are
{\tt\$reduce/xr/doc/packages}, which enumerates the available
\REDUCE\ fast load packages, and {\tt\$reduce/xr/doc/defaults}, which
lists the possible locations to which the Font Selector can write new
font resources.  You can change both if you wish, but if their format
isn't obvious to you, you probably shouldn't be fiddling about with
them\footnote{The final field in each line of the defaults list file
is redundant, but must be present.}.  User-specific copies of these
two files may, if desired, be kept in {\tt\$HOME/.xrpackages} and
{\tt\$HOME/.xrxdefaults} respectively; these take precedence over the
system-wide copies.

Once the {\tt XR} file has been moved to a suitable directory, it
should be possible to fire up \xr\ by executing
{\tt\$reduce/xr/bin/xr}.

\subsection{Installing \xr\ from sources}

Source installation can currently be rather more complicated than it
need be, particularly if for some reason the heuristics in the
provided automated installation script fail for your system's setup,
or if you have to make extensive changes to the Makefile.

At the moment there are, for most systems, these basic steps.

\begin{enumerate}

\item Ensure that the {\tt Makefile} is accurate for your system,
particularly with regard to the {\tt CC}, {\tt CFLAGS} and {\tt
LDFLAGS} variables.  For many systems this is the most tricky bit of
the installation.  Note that all the C code in \xr\ is pure K\&R, and
that it refers to one or two BSD functions, such as {\tt
socketpair(2)}, and to the {\tt m} library.  Most of the MIT
distribution X libraries, or your vendor's equivalent, are needed
(X11, Xt, Xaw, Xmu and Xext are a likely selection), but \xr\ makes no
use of proprietary widget code such as the OSF's Motif or Sun OLIT.
Note that for the Solaris you will probably need to define the {\tt
SOLARIS} flag, and whether or not you can define {\tt
HAVE\_VIEWPORT\_SET\_FUNCTIONS} will depend on the X libraries.  Try
it and see.

\item Whilst in the directory {\tt\$reduce/xr/src}, run the
shell-script {\tt ./Install.sh}.  This script attempts to do several
things: produce the suitably-modified installation files {\tt XR} and
{\tt xr.red} from their {\tt .orig} equivalents; {\tt make} the C
code; run \REDUCE\ to compile the RLisp code; run \xr\ to create a
Help Index file ({\tt rhelp.hnx}); and move the files to the correct
directories.  In order to do all this, it requires several questions
to be answered by the installer, and for best results the environment
variable {\tt\$reduce} should be correctly set before running it.  It
also helps if the {\tt\$DISPLAY} variable is set to an openable X
display.

\item Copy the file {\tt XR} (created by the shell-script) to a
directory such as {\tt\$HOME} or {\tt\$XAPPLRESDIR}, as described in
section \ref{executable} above, where it will be found by the X
resource manager when \xr\ is run.  \end{enumerate}

\section{Controlling {\tt XR's} REDUCE output}

The interconnection of LISP symbols to the {\tt XR} print style 
is realized by the properties {\tt fancy!-special!-symbol} and
{\tt fancy!-pprifn}, which maps a symbol to a special character, e.g.
\begin{verbatim}
prop 'kappa;
           ((fancy!-special!-symbol . "\kappa")) 
\end{verbatim}
or to a special printfunction, e.g.
\begin{verbatim}
prop 'meijerg;

            ((fancy!-prifn . fancy!-meijerg)) 
\end{verbatim}

For controlling the output produce by  {\tt XR} there are some additional
facilities:
\begin{itemize}
\item  the share variable $fancy\_print\_df$ can be set to one of
  the values $PARTIAL$, $TOTAL$ or $INDEXED$; this influences the
  printing of differential expressions. E.g. $df(f,x,2,y)$ will be
  printed as
  \begin{itemize}
  \item{$PARTIAL$}: $\frac{\partial ^3 f}{\partial x^2 \partial y}$
  \item{$TOTAL$}: $\frac{d ^3 f}{d x^2 d y}$
  \item{$INDEXED$}: $f_{xxy}$
  \end{itemize}
\item An operator can be declared ``print\_indexed"; its arguments
   then will be printed as indexes. E.g.
   \begin{verbatim}
      print_indexed a;
   \end{verbatim}
   then $a(i,2)$ will be printed as $a_{i 2}$
\item When the switch $fancy\_tex$ is set $ON$ the internal form
   is displayed instead of the typsetting form.
\end{itemize}

\newpage
\section{Problems}

\subsection{Error Messages}

\subsubsection{General \xr\ Errors}

The following table lists some of the error messages \xr\ may produce,
excluding those generated by the Help module. If the {\tt -quiet}
command-line option is used, some of these may be shortened or
suppressed.

\begin{description}

\item[{\tt Unknown command line option~{\it x\/}}]\hskip 0.6pc\relax
You have given \xr\ a command-line option it doesn't recognise.  If
this was intentional and you only wanted the option passed on to the
\REDUCE{} process, you should use the {\tt -continue} or {\tt -quiet}
option as well.  You can obtain a list of recognised options by using
the {\tt -query} option.

\item[{\tt Cannot find X defaults file}]\hskip 0.6pc\relax Wherever
the {\tt XR} X resource file was placed in the installation process,
the X resource manager didn't find it; see section \ref{executable}
above.  This error also arises if the resources are found, but fail to
set the {\tt foundDefaults} resource to {\tt True}; if this was the
case, you'd probably know.

\item[{\tt Error in resource specifications:~{\it x\/}}]\hskip
0.6pc\relax An invalid value has been given in one of \xr's resources
or on the command line.  The {\it x\/} will probably indicate what it
was.

\item[{\tt No Reduce binary {\it (or {\tt image}) x\/}}]\hskip
0.6pc\relax The \REDUCE\ binary (or image) file given by the {\tt
reducePath} and {\tt reduceName} (or {\tt reduceImageFile}) resources
is absent or unreadable.  Check the given values in the X~resource
file; the error message should tell you where \xr{} is looking.

\item[{\tt Could not open directory~{\it x}}]\hskip 0.6pc\relax The
directory in which \xr\ looks for the \REDUCE\ or RLisp executable
doesn't exist or can't be opened.  Check the specification of the {\tt
reducePath} resource in the X~resource file.

\item[{\tt Could not open Reduce initialisation file}]\hskip
0.6pc\relax The file given in the {\tt reduceInitFile} resource or as
the argument to the {\tt -rinitfile} command line option doesn't exist
or is lacking read permission.

\item[{\tt Couldn't get value of environment variable~{\it x}}]\hskip
0.6pc\relax The specified environment variable is not set, or is not
marked for export.  If you think the variable shouldn't be needed,
check the \xr\ initialisation and default files for references to it.

\item[{\tt Could not execve()}]\hskip 0.6pc\relax \xr\ is for some
reason unable to execute the \REDUCE\ executable.  Check that it does
actually work.

\item[{\tt Caught a Terminate signal}]\hskip 0.6pc\relax \xr\ has been
sent a Terminate signal; this sometimes happens at startup if your
system has run out of processes or memory and cannot run both \xr\ and
\REDUCE.  There's not much I can do about that case.

\item[{\tt Reduce process has exited;~tidying up and leaving}]\hskip
0.6pc\relax This is produced when the user types {\tt bye} or {\tt
quit} into the \REDUCE{} window; it may also appear if for some reason
the initial execution of the \REDUCE\ process fails.  (This often
happens when the \REDUCE\ executable is a shell script which is
lacking an opening line such as {\tt \#!/bin/csh} to tell {\tt
execve(2)} to run it in a shell.  If you get such an error and can't
work it out, please do ask.)

\item[{\tt X Warning:~{\it x\/} on display {\it y\/}}]\hskip
0.6pc\relax This is a report of a failed request on the X server.  If
it is reproducible, it's probably a bug; unfortunately, sometimes
these errors are caused by side conditions and are not directly
traceable to \xr.

\item[{\tt My relationship with the X server has been cruelly
severed}]\hskip 0.6pc\relax This is normally caused by the use of a
window manager's `kill' function or the {\tt xkill} utility on \xr.

\item[{\tt No font~{\it x\/}, trying~{\it y}}]\hskip 0.6pc\relax A
font has been requested that isn't available on your server --- even
though it may be listed as available.

\item[{\tt Couldn't load default font}]\hskip 0.6pc\relax The font
(normally {\tt\verb+"+fixed\verb+"+}) that \xr\ uses as a default when
it can't find one of the specified ones is apparently absent from the
server.

\item[{\tt Can't locate font~{\it x\/}; trying~{\it y\/}}]\hskip
0.6pc\relax This sometimes results from a known (but persistent) bug
in the Font Selector.

\item[{\tt Can't locate default font.  This is probably my fault;
sorry}]\hskip 0.6pc\relax This arises from the same bug, and really
should not happen.

\item[{\tt Can't locate font alias~{\it x\/}; using~{\it y\/}}]\hskip
0.6pc\relax Probably a bug, though normally a fairly benevolent one.

\item[{\tt Too many font families}]\hskip 0.6pc\relax Your font server
is too well-equipped.

\item[{\tt Normal-size Greek characters will be in non-Greek
font}]\hskip 0.6pc\relax Errors like this indicate that the Greek
character font specified in the resources could not be loaded.  Either
specify a different Greek font (using the Font Selector or by changing
the resource file), turn off the `Use Greek Fonts' option (using the
Option Box or the resource file), or put up with the ensuing ugly
output.

\item[{\tt No symbol font:~drawing my own mathematical symbols}]\hskip
0.6pc\relax The font specified for mathematical symbols in the
resources could not be loaded.  Either specify a different symbol
font, or put up with \xr's own attempts at drawing mathematical
characters (in which case you may wish to set the {\tt useSymbolFont}
resource to {\tt False}).  \end{description}

If you see any other (apparently serious, or fatal) error message, or
if you are driven to the brink of madness by any of the above ones,
please report the problem.

\subsubsection{Help Module Errors}

There are several non-fatal errors which can be produced by the Help
module.  Most of these will be accompanied by further explanation
messages; some of the more likely or nasty ones are:

\begin{description} \item[{\tt Cannot open help file}]\hskip
0.6pc\relax The help file specified in the resources is absent, or has
no read permission.  Check your X~resource file.

\item[{\tt Cannot open the help index file}]\hskip 0.6pc\relax The
help file appears not to be accompanied by a readable index ({\tt
.hnx}) file. This is probably a problem in installation; it means that
the Search facility will be absent from the Help module.

\item[{\tt Could not read TOC location from index file}]\hskip
0.6pc\relax The index file is apparently wrong, though not fatally.
It will still be used for searching, though it's possible that the
results may be inaccurate.  You could try running \xr\ with the {\tt
-writehelpix} option to remake the index file; if this doesn't put it
right, report the problem.

\item[{\tt Unexpected EOF in help file~{\it and\/} Error reading help
file}]\hskip 0.6pc\relax The help info file is corrupted.  Complain.

\item[{\tt No such node as~{\it x\/}}]\hskip 0.6pc\relax The help info
file is corrupted or wrong.  Complain.

\item[{\tt Unknown or unexpected node tag~{\it x\/}}]\hskip
0.6pc\relax The help info file is corrupted or wrong.  Complain.

\item[{\tt Too many nodes in helpfile}]\hskip 0.6pc\relax The help
info file is simply too big.  This should currently not appear, as the
distributed help file is known to be within reasonable limits.

\item[{\tt Maximum text per page exceeded}]\hskip 0.6pc\relax A help
page is too long.  If you get this, please report it.

\item[{\tt Can't install Help facility}]\hskip 0.6pc\relax It's about
to give up, after some other errors were caught.  If this appears
without any other warnings preceding it, something is being strange
somewhere.

\end{description}

\subsection{Bug List}
At the time of writing, known bugs are several and rampant, and
include:

\begin{itemize}

\item On some versions\footnote{Actually those compiled on machines
whose installation of X precludes using the preprocessor flag {\tt
HAVE\_VIEWPORT\_SET\_FUNCTIONS} --- if you can compile with this flag,
it's probably better to do so.}, the scrollbar on the output history
window doesn't quite sit at the bottom, and is rather inaccurate; also
the font name highlighting is sometimes wrong.

\item The Solaris version will probably spew lots of errors, fail to
display output and probably eventually crash unless the -DSOLARIS flag
is set during compilation.  Note that it is safe to leave this flag
defined when compiling on other machines, if you wish.

\item Behaviour of the Font Selector is shaky.  Depending upon your
machine and the fonts available on your display server, attempting to
call up the Font Selector may simply fail.  It is, however, more
reliable than it used to be.

\item Whether or not Interrupt works depends entirely on how \REDUCE\
handles Interrupt signals on your machine's version.  Here it works on
most versions, abandons the whole \REDUCE{}/\xr\ setup on the HP (not
the most friendly of behaviour), and is ignored on the IBM RS.

\item It is sometimes possible to get \xr\ out of sync by strange use
of Interrupt or by over-zealous use of Page; the effect is usually to
cause commands to appear one prompt later than they should.  This is
very annoying.

\item Very long textual output lines may disappear off the right side
of the window, and --- unlike graphical output --- cannot be viewed no
matter how wide the window is.

\item Some stupid display hardware requires that the Reverse Video
switch be used; this is probably beyond my control.

\item Page Mode very often crashes, or at least gives dozens of
warnings.

\item There may be lapses in screen updating when the size of the
input or output font is changed.

\item Because the Print option just transcribes the user input and
actual returned result (ie. workspace value) for each prompt, it will
not reproduce output from reading in files, or from \REDUCE\ {\tt
write} commands.

\item The PostScript output produced by the Print option is flaky.
In particular, it doesn't check for the output's running off the right
hand side of the page.  Page numbering might also be desirable.

\item \LaTeX{} output has a few bugs; for instance, it may
occasionally produce nonexistent macro codes.  Generally the \LaTeX{}
is less than beautiful, and although it should get better, at the
moment you'll almost certainly need to do some fine tuning on the
results.  Comments on how to improve the \LaTeX{} output are welcome.

\end{itemize}

\subsection{Contacts}

If you find any interesting new bugs in \xr , or if there is any
general or particular aspect of the program about which you want to
complain, please feel free to contact Herbert Melenk at one of the
following addresses.
\begin{description}

\item[Electronic Mail] \ {\tt melenk@sc.ZIB-Berlin.DE}
\item[Real, Genuine Tangible Mail] \
\\Konrad-Zuse-Zentrum \\
\\Symbolik\\
\\Takustrasse 7\\
\\D-14195 Berlin-Dahlem\\
Fed. Rep. of Germany

\end{description}

\xr\ was written mostly by Chris Cannam and Herbert Melenk at ZIB
Berlin.  This document was prepared by Chris Cannam, updated by Winfried Neun
in October 1993 and October 1996.

\end{document}

